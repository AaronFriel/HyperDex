\chapter{Asynchronous Operations}
\label{chap:async-ops}

Thus far, all HyperDex operations we've seen are {\em synchronous}, that is, the
application calls the HyperDex operation and blocks until the operation is
complete.  HyperDex is very fast, but even so, clients may spend a non-trivial
amount of time waiting on HyperDex operations.  For each operation in HyperDex,
there is an {\em asynchronous} variant that accomplishes the same task, but
allows the client to perform other work while the operation completes.

In this chapter, we'll look at the asynchronous operations HyperDex supports,
and see different ways to use them to increase concurrency in the system.

\section{Setup}
\label{sec:async-ops:setup}

As in the previous chapters, the first step is to deploy the cluster and connect
a client.   First we launch and initialize the coordinator:

\begin{consolecode}
hyperdex coordinator -f -l 127.0.0.1 -p 1982
\end{consolecode}

Next, let's launch a few daemon processes to store data.  Execute the following
commands (note that each instance binds to a different port and has a different
\code{/path/to/data}):

\begin{consolecode}
hyperdex daemon -f --listen=127.0.0.1 --listen-port=2012 \
                   --coordinator=127.0.0.1 --coordinator-port=1982 --data=/path/to/data1
hyperdex daemon -f --listen=127.0.0.1 --listen-port=2013 \
                   --coordinator=127.0.0.1 --coordinator-port=1982 --data=/path/to/data2
hyperdex daemon -f --listen=127.0.0.1 --listen-port=2014 \
                   --coordinator=127.0.0.1 --coordinator-port=1982 --data=/path/to/data3
\end{consolecode}

We now have three different daemons ready to serve in the HyperDex cluster.
Finally, we create a space which makes use of all three systems in the cluster.
In this example, let's create a space that may be suitable for storing friend
lists in a social network:

\begin{pythoncode}
>>> import hyperdex.admin
>>> a = hyperdex.admin.Admin('127.0.0.1', 1982)
>>> a.add_space('''
... space friendlists
... key username
... attributes
...    string first,
...    string last,
...    set(string) friends
... ''')
True
>>> import hyperdex.client
>>> c = hyperdex.client.Client('127.0.0.1', 1982)
\end{pythoncode}

Finally, our object for John Smith and some others

\begin{pythoncode}
>>> c.put('friendlists', 'jsmith1', {'first': 'John', 'last': 'Smith'})
True
>>> c.put('friendlists', 'jd', {'first': 'John', 'last': 'Doe'})
True
>>> c.put('friendlists', 'bjones1', {'first': 'Brian', 'last': 'Jones'})
True
\end{pythoncode}

\section{Asynchronous Operations}
\label{sec:async-ops:ops}

Asynchronous operations permit applications to retrieve or modify multiple
objects simultaneously and to perform local computation while doing the same.
In previous chapters, we submitted synchronous operations to the key-value
store, where each client had just a single outstanding request, and waited
patiently for that request to complete.  In high-throughput applications,
clients may have a bactch of operations that may be performed simultaneously.
The standard practice in such cases is to issue *asynchronous* operations, where
the client does not immediately wait for each individual operation to complete.
HyperDex has a very versatile interface for supporting this use case.

Asynchronous operations separate the request and response portions of a single
operation into two separate parts.  Each asynchronous operation returns a small
token that identifies the outstanding operation, which can then be used by the
client, if and when needed, to wait for the completion of the selected
operation.

Every API method covered in the tutorials so far (e.g. \code{get}) has a
corresponding asynchronous version, usually prefixed with \code{async\_} (e.g.
\code{async\_get}), for performing asynchronous operations.  Those without an
\code{async\_} prefix are natively asynchronous.  The basic pattern of usage for
asynchronous operations is:

 * Initiate the asynchronous operation
 * Do some work and perhaps issue more operations, async or otherwise,
 * Wait for selected asynchronous operations to complete

This enables the application to continue doing other work while HyperDex
performs the requested operations.  Here's how we could insert an object for
user John Jackson asynchronously:

\begin{pythoncode}
>>> d = c.async_put('friendlists', 'jj', {'first': 'John', 'last': 'Jackson'})
>>> d
<hyperdex.client.Deferred object at ...>
>>> # do some work
>>> d.wait()
True
>>> d = c.async_get('friendlists', 'jj')
>>> d.wait()
{'first': 'John', 'last': 'Jackson', 'friends': set([])}
\end{pythoncode}

Notice that the return value of the first \code{d.wait()} is \code{True}.  This
is the same return value that would have come from performing \code{c.put(...)},
except the client was free to do other computations while HyperDex servers were
processing the \code{put} request.  Similarly, the second asynchronous
operation, \code{async\_get}, queues up the request on the servers, frees the
client to perform other work, and yields its results only when \code{wait} is
called.  In fact, the Python bindings implement all synchronous operations using
their asynchronous equivalents.  For example, here's a sample definition of
\code{get}:

\begin{pythoncode}
>>> def get(client, space, key):
...     return client.async_get(space, key).wait()
...
>>> get(c, 'friendlists', 'jj')
{'first': 'John', 'last': 'Jackson', 'friends': set([])}
\end{pythoncode}

By itself, an asynchronous operation is not very useful if it is waited on right
away because that makes it equivalent to a synchronous operation.  The true
power comes from requesting multiple concurrent operations.  For example, to
establish a bidirectional friendship between John Smith and John Jackson:

\begin{pythoncode}
>>> d1 = c.async_set_add('friendlists', 'jj', {'friends': 'jsmith1'})
>>> d2 = c.async_set_add('friendlists', 'jsmith1', {'friends': 'jj'})
>>> d1.wait()
True
>>> d2.wait()
True
\end{pythoncode}

Note that the order in which operations are waited on does not matter.  We could
just as easily execute them in a different order, and still get the desired
effect.  Similarly, we could concurrently add multiple friends for John Smith:

\begin{pythoncode}
>>> d1 = c.async_set_add('friendlists', 'jsmith1', {'friends': 'bjones1'})
>>> d2 = c.async_set_add('friendlists', 'bjones1', {'friends': 'jsmith1'})
>>> d3 = c.async_set_add('friendlists', 'jsmith1', {'friends': 'jd'})
>>> d4 = c.async_set_add('friendlists', 'jd', {'friends': 'jsmith1'})
>>> d1.wait()
True
>>> d2.wait()
True
>>> d3.wait()
True
>>> d4.wait()
True
\end{pythoncode}

This allows for powerful applications.  Going a step further, HyperDex allows a
client to wait for the next operation to complete, without specifying an order
among asynchronous operations.  For instance, fetching the first names of every
friend of John can be done in parallel:

\begin{pythoncode}
>>> friends_usernames = c.get('friendlists', 'jsmith1')['friends']
>>> outstanding = set()
>>> for username in friends_usernames:
...     outstanding.add(c.async_get('friendlists', username))
...
>>> friends = []
>>> while outstanding:
...     d = c.loop()
...     outstanding.remove(d)
...     friend = d.wait()['first']
...     friends.append(friend)
...
>>> sorted(friends)
['Brian', 'John', 'John']
\end{pythoncode}

Using the \code{loop()} method, it is possible to issue thousands of requests
and then wait for each one in turn without having to serialize the round trips
to the server.

\section{Potential Pitfalls}
\label{sec:async-ops:pitfalls}

Keep in mind that HyperDex may (and will!) maximize performance by executing
concurrent asynchronous operations in any order.  If a client wants to ensure
that an operation B is performed only after asynchronous operation A is
committed to the data store, that client needs to call wait() on A. When wait()
returns success, the client is guaranteed that HyperDex has committed the data
to sufficiently many replicas to tolerate the number of failures in the space
definition.

On a related note, a client that has not explicitly performed a wait() on an
outstanding asynchronous operation should not assume anything about the
disposition of those operations. Specifically, the queued asynchronous
operations may not have made it to any of the servers, and therefore may not be
committed anywhere. A client that simply issues asynchronous requests and
terminates without calling wait() on them is not guaranteed to have any of those
asynchronous operations execute. Unless the system explicitly tells a client
that the data is committed, it is not safe to assume that it will be committed
behind the scenes. (The flip-side of this is that, when HyperDex says the data
is committed, it really is committed to sufficiently many replicas to withstand
the number of simultaneous failures in the space specification).

\section{A Common Window Pattern}
\label{sec:async-ops:window}

The last point necessitates a common pattern, where a client will want to keep a
fixed-size window of outstanding requests so as to not issue too many operations
concurrently, and will appropriately wait for all asynchronous operations to
complete. Such code will look, approximately, like this:

 \begin{pythoncode}
c = hyperclient.Client(...)
outstanding = 0
for line in file:
    while outstanding >= 1024: # 1024 is window size
        d = c.loop()
        d.wait()
        outstanding -= 1
    parse line and issue put
    outstanding += 1
# flush remaining
while outstanding > 0:
    d = c.loop()
    d.wait()
    outstanding -= 1
\end{pythoncode}
