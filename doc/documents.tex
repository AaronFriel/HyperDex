\chapter{Document Storage}
\label{chap:documents}

Thus far, we have assumed that you know the structure of your data in advance.
What do you do when you don't know what your data looks like in advance?
HyperDex provides a document type for just this scenario.  Documents are JSON
objects, stored within HyperDex, that can be updated and queried just like
regular objects.

Let's see how to use documents in practice.  The setup below is similar to the
previous chapter, so if you have a running cluster, you can skip to the space
creation step.

\section{Setup}
\label{sec:documents:setup}

As in the previous chapters, the first step is to deploy the cluster and connect
a client.   First we launch and initialize the coordinator:

\begin{consolecode}
hyperdex coordinator -f -l 127.0.0.1 -p 1982
\end{consolecode}

Next, let's launch a daemon process to store data.  Execute the following
command:

\begin{consolecode}
hyperdex daemon -f --listen=127.0.0.1 --listen-port=2012 \
                   --coordinator=127.0.0.1 --coordinator-port=1982 --data=/path/to/data
\end{consolecode}

We now have a HyperDex cluster ready to serve our data.  Finally, we create a
space which makes use of the cluster.  In this example, let's create a space
that may be suitable for storing social network profiles.

\begin{pythoncode}
>>> import hyperdex.admin
>>> a = hyperdex.admin.Admin('127.0.0.1', 1982)
>>> a.add_space('''
... space profiles
... key username
... attributes
...    document profile
... ''')
True
>>> import hyperdex.client
>>> c = hyperdex.client.Client('127.0.0.1', 1982)
\end{pythoncode}

\section{Working with Documents}

It's easy to see how documents enable a wide array of applications.  Consider a
social networking application that stores each user's profile as a document in
HyperDex.  Users which provide very little information to the social network
will have a relatively sparse profile like this:

\begin{jsoncode}
{"name": "John Smith"}
\end{jsoncode}

You can store this simple document almost exactly like you would store any other
data in HyperDex.

\begin{pythoncode}
>>> Document = hyperdex.client.Document
>>> c.put('profiles', 'jsmith1', {'profile': Document({"name": "John Smith"})})
True
\end{pythoncode}

Of course, nothing prevents users from having more-complex versions of the
profile like so:

\begin{pythoncode}
>>> c.put('profiles', 'jd', {'profile': Document({
...     "name": "John Doe",
...     "www": "http://example.org",
...     "email": "doe@example.org",
...     "friends": ["John Smith"]
... })})
True
\end{pythoncode}

You can search over documents just like you can search over regular HyperDex
attributes.  For example, to retrieve the objects for all people named John Doe,
you can do a search on \code{profile.name} to retrieve all such objects:

\begin{pythoncode}
>>> print [x for x in c.search('profiles', {'profile.name': 'John Doe'})]
[{'username': 'jd', 'profile': Document({"www": "http://example.org", "friends":
["John Smith"], "name": "John Doe", "email": "doe@example.org"})}]
\end{pythoncode}

You can even do HyperDex's more-complex queries.  For example, to find everyone
whose name starts with ``John'', you can do a regular expression search like
this:

\begin{pythoncode}
>>> print [x for x in c.search('profiles', {'profile.name': hyperdex.client.Regex('John Doe')})]
[{'username': 'jd', 'profile': Document({"www": "http://example.org", "friends":
["John Smith"], "name": "John Doe", "email": "doe@example.org"})}]
\end{pythoncode}

\section{Indexing Documents}

Documents, by their very nature, impose no structure on the data they contain.
Your application is free to store any JSON it wants as a document, and HyperDex
will search it just fine.  Sometimes though, your documents {\em do} have some
structure, and your queries will often look at the same elements of the
documents.  For these situations, it is possible to construct an index on the
document that significantly speeds up many popular search queries.

With our example queries above, we could create an index on the
\code{profile.name} attribute.  This index may be consulted by equality and
range searches that include the \code{profile.name} element.

To create this index, use the \code{add-index} command, specifying the space
``profiles'' and the attribute ``profile.name''.

\begin{consolecode}
hyperdex add-index profiles profile.name
\end{consolecode}

This will instruct HyperDex to add the new index, and will index all existing
data using the new index.  You can add and remove indices at any time as your
documents change.
