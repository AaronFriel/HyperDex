\chapter{Data Types}

% .. _datatypes:
% 
% Datatypes
% =========
% 
% As we saw in the previous chapter, HyperDex offers support for the basic
% ``get``, ``put``, and ``search`` operations on strings and integers.  In this
% chapter, we explore HyperDex's richer datatypes and atomic operations on lists,
% sets, and maps.  We will see how providing efficient atomic operations on these
% rich, native datastructures greatly simplifies design for applications with
% complicated data layout requirements.
% 
% By the end of this chapter you'll be familiar with all datatypes provided by
% HyperDex, and a number of atomic operations.
% 
% Setup
% -----
% 
% As in the previous chapter, the first step is to deploy the cluster and connect
% a client.   First we launch and initialize the coordinator:
% 
% .. sourcecode:: console
% 
%    $ hyperdex coordinator -f -l 127.0.0.1 -p 1982
% 
% Next, let's launch a few daemon processes to store data.  Execute the following
% commands (note that each instance binds to a different port and has a different ``/path/to/data``):
% 
% .. sourcecode:: console
% 
%    $ hyperdex daemon -f --listen=127.0.0.1 --listen-port=2012 \
%                         --coordinator=127.0.0.1 --coordinator-port=1982 --data=/path/to/data1
%    $ hyperdex daemon -f --listen=127.0.0.1 --listen-port=2013 \
%                         --coordinator=127.0.0.1 --coordinator-port=1982 --data=/path/to/data2
%    $ hyperdex daemon -f --listen=127.0.0.1 --listen-port=2014 \
%                         --coordinator=127.0.0.1 --coordinator-port=1982 --data=/path/to/data3
% 
% This brings up three different daemons ready to serve in the HyperDex cluster.
% Finally, we create a space which makes use of all three systems in the cluster.
% In this example, let's create a space that may be suitable for storing profiles
% in a social network:
% 
% .. sourcecode:: console
% 
%    >>> import hyperclient
%    >>> c = hyperclient.Client('127.0.0.1', 1982)
%    >>> c.add_space('''
%    ... space profiles
%    ... key username
%    ... attributes
%    ...    string first,
%    ...    string last,
%    ...    int profile_views,
%    ...    list(string) pending_requests,
%    ...    set(string) hobbies,
%    ...    map(string, string) unread_messages,
%    ...    map(string, int) upvotes
%    ... subspace first, last
%    ... subspace profile_views
%    ... ''')
% 
% Finally, let's create a profile for John Smith that we can use in the rest of
% this chapter.
% 
% .. sourcecode:: pycon
% 
%    >>> c.put('profiles', 'jsmith1', {'first': 'John', 'last': 'Smith'})
%    True
%    >>> c.get('profiles', 'jsmith1')
%    {'first': 'John', 'last': 'Smith',
%     'profile_views': 0,
%     'pending_requests': [],
%     'hobbies': set([]),
%     'unread_messages': {},
%     'upvotes': {}}
% 
% Strings
% -------
% 
% The basic datatype in HyperDex is a byte string.  If you don't specify the type
% of an attribute when creating a space, it is automatically treated as an 8-bit
% bytestring.  This means that you'll have to encode and decode unicode strings as
% appropriate.  For example, if John wanted to add an accent to his name on his
% social network page, the code could look like:
% 
% .. sourcecode:: pycon
% 
%    >>> c.put('profiles', 'jsmith1', {'first': u'Jóhn'.encode('utf8')})
%    True
% 
% This encodes the string to raw bytes using UTF-8.  When fetching his profile it
% is necessary to decode the UTF-8:
% 
% .. sourcecode:: pycon
% 
%    >>> c.get('profiles', 'jsmith1')['first']
%    b'J\xc3\xb3hn'
%    >>> c.get('profiles', 'jsmith1')['first'].decode('utf8')
%    'Jóhn'
%    >>> c.put('profiles', 'jsmith1', {'first': 'John', 'last': 'Smith'})
%    True
% 
% Of course, it's always possible to change John's name back to its unaccented
% form:
% 
% .. sourcecode:: pycon
% 
%    >>> c.put('profiles', 'jsmith1', {'first': 'John', 'last': 'Smith'})
%    True
%    >>> c.get('profiles', 'jsmith1')['first']
%    'John'
% 
% HyperDex knows nothing about encodings, so it is up to the application to encode
% or decode data appropriately.
% 
% Integers
% --------
% 
% As we've already seen, HyperDex supports ``get`` and ``put`` operations on
% integers.  In addition to these basic operations, HyperDex provides atomic
% opertions to manipulate integers using basic math operations.  This is useful
% when implementing features such as page-view counters.  Let's add support for
% tracking the profile_views of a page by incrementing the counter:
% 
% .. sourcecode:: pycon
% 
%    >>> c.atomic_add('profiles', 'jsmith1', {'profile_views': 1})
%    True
%    >>> c.get('profiles', 'jsmith1')
%    {'first': 'John', 'last': 'Smith',
%     'profile_views': 1,
%     'pending_requests': [],
%     'hobbies': set([]),
%     'unread_messages': {},
%     'upvotes': {}}
% 
% Note that this change required just one request to HyperDex.  The server
% atomically examines the current value, and changes it by the amount specified.
% In this case, the ``profile_views`` attribute is incremented by one.
% 
% HyperDex supports a full range of basic operations including 
% :py:meth:`hyperclient.Client.atomic_add`,
% :py:meth:`hyperclient.Client.atomic_sub`,
% :py:meth:`hyperclient.Client.atomic_mul`,
% :py:meth:`hyperclient.Client.atomic_div`,
% :py:meth:`hyperclient.Client.atomic_mod`,
% :py:meth:`hyperclient.Client.atomic_and`,
% :py:meth:`hyperclient.Client.atomic_or`, and
% :py:meth:`hyperclient.Client.atomic_xor`.
% 
% Floats
% ------
% 
% HyperDex also supports double-precision floating point types.  Like integers,
% floats support range searches and atomic operations.
% 
% Lists
% -----
% 
% Let's add support for friend requests using HyperDex lists the basis of the
% feature.  For this we'll use the ``pending_requests`` attribute in the
% ``profiles`` space.
% 
% Imagine that shortly after joining, John Smith receives a friend request from
% his friend Brian Jones.  Behind the scenes, this could be implemented with a 
% simple list operation, pushing the friend request onto John's ``pending_requests``:
% 
% .. sourcecode:: pycon
% 
%    >>> c.list_rpush('profiles', 'jsmith1', {'pending_requests': 'bjones1'})
%    True
%    >>> c.get('profiles', 'jsmith1')['pending_requests']
%    ['bjones1']
% 
% The operation ``list_rpush`` is guaranteed to be performed atomically, and will
% be applied consistently with respect to all other operations on the same list.
% 
% .. todo::
% 
%    XXX Note that lists provide both an lpush and rpush operation. The former
%    adds the new element at the head of the list, while the latter adds at the
%    tail. They also provide lpop operation for taking an element off the head of
%    the list. Coupled together, these operations provide a comprehensive list
%    datatype that can be used to implement fault-tolerant lists of all kinds. For
%    instnace, one can implement work queues and generalized producer-consumer
%    patterns on top of HyperDex lists in a pretty straightforward fashion. In
%    this case, producers would push at one end of the list (the tail, with an
%    rpush) and consumers would pop from the other (the head, with a pop). Since
%    the operations are atomic, no additional synchronization would be necessary,
%    enabling a high-performance implementation.
% 
% 
% Sets
% ----
% 
% Our social networking app captures the notion of hobbies.  A set of strings is a
% natural representation for a user's hobbies, as each hobby is represented just
% once and may be added or removed.
% 
% Let's add some hobbies to John's profile:
% 
% .. sourcecode:: pycon
% 
%    >>> hobbies = set(['hockey', 'basket weaving', 'hacking',
%    ...                'air guitar rocking'])
%    >>> c.set_union('profiles', 'jsmith1', {'hobbies': hobbies})
%    True
%    >>> c.set_add('profiles', 'jsmith1', {'hobbies': 'gaming'})
%    True
%    >>> c.get('profiles', 'jsmith1')['hobbies']
%    set(['hacking', 'air guitar rocking', 'hockey', 'gaming', 'basket weaving'])
% 
% If John Smith decides that his life's dream is to just write code, he may decide
% to join a group on the social network filled with like-minded individuals.  We can 
% use HyperDex's intersect primitive to narrow down his interests:
% 
% .. sourcecode:: pycon
% 
%    >>> c.set_intersect('profiles', 'jsmith1',
%    ...                 {'hobbies': set(['hacking', 'programming'])})
%    True
%    >>> c.get('profiles', 'jsmith1')['hobbies']
%    set(['hacking'])
% 
% Notice how John's hobbies become the intersection of his previous hobbies and the 
% ones named in the operation.
% 
% Overall, HyperDex supports simple set assignment (using the ``put`` interface),
% adding and removing elements with :py:meth:`Client.set_add` and
% :py:meth:`hyperclient.Client.set_remove`, taking the union of a set with
% :py:meth:`hyperclient.Client.set_union` and storing the intersection of a set with
% :py:meth:`hyperclient.Client.set_intersect`.
% 
% Maps
% ----
% 
% Lastly, our social networking system needs a means for allowing users to
% exchange messages.  Let's demonstrate how we can accomplish this with the
% ``unread_messages`` attribute. In this contrived example, we're going to use an
% object attribute as a map (aka dictionary) to map from a user name to a string
% that contains the message from that user, as follows:
% 
% .. sourcecode:: pycon
% 
%    >>> c.map_add('profiles', 'jsmith1',
%    ...           {'unread_messages' : {'bjones1' : 'Hi John'}})
%    True
%    >>> c.map_add('profiles', 'jsmith1',
%    ...           {'unread_messages' : {'timmy' : 'Lunch?'}})
%    True
%    >>> c.get('profiles', 'jsmith1')['unread_messages']
%    {'timmy': 'Lunch?', 'bjones1': 'Hi John'}
% 
% HyperDex enables map contents to be modified in-place within the map.  For example, if Brian sent
% another message to John, we could separate the messages with "|" and just append
% the new message:
% 
% .. sourcecode:: pycon
% 
%    >>> c.map_string_append('profiles', 'jsmith1',
%    ...                      {'unread_messages' : {'bjones1' : ' | Want to hang out?'}})
%    True
%    >>> c.get('profiles', 'jsmith1')['unread_messages']
%    {'timmy': 'Lunch?', 'bjones1': 'Hi John | Want to hang out?'}
% 
% Note that maps may have strings or integers as values, and every atomic
% operation available for strings and integers is also available in map form,
% operating on the values of the map.
% 
% For the sake of illustrating maps involving integers, let's imagine that we will use a map to keep track
% of the plus-one's and like/dislike's on John's status updates. 
% 
% First, let's create some counters that will keep the net count of up and down votes 
% corresponding to John's link posts, with ids "http://url1.com" and "http://url2.com". 
% 
% .. sourcecode:: pycon
% 
%    >>> url1 = "http://url1.com"
%    >>> url2 = "http://url2.com"
%    >>> c.map_add('profiles', 'jsmith1',
%    ...           {'upvotes' : {url1 : 1, url2: 1}})
%    True
% 
% So John's posts start out with a counter set to 1, as shown above. 
% 
% Imagine that two other users, Jane and Elaine, upvote John's first link post,
% we would implement it like this:
% 
% .. sourcecode:: pycon
% 
%    >>> c.map_atomic_add('profiles', 'jsmith1', {'upvotes' : {url1: 1}})
%    True
%    >>> c.map_atomic_add('profiles', 'jsmith1', {'upvotes' : {url1: 1}})
%    True
% 
% Charlie, sworn enemy of John, can downvote both of John's urls like this:
% 
% .. sourcecode:: pycon
% 
%    >>> c.map_atomic_add('profiles', 'jsmith1', {'upvotes' : {url1: -1, url2: -1}})
%    True
% 
% This shows that any map operation can operate atomically on a group of map
% attributes at the same time. This is fully transactional; all such operations
% will be ordered in exactly the same way on all replicas, and there is no
% opportunity for divergence, even through failures.
% 
% Checking where we stand:
% 
% .. sourcecode:: pycon
% 
%    >>> c.get('profiles', 'jsmith1')['upvotes']
%    {'http://url1.com': 2, 'http://url2.com': 0}
% 
% All of the preceding operations could have been issued concurrently -- the results
% will be the same because they commute with each other and are executed atomically.
% 
% .. todo::
% 
%    .. sourcecode:: pycon
% 
%       >>> c.rm_space('profiles')
