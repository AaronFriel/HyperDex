\chapter{Data Types}
\label{chap:data-types}

As we saw in the previous chapter, HyperDex offers support for the basic
\code{get}, \code{put}, and \code{search} operations on strings and integers.
In this chapter, we explore HyperDex's richer datatypes and atomic operations on
lists, sets, and maps.  We will see how providing efficient atomic operations on
these rich, native datastructures greatly simplifies design for applications
with complicated data layout requirements.

By the end of this chapter you'll be familiar with all datatypes provided by
HyperDex, and a number of atomic operations.

\section{Setup}
\label{sec:data-types:setup}

As in the previous chapter, the first step is to deploy the cluster and connect
a client.   First we launch and initialize the coordinator:

\begin{consolecode}
hyperdex coordinator -f -l 127.0.0.1 -p 1982
\end{consolecode}

Next, let's launch a few daemon processes to store data.  Execute the following
commands (note that each instance binds to a different port and has a different
\code{/path/to/data}):

\begin{consolecode}
hyperdex daemon -f --listen=127.0.0.1 --listen-port=2012 \
                   --coordinator=127.0.0.1 --coordinator-port=1982 --data=/path/to/data1
hyperdex daemon -f --listen=127.0.0.1 --listen-port=2013 \
                   --coordinator=127.0.0.1 --coordinator-port=1982 --data=/path/to/data2
hyperdex daemon -f --listen=127.0.0.1 --listen-port=2014 \
                   --coordinator=127.0.0.1 --coordinator-port=1982 --data=/path/to/data3
\end{consolecode}

This brings up three different daemons ready to serve in the HyperDex cluster.
Finally, we create a space which makes use of all three systems in the cluster.
In this example, let's create a space that may be suitable for storing profiles
in a social network:

\begin{pythoncode}
>>> import hyperdex.admin
>>> a = hyperdex.admin.Admin('127.0.0.1', 1982)
>>> a.add_space('''
... space profiles
... key username
... attributes
...    string first,
...    string last,
...    int profile_views,
...    list(string) pending_requests,
...    set(string) hobbies,
...    map(string, string) unread_messages,
...    map(string, int) upvotes
... subspace first, last
... subspace profile_views
... ''')
True
>>> import hyperdex.client
>>> c = hyperdex.client.Client('127.0.0.1', 1982)
\end{pythoncode}

Finally, let's create a profile for John Smith that we can use in the rest of
this chapter.

\begin{pythoncode}
>>> c.put('profiles', 'jsmith1', {'first': 'John', 'last': 'Smith'})
True
>>> c.get('profiles', 'jsmith1')
{'first': 'John', 'last': 'Smith',
 'profile_views': 0,
 'pending_requests': [],
 'hobbies': set([]),
 'unread_messages': {},
 'upvotes': {}}
\end{pythoncode}

\section{Strings}
\label{sec:data-types:strings}

The basic datatype in HyperDex is a byte string.  If you don't specify the type
of an attribute when creating a space, it is automatically treated as an 8-bit
bytestring.  This means that you'll have to encode and decode unicode strings as
appropriate.  For example, if John wanted to add an accent to his name on his
social network page, the code could look like:

\begin{pythoncode}
>>> c.put('profiles', 'jsmith1', {'first': u'Jóhn'.encode('utf8')})
True
\end{pythoncode}

This encodes the string to raw bytes using UTF-8.  When fetching his profile it
is necessary to decode the UTF-8:

\begin{pythoncode}
>>> c.get('profiles', 'jsmith1')['first']
'J\xc3\x83\xc2\xb3hn'
>>> c.get('profiles', 'jsmith1')['first'].decode('utf8')
u'Jóhn'
>>> c.put('profiles', 'jsmith1', {'first': 'John', 'last': 'Smith'})
True
\end{pythoncode}

Of course, it's always possible to change John's name back to its unaccented
form:

\begin{pythoncode}
>>> c.put('profiles', 'jsmith1', {'first': 'John', 'last': 'Smith'})
True
>>> c.get('profiles', 'jsmith1')['first']
'John'
\end{pythoncode}

HyperDex knows nothing about encodings, so it is up to the application to encode
or decode data appropriately.

\section{Integers}
\label{sec:data-types:integers}

As we've already seen, HyperDex supports \code{get} and \code{put} operations on
integers.  In addition to these basic operations, HyperDex provides atomic
opertions to manipulate integers using basic math operations.  This is useful
when implementing features such as page-view counters.  Let's add support for
tracking the profile views of a page by incrementing the counter:

\begin{pythoncode}
>>> c.atomic_add('profiles', 'jsmith1', {'profile_views': 1})
True
>>> c.get('profiles', 'jsmith1')
{'first': 'John', 'last': 'Smith',
 'profile_views': 1,
 'pending_requests': [],
 'hobbies': set([]),
 'unread_messages': {},
 'upvotes': {}}
\end{pythoncode}

Note that this change required just one request to HyperDex.  The server
atomically examines the current value, and changes it by the amount specified.
In this case, the \code{profile\_views} attribute is incremented by one.

HyperDex supports a full range of basic operations including:
\begin{itemize}[noitemsep,nolistsep]
\item \code{atomic\_add}
\item \code{atomic\_sub}
\item \code{atomic\_mul}
\item \code{atomic\_div}
\item \code{atomic\_mod}
\item \code{atomic\_and}
\item \code{atomic\_or}
\item \code{atomic\_xor}
\end{itemize}

\section{Floats}
\label{sec:data-types:floats}

HyperDex also supports double-precision floating point types.  Like integers,
floats support range searches and atomic operations.

\section{Lists}
\label{sec:data-types:lists}

Let's add support for friend requests using HyperDex lists the basis of the
feature.  For this we'll use the \code{pending\_requests} attribute in the
\code{profiles} space.
 
Imagine that shortly after joining, John Smith receives a friend request from
his friend Brian Jones.  Behind the scenes, this could be implemented with a
simple list operation, pushing the friend request onto John's
\code{pending\_requests}:

\begin{pythoncode}
>>> c.list_rpush('profiles', 'jsmith1', {'pending_requests': 'bjones1'})
True
>>> c.get('profiles', 'jsmith1')['pending_requests']
['bjones1']
\end{pythoncode}

The operation \code{list\_rpush} is guaranteed to be performed atomically, and
will be applied consistently with respect to all other operations on the same
list.

\begin{comment} % XXX
Note that lists provide both an lpush and rpush operation. The former adds the
new element at the head of the list, while the latter adds at the tail. They
also provide lpop operation for taking an element off the head of the list.
Coupled together, these operations provide a comprehensive list datatype that
can be used to implement fault-tolerant lists of all kinds. For instnace, one
can implement work queues and generalized producer-consumer patterns on top of
HyperDex lists in a pretty straightforward fashion. In this case, producers
would push at one end of the list (the tail, with an rpush) and consumers would
pop from the other (the head, with a pop). Since the operations are atomic, no
additional synchronization would be necessary, enabling a high-performance
implementation.
\end{comment}

\section{Sets}
\label{sec:data-types:sets}

Our social networking app captures the notion of hobbies.  A set of strings is a
natural representation for a user's hobbies, as each hobby is represented just
once and may be added or removed.

Let's add some hobbies to John's profile:

\begin{pythoncode}
>>> hobbies = set(['hockey', 'basket weaving', 'hacking',
...                'air guitar rocking'])
>>> c.set_union('profiles', 'jsmith1', {'hobbies': hobbies})
True
>>> c.set_add('profiles', 'jsmith1', {'hobbies': 'gaming'})
True
>>> c.get('profiles', 'jsmith1')['hobbies']
set(['hacking', 'air guitar rocking', 'hockey', 'gaming', 'basket weaving'])
\end{pythoncode}

If John Smith decides that his life's dream is to just write code, he may decide
to join a group on the social network filled with like-minded individuals.  We
can use HyperDex's intersect primitive to narrow down his interests:

\begin{pythoncode}
>>> c.set_intersect('profiles', 'jsmith1',
...                 {'hobbies': set(['hacking', 'programming'])})
True
>>> c.get('profiles', 'jsmith1')['hobbies']
set(['hacking'])
\end{pythoncode}

Notice how John's hobbies become the intersection of his previous hobbies and
the ones named in the operation.

Overall, HyperDex supports simple set assignment (using the \code{put}
interface), adding and removing elements with \code{set\_add} and
\code{set\_remove}, taking the union of a set with \code{set\_union} and storing
the intersection of a set with \code{set\_intersect}.

\section{Maps}
\label{sec:data-types:maps}

Lastly, our social networking system needs a means for allowing users to
exchange messages.  Let's demonstrate how we can accomplish this with the
\code{unread\_messages} attribute. In this contrived example, we're going to use
an object attribute as a map (aka dictionary) to map from a user name to a
string that contains the message from that user, as follows:

\begin{pythoncode}
>>> c.map_add('profiles', 'jsmith1',
...           {'unread_messages' : {'bjones1' : 'Hi John'}})
True
>>> c.map_add('profiles', 'jsmith1',
...           {'unread_messages' : {'timmy' : 'Lunch?'}})
True
>>> c.get('profiles', 'jsmith1')['unread_messages']
{'timmy': 'Lunch?', 'bjones1': 'Hi John'}
\end{pythoncode}

HyperDex enables map contents to be modified in-place within the map.  For
example, if Brian sent another message to John, we could separate the messages
with "|" and just append the new message:

\begin{pythoncode}
>>> c.map_string_append('profiles', 'jsmith1',
...                      {'unread_messages' : {'bjones1' : ' | Want to hang out?'}})
True
>>> c.get('profiles', 'jsmith1')['unread_messages']
{'timmy': 'Lunch?', 'bjones1': 'Hi John | Want to hang out?'}
\end{pythoncode}

Note that maps may have strings or integers as values, and every atomic
operation available for strings and integers is also available in map form,
operating on the values of the map.

For the sake of illustrating maps involving integers, let's imagine that we will
use a map to keep track of the plus-one's and like/dislike's on John's status
updates. 

First, let's create some counters that will keep the net count of up and down
votes corresponding to John's link posts, with ids "http://url1.com" and
"http://url2.com". 

\begin{pythoncode}
>>> url1 = "http://url1.com"
>>> url2 = "http://url2.com"
>>> c.map_add('profiles', 'jsmith1',
...           {'upvotes' : {url1 : 1, url2: 1}})
True
\end{pythoncode}

So John's posts start out with a counter set to 1, as shown above. 

Imagine that two other users, Jane and Elaine, upvote John's first link post, we
would implement it like this:

\begin{pythoncode}
>>> c.map_atomic_add('profiles', 'jsmith1', {'upvotes' : {url1: 1}})
True
>>> c.map_atomic_add('profiles', 'jsmith1', {'upvotes' : {url1: 1}})
True
\end{pythoncode}

Charlie, sworn enemy of John, can downvote both of John's urls like this:

\begin{pythoncode}
>>> c.map_atomic_add('profiles', 'jsmith1', {'upvotes' : {url1: -1, url2: -1}})
True
\end{pythoncode}

This shows that any map operation can operate atomically on a group of map
attributes at the same time. This is fully transactional; all such operations
will be ordered in exactly the same way on all replicas, and there is no
opportunity for divergence, even through failures.

Checking where we stand:

\begin{pythoncode}
>>> c.get('profiles', 'jsmith1')['upvotes']
{'http://url1.com': 2, 'http://url2.com': 0}
\end{pythoncode}

All of the preceding operations could have been issued concurrently -- the
results will be the same because they commute with each other and are executed
atomically.
