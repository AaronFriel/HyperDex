\chapter{Java API}
\label{chap:api:java}

\section{Client Library}
\label{sec:api:java:client}

HyperDex provides Java bindings in the package \code{org.hyperdex.client}.  This
package wraps the HyperDex C Client library and enables the use of native Java
data types.

This library was brought up-to-date following the 1.0.5 release.

\subsection{Building the HyperDex Java Binding}
\label{sec:api:java:building}

The HyperDex Java Binding must be requested at configure time as it is not
automatically built.  You can ensure that the Java bindings are always built by
providing the \code{--enable-java-bindings} option to \code{./configure} like
so:

\begin{consolecode}
% ./configure --enable-client --enable-java-bindings
\end{consolecode}

\subsection{Using Java Within Your Application}
\label{sec:api:java:using}

All client operation are defined in the \code{org.hyperdex.client} package.  You
can access this in your program with:

\begin{javacode}
import org.hyperdex.client.*;
\end{javacode}

\subsection{Hello World}
\label{sec:api:java:hello-world}

The following is a minimal application that stores the value "Hello World" and
then immediately retrieves the value:

\inputminted{java}{\topdir/api/java/HelloWorld.java}

You can run this example with:

\begin{consolecode}
% javac HelloWorld.java
% java -Djava.library.path=/usr/local/lib HelloWorld
put: true
got: {v=Hello World!}
\end{consolecode}

Right away, there are several points worth noting in this example:

\begin{itemize}
\item Every operation is synchronous.  The PUT and GET operations run to
completion by default.

\item Java types are automatically converted to HyperDex types.  There's no need
to specify information such as the length of each string, as one would do with
the C API.

\item We specify -Djava.library.path=/usr/local/lib.  This is necessary for
builds from source, but should not be necessary for Java bindings installed
using binary packages.
\end{itemize}

\subsection{Asynchronous Operations}
\label{sec:api:java:async-ops}

For convenience, the Java bindings treat every operation as synchronous.  This
enables you to write short programs without concern for asynchronous operations.
Most operations come with an asynchronous form, denoted by the \code{async\_}
prefix.  For example, the above Hello World example could be rewritten in
asynchronous fashion as such:

\inputminted{java}{\topdir/api/java/HelloWorldAsyncWait.java}

This enables applications to issue multiple requests simultaneously and wait for
their completion in an application-specific order.  It's also possible to use
the \code{loop} method on the client object to wait for the next request to
complete:

\inputminted{java}{\topdir/api/java/HelloWorldAsyncLoop.java}

\subsection{Data Structures}
\label{sec:api:java:data-structures}

The Java bindings automatically manage conversion of data types from Java to
HyperDex types, enabling applications to be written in idiomatic Java.

\subsubsection{Examples}
\label{sec:api:java:examples}

This section shows examples of Java data structures that are recognized by
HyperDex.  The examples here are for illustration purposes and are not
exhaustive.

\paragraph{Strings}

The HyperDex client recognizes Java's strings and automatically converts them to
HyperDex strings.  For example, the following call stores a string:
equivalent and have the same effect:

\begin{javacode}
Map<String, Object> attrs = new HashMap<String, Object>();
attrs.put("v", "someattrs");
c.put("kv", "somekey", attrs);
\end{javacode}

\paragraph{Integers}

The HyperDex client recognizes Java's integers and automatically converts them
to HyperDex integers.  For example:

\begin{javacode}
Map<String, Object> attrs = new HashMap<String, Object>();
attrs.put("v", 42);
c.put("kv", "somekey", attrs);
\end{javacode}

\paragraph{Floats}

The HyperDex client recognizes Java's floating point numbers and automatically
converts them to HyperDex floats.  For example:

\begin{javacode}
Map<String, Object> attrs = new HashMap<String, Object>();
attrs.put("v", 3.1415);
c.put("kv", "somekey", attrs);
\end{javacode}

\paragraph{Lists}

The HyperDex client recognizes Java lists and automatically converts them to
HyperDex lists.  For example:

\begin{javacode}
List<Object> list = new ArrayList<Object>();
list.add("a");
list.add("b");
list.add("c");
Map<String, Object> attrs = new HashMap<String, Object>();
attrs.put("v", list);
c.put("kv", "somekey", attrs);
\end{javacode}

\paragraph{Sets}

The HyperDex client recognizes Java sets and automatically converts them to
HyperDex sets.  For example:

\begin{javacode}
Set<Object> set = new HashSet<Object>();
set.add("a");
set.add("b");
set.add("c");
Map<String, Object> attrs = new HashMap<String, Object>();
attrs.put("v", set);
c.put("kv", "somekey", attrs);
\end{javacode}

\paragraph{Maps}

The HyperDex client recognizes Java maps and automatically converts them to
HyperDex maps.  For example:

\begin{javacode}
Map<Object, Object> map = new HashMap<Object, Object>();
map.put("k", "v");
Map<String, Object> attrs = new HashMap<String, Object>();
attrs.put("v", map);
c.put("kv", "somekey", attrs);
\end{javacode}

\subsection{Attributes}
\label{sec:api:java:attributes}

Attributes in Java are specified in the form of a map from attribute names to
their values.  As you can see in the examples above, attributes are specified in
the form:

\begin{javacode}
Map<String, Object> attrs = new HashMap<String, Object>();
\end{javacode}

\subsection{Map Attributes}
\label{sec:api:java:map-attributes}

Map attributes in Java are specified in the form of a nested map.  The outer
map key specifies the name, while the inner map key-value pair's specify the
key-value pair of the map.  For example:

\begin{javacode}
Map<String, Map<Object, Object>> mapattrs = new HashMap<String, Map<Object, Object>>();
\end{javacode}

\subsection{Predicates}
\label{sec:api:java:predicates}

Predicates in Java are specified in the form of a hash from attribute names to
their predicates.  In the simple case, the predicate is just a value to be
compared against:

\begin{javacode}
Map<String, Object> checks = new HashMap<String, Object>();
checks.put("v", "value");
\end{javacode}

This is the same as saying:

\begin{javacode}
Map<String, Object> checks = new HashMap<String, Object>();
checks.put("v", new Equals("value"));
\end{javacode}

The Java bindings support the full range of predicates supported by HyperDex
itself.  For example:

\begin{javacode}
checks.put("v", new LessEqual(5));
checks.put("v", new GreaterEqual(5));
checks.put("v", new RangeEqual(5, 10));
checks.put("v", new Regex("^s.*"));
checks.put("v", new LengthEquals(5));
checks.put("v", new LengthLessEqual(5));
checks.put("v", new LengthGreaterEqual(5));
checks.put("v", new Contains('value'));
\end{javacode}

\subsection{Error Handling}
\label{sec:api:java:error-handling}

All error handling within the Java bindings is done via the
\code{try}/\code{catch} mechanism of Java.  Errors will be thrown by the package
and should be handled by your application.  For example, if we were trying to
store an integer (5) as attribute \code{"v"}, where \code{"v"} is actually a
string, we'd generate an error.

\begin{javacode}
try
{
    attrs.put("v", 5);
    System.out.println("put: " + c.put("kv", "k", attrs));
}
catch (HyperDexClientException e)
{
    System.out.println(e.status());
    System.out.println(e.symbol());
    System.out.println(e.message());
}
\end{javacode}

Errors of type \code{HyperDexClientException} will contain both a message
indicating what went wrong, as well as the underlying \code{enum
hyperdex\_client\_returncode}.  The member \code{status} indicates the numeric
value of this enum, while \code{symbol} returns the enum as a string.  The above
code will fail with the following output:

\begin{verbatim}
8525
HYPERDEX_CLIENT_WRONGTYPE
invalid attribute "v": attribute has the wrong type
\end{verbatim}

\subsection{Operations}
\label{sec:api:java:ops}

% Copyright (c) 2014, Cornell University
% All rights reserved.
%
% Redistribution and use in source and binary forms, with or without
% modification, are permitted provided that the following conditions are met:
%
%     * Redistributions of source code must retain the above copyright notice,
%       this list of conditions and the following disclaimer.
%     * Redistributions in binary form must reproduce the above copyright
%       notice, this list of conditions and the following disclaimer in the
%       documentation and/or other materials provided with the distribution.
%     * Neither the name of HyperDex nor the names of its contributors may be
%       used to endorse or promote products derived from this software without
%       specific prior written permission.
%
% THIS SOFTWARE IS PROVIDED BY THE COPYRIGHT HOLDERS AND CONTRIBUTORS "AS IS"
% AND ANY EXPRESS OR IMPLIED WARRANTIES, INCLUDING, BUT NOT LIMITED TO, THE
% IMPLIED WARRANTIES OF MERCHANTABILITY AND FITNESS FOR A PARTICULAR PURPOSE ARE
% DISCLAIMED. IN NO EVENT SHALL THE COPYRIGHT OWNER OR CONTRIBUTORS BE LIABLE
% FOR ANY DIRECT, INDIRECT, INCIDENTAL, SPECIAL, EXEMPLARY, OR CONSEQUENTIAL
% DAMAGES (INCLUDING, BUT NOT LIMITED TO, PROCUREMENT OF SUBSTITUTE GOODS OR
% SERVICES; LOSS OF USE, DATA, OR PROFITS; OR BUSINESS INTERRUPTION) HOWEVER
% CAUSED AND ON ANY THEORY OF LIABILITY, WHETHER IN CONTRACT, STRICT LIABILITY,
% OR TORT (INCLUDING NEGLIGENCE OR OTHERWISE) ARISING IN ANY WAY OUT OF THE USE
% OF THIS SOFTWARE, EVEN IF ADVISED OF THE POSSIBILITY OF SUCH DAMAGE.

% This LaTeX file is generated by bindings/java.py

%%%%%%%%%%%%%%%%%%%% get %%%%%%%%%%%%%%%%%%%%
\pagebreak
\subsubsection{\code{get}}
\label{api:java:get}
\index{get!Java API}
Get an object by key.

\paragraph{Behavior:}
\begin{itemize}[noitemsep]
\item XXX % XXX

\end{itemize}


\paragraph{Definition:}
\begin{javacode}
public Map<String, Object> get(
        String spacename,
        Object key) throws HyperDexClientException
\end{javacode}

\paragraph{Parameters:}
\begin{itemize}[noitemsep]
\item \code{String spacename}\\
The name of the space as a string or symbol.

\item \code{Object key}\\
The key for the operation as a Python type.

\end{itemize}

\paragraph{Returns:}
This function returns via the provided callback.  In the normal case, the first
argument will indicate success or failure of the operation with one of the
following values:

\begin{itemize}[noitemsep]
\item A Javascript representation of the stored object.
\item \code{null} if the operation is a retrieval operation and no object was
    found.
\end{itemize}

If the operation encounters any error, the error argument will be provided and
will specify the error, in which case the first argument is undefined.


\pagebreak
\subsubsection{\code{async\_get}}
\label{api:java:async_get}
\index{async\_get!Java API}
Get an object by key.

\paragraph{Behavior:}
\begin{itemize}[noitemsep]
\item XXX % XXX

\end{itemize}


\paragraph{Definition:}
\begin{javacode}
public Deferred async_get(
        String spacename,
        Object key) throws HyperDexClientException
\end{javacode}

\paragraph{Parameters:}
\begin{itemize}[noitemsep]
\item \code{String spacename}\\
The name of the space as a string or symbol.

\item \code{Object key}\\
The key for the operation as a Python type.

\end{itemize}

\paragraph{Returns:}
A \code{Deferred} object with a \code{wait} method that returns the object if
found, \code{None} if not found.  Raises exception on error.


\paragraph{See also:}  This is the asynchronous form of \code{get}.

%%%%%%%%%%%%%%%%%%%% get_partial %%%%%%%%%%%%%%%%%%%%
\pagebreak
\subsubsection{\code{get\_partial}}
\label{api:java:get_partial}
\index{get\_partial!Java API}
Get part of an object by key.  This will return only the listed attribute names.

\paragraph{Behavior:}
\begin{itemize}[noitemsep]
\item XXX % XXX

\end{itemize}


\paragraph{Definition:}
\begin{javacode}
public Map<String, Object> get_partial(
        String spacename,
        Object key,
        List<String> attributenames) throws HyperDexClientException
\end{javacode}

\paragraph{Parameters:}
\begin{itemize}[noitemsep]
\item \code{String spacename}\\
The name of the space as a string or symbol.

\item \code{Object key}\\
The key for the operation as a Python type.

\item \code{List<String> attributenames}\\
A list of attributes to return.  \code{attrnames} is a \code{List<String>}.

\end{itemize}

\paragraph{Returns:}
This function returns via the provided callback.  In the normal case, the first
argument will indicate success or failure of the operation with one of the
following values:

\begin{itemize}[noitemsep]
\item A Javascript representation of the stored object.
\item \code{null} if the operation is a retrieval operation and no object was
    found.
\end{itemize}

If the operation encounters any error, the error argument will be provided and
will specify the error, in which case the first argument is undefined.


\pagebreak
\subsubsection{\code{async\_get\_partial}}
\label{api:java:async_get_partial}
\index{async\_get\_partial!Java API}
Get part of an object by key.  This will return only the listed attribute names.

\paragraph{Behavior:}
\begin{itemize}[noitemsep]
\item XXX % XXX

\end{itemize}


\paragraph{Definition:}
\begin{javacode}
public Deferred async_get_partial(
        String spacename,
        Object key,
        List<String> attributenames) throws HyperDexClientException
\end{javacode}

\paragraph{Parameters:}
\begin{itemize}[noitemsep]
\item \code{String spacename}\\
The name of the space as a string or symbol.

\item \code{Object key}\\
The key for the operation as a Python type.

\item \code{List<String> attributenames}\\
A list of attributes to return.  \code{attrnames} is a \code{List<String>}.

\end{itemize}

\paragraph{Returns:}
A \code{Deferred} object with a \code{wait} method that returns the object if
found, \code{None} if not found.  Raises exception on error.


\paragraph{See also:}  This is the asynchronous form of \code{get\_partial}.

%%%%%%%%%%%%%%%%%%%% put %%%%%%%%%%%%%%%%%%%%
\pagebreak
\subsubsection{\code{put}}
\label{api:java:put}
\index{put!Java API}
Store or update an object by key.  The object's attributes will be set to the
values specified by \code{attrs}.
\item An existing object will be updated by the operation.  If no object does
    exists, a new object will be created, with attributes initialized to their
    default values.



\paragraph{Definition:}
\begin{javacode}
public Boolean put(
        String spacename,
        Object key,
        Map<String, Object> attributes) throws HyperDexClientException
\end{javacode}

\paragraph{Parameters:}
\begin{itemize}[noitemsep]
\item \code{String spacename}\\
The name of the space as a string or symbol.

\item \code{Object key}\\
The key for the operation as a Python type.

\item \code{Map<String, Object> attributes}\\
The set of attributes to modify and their respective values.  \code{attrs} is a
map from the attributes' names to their values.

\end{itemize}

\paragraph{Returns:}
This function returns an object indicating the success or failure of the
operation.  Valid values to be returned are:

\begin{itemize}[noitemsep]
\item \code{True} if the operation succeeded
\item \code{False} if any provided predicates failed.
\item \code{null} if the operation requires an existing value and none exists
\end{itemize}

On error, this function will raise a \code{HyperDexClientException} describing
the error.


\pagebreak
\subsubsection{\code{async\_put}}
\label{api:java:async_put}
\index{async\_put!Java API}
Store or update an object by key.  The object's attributes will be set to the
values specified by \code{attrs}.
\item An existing object will be updated by the operation.  If no object does
    exists, a new object will be created, with attributes initialized to their
    default values.



\paragraph{Definition:}
\begin{javacode}
public Deferred async_put(
        String spacename,
        Object key,
        Map<String, Object> attributes) throws HyperDexClientException
\end{javacode}

\paragraph{Parameters:}
\begin{itemize}[noitemsep]
\item \code{String spacename}\\
The name of the space as a string or symbol.

\item \code{Object key}\\
The key for the operation as a Python type.

\item \code{Map<String, Object> attributes}\\
The set of attributes to modify and their respective values.  \code{attrs} is a
map from the attributes' names to their values.

\end{itemize}

\paragraph{Returns:}
A Deferred object with a \code{wait} method that returns True if the operation
succeeded or False if any provided predicates failed.  Raises an exception on
error.


\paragraph{See also:}  This is the asynchronous form of \code{put}.

%%%%%%%%%%%%%%%%%%%% cond_put %%%%%%%%%%%%%%%%%%%%
\pagebreak
\subsubsection{\code{cond\_put}}
\label{api:java:cond_put}
\index{cond\_put!Java API}
Conditionally write the specified attributes to the object in space "space" under key "key".

The operation will modify the object if and only if all \texttt{checks} are true
for the latest version of the object.  This test is atomic with the write.  If
the object does not exist, the checks will fail.

The attributes specified by \texttt{attrs} will be overwritten with their
respective values.  Any attributes not specified by \texttt{attrs} will be
preserved.

\noindent\textbf{Cost:}  Approximately one traversal of the value-dependent
chain.


\noindent\textbf{Consistency:}  Linearizable



\paragraph{Definition:}
\begin{javacode}
public Boolean cond_put(
        String spacename,
        Object key,
        Map<String, Object> predicates,
        Map<String, Object> attributes) throws HyperDexClientException
\end{javacode}

\paragraph{Parameters:}
\begin{itemize}[noitemsep]
\item \code{String spacename}\\
The name of the space as a string or symbol.

\item \code{Object key}\\
The key for the operation as a Python type.

\item \code{Map<String, Object> predicates}\\
A hash of predicates to check against.

\item \code{Map<String, Object> attributes}\\
The set of attributes to modify and their respective values.  \code{attrs} is a
map from the attributes' names to their values.

\end{itemize}

\paragraph{Returns:}
This function returns an object indicating the success or failure of the
operation.  Valid values to be returned are:

\begin{itemize}[noitemsep]
\item \code{True} if the operation succeeded
\item \code{False} if any provided predicates failed.
\item \code{null} if the operation requires an existing value and none exists
\end{itemize}

On error, this function will raise a \code{HyperDexClientException} describing
the error.


\pagebreak
\subsubsection{\code{async\_cond\_put}}
\label{api:java:async_cond_put}
\index{async\_cond\_put!Java API}
Conditionally write the specified attributes to the object in space "space" under key "key".

The operation will modify the object if and only if all \texttt{checks} are true
for the latest version of the object.  This test is atomic with the write.  If
the object does not exist, the checks will fail.

The attributes specified by \texttt{attrs} will be overwritten with their
respective values.  Any attributes not specified by \texttt{attrs} will be
preserved.

\noindent\textbf{Cost:}  Approximately one traversal of the value-dependent
chain.


\noindent\textbf{Consistency:}  Linearizable



\paragraph{Definition:}
\begin{javacode}
public Deferred async_cond_put(
        String spacename,
        Object key,
        Map<String, Object> predicates,
        Map<String, Object> attributes) throws HyperDexClientException
\end{javacode}

\paragraph{Parameters:}
\begin{itemize}[noitemsep]
\item \code{String spacename}\\
The name of the space as a string or symbol.

\item \code{Object key}\\
The key for the operation as a Python type.

\item \code{Map<String, Object> predicates}\\
A hash of predicates to check against.

\item \code{Map<String, Object> attributes}\\
The set of attributes to modify and their respective values.  \code{attrs} is a
map from the attributes' names to their values.

\end{itemize}

\paragraph{Returns:}
A Deferred object with a \code{wait} method that returns True if the operation
succeeded or False if any provided predicates failed.  Raises an exception on
error.


\paragraph{See also:}  This is the asynchronous form of \code{cond\_put}.

%%%%%%%%%%%%%%%%%%%% put_if_not_exist %%%%%%%%%%%%%%%%%%%%
\pagebreak
\subsubsection{\code{put\_if\_not\_exist}}
\label{api:java:put_if_not_exist}
\index{put\_if\_not\_exist!Java API}
Store or object under \code{key} in \code{space} if and only if the operation
creates a new object.  The object's attributes will be set to the values
specified by \code{attrs}; any attributes not specified by \code{attrs} will be
initialized to their defaults.  If the object exists, the operation will fail
with \code{CMPFAIL}.


\paragraph{Definition:}
\begin{javacode}
public Boolean put_if_not_exist(
        String spacename,
        Object key,
        Map<String, Object> attributes) throws HyperDexClientException
\end{javacode}

\paragraph{Parameters:}
\begin{itemize}[noitemsep]
\item \code{String spacename}\\
The name of the space as a string or symbol.

\item \code{Object key}\\
The key for the operation as a Python type.

\item \code{Map<String, Object> attributes}\\
The set of attributes to modify and their respective values.  \code{attrs} is a
map from the attributes' names to their values.

\end{itemize}

\paragraph{Returns:}
This function returns an object indicating the success or failure of the
operation.  Valid values to be returned are:

\begin{itemize}[noitemsep]
\item \code{True} if the operation succeeded
\item \code{False} if any provided predicates failed.
\item \code{null} if the operation requires an existing value and none exists
\end{itemize}

On error, this function will raise a \code{HyperDexClientException} describing
the error.


\pagebreak
\subsubsection{\code{async\_put\_if\_not\_exist}}
\label{api:java:async_put_if_not_exist}
\index{async\_put\_if\_not\_exist!Java API}
Store or object under \code{key} in \code{space} if and only if the operation
creates a new object.  The object's attributes will be set to the values
specified by \code{attrs}; any attributes not specified by \code{attrs} will be
initialized to their defaults.  If the object exists, the operation will fail
with \code{CMPFAIL}.


\paragraph{Definition:}
\begin{javacode}
public Deferred async_put_if_not_exist(
        String spacename,
        Object key,
        Map<String, Object> attributes) throws HyperDexClientException
\end{javacode}

\paragraph{Parameters:}
\begin{itemize}[noitemsep]
\item \code{String spacename}\\
The name of the space as a string or symbol.

\item \code{Object key}\\
The key for the operation as a Python type.

\item \code{Map<String, Object> attributes}\\
The set of attributes to modify and their respective values.  \code{attrs} is a
map from the attributes' names to their values.

\end{itemize}

\paragraph{Returns:}
A Deferred object with a \code{wait} method that returns True if the operation
succeeded or False if any provided predicates failed.  Raises an exception on
error.


\paragraph{See also:}  This is the asynchronous form of \code{put\_if\_not\_exist}.

%%%%%%%%%%%%%%%%%%%% del %%%%%%%%%%%%%%%%%%%%
\pagebreak
\subsubsection{\code{del}}
\label{api:java:del}
\index{del!Java API}
Delete an object by key.

%%% Generated below here
\paragraph{Behavior:}
\begin{itemize}[noitemsep]
If no object exists, the operation will fail with \code{NOTFOUND}.

\end{itemize}


\paragraph{Definition:}
\begin{javacode}
public Boolean del(String spacename, Object key) throws HyperDexClientException
\end{javacode}

\paragraph{Parameters:}
\begin{itemize}[noitemsep]
\item \code{String spacename}\\
The name of the space as a string or symbol.

\item \code{Object key}\\
The key for the operation as a Python type.

\end{itemize}

\paragraph{Returns:}
This function returns an object indicating the success or failure of the
operation.  Valid values to be returned are:

\begin{itemize}[noitemsep]
\item \code{True} if the operation succeeded
\item \code{False} if any provided predicates failed.
\item \code{null} if the operation requires an existing value and none exists
\end{itemize}

On error, this function will raise a \code{HyperDexClientException} describing
the error.


\pagebreak
\subsubsection{\code{async\_del}}
\label{api:java:async_del}
\index{async\_del!Java API}
Delete an object by key.

%%% Generated below here
\paragraph{Behavior:}
\begin{itemize}[noitemsep]
If no object exists, the operation will fail with \code{NOTFOUND}.

\end{itemize}


\paragraph{Definition:}
\begin{javacode}
public Deferred async_del(
        String spacename,
        Object key) throws HyperDexClientException
\end{javacode}

\paragraph{Parameters:}
\begin{itemize}[noitemsep]
\item \code{String spacename}\\
The name of the space as a string or symbol.

\item \code{Object key}\\
The key for the operation as a Python type.

\end{itemize}

\paragraph{Returns:}
A Deferred object with a \code{wait} method that returns True if the operation
succeeded or False if any provided predicates failed.  Raises an exception on
error.


\paragraph{See also:}  This is the asynchronous form of \code{del}.

%%%%%%%%%%%%%%%%%%%% cond_del %%%%%%%%%%%%%%%%%%%%
\pagebreak
\subsubsection{\code{cond\_del}}
\label{api:java:cond_del}
\index{cond\_del!Java API}
Conditionally delete the object stored under \code{key} from \code{space}.
If no object exists, the operation will fail with \code{NOTFOUND}.


This operation will succeed if and only if the predicates specified by
\code{checks} hold on the pre-existing object.  If any of the predicates are not
true for the existing object, then the operation will have no effect and fail
with \code{CMPFAIL}.

All checks are atomic with the write.  HyperDex guarantees that no other
operation will come between validating the checks, and writing the new version
of the object.



\paragraph{Definition:}
\begin{javacode}
public Boolean cond_del(
        String spacename,
        Object key,
        Map<String, Object> predicates) throws HyperDexClientException
\end{javacode}

\paragraph{Parameters:}
\begin{itemize}[noitemsep]
\item \code{String spacename}\\
The name of the space as a string or symbol.

\item \code{Object key}\\
The key for the operation as a Python type.

\item \code{Map<String, Object> predicates}\\
A hash of predicates to check against.

\end{itemize}

\paragraph{Returns:}
This function returns an object indicating the success or failure of the
operation.  Valid values to be returned are:

\begin{itemize}[noitemsep]
\item \code{True} if the operation succeeded
\item \code{False} if any provided predicates failed.
\item \code{null} if the operation requires an existing value and none exists
\end{itemize}

On error, this function will raise a \code{HyperDexClientException} describing
the error.


\pagebreak
\subsubsection{\code{async\_cond\_del}}
\label{api:java:async_cond_del}
\index{async\_cond\_del!Java API}
Conditionally delete the object stored under \code{key} from \code{space}.
If no object exists, the operation will fail with \code{NOTFOUND}.


This operation will succeed if and only if the predicates specified by
\code{checks} hold on the pre-existing object.  If any of the predicates are not
true for the existing object, then the operation will have no effect and fail
with \code{CMPFAIL}.

All checks are atomic with the write.  HyperDex guarantees that no other
operation will come between validating the checks, and writing the new version
of the object.



\paragraph{Definition:}
\begin{javacode}
public Deferred async_cond_del(
        String spacename,
        Object key,
        Map<String, Object> predicates) throws HyperDexClientException
\end{javacode}

\paragraph{Parameters:}
\begin{itemize}[noitemsep]
\item \code{String spacename}\\
The name of the space as a string or symbol.

\item \code{Object key}\\
The key for the operation as a Python type.

\item \code{Map<String, Object> predicates}\\
A hash of predicates to check against.

\end{itemize}

\paragraph{Returns:}
A Deferred object with a \code{wait} method that returns True if the operation
succeeded or False if any provided predicates failed.  Raises an exception on
error.


\paragraph{See also:}  This is the asynchronous form of \code{cond\_del}.

%%%%%%%%%%%%%%%%%%%% atomic_add %%%%%%%%%%%%%%%%%%%%
\pagebreak
\subsubsection{\code{atomic\_add}}
\label{api:java:atomic_add}
\index{atomic\_add!Java API}
Add the specified number to the existing value for each attribute.

%%% Generated below here
\paragraph{Behavior:}
\begin{itemize}[noitemsep]
This operation requires a pre-existing object in order to complete successfully.
If no object exists, the operation will fail with \code{NOTFOUND}.

\end{itemize}


\paragraph{Definition:}
\begin{javacode}
public Boolean atomic_add(
        String spacename,
        Object key,
        Map<String, Object> attributes) throws HyperDexClientException
\end{javacode}

\paragraph{Parameters:}
\begin{itemize}[noitemsep]
\item \code{String spacename}\\
The name of the space as a string or symbol.

\item \code{Object key}\\
The key for the operation as a Python type.

\item \code{Map<String, Object> attributes}\\
The set of attributes to modify and their respective values.  \code{attrs} is a
map from the attributes' names to their values.

\end{itemize}

\paragraph{Returns:}
This function returns an object indicating the success or failure of the
operation.  Valid values to be returned are:

\begin{itemize}[noitemsep]
\item \code{True} if the operation succeeded
\item \code{False} if any provided predicates failed.
\item \code{null} if the operation requires an existing value and none exists
\end{itemize}

On error, this function will raise a \code{HyperDexClientException} describing
the error.


\pagebreak
\subsubsection{\code{async\_atomic\_add}}
\label{api:java:async_atomic_add}
\index{async\_atomic\_add!Java API}
Add the specified number to the existing value for each attribute.

%%% Generated below here
\paragraph{Behavior:}
\begin{itemize}[noitemsep]
This operation requires a pre-existing object in order to complete successfully.
If no object exists, the operation will fail with \code{NOTFOUND}.

\end{itemize}


\paragraph{Definition:}
\begin{javacode}
public Deferred async_atomic_add(
        String spacename,
        Object key,
        Map<String, Object> attributes) throws HyperDexClientException
\end{javacode}

\paragraph{Parameters:}
\begin{itemize}[noitemsep]
\item \code{String spacename}\\
The name of the space as a string or symbol.

\item \code{Object key}\\
The key for the operation as a Python type.

\item \code{Map<String, Object> attributes}\\
The set of attributes to modify and their respective values.  \code{attrs} is a
map from the attributes' names to their values.

\end{itemize}

\paragraph{Returns:}
A Deferred object with a \code{wait} method that returns True if the operation
succeeded or False if any provided predicates failed.  Raises an exception on
error.


\paragraph{See also:}  This is the asynchronous form of \code{atomic\_add}.

%%%%%%%%%%%%%%%%%%%% group_atomic_add %%%%%%%%%%%%%%%%%%%%
\pagebreak
\subsubsection{\code{group\_atomic\_add}}
\label{api:java:group_atomic_add}
\index{group\_atomic\_add!Java API}
Add the specified number to the existing value for each object in \code{space}
that matches \code{checks}.

This operation will only affect objects that match the provided \code{checks}.
Objects that do not match \code{checks} will be unaffected by the group call.
Each object that matches \code{checks} will be atomically updated with the check
on the object.  HyperDex guarantees that no object will be altered if the
\code{checks} do not pass at the time of the write.  Objects that are updated
concurrently with the group call may or may not be updated; however, regardless
of any other concurrent operations, the preceding guarantee will always hold.



\paragraph{Definition:}
\begin{javacode}
public Long group_atomic_add(
        String spacename,
        Map<String, Object> predicates,
        Map<String, Object> attributes) throws HyperDexClientException
\end{javacode}

\paragraph{Parameters:}
\begin{itemize}[noitemsep]
\item \code{String spacename}\\
The name of the space as a string or symbol.

\item \code{Map<String, Object> predicates}\\
A hash of predicates to check against.

\item \code{Map<String, Object> attributes}\\
The set of attributes to modify and their respective values.  \code{attrs} is a
map from the attributes' names to their values.

\end{itemize}

\paragraph{Returns:}
A count of the number of objects, and a \code{client.Error} object indicating
the status of the operation.


\pagebreak
\subsubsection{\code{async\_group\_atomic\_add}}
\label{api:java:async_group_atomic_add}
\index{async\_group\_atomic\_add!Java API}
Add the specified number to the existing value for each object in \code{space}
that matches \code{checks}.

This operation will only affect objects that match the provided \code{checks}.
Objects that do not match \code{checks} will be unaffected by the group call.
Each object that matches \code{checks} will be atomically updated with the check
on the object.  HyperDex guarantees that no object will be altered if the
\code{checks} do not pass at the time of the write.  Objects that are updated
concurrently with the group call may or may not be updated; however, regardless
of any other concurrent operations, the preceding guarantee will always hold.



\paragraph{Definition:}
\begin{javacode}
public Deferred async_group_atomic_add(
        String spacename,
        Map<String, Object> predicates,
        Map<String, Object> attributes) throws HyperDexClientException
\end{javacode}

\paragraph{Parameters:}
\begin{itemize}[noitemsep]
\item \code{String spacename}\\
The name of the space as a string or symbol.

\item \code{Map<String, Object> predicates}\\
A hash of predicates to check against.

\item \code{Map<String, Object> attributes}\\
The set of attributes to modify and their respective values.  \code{attrs} is a
map from the attributes' names to their values.

\end{itemize}

\paragraph{Returns:}
A Deferred object with a \code{wait} method that returns the number of objects
found.  Raises exception on error.


\paragraph{See also:}  This is the asynchronous form of \code{group\_atomic\_add}.

%%%%%%%%%%%%%%%%%%%% cond_atomic_add %%%%%%%%%%%%%%%%%%%%
\pagebreak
\subsubsection{\code{cond\_atomic\_add}}
\label{api:java:cond_atomic_add}
\index{cond\_atomic\_add!Java API}
Conditionally add the specified number to the existing value for each attribute.

%%% Generated below here
\paragraph{Behavior:}
\begin{itemize}[noitemsep]
This operation requires a pre-existing object in order to complete successfully.
If no object exists, the operation will fail with \code{NOTFOUND}.

This operation will succeed if and only if the predicates specified by
\code{checks} hold on the pre-existing object.  If any of the predicates are not
true for the existing object, then the operation will have no effect and fail
with \code{CMPFAIL}.

All checks are atomic with the write.  HyperDex guarantees that no other
operation will come between validating the checks, and writing the new version
of the object.

\end{itemize}


\paragraph{Definition:}
\begin{javacode}
public Boolean cond_atomic_add(
        String spacename,
        Object key,
        Map<String, Object> predicates,
        Map<String, Object> attributes) throws HyperDexClientException
\end{javacode}

\paragraph{Parameters:}
\begin{itemize}[noitemsep]
\item \code{String spacename}\\
The name of the space as a string or symbol.

\item \code{Object key}\\
The key for the operation as a Python type.

\item \code{Map<String, Object> predicates}\\
A hash of predicates to check against.

\item \code{Map<String, Object> attributes}\\
The set of attributes to modify and their respective values.  \code{attrs} is a
map from the attributes' names to their values.

\end{itemize}

\paragraph{Returns:}
This function returns an object indicating the success or failure of the
operation.  Valid values to be returned are:

\begin{itemize}[noitemsep]
\item \code{True} if the operation succeeded
\item \code{False} if any provided predicates failed.
\item \code{null} if the operation requires an existing value and none exists
\end{itemize}

On error, this function will raise a \code{HyperDexClientException} describing
the error.


\pagebreak
\subsubsection{\code{async\_cond\_atomic\_add}}
\label{api:java:async_cond_atomic_add}
\index{async\_cond\_atomic\_add!Java API}
Conditionally add the specified number to the existing value for each attribute.

%%% Generated below here
\paragraph{Behavior:}
\begin{itemize}[noitemsep]
This operation requires a pre-existing object in order to complete successfully.
If no object exists, the operation will fail with \code{NOTFOUND}.

This operation will succeed if and only if the predicates specified by
\code{checks} hold on the pre-existing object.  If any of the predicates are not
true for the existing object, then the operation will have no effect and fail
with \code{CMPFAIL}.

All checks are atomic with the write.  HyperDex guarantees that no other
operation will come between validating the checks, and writing the new version
of the object.

\end{itemize}


\paragraph{Definition:}
\begin{javacode}
public Deferred async_cond_atomic_add(
        String spacename,
        Object key,
        Map<String, Object> predicates,
        Map<String, Object> attributes) throws HyperDexClientException
\end{javacode}

\paragraph{Parameters:}
\begin{itemize}[noitemsep]
\item \code{String spacename}\\
The name of the space as a string or symbol.

\item \code{Object key}\\
The key for the operation as a Python type.

\item \code{Map<String, Object> predicates}\\
A hash of predicates to check against.

\item \code{Map<String, Object> attributes}\\
The set of attributes to modify and their respective values.  \code{attrs} is a
map from the attributes' names to their values.

\end{itemize}

\paragraph{Returns:}
A Deferred object with a \code{wait} method that returns True if the operation
succeeded or False if any provided predicates failed.  Raises an exception on
error.


\paragraph{See also:}  This is the asynchronous form of \code{cond\_atomic\_add}.

%%%%%%%%%%%%%%%%%%%% atomic_sub %%%%%%%%%%%%%%%%%%%%
\pagebreak
\subsubsection{\code{atomic\_sub}}
\label{api:java:atomic_sub}
\index{atomic\_sub!Java API}
Subtract the specified number from the existing value for each attribute.

%%% Generated below here
\paragraph{Behavior:}
\begin{itemize}[noitemsep]
This operation requires a pre-existing object in order to complete successfully.
If no object exists, the operation will fail with \code{NOTFOUND}.

\end{itemize}


\paragraph{Definition:}
\begin{javacode}
public Boolean atomic_sub(
        String spacename,
        Object key,
        Map<String, Object> attributes) throws HyperDexClientException
\end{javacode}

\paragraph{Parameters:}
\begin{itemize}[noitemsep]
\item \code{String spacename}\\
The name of the space as a string or symbol.

\item \code{Object key}\\
The key for the operation as a Python type.

\item \code{Map<String, Object> attributes}\\
The set of attributes to modify and their respective values.  \code{attrs} is a
map from the attributes' names to their values.

\end{itemize}

\paragraph{Returns:}
This function returns an object indicating the success or failure of the
operation.  Valid values to be returned are:

\begin{itemize}[noitemsep]
\item \code{True} if the operation succeeded
\item \code{False} if any provided predicates failed.
\item \code{null} if the operation requires an existing value and none exists
\end{itemize}

On error, this function will raise a \code{HyperDexClientException} describing
the error.


\pagebreak
\subsubsection{\code{async\_atomic\_sub}}
\label{api:java:async_atomic_sub}
\index{async\_atomic\_sub!Java API}
Subtract the specified number from the existing value for each attribute.

%%% Generated below here
\paragraph{Behavior:}
\begin{itemize}[noitemsep]
This operation requires a pre-existing object in order to complete successfully.
If no object exists, the operation will fail with \code{NOTFOUND}.

\end{itemize}


\paragraph{Definition:}
\begin{javacode}
public Deferred async_atomic_sub(
        String spacename,
        Object key,
        Map<String, Object> attributes) throws HyperDexClientException
\end{javacode}

\paragraph{Parameters:}
\begin{itemize}[noitemsep]
\item \code{String spacename}\\
The name of the space as a string or symbol.

\item \code{Object key}\\
The key for the operation as a Python type.

\item \code{Map<String, Object> attributes}\\
The set of attributes to modify and their respective values.  \code{attrs} is a
map from the attributes' names to their values.

\end{itemize}

\paragraph{Returns:}
A Deferred object with a \code{wait} method that returns True if the operation
succeeded or False if any provided predicates failed.  Raises an exception on
error.


\paragraph{See also:}  This is the asynchronous form of \code{atomic\_sub}.

%%%%%%%%%%%%%%%%%%%% cond_atomic_sub %%%%%%%%%%%%%%%%%%%%
\pagebreak
\subsubsection{\code{cond\_atomic\_sub}}
\label{api:java:cond_atomic_sub}
\index{cond\_atomic\_sub!Java API}
Conditionally subtract the specified number from the existing value for each attribute.

%%% Generated below here
\paragraph{Behavior:}
\begin{itemize}[noitemsep]
This operation requires a pre-existing object in order to complete successfully.
If no object exists, the operation will fail with \code{NOTFOUND}.

This operation will succeed if and only if the predicates specified by
\code{checks} hold on the pre-existing object.  If any of the predicates are not
true for the existing object, then the operation will have no effect and fail
with \code{CMPFAIL}.

All checks are atomic with the write.  HyperDex guarantees that no other
operation will come between validating the checks, and writing the new version
of the object.

\end{itemize}


\paragraph{Definition:}
\begin{javacode}
public Boolean cond_atomic_sub(
        String spacename,
        Object key,
        Map<String, Object> predicates,
        Map<String, Object> attributes) throws HyperDexClientException
\end{javacode}

\paragraph{Parameters:}
\begin{itemize}[noitemsep]
\item \code{String spacename}\\
The name of the space as a string or symbol.

\item \code{Object key}\\
The key for the operation as a Python type.

\item \code{Map<String, Object> predicates}\\
A hash of predicates to check against.

\item \code{Map<String, Object> attributes}\\
The set of attributes to modify and their respective values.  \code{attrs} is a
map from the attributes' names to their values.

\end{itemize}

\paragraph{Returns:}
This function returns an object indicating the success or failure of the
operation.  Valid values to be returned are:

\begin{itemize}[noitemsep]
\item \code{True} if the operation succeeded
\item \code{False} if any provided predicates failed.
\item \code{null} if the operation requires an existing value and none exists
\end{itemize}

On error, this function will raise a \code{HyperDexClientException} describing
the error.


\pagebreak
\subsubsection{\code{async\_cond\_atomic\_sub}}
\label{api:java:async_cond_atomic_sub}
\index{async\_cond\_atomic\_sub!Java API}
Conditionally subtract the specified number from the existing value for each attribute.

%%% Generated below here
\paragraph{Behavior:}
\begin{itemize}[noitemsep]
This operation requires a pre-existing object in order to complete successfully.
If no object exists, the operation will fail with \code{NOTFOUND}.

This operation will succeed if and only if the predicates specified by
\code{checks} hold on the pre-existing object.  If any of the predicates are not
true for the existing object, then the operation will have no effect and fail
with \code{CMPFAIL}.

All checks are atomic with the write.  HyperDex guarantees that no other
operation will come between validating the checks, and writing the new version
of the object.

\end{itemize}


\paragraph{Definition:}
\begin{javacode}
public Deferred async_cond_atomic_sub(
        String spacename,
        Object key,
        Map<String, Object> predicates,
        Map<String, Object> attributes) throws HyperDexClientException
\end{javacode}

\paragraph{Parameters:}
\begin{itemize}[noitemsep]
\item \code{String spacename}\\
The name of the space as a string or symbol.

\item \code{Object key}\\
The key for the operation as a Python type.

\item \code{Map<String, Object> predicates}\\
A hash of predicates to check against.

\item \code{Map<String, Object> attributes}\\
The set of attributes to modify and their respective values.  \code{attrs} is a
map from the attributes' names to their values.

\end{itemize}

\paragraph{Returns:}
A Deferred object with a \code{wait} method that returns True if the operation
succeeded or False if any provided predicates failed.  Raises an exception on
error.


\paragraph{See also:}  This is the asynchronous form of \code{cond\_atomic\_sub}.

%%%%%%%%%%%%%%%%%%%% atomic_mul %%%%%%%%%%%%%%%%%%%%
\pagebreak
\subsubsection{\code{atomic\_mul}}
\label{api:java:atomic_mul}
\index{atomic\_mul!Java API}
Multiply the existing value by the number specified for each attribute.

The multiplication is atomic with the write.  If the object does not exist, the
operation will fail.

\noindent\textbf{Cost:}  Approximately one traversal of the value-dependent
chain.


\noindent\textbf{Consistency:}  Linearizable



\paragraph{Definition:}
\begin{javacode}
public Boolean atomic_mul(
        String spacename,
        Object key,
        Map<String, Object> attributes) throws HyperDexClientException
\end{javacode}

\paragraph{Parameters:}
\begin{itemize}[noitemsep]
\item \code{String spacename}\\
The name of the space as a string or symbol.

\item \code{Object key}\\
The key for the operation as a Python type.

\item \code{Map<String, Object> attributes}\\
The set of attributes to modify and their respective values.  \code{attrs} is a
map from the attributes' names to their values.

\end{itemize}

\paragraph{Returns:}
This function returns an object indicating the success or failure of the
operation.  Valid values to be returned are:

\begin{itemize}[noitemsep]
\item \code{True} if the operation succeeded
\item \code{False} if any provided predicates failed.
\item \code{null} if the operation requires an existing value and none exists
\end{itemize}

On error, this function will raise a \code{HyperDexClientException} describing
the error.


\pagebreak
\subsubsection{\code{async\_atomic\_mul}}
\label{api:java:async_atomic_mul}
\index{async\_atomic\_mul!Java API}
Multiply the existing value by the number specified for each attribute.

The multiplication is atomic with the write.  If the object does not exist, the
operation will fail.

\noindent\textbf{Cost:}  Approximately one traversal of the value-dependent
chain.


\noindent\textbf{Consistency:}  Linearizable



\paragraph{Definition:}
\begin{javacode}
public Deferred async_atomic_mul(
        String spacename,
        Object key,
        Map<String, Object> attributes) throws HyperDexClientException
\end{javacode}

\paragraph{Parameters:}
\begin{itemize}[noitemsep]
\item \code{String spacename}\\
The name of the space as a string or symbol.

\item \code{Object key}\\
The key for the operation as a Python type.

\item \code{Map<String, Object> attributes}\\
The set of attributes to modify and their respective values.  \code{attrs} is a
map from the attributes' names to their values.

\end{itemize}

\paragraph{Returns:}
A Deferred object with a \code{wait} method that returns True if the operation
succeeded or False if any provided predicates failed.  Raises an exception on
error.


\paragraph{See also:}  This is the asynchronous form of \code{atomic\_mul}.

%%%%%%%%%%%%%%%%%%%% cond_atomic_mul %%%%%%%%%%%%%%%%%%%%
\pagebreak
\subsubsection{\code{cond\_atomic\_mul}}
\label{api:java:cond_atomic_mul}
\index{cond\_atomic\_mul!Java API}
Conditionally multiply the existing value by the specified number for each
attribute.

%%% Generated below here
\paragraph{Behavior:}
\begin{itemize}[noitemsep]
This operation requires a pre-existing object in order to complete successfully.
If no object exists, the operation will fail with \code{NOTFOUND}.

This operation will succeed if and only if the predicates specified by
\code{checks} hold on the pre-existing object.  If any of the predicates are not
true for the existing object, then the operation will have no effect and fail
with \code{CMPFAIL}.

All checks are atomic with the write.  HyperDex guarantees that no other
operation will come between validating the checks, and writing the new version
of the object.

\end{itemize}


\paragraph{Definition:}
\begin{javacode}
public Boolean cond_atomic_mul(
        String spacename,
        Object key,
        Map<String, Object> predicates,
        Map<String, Object> attributes) throws HyperDexClientException
\end{javacode}

\paragraph{Parameters:}
\begin{itemize}[noitemsep]
\item \code{String spacename}\\
The name of the space as a string or symbol.

\item \code{Object key}\\
The key for the operation as a Python type.

\item \code{Map<String, Object> predicates}\\
A hash of predicates to check against.

\item \code{Map<String, Object> attributes}\\
The set of attributes to modify and their respective values.  \code{attrs} is a
map from the attributes' names to their values.

\end{itemize}

\paragraph{Returns:}
This function returns an object indicating the success or failure of the
operation.  Valid values to be returned are:

\begin{itemize}[noitemsep]
\item \code{True} if the operation succeeded
\item \code{False} if any provided predicates failed.
\item \code{null} if the operation requires an existing value and none exists
\end{itemize}

On error, this function will raise a \code{HyperDexClientException} describing
the error.


\pagebreak
\subsubsection{\code{async\_cond\_atomic\_mul}}
\label{api:java:async_cond_atomic_mul}
\index{async\_cond\_atomic\_mul!Java API}
Conditionally multiply the existing value by the specified number for each
attribute.

%%% Generated below here
\paragraph{Behavior:}
\begin{itemize}[noitemsep]
This operation requires a pre-existing object in order to complete successfully.
If no object exists, the operation will fail with \code{NOTFOUND}.

This operation will succeed if and only if the predicates specified by
\code{checks} hold on the pre-existing object.  If any of the predicates are not
true for the existing object, then the operation will have no effect and fail
with \code{CMPFAIL}.

All checks are atomic with the write.  HyperDex guarantees that no other
operation will come between validating the checks, and writing the new version
of the object.

\end{itemize}


\paragraph{Definition:}
\begin{javacode}
public Deferred async_cond_atomic_mul(
        String spacename,
        Object key,
        Map<String, Object> predicates,
        Map<String, Object> attributes) throws HyperDexClientException
\end{javacode}

\paragraph{Parameters:}
\begin{itemize}[noitemsep]
\item \code{String spacename}\\
The name of the space as a string or symbol.

\item \code{Object key}\\
The key for the operation as a Python type.

\item \code{Map<String, Object> predicates}\\
A hash of predicates to check against.

\item \code{Map<String, Object> attributes}\\
The set of attributes to modify and their respective values.  \code{attrs} is a
map from the attributes' names to their values.

\end{itemize}

\paragraph{Returns:}
A Deferred object with a \code{wait} method that returns True if the operation
succeeded or False if any provided predicates failed.  Raises an exception on
error.


\paragraph{See also:}  This is the asynchronous form of \code{cond\_atomic\_mul}.

%%%%%%%%%%%%%%%%%%%% atomic_div %%%%%%%%%%%%%%%%%%%%
\pagebreak
\subsubsection{\code{atomic\_div}}
\label{api:java:atomic_div}
\index{atomic\_div!Java API}
Divide the existing value by the specified number for each attribute.
This operation requires a pre-existing object in order to complete successfully.
If no object exists, the operation will fail with \code{NOTFOUND}.



\paragraph{Definition:}
\begin{javacode}
public Boolean atomic_div(
        String spacename,
        Object key,
        Map<String, Object> attributes) throws HyperDexClientException
\end{javacode}

\paragraph{Parameters:}
\begin{itemize}[noitemsep]
\item \code{String spacename}\\
The name of the space as a string or symbol.

\item \code{Object key}\\
The key for the operation as a Python type.

\item \code{Map<String, Object> attributes}\\
The set of attributes to modify and their respective values.  \code{attrs} is a
map from the attributes' names to their values.

\end{itemize}

\paragraph{Returns:}
This function returns an object indicating the success or failure of the
operation.  Valid values to be returned are:

\begin{itemize}[noitemsep]
\item \code{True} if the operation succeeded
\item \code{False} if any provided predicates failed.
\item \code{null} if the operation requires an existing value and none exists
\end{itemize}

On error, this function will raise a \code{HyperDexClientException} describing
the error.


\pagebreak
\subsubsection{\code{async\_atomic\_div}}
\label{api:java:async_atomic_div}
\index{async\_atomic\_div!Java API}
Divide the existing value by the specified number for each attribute.
This operation requires a pre-existing object in order to complete successfully.
If no object exists, the operation will fail with \code{NOTFOUND}.



\paragraph{Definition:}
\begin{javacode}
public Deferred async_atomic_div(
        String spacename,
        Object key,
        Map<String, Object> attributes) throws HyperDexClientException
\end{javacode}

\paragraph{Parameters:}
\begin{itemize}[noitemsep]
\item \code{String spacename}\\
The name of the space as a string or symbol.

\item \code{Object key}\\
The key for the operation as a Python type.

\item \code{Map<String, Object> attributes}\\
The set of attributes to modify and their respective values.  \code{attrs} is a
map from the attributes' names to their values.

\end{itemize}

\paragraph{Returns:}
A Deferred object with a \code{wait} method that returns True if the operation
succeeded or False if any provided predicates failed.  Raises an exception on
error.


\paragraph{See also:}  This is the asynchronous form of \code{atomic\_div}.

%%%%%%%%%%%%%%%%%%%% cond_atomic_div %%%%%%%%%%%%%%%%%%%%
\pagebreak
\subsubsection{\code{cond\_atomic\_div}}
\label{api:java:cond_atomic_div}
\index{cond\_atomic\_div!Java API}
Conditionally divide the existing value by the specified number for each
attribute.

%%% Generated below here
\paragraph{Behavior:}
\begin{itemize}[noitemsep]
This operation requires a pre-existing object in order to complete successfully.
If no object exists, the operation will fail with \code{NOTFOUND}.

This operation will succeed if and only if the predicates specified by
\code{checks} hold on the pre-existing object.  If any of the predicates are not
true for the existing object, then the operation will have no effect and fail
with \code{CMPFAIL}.

All checks are atomic with the write.  HyperDex guarantees that no other
operation will come between validating the checks, and writing the new version
of the object.

\end{itemize}


\paragraph{Definition:}
\begin{javacode}
public Boolean cond_atomic_div(
        String spacename,
        Object key,
        Map<String, Object> predicates,
        Map<String, Object> attributes) throws HyperDexClientException
\end{javacode}

\paragraph{Parameters:}
\begin{itemize}[noitemsep]
\item \code{String spacename}\\
The name of the space as a string or symbol.

\item \code{Object key}\\
The key for the operation as a Python type.

\item \code{Map<String, Object> predicates}\\
A hash of predicates to check against.

\item \code{Map<String, Object> attributes}\\
The set of attributes to modify and their respective values.  \code{attrs} is a
map from the attributes' names to their values.

\end{itemize}

\paragraph{Returns:}
This function returns an object indicating the success or failure of the
operation.  Valid values to be returned are:

\begin{itemize}[noitemsep]
\item \code{True} if the operation succeeded
\item \code{False} if any provided predicates failed.
\item \code{null} if the operation requires an existing value and none exists
\end{itemize}

On error, this function will raise a \code{HyperDexClientException} describing
the error.


\pagebreak
\subsubsection{\code{async\_cond\_atomic\_div}}
\label{api:java:async_cond_atomic_div}
\index{async\_cond\_atomic\_div!Java API}
Conditionally divide the existing value by the specified number for each
attribute.

%%% Generated below here
\paragraph{Behavior:}
\begin{itemize}[noitemsep]
This operation requires a pre-existing object in order to complete successfully.
If no object exists, the operation will fail with \code{NOTFOUND}.

This operation will succeed if and only if the predicates specified by
\code{checks} hold on the pre-existing object.  If any of the predicates are not
true for the existing object, then the operation will have no effect and fail
with \code{CMPFAIL}.

All checks are atomic with the write.  HyperDex guarantees that no other
operation will come between validating the checks, and writing the new version
of the object.

\end{itemize}


\paragraph{Definition:}
\begin{javacode}
public Deferred async_cond_atomic_div(
        String spacename,
        Object key,
        Map<String, Object> predicates,
        Map<String, Object> attributes) throws HyperDexClientException
\end{javacode}

\paragraph{Parameters:}
\begin{itemize}[noitemsep]
\item \code{String spacename}\\
The name of the space as a string or symbol.

\item \code{Object key}\\
The key for the operation as a Python type.

\item \code{Map<String, Object> predicates}\\
A hash of predicates to check against.

\item \code{Map<String, Object> attributes}\\
The set of attributes to modify and their respective values.  \code{attrs} is a
map from the attributes' names to their values.

\end{itemize}

\paragraph{Returns:}
A Deferred object with a \code{wait} method that returns True if the operation
succeeded or False if any provided predicates failed.  Raises an exception on
error.


\paragraph{See also:}  This is the asynchronous form of \code{cond\_atomic\_div}.

%%%%%%%%%%%%%%%%%%%% atomic_mod %%%%%%%%%%%%%%%%%%%%
\pagebreak
\subsubsection{\code{atomic\_mod}}
\label{api:java:atomic_mod}
\index{atomic\_mod!Java API}
Store the existing value modulo the specified number for each attribute.

%%% Generated below here
\paragraph{Behavior:}
\begin{itemize}[noitemsep]
This operation requires a pre-existing object in order to complete successfully.
If no object exists, the operation will fail with \code{NOTFOUND}.

\end{itemize}


\paragraph{Definition:}
\begin{javacode}
public Boolean atomic_mod(
        String spacename,
        Object key,
        Map<String, Object> attributes) throws HyperDexClientException
\end{javacode}

\paragraph{Parameters:}
\begin{itemize}[noitemsep]
\item \code{String spacename}\\
The name of the space as a string or symbol.

\item \code{Object key}\\
The key for the operation as a Python type.

\item \code{Map<String, Object> attributes}\\
The set of attributes to modify and their respective values.  \code{attrs} is a
map from the attributes' names to their values.

\end{itemize}

\paragraph{Returns:}
This function returns an object indicating the success or failure of the
operation.  Valid values to be returned are:

\begin{itemize}[noitemsep]
\item \code{True} if the operation succeeded
\item \code{False} if any provided predicates failed.
\item \code{null} if the operation requires an existing value and none exists
\end{itemize}

On error, this function will raise a \code{HyperDexClientException} describing
the error.


\pagebreak
\subsubsection{\code{async\_atomic\_mod}}
\label{api:java:async_atomic_mod}
\index{async\_atomic\_mod!Java API}
Store the existing value modulo the specified number for each attribute.

%%% Generated below here
\paragraph{Behavior:}
\begin{itemize}[noitemsep]
This operation requires a pre-existing object in order to complete successfully.
If no object exists, the operation will fail with \code{NOTFOUND}.

\end{itemize}


\paragraph{Definition:}
\begin{javacode}
public Deferred async_atomic_mod(
        String spacename,
        Object key,
        Map<String, Object> attributes) throws HyperDexClientException
\end{javacode}

\paragraph{Parameters:}
\begin{itemize}[noitemsep]
\item \code{String spacename}\\
The name of the space as a string or symbol.

\item \code{Object key}\\
The key for the operation as a Python type.

\item \code{Map<String, Object> attributes}\\
The set of attributes to modify and their respective values.  \code{attrs} is a
map from the attributes' names to their values.

\end{itemize}

\paragraph{Returns:}
A Deferred object with a \code{wait} method that returns True if the operation
succeeded or False if any provided predicates failed.  Raises an exception on
error.


\paragraph{See also:}  This is the asynchronous form of \code{atomic\_mod}.

%%%%%%%%%%%%%%%%%%%% cond_atomic_mod %%%%%%%%%%%%%%%%%%%%
\pagebreak
\subsubsection{\code{cond\_atomic\_mod}}
\label{api:java:cond_atomic_mod}
\index{cond\_atomic\_mod!Java API}
Conditionally store the existing value modulo the specified number for each
attribute.

%%% Generated below here
\paragraph{Behavior:}
\begin{itemize}[noitemsep]
This operation requires a pre-existing object in order to complete successfully.
If no object exists, the operation will fail with \code{NOTFOUND}.

This operation will succeed if and only if the predicates specified by
\code{checks} hold on the pre-existing object.  If any of the predicates are not
true for the existing object, then the operation will have no effect and fail
with \code{CMPFAIL}.

All checks are atomic with the write.  HyperDex guarantees that no other
operation will come between validating the checks, and writing the new version
of the object.

\end{itemize}


\paragraph{Definition:}
\begin{javacode}
public Boolean cond_atomic_mod(
        String spacename,
        Object key,
        Map<String, Object> predicates,
        Map<String, Object> attributes) throws HyperDexClientException
\end{javacode}

\paragraph{Parameters:}
\begin{itemize}[noitemsep]
\item \code{String spacename}\\
The name of the space as a string or symbol.

\item \code{Object key}\\
The key for the operation as a Python type.

\item \code{Map<String, Object> predicates}\\
A hash of predicates to check against.

\item \code{Map<String, Object> attributes}\\
The set of attributes to modify and their respective values.  \code{attrs} is a
map from the attributes' names to their values.

\end{itemize}

\paragraph{Returns:}
This function returns an object indicating the success or failure of the
operation.  Valid values to be returned are:

\begin{itemize}[noitemsep]
\item \code{True} if the operation succeeded
\item \code{False} if any provided predicates failed.
\item \code{null} if the operation requires an existing value and none exists
\end{itemize}

On error, this function will raise a \code{HyperDexClientException} describing
the error.


\pagebreak
\subsubsection{\code{async\_cond\_atomic\_mod}}
\label{api:java:async_cond_atomic_mod}
\index{async\_cond\_atomic\_mod!Java API}
Conditionally store the existing value modulo the specified number for each
attribute.

%%% Generated below here
\paragraph{Behavior:}
\begin{itemize}[noitemsep]
This operation requires a pre-existing object in order to complete successfully.
If no object exists, the operation will fail with \code{NOTFOUND}.

This operation will succeed if and only if the predicates specified by
\code{checks} hold on the pre-existing object.  If any of the predicates are not
true for the existing object, then the operation will have no effect and fail
with \code{CMPFAIL}.

All checks are atomic with the write.  HyperDex guarantees that no other
operation will come between validating the checks, and writing the new version
of the object.

\end{itemize}


\paragraph{Definition:}
\begin{javacode}
public Deferred async_cond_atomic_mod(
        String spacename,
        Object key,
        Map<String, Object> predicates,
        Map<String, Object> attributes) throws HyperDexClientException
\end{javacode}

\paragraph{Parameters:}
\begin{itemize}[noitemsep]
\item \code{String spacename}\\
The name of the space as a string or symbol.

\item \code{Object key}\\
The key for the operation as a Python type.

\item \code{Map<String, Object> predicates}\\
A hash of predicates to check against.

\item \code{Map<String, Object> attributes}\\
The set of attributes to modify and their respective values.  \code{attrs} is a
map from the attributes' names to their values.

\end{itemize}

\paragraph{Returns:}
A Deferred object with a \code{wait} method that returns True if the operation
succeeded or False if any provided predicates failed.  Raises an exception on
error.


\paragraph{See also:}  This is the asynchronous form of \code{cond\_atomic\_mod}.

%%%%%%%%%%%%%%%%%%%% atomic_and %%%%%%%%%%%%%%%%%%%%
\pagebreak
\subsubsection{\code{atomic\_and}}
\label{api:java:atomic_and}
\index{atomic\_and!Java API}
Store the bitwise AND of the existing value and the specified number for
each attribute.
This operation requires a pre-existing object in order to complete successfully.
If no object exists, the operation will fail with \code{NOTFOUND}.



\paragraph{Definition:}
\begin{javacode}
public Boolean atomic_and(
        String spacename,
        Object key,
        Map<String, Object> attributes) throws HyperDexClientException
\end{javacode}

\paragraph{Parameters:}
\begin{itemize}[noitemsep]
\item \code{String spacename}\\
The name of the space as a string or symbol.

\item \code{Object key}\\
The key for the operation as a Python type.

\item \code{Map<String, Object> attributes}\\
The set of attributes to modify and their respective values.  \code{attrs} is a
map from the attributes' names to their values.

\end{itemize}

\paragraph{Returns:}
This function returns an object indicating the success or failure of the
operation.  Valid values to be returned are:

\begin{itemize}[noitemsep]
\item \code{True} if the operation succeeded
\item \code{False} if any provided predicates failed.
\item \code{null} if the operation requires an existing value and none exists
\end{itemize}

On error, this function will raise a \code{HyperDexClientException} describing
the error.


\pagebreak
\subsubsection{\code{async\_atomic\_and}}
\label{api:java:async_atomic_and}
\index{async\_atomic\_and!Java API}
Store the bitwise AND of the existing value and the specified number for
each attribute.
This operation requires a pre-existing object in order to complete successfully.
If no object exists, the operation will fail with \code{NOTFOUND}.



\paragraph{Definition:}
\begin{javacode}
public Deferred async_atomic_and(
        String spacename,
        Object key,
        Map<String, Object> attributes) throws HyperDexClientException
\end{javacode}

\paragraph{Parameters:}
\begin{itemize}[noitemsep]
\item \code{String spacename}\\
The name of the space as a string or symbol.

\item \code{Object key}\\
The key for the operation as a Python type.

\item \code{Map<String, Object> attributes}\\
The set of attributes to modify and their respective values.  \code{attrs} is a
map from the attributes' names to their values.

\end{itemize}

\paragraph{Returns:}
A Deferred object with a \code{wait} method that returns True if the operation
succeeded or False if any provided predicates failed.  Raises an exception on
error.


\paragraph{See also:}  This is the asynchronous form of \code{atomic\_and}.

%%%%%%%%%%%%%%%%%%%% cond_atomic_and %%%%%%%%%%%%%%%%%%%%
\pagebreak
\subsubsection{\code{cond\_atomic\_and}}
\label{api:java:cond_atomic_and}
\index{cond\_atomic\_and!Java API}
Conditionally store the bitwise AND of the existing value and the specified
number for each attribute.

%%% Generated below here
\paragraph{Behavior:}
\begin{itemize}[noitemsep]
This operation requires a pre-existing object in order to complete successfully.
If no object exists, the operation will fail with \code{NOTFOUND}.

This operation will succeed if and only if the predicates specified by
\code{checks} hold on the pre-existing object.  If any of the predicates are not
true for the existing object, then the operation will have no effect and fail
with \code{CMPFAIL}.

All checks are atomic with the write.  HyperDex guarantees that no other
operation will come between validating the checks, and writing the new version
of the object.

\end{itemize}


\paragraph{Definition:}
\begin{javacode}
public Boolean cond_atomic_and(
        String spacename,
        Object key,
        Map<String, Object> predicates,
        Map<String, Object> attributes) throws HyperDexClientException
\end{javacode}

\paragraph{Parameters:}
\begin{itemize}[noitemsep]
\item \code{String spacename}\\
The name of the space as a string or symbol.

\item \code{Object key}\\
The key for the operation as a Python type.

\item \code{Map<String, Object> predicates}\\
A hash of predicates to check against.

\item \code{Map<String, Object> attributes}\\
The set of attributes to modify and their respective values.  \code{attrs} is a
map from the attributes' names to their values.

\end{itemize}

\paragraph{Returns:}
This function returns an object indicating the success or failure of the
operation.  Valid values to be returned are:

\begin{itemize}[noitemsep]
\item \code{True} if the operation succeeded
\item \code{False} if any provided predicates failed.
\item \code{null} if the operation requires an existing value and none exists
\end{itemize}

On error, this function will raise a \code{HyperDexClientException} describing
the error.


\pagebreak
\subsubsection{\code{async\_cond\_atomic\_and}}
\label{api:java:async_cond_atomic_and}
\index{async\_cond\_atomic\_and!Java API}
Conditionally store the bitwise AND of the existing value and the specified
number for each attribute.

%%% Generated below here
\paragraph{Behavior:}
\begin{itemize}[noitemsep]
This operation requires a pre-existing object in order to complete successfully.
If no object exists, the operation will fail with \code{NOTFOUND}.

This operation will succeed if and only if the predicates specified by
\code{checks} hold on the pre-existing object.  If any of the predicates are not
true for the existing object, then the operation will have no effect and fail
with \code{CMPFAIL}.

All checks are atomic with the write.  HyperDex guarantees that no other
operation will come between validating the checks, and writing the new version
of the object.

\end{itemize}


\paragraph{Definition:}
\begin{javacode}
public Deferred async_cond_atomic_and(
        String spacename,
        Object key,
        Map<String, Object> predicates,
        Map<String, Object> attributes) throws HyperDexClientException
\end{javacode}

\paragraph{Parameters:}
\begin{itemize}[noitemsep]
\item \code{String spacename}\\
The name of the space as a string or symbol.

\item \code{Object key}\\
The key for the operation as a Python type.

\item \code{Map<String, Object> predicates}\\
A hash of predicates to check against.

\item \code{Map<String, Object> attributes}\\
The set of attributes to modify and their respective values.  \code{attrs} is a
map from the attributes' names to their values.

\end{itemize}

\paragraph{Returns:}
A Deferred object with a \code{wait} method that returns True if the operation
succeeded or False if any provided predicates failed.  Raises an exception on
error.


\paragraph{See also:}  This is the asynchronous form of \code{cond\_atomic\_and}.

%%%%%%%%%%%%%%%%%%%% atomic_or %%%%%%%%%%%%%%%%%%%%
\pagebreak
\subsubsection{\code{atomic\_or}}
\label{api:java:atomic_or}
\index{atomic\_or!Java API}
Store the bitwise OR of the existing value and the specified number for each
attribute.

%%% Generated below here
\paragraph{Behavior:}
\begin{itemize}[noitemsep]
This operation requires a pre-existing object in order to complete successfully.
If no object exists, the operation will fail with \code{NOTFOUND}.

\end{itemize}


\paragraph{Definition:}
\begin{javacode}
public Boolean atomic_or(
        String spacename,
        Object key,
        Map<String, Object> attributes) throws HyperDexClientException
\end{javacode}

\paragraph{Parameters:}
\begin{itemize}[noitemsep]
\item \code{String spacename}\\
The name of the space as a string or symbol.

\item \code{Object key}\\
The key for the operation as a Python type.

\item \code{Map<String, Object> attributes}\\
The set of attributes to modify and their respective values.  \code{attrs} is a
map from the attributes' names to their values.

\end{itemize}

\paragraph{Returns:}
This function returns an object indicating the success or failure of the
operation.  Valid values to be returned are:

\begin{itemize}[noitemsep]
\item \code{True} if the operation succeeded
\item \code{False} if any provided predicates failed.
\item \code{null} if the operation requires an existing value and none exists
\end{itemize}

On error, this function will raise a \code{HyperDexClientException} describing
the error.


\pagebreak
\subsubsection{\code{async\_atomic\_or}}
\label{api:java:async_atomic_or}
\index{async\_atomic\_or!Java API}
Store the bitwise OR of the existing value and the specified number for each
attribute.

%%% Generated below here
\paragraph{Behavior:}
\begin{itemize}[noitemsep]
This operation requires a pre-existing object in order to complete successfully.
If no object exists, the operation will fail with \code{NOTFOUND}.

\end{itemize}


\paragraph{Definition:}
\begin{javacode}
public Deferred async_atomic_or(
        String spacename,
        Object key,
        Map<String, Object> attributes) throws HyperDexClientException
\end{javacode}

\paragraph{Parameters:}
\begin{itemize}[noitemsep]
\item \code{String spacename}\\
The name of the space as a string or symbol.

\item \code{Object key}\\
The key for the operation as a Python type.

\item \code{Map<String, Object> attributes}\\
The set of attributes to modify and their respective values.  \code{attrs} is a
map from the attributes' names to their values.

\end{itemize}

\paragraph{Returns:}
A Deferred object with a \code{wait} method that returns True if the operation
succeeded or False if any provided predicates failed.  Raises an exception on
error.


\paragraph{See also:}  This is the asynchronous form of \code{atomic\_or}.

%%%%%%%%%%%%%%%%%%%% cond_atomic_or %%%%%%%%%%%%%%%%%%%%
\pagebreak
\subsubsection{\code{cond\_atomic\_or}}
\label{api:java:cond_atomic_or}
\index{cond\_atomic\_or!Java API}
Conditionally store the bitwise OR of the existing value and the specified
number for each attribute.

%%% Generated below here
\paragraph{Behavior:}
\begin{itemize}[noitemsep]
This operation requires a pre-existing object in order to complete successfully.
If no object exists, the operation will fail with \code{NOTFOUND}.

This operation will succeed if and only if the predicates specified by
\code{checks} hold on the pre-existing object.  If any of the predicates are not
true for the existing object, then the operation will have no effect and fail
with \code{CMPFAIL}.

All checks are atomic with the write.  HyperDex guarantees that no other
operation will come between validating the checks, and writing the new version
of the object.

\end{itemize}


\paragraph{Definition:}
\begin{javacode}
public Boolean cond_atomic_or(
        String spacename,
        Object key,
        Map<String, Object> predicates,
        Map<String, Object> attributes) throws HyperDexClientException
\end{javacode}

\paragraph{Parameters:}
\begin{itemize}[noitemsep]
\item \code{String spacename}\\
The name of the space as a string or symbol.

\item \code{Object key}\\
The key for the operation as a Python type.

\item \code{Map<String, Object> predicates}\\
A hash of predicates to check against.

\item \code{Map<String, Object> attributes}\\
The set of attributes to modify and their respective values.  \code{attrs} is a
map from the attributes' names to their values.

\end{itemize}

\paragraph{Returns:}
This function returns an object indicating the success or failure of the
operation.  Valid values to be returned are:

\begin{itemize}[noitemsep]
\item \code{True} if the operation succeeded
\item \code{False} if any provided predicates failed.
\item \code{null} if the operation requires an existing value and none exists
\end{itemize}

On error, this function will raise a \code{HyperDexClientException} describing
the error.


\pagebreak
\subsubsection{\code{async\_cond\_atomic\_or}}
\label{api:java:async_cond_atomic_or}
\index{async\_cond\_atomic\_or!Java API}
Conditionally store the bitwise OR of the existing value and the specified
number for each attribute.

%%% Generated below here
\paragraph{Behavior:}
\begin{itemize}[noitemsep]
This operation requires a pre-existing object in order to complete successfully.
If no object exists, the operation will fail with \code{NOTFOUND}.

This operation will succeed if and only if the predicates specified by
\code{checks} hold on the pre-existing object.  If any of the predicates are not
true for the existing object, then the operation will have no effect and fail
with \code{CMPFAIL}.

All checks are atomic with the write.  HyperDex guarantees that no other
operation will come between validating the checks, and writing the new version
of the object.

\end{itemize}


\paragraph{Definition:}
\begin{javacode}
public Deferred async_cond_atomic_or(
        String spacename,
        Object key,
        Map<String, Object> predicates,
        Map<String, Object> attributes) throws HyperDexClientException
\end{javacode}

\paragraph{Parameters:}
\begin{itemize}[noitemsep]
\item \code{String spacename}\\
The name of the space as a string or symbol.

\item \code{Object key}\\
The key for the operation as a Python type.

\item \code{Map<String, Object> predicates}\\
A hash of predicates to check against.

\item \code{Map<String, Object> attributes}\\
The set of attributes to modify and their respective values.  \code{attrs} is a
map from the attributes' names to their values.

\end{itemize}

\paragraph{Returns:}
A Deferred object with a \code{wait} method that returns True if the operation
succeeded or False if any provided predicates failed.  Raises an exception on
error.


\paragraph{See also:}  This is the asynchronous form of \code{cond\_atomic\_or}.

%%%%%%%%%%%%%%%%%%%% atomic_xor %%%%%%%%%%%%%%%%%%%%
\pagebreak
\subsubsection{\code{atomic\_xor}}
\label{api:java:atomic_xor}
\index{atomic\_xor!Java API}
Store the bitwise XOR of the existing value and the specified number for each
attribute.

%%% Generated below here
\paragraph{Behavior:}
\begin{itemize}[noitemsep]
This operation requires a pre-existing object in order to complete successfully.
If no object exists, the operation will fail with \code{NOTFOUND}.

\end{itemize}


\paragraph{Definition:}
\begin{javacode}
public Boolean atomic_xor(
        String spacename,
        Object key,
        Map<String, Object> attributes) throws HyperDexClientException
\end{javacode}

\paragraph{Parameters:}
\begin{itemize}[noitemsep]
\item \code{String spacename}\\
The name of the space as a string or symbol.

\item \code{Object key}\\
The key for the operation as a Python type.

\item \code{Map<String, Object> attributes}\\
The set of attributes to modify and their respective values.  \code{attrs} is a
map from the attributes' names to their values.

\end{itemize}

\paragraph{Returns:}
This function returns an object indicating the success or failure of the
operation.  Valid values to be returned are:

\begin{itemize}[noitemsep]
\item \code{True} if the operation succeeded
\item \code{False} if any provided predicates failed.
\item \code{null} if the operation requires an existing value and none exists
\end{itemize}

On error, this function will raise a \code{HyperDexClientException} describing
the error.


\pagebreak
\subsubsection{\code{async\_atomic\_xor}}
\label{api:java:async_atomic_xor}
\index{async\_atomic\_xor!Java API}
Store the bitwise XOR of the existing value and the specified number for each
attribute.

%%% Generated below here
\paragraph{Behavior:}
\begin{itemize}[noitemsep]
This operation requires a pre-existing object in order to complete successfully.
If no object exists, the operation will fail with \code{NOTFOUND}.

\end{itemize}


\paragraph{Definition:}
\begin{javacode}
public Deferred async_atomic_xor(
        String spacename,
        Object key,
        Map<String, Object> attributes) throws HyperDexClientException
\end{javacode}

\paragraph{Parameters:}
\begin{itemize}[noitemsep]
\item \code{String spacename}\\
The name of the space as a string or symbol.

\item \code{Object key}\\
The key for the operation as a Python type.

\item \code{Map<String, Object> attributes}\\
The set of attributes to modify and their respective values.  \code{attrs} is a
map from the attributes' names to their values.

\end{itemize}

\paragraph{Returns:}
A Deferred object with a \code{wait} method that returns True if the operation
succeeded or False if any provided predicates failed.  Raises an exception on
error.


\paragraph{See also:}  This is the asynchronous form of \code{atomic\_xor}.

%%%%%%%%%%%%%%%%%%%% cond_atomic_xor %%%%%%%%%%%%%%%%%%%%
\pagebreak
\subsubsection{\code{cond\_atomic\_xor}}
\label{api:java:cond_atomic_xor}
\index{cond\_atomic\_xor!Java API}
Conditionally store the bitwise XOR of the existing value and the specified
number for each attribute.

%%% Generated below here
\paragraph{Behavior:}
\begin{itemize}[noitemsep]
This operation requires a pre-existing object in order to complete successfully.
If no object exists, the operation will fail with \code{NOTFOUND}.

This operation will succeed if and only if the predicates specified by
\code{checks} hold on the pre-existing object.  If any of the predicates are not
true for the existing object, then the operation will have no effect and fail
with \code{CMPFAIL}.

All checks are atomic with the write.  HyperDex guarantees that no other
operation will come between validating the checks, and writing the new version
of the object.

\end{itemize}


\paragraph{Definition:}
\begin{javacode}
public Boolean cond_atomic_xor(
        String spacename,
        Object key,
        Map<String, Object> predicates,
        Map<String, Object> attributes) throws HyperDexClientException
\end{javacode}

\paragraph{Parameters:}
\begin{itemize}[noitemsep]
\item \code{String spacename}\\
The name of the space as a string or symbol.

\item \code{Object key}\\
The key for the operation as a Python type.

\item \code{Map<String, Object> predicates}\\
A hash of predicates to check against.

\item \code{Map<String, Object> attributes}\\
The set of attributes to modify and their respective values.  \code{attrs} is a
map from the attributes' names to their values.

\end{itemize}

\paragraph{Returns:}
This function returns an object indicating the success or failure of the
operation.  Valid values to be returned are:

\begin{itemize}[noitemsep]
\item \code{True} if the operation succeeded
\item \code{False} if any provided predicates failed.
\item \code{null} if the operation requires an existing value and none exists
\end{itemize}

On error, this function will raise a \code{HyperDexClientException} describing
the error.


\pagebreak
\subsubsection{\code{async\_cond\_atomic\_xor}}
\label{api:java:async_cond_atomic_xor}
\index{async\_cond\_atomic\_xor!Java API}
Conditionally store the bitwise XOR of the existing value and the specified
number for each attribute.

%%% Generated below here
\paragraph{Behavior:}
\begin{itemize}[noitemsep]
This operation requires a pre-existing object in order to complete successfully.
If no object exists, the operation will fail with \code{NOTFOUND}.

This operation will succeed if and only if the predicates specified by
\code{checks} hold on the pre-existing object.  If any of the predicates are not
true for the existing object, then the operation will have no effect and fail
with \code{CMPFAIL}.

All checks are atomic with the write.  HyperDex guarantees that no other
operation will come between validating the checks, and writing the new version
of the object.

\end{itemize}


\paragraph{Definition:}
\begin{javacode}
public Deferred async_cond_atomic_xor(
        String spacename,
        Object key,
        Map<String, Object> predicates,
        Map<String, Object> attributes) throws HyperDexClientException
\end{javacode}

\paragraph{Parameters:}
\begin{itemize}[noitemsep]
\item \code{String spacename}\\
The name of the space as a string or symbol.

\item \code{Object key}\\
The key for the operation as a Python type.

\item \code{Map<String, Object> predicates}\\
A hash of predicates to check against.

\item \code{Map<String, Object> attributes}\\
The set of attributes to modify and their respective values.  \code{attrs} is a
map from the attributes' names to their values.

\end{itemize}

\paragraph{Returns:}
A Deferred object with a \code{wait} method that returns True if the operation
succeeded or False if any provided predicates failed.  Raises an exception on
error.


\paragraph{See also:}  This is the asynchronous form of \code{cond\_atomic\_xor}.

%%%%%%%%%%%%%%%%%%%% atomic_min %%%%%%%%%%%%%%%%%%%%
\pagebreak
\subsubsection{\code{atomic\_min}}
\label{api:java:atomic_min}
\index{atomic\_min!Java API}
Store the minimum of the existing value and the provided value for each
attribute.
This operation requires a pre-existing object in order to complete successfully.
If no object exists, the operation will fail with \code{NOTFOUND}.



\paragraph{Definition:}
\begin{javacode}
public Boolean atomic_min(
        String spacename,
        Object key,
        Map<String, Object> attributes) throws HyperDexClientException
\end{javacode}

\paragraph{Parameters:}
\begin{itemize}[noitemsep]
\item \code{String spacename}\\
The name of the space as a string or symbol.

\item \code{Object key}\\
The key for the operation as a Python type.

\item \code{Map<String, Object> attributes}\\
The set of attributes to modify and their respective values.  \code{attrs} is a
map from the attributes' names to their values.

\end{itemize}

\paragraph{Returns:}
This function returns an object indicating the success or failure of the
operation.  Valid values to be returned are:

\begin{itemize}[noitemsep]
\item \code{True} if the operation succeeded
\item \code{False} if any provided predicates failed.
\item \code{null} if the operation requires an existing value and none exists
\end{itemize}

On error, this function will raise a \code{HyperDexClientException} describing
the error.


\pagebreak
\subsubsection{\code{async\_atomic\_min}}
\label{api:java:async_atomic_min}
\index{async\_atomic\_min!Java API}
Store the minimum of the existing value and the provided value for each
attribute.
This operation requires a pre-existing object in order to complete successfully.
If no object exists, the operation will fail with \code{NOTFOUND}.



\paragraph{Definition:}
\begin{javacode}
public Deferred async_atomic_min(
        String spacename,
        Object key,
        Map<String, Object> attributes) throws HyperDexClientException
\end{javacode}

\paragraph{Parameters:}
\begin{itemize}[noitemsep]
\item \code{String spacename}\\
The name of the space as a string or symbol.

\item \code{Object key}\\
The key for the operation as a Python type.

\item \code{Map<String, Object> attributes}\\
The set of attributes to modify and their respective values.  \code{attrs} is a
map from the attributes' names to their values.

\end{itemize}

\paragraph{Returns:}
A Deferred object with a \code{wait} method that returns True if the operation
succeeded or False if any provided predicates failed.  Raises an exception on
error.


\paragraph{See also:}  This is the asynchronous form of \code{atomic\_min}.

%%%%%%%%%%%%%%%%%%%% cond_atomic_min %%%%%%%%%%%%%%%%%%%%
\pagebreak
\subsubsection{\code{cond\_atomic\_min}}
\label{api:java:cond_atomic_min}
\index{cond\_atomic\_min!Java API}
Store the minimum of the existing value and the provided value for each
attribute if and only if \code{checks} hold on the object.
This operation requires a pre-existing object in order to complete successfully.
If no object exists, the operation will fail with \code{NOTFOUND}.


This operation will succeed if and only if the predicates specified by
\code{checks} hold on the pre-existing object.  If any of the predicates are not
true for the existing object, then the operation will have no effect and fail
with \code{CMPFAIL}.

All checks are atomic with the write.  HyperDex guarantees that no other
operation will come between validating the checks, and writing the new version
of the object.



\paragraph{Definition:}
\begin{javacode}
public Boolean cond_atomic_min(
        String spacename,
        Object key,
        Map<String, Object> predicates,
        Map<String, Object> attributes) throws HyperDexClientException
\end{javacode}

\paragraph{Parameters:}
\begin{itemize}[noitemsep]
\item \code{String spacename}\\
The name of the space as a string or symbol.

\item \code{Object key}\\
The key for the operation as a Python type.

\item \code{Map<String, Object> predicates}\\
A hash of predicates to check against.

\item \code{Map<String, Object> attributes}\\
The set of attributes to modify and their respective values.  \code{attrs} is a
map from the attributes' names to their values.

\end{itemize}

\paragraph{Returns:}
This function returns an object indicating the success or failure of the
operation.  Valid values to be returned are:

\begin{itemize}[noitemsep]
\item \code{True} if the operation succeeded
\item \code{False} if any provided predicates failed.
\item \code{null} if the operation requires an existing value and none exists
\end{itemize}

On error, this function will raise a \code{HyperDexClientException} describing
the error.


\pagebreak
\subsubsection{\code{async\_cond\_atomic\_min}}
\label{api:java:async_cond_atomic_min}
\index{async\_cond\_atomic\_min!Java API}
Store the minimum of the existing value and the provided value for each
attribute if and only if \code{checks} hold on the object.
This operation requires a pre-existing object in order to complete successfully.
If no object exists, the operation will fail with \code{NOTFOUND}.


This operation will succeed if and only if the predicates specified by
\code{checks} hold on the pre-existing object.  If any of the predicates are not
true for the existing object, then the operation will have no effect and fail
with \code{CMPFAIL}.

All checks are atomic with the write.  HyperDex guarantees that no other
operation will come between validating the checks, and writing the new version
of the object.



\paragraph{Definition:}
\begin{javacode}
public Deferred async_cond_atomic_min(
        String spacename,
        Object key,
        Map<String, Object> predicates,
        Map<String, Object> attributes) throws HyperDexClientException
\end{javacode}

\paragraph{Parameters:}
\begin{itemize}[noitemsep]
\item \code{String spacename}\\
The name of the space as a string or symbol.

\item \code{Object key}\\
The key for the operation as a Python type.

\item \code{Map<String, Object> predicates}\\
A hash of predicates to check against.

\item \code{Map<String, Object> attributes}\\
The set of attributes to modify and their respective values.  \code{attrs} is a
map from the attributes' names to their values.

\end{itemize}

\paragraph{Returns:}
A Deferred object with a \code{wait} method that returns True if the operation
succeeded or False if any provided predicates failed.  Raises an exception on
error.


\paragraph{See also:}  This is the asynchronous form of \code{cond\_atomic\_min}.

%%%%%%%%%%%%%%%%%%%% atomic_max %%%%%%%%%%%%%%%%%%%%
\pagebreak
\subsubsection{\code{atomic\_max}}
\label{api:java:atomic_max}
\index{atomic\_max!Java API}
Store the maximum of the existing value and the provided value for each
attribute.
This operation requires a pre-existing object in order to complete successfully.
If no object exists, the operation will fail with \code{NOTFOUND}.



\paragraph{Definition:}
\begin{javacode}
public Boolean atomic_max(
        String spacename,
        Object key,
        Map<String, Object> attributes) throws HyperDexClientException
\end{javacode}

\paragraph{Parameters:}
\begin{itemize}[noitemsep]
\item \code{String spacename}\\
The name of the space as a string or symbol.

\item \code{Object key}\\
The key for the operation as a Python type.

\item \code{Map<String, Object> attributes}\\
The set of attributes to modify and their respective values.  \code{attrs} is a
map from the attributes' names to their values.

\end{itemize}

\paragraph{Returns:}
This function returns an object indicating the success or failure of the
operation.  Valid values to be returned are:

\begin{itemize}[noitemsep]
\item \code{True} if the operation succeeded
\item \code{False} if any provided predicates failed.
\item \code{null} if the operation requires an existing value and none exists
\end{itemize}

On error, this function will raise a \code{HyperDexClientException} describing
the error.


\pagebreak
\subsubsection{\code{async\_atomic\_max}}
\label{api:java:async_atomic_max}
\index{async\_atomic\_max!Java API}
Store the maximum of the existing value and the provided value for each
attribute.
This operation requires a pre-existing object in order to complete successfully.
If no object exists, the operation will fail with \code{NOTFOUND}.



\paragraph{Definition:}
\begin{javacode}
public Deferred async_atomic_max(
        String spacename,
        Object key,
        Map<String, Object> attributes) throws HyperDexClientException
\end{javacode}

\paragraph{Parameters:}
\begin{itemize}[noitemsep]
\item \code{String spacename}\\
The name of the space as a string or symbol.

\item \code{Object key}\\
The key for the operation as a Python type.

\item \code{Map<String, Object> attributes}\\
The set of attributes to modify and their respective values.  \code{attrs} is a
map from the attributes' names to their values.

\end{itemize}

\paragraph{Returns:}
A Deferred object with a \code{wait} method that returns True if the operation
succeeded or False if any provided predicates failed.  Raises an exception on
error.


\paragraph{See also:}  This is the asynchronous form of \code{atomic\_max}.

%%%%%%%%%%%%%%%%%%%% cond_atomic_max %%%%%%%%%%%%%%%%%%%%
\pagebreak
\subsubsection{\code{cond\_atomic\_max}}
\label{api:java:cond_atomic_max}
\index{cond\_atomic\_max!Java API}
Store the maximum of the existing value and the provided value for each
attribute if and only if \code{checks} hold on the object.
This operation requires a pre-existing object in order to complete successfully.
If no object exists, the operation will fail with \code{NOTFOUND}.


This operation will succeed if and only if the predicates specified by
\code{checks} hold on the pre-existing object.  If any of the predicates are not
true for the existing object, then the operation will have no effect and fail
with \code{CMPFAIL}.

All checks are atomic with the write.  HyperDex guarantees that no other
operation will come between validating the checks, and writing the new version
of the object.



\paragraph{Definition:}
\begin{javacode}
public Boolean cond_atomic_max(
        String spacename,
        Object key,
        Map<String, Object> predicates,
        Map<String, Object> attributes) throws HyperDexClientException
\end{javacode}

\paragraph{Parameters:}
\begin{itemize}[noitemsep]
\item \code{String spacename}\\
The name of the space as a string or symbol.

\item \code{Object key}\\
The key for the operation as a Python type.

\item \code{Map<String, Object> predicates}\\
A hash of predicates to check against.

\item \code{Map<String, Object> attributes}\\
The set of attributes to modify and their respective values.  \code{attrs} is a
map from the attributes' names to their values.

\end{itemize}

\paragraph{Returns:}
This function returns an object indicating the success or failure of the
operation.  Valid values to be returned are:

\begin{itemize}[noitemsep]
\item \code{True} if the operation succeeded
\item \code{False} if any provided predicates failed.
\item \code{null} if the operation requires an existing value and none exists
\end{itemize}

On error, this function will raise a \code{HyperDexClientException} describing
the error.


\pagebreak
\subsubsection{\code{async\_cond\_atomic\_max}}
\label{api:java:async_cond_atomic_max}
\index{async\_cond\_atomic\_max!Java API}
Store the maximum of the existing value and the provided value for each
attribute if and only if \code{checks} hold on the object.
This operation requires a pre-existing object in order to complete successfully.
If no object exists, the operation will fail with \code{NOTFOUND}.


This operation will succeed if and only if the predicates specified by
\code{checks} hold on the pre-existing object.  If any of the predicates are not
true for the existing object, then the operation will have no effect and fail
with \code{CMPFAIL}.

All checks are atomic with the write.  HyperDex guarantees that no other
operation will come between validating the checks, and writing the new version
of the object.



\paragraph{Definition:}
\begin{javacode}
public Deferred async_cond_atomic_max(
        String spacename,
        Object key,
        Map<String, Object> predicates,
        Map<String, Object> attributes) throws HyperDexClientException
\end{javacode}

\paragraph{Parameters:}
\begin{itemize}[noitemsep]
\item \code{String spacename}\\
The name of the space as a string or symbol.

\item \code{Object key}\\
The key for the operation as a Python type.

\item \code{Map<String, Object> predicates}\\
A hash of predicates to check against.

\item \code{Map<String, Object> attributes}\\
The set of attributes to modify and their respective values.  \code{attrs} is a
map from the attributes' names to their values.

\end{itemize}

\paragraph{Returns:}
A Deferred object with a \code{wait} method that returns True if the operation
succeeded or False if any provided predicates failed.  Raises an exception on
error.


\paragraph{See also:}  This is the asynchronous form of \code{cond\_atomic\_max}.

%%%%%%%%%%%%%%%%%%%% string_prepend %%%%%%%%%%%%%%%%%%%%
\pagebreak
\subsubsection{\code{string\_prepend}}
\label{api:java:string_prepend}
\index{string\_prepend!Java API}
Prepend the specified string to the existing value for each attribute.

%%% Generated below here
\paragraph{Behavior:}
\begin{itemize}[noitemsep]
This operation requires a pre-existing object in order to complete successfully.
If no object exists, the operation will fail with \code{NOTFOUND}.

\end{itemize}


\paragraph{Definition:}
\begin{javacode}
public Boolean string_prepend(
        String spacename,
        Object key,
        Map<String, Object> attributes) throws HyperDexClientException
\end{javacode}

\paragraph{Parameters:}
\begin{itemize}[noitemsep]
\item \code{String spacename}\\
The name of the space as a string or symbol.

\item \code{Object key}\\
The key for the operation as a Python type.

\item \code{Map<String, Object> attributes}\\
The set of attributes to modify and their respective values.  \code{attrs} is a
map from the attributes' names to their values.

\end{itemize}

\paragraph{Returns:}
This function returns an object indicating the success or failure of the
operation.  Valid values to be returned are:

\begin{itemize}[noitemsep]
\item \code{True} if the operation succeeded
\item \code{False} if any provided predicates failed.
\item \code{null} if the operation requires an existing value and none exists
\end{itemize}

On error, this function will raise a \code{HyperDexClientException} describing
the error.


\pagebreak
\subsubsection{\code{async\_string\_prepend}}
\label{api:java:async_string_prepend}
\index{async\_string\_prepend!Java API}
Prepend the specified string to the existing value for each attribute.

%%% Generated below here
\paragraph{Behavior:}
\begin{itemize}[noitemsep]
This operation requires a pre-existing object in order to complete successfully.
If no object exists, the operation will fail with \code{NOTFOUND}.

\end{itemize}


\paragraph{Definition:}
\begin{javacode}
public Deferred async_string_prepend(
        String spacename,
        Object key,
        Map<String, Object> attributes) throws HyperDexClientException
\end{javacode}

\paragraph{Parameters:}
\begin{itemize}[noitemsep]
\item \code{String spacename}\\
The name of the space as a string or symbol.

\item \code{Object key}\\
The key for the operation as a Python type.

\item \code{Map<String, Object> attributes}\\
The set of attributes to modify and their respective values.  \code{attrs} is a
map from the attributes' names to their values.

\end{itemize}

\paragraph{Returns:}
A Deferred object with a \code{wait} method that returns True if the operation
succeeded or False if any provided predicates failed.  Raises an exception on
error.


\paragraph{See also:}  This is the asynchronous form of \code{string\_prepend}.

%%%%%%%%%%%%%%%%%%%% cond_string_prepend %%%%%%%%%%%%%%%%%%%%
\pagebreak
\subsubsection{\code{cond\_string\_prepend}}
\label{api:java:cond_string_prepend}
\index{cond\_string\_prepend!Java API}
Conditionally prepend the specified string to the existing value for each
attribute.

%%% Generated below here
\paragraph{Behavior:}
\begin{itemize}[noitemsep]
This operation requires a pre-existing object in order to complete successfully.
If no object exists, the operation will fail with \code{NOTFOUND}.

This operation will succeed if and only if the predicates specified by
\code{checks} hold on the pre-existing object.  If any of the predicates are not
true for the existing object, then the operation will have no effect and fail
with \code{CMPFAIL}.

All checks are atomic with the write.  HyperDex guarantees that no other
operation will come between validating the checks, and writing the new version
of the object.

\end{itemize}


\paragraph{Definition:}
\begin{javacode}
public Boolean cond_string_prepend(
        String spacename,
        Object key,
        Map<String, Object> predicates,
        Map<String, Object> attributes) throws HyperDexClientException
\end{javacode}

\paragraph{Parameters:}
\begin{itemize}[noitemsep]
\item \code{String spacename}\\
The name of the space as a string or symbol.

\item \code{Object key}\\
The key for the operation as a Python type.

\item \code{Map<String, Object> predicates}\\
A hash of predicates to check against.

\item \code{Map<String, Object> attributes}\\
The set of attributes to modify and their respective values.  \code{attrs} is a
map from the attributes' names to their values.

\end{itemize}

\paragraph{Returns:}
This function returns an object indicating the success or failure of the
operation.  Valid values to be returned are:

\begin{itemize}[noitemsep]
\item \code{True} if the operation succeeded
\item \code{False} if any provided predicates failed.
\item \code{null} if the operation requires an existing value and none exists
\end{itemize}

On error, this function will raise a \code{HyperDexClientException} describing
the error.


\pagebreak
\subsubsection{\code{async\_cond\_string\_prepend}}
\label{api:java:async_cond_string_prepend}
\index{async\_cond\_string\_prepend!Java API}
Conditionally prepend the specified string to the existing value for each
attribute.

%%% Generated below here
\paragraph{Behavior:}
\begin{itemize}[noitemsep]
This operation requires a pre-existing object in order to complete successfully.
If no object exists, the operation will fail with \code{NOTFOUND}.

This operation will succeed if and only if the predicates specified by
\code{checks} hold on the pre-existing object.  If any of the predicates are not
true for the existing object, then the operation will have no effect and fail
with \code{CMPFAIL}.

All checks are atomic with the write.  HyperDex guarantees that no other
operation will come between validating the checks, and writing the new version
of the object.

\end{itemize}


\paragraph{Definition:}
\begin{javacode}
public Deferred async_cond_string_prepend(
        String spacename,
        Object key,
        Map<String, Object> predicates,
        Map<String, Object> attributes) throws HyperDexClientException
\end{javacode}

\paragraph{Parameters:}
\begin{itemize}[noitemsep]
\item \code{String spacename}\\
The name of the space as a string or symbol.

\item \code{Object key}\\
The key for the operation as a Python type.

\item \code{Map<String, Object> predicates}\\
A hash of predicates to check against.

\item \code{Map<String, Object> attributes}\\
The set of attributes to modify and their respective values.  \code{attrs} is a
map from the attributes' names to their values.

\end{itemize}

\paragraph{Returns:}
A Deferred object with a \code{wait} method that returns True if the operation
succeeded or False if any provided predicates failed.  Raises an exception on
error.


\paragraph{See also:}  This is the asynchronous form of \code{cond\_string\_prepend}.

%%%%%%%%%%%%%%%%%%%% string_append %%%%%%%%%%%%%%%%%%%%
\pagebreak
\subsubsection{\code{string\_append}}
\label{api:java:string_append}
\index{string\_append!Java API}
Append the specified string to the existing value for each attribute.

%%% Generated below here
\paragraph{Behavior:}
\begin{itemize}[noitemsep]
This operation requires a pre-existing object in order to complete successfully.
If no object exists, the operation will fail with \code{NOTFOUND}.

\end{itemize}


\paragraph{Definition:}
\begin{javacode}
public Boolean string_append(
        String spacename,
        Object key,
        Map<String, Object> attributes) throws HyperDexClientException
\end{javacode}

\paragraph{Parameters:}
\begin{itemize}[noitemsep]
\item \code{String spacename}\\
The name of the space as a string or symbol.

\item \code{Object key}\\
The key for the operation as a Python type.

\item \code{Map<String, Object> attributes}\\
The set of attributes to modify and their respective values.  \code{attrs} is a
map from the attributes' names to their values.

\end{itemize}

\paragraph{Returns:}
This function returns an object indicating the success or failure of the
operation.  Valid values to be returned are:

\begin{itemize}[noitemsep]
\item \code{True} if the operation succeeded
\item \code{False} if any provided predicates failed.
\item \code{null} if the operation requires an existing value and none exists
\end{itemize}

On error, this function will raise a \code{HyperDexClientException} describing
the error.


\pagebreak
\subsubsection{\code{async\_string\_append}}
\label{api:java:async_string_append}
\index{async\_string\_append!Java API}
Append the specified string to the existing value for each attribute.

%%% Generated below here
\paragraph{Behavior:}
\begin{itemize}[noitemsep]
This operation requires a pre-existing object in order to complete successfully.
If no object exists, the operation will fail with \code{NOTFOUND}.

\end{itemize}


\paragraph{Definition:}
\begin{javacode}
public Deferred async_string_append(
        String spacename,
        Object key,
        Map<String, Object> attributes) throws HyperDexClientException
\end{javacode}

\paragraph{Parameters:}
\begin{itemize}[noitemsep]
\item \code{String spacename}\\
The name of the space as a string or symbol.

\item \code{Object key}\\
The key for the operation as a Python type.

\item \code{Map<String, Object> attributes}\\
The set of attributes to modify and their respective values.  \code{attrs} is a
map from the attributes' names to their values.

\end{itemize}

\paragraph{Returns:}
A Deferred object with a \code{wait} method that returns True if the operation
succeeded or False if any provided predicates failed.  Raises an exception on
error.


\paragraph{See also:}  This is the asynchronous form of \code{string\_append}.

%%%%%%%%%%%%%%%%%%%% cond_string_append %%%%%%%%%%%%%%%%%%%%
\pagebreak
\subsubsection{\code{cond\_string\_append}}
\label{api:java:cond_string_append}
\index{cond\_string\_append!Java API}
Conditionally append the specified string to the existing value for each
attribute.

%%% Generated below here
\paragraph{Behavior:}
\begin{itemize}[noitemsep]
This operation requires a pre-existing object in order to complete successfully.
If no object exists, the operation will fail with \code{NOTFOUND}.

This operation will succeed if and only if the predicates specified by
\code{checks} hold on the pre-existing object.  If any of the predicates are not
true for the existing object, then the operation will have no effect and fail
with \code{CMPFAIL}.

All checks are atomic with the write.  HyperDex guarantees that no other
operation will come between validating the checks, and writing the new version
of the object.

\end{itemize}


\paragraph{Definition:}
\begin{javacode}
public Boolean cond_string_append(
        String spacename,
        Object key,
        Map<String, Object> predicates,
        Map<String, Object> attributes) throws HyperDexClientException
\end{javacode}

\paragraph{Parameters:}
\begin{itemize}[noitemsep]
\item \code{String spacename}\\
The name of the space as a string or symbol.

\item \code{Object key}\\
The key for the operation as a Python type.

\item \code{Map<String, Object> predicates}\\
A hash of predicates to check against.

\item \code{Map<String, Object> attributes}\\
The set of attributes to modify and their respective values.  \code{attrs} is a
map from the attributes' names to their values.

\end{itemize}

\paragraph{Returns:}
This function returns an object indicating the success or failure of the
operation.  Valid values to be returned are:

\begin{itemize}[noitemsep]
\item \code{True} if the operation succeeded
\item \code{False} if any provided predicates failed.
\item \code{null} if the operation requires an existing value and none exists
\end{itemize}

On error, this function will raise a \code{HyperDexClientException} describing
the error.


\pagebreak
\subsubsection{\code{async\_cond\_string\_append}}
\label{api:java:async_cond_string_append}
\index{async\_cond\_string\_append!Java API}
Conditionally append the specified string to the existing value for each
attribute.

%%% Generated below here
\paragraph{Behavior:}
\begin{itemize}[noitemsep]
This operation requires a pre-existing object in order to complete successfully.
If no object exists, the operation will fail with \code{NOTFOUND}.

This operation will succeed if and only if the predicates specified by
\code{checks} hold on the pre-existing object.  If any of the predicates are not
true for the existing object, then the operation will have no effect and fail
with \code{CMPFAIL}.

All checks are atomic with the write.  HyperDex guarantees that no other
operation will come between validating the checks, and writing the new version
of the object.

\end{itemize}


\paragraph{Definition:}
\begin{javacode}
public Deferred async_cond_string_append(
        String spacename,
        Object key,
        Map<String, Object> predicates,
        Map<String, Object> attributes) throws HyperDexClientException
\end{javacode}

\paragraph{Parameters:}
\begin{itemize}[noitemsep]
\item \code{String spacename}\\
The name of the space as a string or symbol.

\item \code{Object key}\\
The key for the operation as a Python type.

\item \code{Map<String, Object> predicates}\\
A hash of predicates to check against.

\item \code{Map<String, Object> attributes}\\
The set of attributes to modify and their respective values.  \code{attrs} is a
map from the attributes' names to their values.

\end{itemize}

\paragraph{Returns:}
A Deferred object with a \code{wait} method that returns True if the operation
succeeded or False if any provided predicates failed.  Raises an exception on
error.


\paragraph{See also:}  This is the asynchronous form of \code{cond\_string\_append}.

%%%%%%%%%%%%%%%%%%%% list_lpush %%%%%%%%%%%%%%%%%%%%
\pagebreak
\subsubsection{\code{list\_lpush}}
\label{api:java:list_lpush}
\index{list\_lpush!Java API}
Push the specified value onto the front of the list for each attribute.

%%% Generated below here
\paragraph{Behavior:}
\begin{itemize}[noitemsep]
This operation requires a pre-existing object in order to complete successfully.
If no object exists, the operation will fail with \code{NOTFOUND}.

\end{itemize}


\paragraph{Definition:}
\begin{javacode}
public Boolean list_lpush(
        String spacename,
        Object key,
        Map<String, Object> attributes) throws HyperDexClientException
\end{javacode}

\paragraph{Parameters:}
\begin{itemize}[noitemsep]
\item \code{String spacename}\\
The name of the space as a string or symbol.

\item \code{Object key}\\
The key for the operation as a Python type.

\item \code{Map<String, Object> attributes}\\
The set of attributes to modify and their respective values.  \code{attrs} is a
map from the attributes' names to their values.

\end{itemize}

\paragraph{Returns:}
This function returns an object indicating the success or failure of the
operation.  Valid values to be returned are:

\begin{itemize}[noitemsep]
\item \code{True} if the operation succeeded
\item \code{False} if any provided predicates failed.
\item \code{null} if the operation requires an existing value and none exists
\end{itemize}

On error, this function will raise a \code{HyperDexClientException} describing
the error.


\pagebreak
\subsubsection{\code{async\_list\_lpush}}
\label{api:java:async_list_lpush}
\index{async\_list\_lpush!Java API}
Push the specified value onto the front of the list for each attribute.

%%% Generated below here
\paragraph{Behavior:}
\begin{itemize}[noitemsep]
This operation requires a pre-existing object in order to complete successfully.
If no object exists, the operation will fail with \code{NOTFOUND}.

\end{itemize}


\paragraph{Definition:}
\begin{javacode}
public Deferred async_list_lpush(
        String spacename,
        Object key,
        Map<String, Object> attributes) throws HyperDexClientException
\end{javacode}

\paragraph{Parameters:}
\begin{itemize}[noitemsep]
\item \code{String spacename}\\
The name of the space as a string or symbol.

\item \code{Object key}\\
The key for the operation as a Python type.

\item \code{Map<String, Object> attributes}\\
The set of attributes to modify and their respective values.  \code{attrs} is a
map from the attributes' names to their values.

\end{itemize}

\paragraph{Returns:}
A Deferred object with a \code{wait} method that returns True if the operation
succeeded or False if any provided predicates failed.  Raises an exception on
error.


\paragraph{See also:}  This is the asynchronous form of \code{list\_lpush}.

%%%%%%%%%%%%%%%%%%%% cond_list_lpush %%%%%%%%%%%%%%%%%%%%
\pagebreak
\subsubsection{\code{cond\_list\_lpush}}
\label{api:java:cond_list_lpush}
\index{cond\_list\_lpush!Java API}
Condtitionally push the specified value onto the front of the list for each
attribute.

%%% Generated below here
\paragraph{Behavior:}
\begin{itemize}[noitemsep]
This operation requires a pre-existing object in order to complete successfully.
If no object exists, the operation will fail with \code{NOTFOUND}.

This operation will succeed if and only if the predicates specified by
\code{checks} hold on the pre-existing object.  If any of the predicates are not
true for the existing object, then the operation will have no effect and fail
with \code{CMPFAIL}.

All checks are atomic with the write.  HyperDex guarantees that no other
operation will come between validating the checks, and writing the new version
of the object.

\end{itemize}


\paragraph{Definition:}
\begin{javacode}
public Boolean cond_list_lpush(
        String spacename,
        Object key,
        Map<String, Object> predicates,
        Map<String, Object> attributes) throws HyperDexClientException
\end{javacode}

\paragraph{Parameters:}
\begin{itemize}[noitemsep]
\item \code{String spacename}\\
The name of the space as a string or symbol.

\item \code{Object key}\\
The key for the operation as a Python type.

\item \code{Map<String, Object> predicates}\\
A hash of predicates to check against.

\item \code{Map<String, Object> attributes}\\
The set of attributes to modify and their respective values.  \code{attrs} is a
map from the attributes' names to their values.

\end{itemize}

\paragraph{Returns:}
This function returns an object indicating the success or failure of the
operation.  Valid values to be returned are:

\begin{itemize}[noitemsep]
\item \code{True} if the operation succeeded
\item \code{False} if any provided predicates failed.
\item \code{null} if the operation requires an existing value and none exists
\end{itemize}

On error, this function will raise a \code{HyperDexClientException} describing
the error.


\pagebreak
\subsubsection{\code{async\_cond\_list\_lpush}}
\label{api:java:async_cond_list_lpush}
\index{async\_cond\_list\_lpush!Java API}
Condtitionally push the specified value onto the front of the list for each
attribute.

%%% Generated below here
\paragraph{Behavior:}
\begin{itemize}[noitemsep]
This operation requires a pre-existing object in order to complete successfully.
If no object exists, the operation will fail with \code{NOTFOUND}.

This operation will succeed if and only if the predicates specified by
\code{checks} hold on the pre-existing object.  If any of the predicates are not
true for the existing object, then the operation will have no effect and fail
with \code{CMPFAIL}.

All checks are atomic with the write.  HyperDex guarantees that no other
operation will come between validating the checks, and writing the new version
of the object.

\end{itemize}


\paragraph{Definition:}
\begin{javacode}
public Deferred async_cond_list_lpush(
        String spacename,
        Object key,
        Map<String, Object> predicates,
        Map<String, Object> attributes) throws HyperDexClientException
\end{javacode}

\paragraph{Parameters:}
\begin{itemize}[noitemsep]
\item \code{String spacename}\\
The name of the space as a string or symbol.

\item \code{Object key}\\
The key for the operation as a Python type.

\item \code{Map<String, Object> predicates}\\
A hash of predicates to check against.

\item \code{Map<String, Object> attributes}\\
The set of attributes to modify and their respective values.  \code{attrs} is a
map from the attributes' names to their values.

\end{itemize}

\paragraph{Returns:}
A Deferred object with a \code{wait} method that returns True if the operation
succeeded or False if any provided predicates failed.  Raises an exception on
error.


\paragraph{See also:}  This is the asynchronous form of \code{cond\_list\_lpush}.

%%%%%%%%%%%%%%%%%%%% list_rpush %%%%%%%%%%%%%%%%%%%%
\pagebreak
\subsubsection{\code{list\_rpush}}
\label{api:java:list_rpush}
\index{list\_rpush!Java API}
Push the specified value onto the back of the list for each attribute.
This operation requires a pre-existing object in order to complete successfully.
If no object exists, the operation will fail with \code{NOTFOUND}.



\paragraph{Definition:}
\begin{javacode}
public Boolean list_rpush(
        String spacename,
        Object key,
        Map<String, Object> attributes) throws HyperDexClientException
\end{javacode}

\paragraph{Parameters:}
\begin{itemize}[noitemsep]
\item \code{String spacename}\\
The name of the space as a string or symbol.

\item \code{Object key}\\
The key for the operation as a Python type.

\item \code{Map<String, Object> attributes}\\
The set of attributes to modify and their respective values.  \code{attrs} is a
map from the attributes' names to their values.

\end{itemize}

\paragraph{Returns:}
This function returns an object indicating the success or failure of the
operation.  Valid values to be returned are:

\begin{itemize}[noitemsep]
\item \code{True} if the operation succeeded
\item \code{False} if any provided predicates failed.
\item \code{null} if the operation requires an existing value and none exists
\end{itemize}

On error, this function will raise a \code{HyperDexClientException} describing
the error.


\pagebreak
\subsubsection{\code{async\_list\_rpush}}
\label{api:java:async_list_rpush}
\index{async\_list\_rpush!Java API}
Push the specified value onto the back of the list for each attribute.
This operation requires a pre-existing object in order to complete successfully.
If no object exists, the operation will fail with \code{NOTFOUND}.



\paragraph{Definition:}
\begin{javacode}
public Deferred async_list_rpush(
        String spacename,
        Object key,
        Map<String, Object> attributes) throws HyperDexClientException
\end{javacode}

\paragraph{Parameters:}
\begin{itemize}[noitemsep]
\item \code{String spacename}\\
The name of the space as a string or symbol.

\item \code{Object key}\\
The key for the operation as a Python type.

\item \code{Map<String, Object> attributes}\\
The set of attributes to modify and their respective values.  \code{attrs} is a
map from the attributes' names to their values.

\end{itemize}

\paragraph{Returns:}
A Deferred object with a \code{wait} method that returns True if the operation
succeeded or False if any provided predicates failed.  Raises an exception on
error.


\paragraph{See also:}  This is the asynchronous form of \code{list\_rpush}.

%%%%%%%%%%%%%%%%%%%% cond_list_rpush %%%%%%%%%%%%%%%%%%%%
\pagebreak
\subsubsection{\code{cond\_list\_rpush}}
\label{api:java:cond_list_rpush}
\index{cond\_list\_rpush!Java API}
Push the specified value onto the back of the list for each attribute if and
only if the \code{checks} hold on the object.
This operation requires a pre-existing object in order to complete successfully.
If no object exists, the operation will fail with \code{NOTFOUND}.


This operation will succeed if and only if the predicates specified by
\code{checks} hold on the pre-existing object.  If any of the predicates are not
true for the existing object, then the operation will have no effect and fail
with \code{CMPFAIL}.

All checks are atomic with the write.  HyperDex guarantees that no other
operation will come between validating the checks, and writing the new version
of the object.



\paragraph{Definition:}
\begin{javacode}
public Boolean cond_list_rpush(
        String spacename,
        Object key,
        Map<String, Object> predicates,
        Map<String, Object> attributes) throws HyperDexClientException
\end{javacode}

\paragraph{Parameters:}
\begin{itemize}[noitemsep]
\item \code{String spacename}\\
The name of the space as a string or symbol.

\item \code{Object key}\\
The key for the operation as a Python type.

\item \code{Map<String, Object> predicates}\\
A hash of predicates to check against.

\item \code{Map<String, Object> attributes}\\
The set of attributes to modify and their respective values.  \code{attrs} is a
map from the attributes' names to their values.

\end{itemize}

\paragraph{Returns:}
This function returns an object indicating the success or failure of the
operation.  Valid values to be returned are:

\begin{itemize}[noitemsep]
\item \code{True} if the operation succeeded
\item \code{False} if any provided predicates failed.
\item \code{null} if the operation requires an existing value and none exists
\end{itemize}

On error, this function will raise a \code{HyperDexClientException} describing
the error.


\pagebreak
\subsubsection{\code{async\_cond\_list\_rpush}}
\label{api:java:async_cond_list_rpush}
\index{async\_cond\_list\_rpush!Java API}
Push the specified value onto the back of the list for each attribute if and
only if the \code{checks} hold on the object.
This operation requires a pre-existing object in order to complete successfully.
If no object exists, the operation will fail with \code{NOTFOUND}.


This operation will succeed if and only if the predicates specified by
\code{checks} hold on the pre-existing object.  If any of the predicates are not
true for the existing object, then the operation will have no effect and fail
with \code{CMPFAIL}.

All checks are atomic with the write.  HyperDex guarantees that no other
operation will come between validating the checks, and writing the new version
of the object.



\paragraph{Definition:}
\begin{javacode}
public Deferred async_cond_list_rpush(
        String spacename,
        Object key,
        Map<String, Object> predicates,
        Map<String, Object> attributes) throws HyperDexClientException
\end{javacode}

\paragraph{Parameters:}
\begin{itemize}[noitemsep]
\item \code{String spacename}\\
The name of the space as a string or symbol.

\item \code{Object key}\\
The key for the operation as a Python type.

\item \code{Map<String, Object> predicates}\\
A hash of predicates to check against.

\item \code{Map<String, Object> attributes}\\
The set of attributes to modify and their respective values.  \code{attrs} is a
map from the attributes' names to their values.

\end{itemize}

\paragraph{Returns:}
A Deferred object with a \code{wait} method that returns True if the operation
succeeded or False if any provided predicates failed.  Raises an exception on
error.


\paragraph{See also:}  This is the asynchronous form of \code{cond\_list\_rpush}.

%%%%%%%%%%%%%%%%%%%% set_add %%%%%%%%%%%%%%%%%%%%
\pagebreak
\subsubsection{\code{set\_add}}
\label{api:java:set_add}
\index{set\_add!Java API}
Add the specified value to the set for each attribute.

%%% Generated below here
\paragraph{Behavior:}
\begin{itemize}[noitemsep]
This operation requires a pre-existing object in order to complete successfully.
If no object exists, the operation will fail with \code{NOTFOUND}.

\end{itemize}


\paragraph{Definition:}
\begin{javacode}
public Boolean set_add(
        String spacename,
        Object key,
        Map<String, Object> attributes) throws HyperDexClientException
\end{javacode}

\paragraph{Parameters:}
\begin{itemize}[noitemsep]
\item \code{String spacename}\\
The name of the space as a string or symbol.

\item \code{Object key}\\
The key for the operation as a Python type.

\item \code{Map<String, Object> attributes}\\
The set of attributes to modify and their respective values.  \code{attrs} is a
map from the attributes' names to their values.

\end{itemize}

\paragraph{Returns:}
This function returns an object indicating the success or failure of the
operation.  Valid values to be returned are:

\begin{itemize}[noitemsep]
\item \code{True} if the operation succeeded
\item \code{False} if any provided predicates failed.
\item \code{null} if the operation requires an existing value and none exists
\end{itemize}

On error, this function will raise a \code{HyperDexClientException} describing
the error.


\pagebreak
\subsubsection{\code{async\_set\_add}}
\label{api:java:async_set_add}
\index{async\_set\_add!Java API}
Add the specified value to the set for each attribute.

%%% Generated below here
\paragraph{Behavior:}
\begin{itemize}[noitemsep]
This operation requires a pre-existing object in order to complete successfully.
If no object exists, the operation will fail with \code{NOTFOUND}.

\end{itemize}


\paragraph{Definition:}
\begin{javacode}
public Deferred async_set_add(
        String spacename,
        Object key,
        Map<String, Object> attributes) throws HyperDexClientException
\end{javacode}

\paragraph{Parameters:}
\begin{itemize}[noitemsep]
\item \code{String spacename}\\
The name of the space as a string or symbol.

\item \code{Object key}\\
The key for the operation as a Python type.

\item \code{Map<String, Object> attributes}\\
The set of attributes to modify and their respective values.  \code{attrs} is a
map from the attributes' names to their values.

\end{itemize}

\paragraph{Returns:}
A Deferred object with a \code{wait} method that returns True if the operation
succeeded or False if any provided predicates failed.  Raises an exception on
error.


\paragraph{See also:}  This is the asynchronous form of \code{set\_add}.

%%%%%%%%%%%%%%%%%%%% cond_set_add %%%%%%%%%%%%%%%%%%%%
\pagebreak
\subsubsection{\code{cond\_set\_add}}
\label{api:java:cond_set_add}
\index{cond\_set\_add!Java API}
Add the specified value to the set for each attribute if and only if the
\code{checks} hold on the object.
This operation requires a pre-existing object in order to complete successfully.
If no object exists, the operation will fail with \code{NOTFOUND}.


This operation will succeed if and only if the predicates specified by
\code{checks} hold on the pre-existing object.  If any of the predicates are not
true for the existing object, then the operation will have no effect and fail
with \code{CMPFAIL}.

All checks are atomic with the write.  HyperDex guarantees that no other
operation will come between validating the checks, and writing the new version
of the object.



\paragraph{Definition:}
\begin{javacode}
public Boolean cond_set_add(
        String spacename,
        Object key,
        Map<String, Object> predicates,
        Map<String, Object> attributes) throws HyperDexClientException
\end{javacode}

\paragraph{Parameters:}
\begin{itemize}[noitemsep]
\item \code{String spacename}\\
The name of the space as a string or symbol.

\item \code{Object key}\\
The key for the operation as a Python type.

\item \code{Map<String, Object> predicates}\\
A hash of predicates to check against.

\item \code{Map<String, Object> attributes}\\
The set of attributes to modify and their respective values.  \code{attrs} is a
map from the attributes' names to their values.

\end{itemize}

\paragraph{Returns:}
This function returns an object indicating the success or failure of the
operation.  Valid values to be returned are:

\begin{itemize}[noitemsep]
\item \code{True} if the operation succeeded
\item \code{False} if any provided predicates failed.
\item \code{null} if the operation requires an existing value and none exists
\end{itemize}

On error, this function will raise a \code{HyperDexClientException} describing
the error.


\pagebreak
\subsubsection{\code{async\_cond\_set\_add}}
\label{api:java:async_cond_set_add}
\index{async\_cond\_set\_add!Java API}
Add the specified value to the set for each attribute if and only if the
\code{checks} hold on the object.
This operation requires a pre-existing object in order to complete successfully.
If no object exists, the operation will fail with \code{NOTFOUND}.


This operation will succeed if and only if the predicates specified by
\code{checks} hold on the pre-existing object.  If any of the predicates are not
true for the existing object, then the operation will have no effect and fail
with \code{CMPFAIL}.

All checks are atomic with the write.  HyperDex guarantees that no other
operation will come between validating the checks, and writing the new version
of the object.



\paragraph{Definition:}
\begin{javacode}
public Deferred async_cond_set_add(
        String spacename,
        Object key,
        Map<String, Object> predicates,
        Map<String, Object> attributes) throws HyperDexClientException
\end{javacode}

\paragraph{Parameters:}
\begin{itemize}[noitemsep]
\item \code{String spacename}\\
The name of the space as a string or symbol.

\item \code{Object key}\\
The key for the operation as a Python type.

\item \code{Map<String, Object> predicates}\\
A hash of predicates to check against.

\item \code{Map<String, Object> attributes}\\
The set of attributes to modify and their respective values.  \code{attrs} is a
map from the attributes' names to their values.

\end{itemize}

\paragraph{Returns:}
A Deferred object with a \code{wait} method that returns True if the operation
succeeded or False if any provided predicates failed.  Raises an exception on
error.


\paragraph{See also:}  This is the asynchronous form of \code{cond\_set\_add}.

%%%%%%%%%%%%%%%%%%%% set_remove %%%%%%%%%%%%%%%%%%%%
\pagebreak
\subsubsection{\code{set\_remove}}
\label{api:java:set_remove}
\index{set\_remove!Java API}
Remove the specified value from the set.  If the value is not contained within
the set, this operation will do nothing.

%%% Generated below here
\paragraph{Behavior:}
\begin{itemize}[noitemsep]
This operation requires a pre-existing object in order to complete successfully.
If no object exists, the operation will fail with \code{NOTFOUND}.

\end{itemize}


\paragraph{Definition:}
\begin{javacode}
public Boolean set_remove(
        String spacename,
        Object key,
        Map<String, Object> attributes) throws HyperDexClientException
\end{javacode}

\paragraph{Parameters:}
\begin{itemize}[noitemsep]
\item \code{String spacename}\\
The name of the space as a string or symbol.

\item \code{Object key}\\
The key for the operation as a Python type.

\item \code{Map<String, Object> attributes}\\
The set of attributes to modify and their respective values.  \code{attrs} is a
map from the attributes' names to their values.

\end{itemize}

\paragraph{Returns:}
This function returns an object indicating the success or failure of the
operation.  Valid values to be returned are:

\begin{itemize}[noitemsep]
\item \code{True} if the operation succeeded
\item \code{False} if any provided predicates failed.
\item \code{null} if the operation requires an existing value and none exists
\end{itemize}

On error, this function will raise a \code{HyperDexClientException} describing
the error.


\pagebreak
\subsubsection{\code{async\_set\_remove}}
\label{api:java:async_set_remove}
\index{async\_set\_remove!Java API}
Remove the specified value from the set.  If the value is not contained within
the set, this operation will do nothing.

%%% Generated below here
\paragraph{Behavior:}
\begin{itemize}[noitemsep]
This operation requires a pre-existing object in order to complete successfully.
If no object exists, the operation will fail with \code{NOTFOUND}.

\end{itemize}


\paragraph{Definition:}
\begin{javacode}
public Deferred async_set_remove(
        String spacename,
        Object key,
        Map<String, Object> attributes) throws HyperDexClientException
\end{javacode}

\paragraph{Parameters:}
\begin{itemize}[noitemsep]
\item \code{String spacename}\\
The name of the space as a string or symbol.

\item \code{Object key}\\
The key for the operation as a Python type.

\item \code{Map<String, Object> attributes}\\
The set of attributes to modify and their respective values.  \code{attrs} is a
map from the attributes' names to their values.

\end{itemize}

\paragraph{Returns:}
A Deferred object with a \code{wait} method that returns True if the operation
succeeded or False if any provided predicates failed.  Raises an exception on
error.


\paragraph{See also:}  This is the asynchronous form of \code{set\_remove}.

%%%%%%%%%%%%%%%%%%%% cond_set_remove %%%%%%%%%%%%%%%%%%%%
\pagebreak
\subsubsection{\code{cond\_set\_remove}}
\label{api:java:cond_set_remove}
\index{cond\_set\_remove!Java API}
Remove the specified value from the set if and only if the \code{checks} hold on
the object.  If the value is not contained within the set, this operation will
do nothing.
This operation requires a pre-existing object in order to complete successfully.
If no object exists, the operation will fail with \code{NOTFOUND}.


This operation will succeed if and only if the predicates specified by
\code{checks} hold on the pre-existing object.  If any of the predicates are not
true for the existing object, then the operation will have no effect and fail
with \code{CMPFAIL}.

All checks are atomic with the write.  HyperDex guarantees that no other
operation will come between validating the checks, and writing the new version
of the object.



\paragraph{Definition:}
\begin{javacode}
public Boolean cond_set_remove(
        String spacename,
        Object key,
        Map<String, Object> predicates,
        Map<String, Object> attributes) throws HyperDexClientException
\end{javacode}

\paragraph{Parameters:}
\begin{itemize}[noitemsep]
\item \code{String spacename}\\
The name of the space as a string or symbol.

\item \code{Object key}\\
The key for the operation as a Python type.

\item \code{Map<String, Object> predicates}\\
A hash of predicates to check against.

\item \code{Map<String, Object> attributes}\\
The set of attributes to modify and their respective values.  \code{attrs} is a
map from the attributes' names to their values.

\end{itemize}

\paragraph{Returns:}
This function returns an object indicating the success or failure of the
operation.  Valid values to be returned are:

\begin{itemize}[noitemsep]
\item \code{True} if the operation succeeded
\item \code{False} if any provided predicates failed.
\item \code{null} if the operation requires an existing value and none exists
\end{itemize}

On error, this function will raise a \code{HyperDexClientException} describing
the error.


\pagebreak
\subsubsection{\code{async\_cond\_set\_remove}}
\label{api:java:async_cond_set_remove}
\index{async\_cond\_set\_remove!Java API}
Remove the specified value from the set if and only if the \code{checks} hold on
the object.  If the value is not contained within the set, this operation will
do nothing.
This operation requires a pre-existing object in order to complete successfully.
If no object exists, the operation will fail with \code{NOTFOUND}.


This operation will succeed if and only if the predicates specified by
\code{checks} hold on the pre-existing object.  If any of the predicates are not
true for the existing object, then the operation will have no effect and fail
with \code{CMPFAIL}.

All checks are atomic with the write.  HyperDex guarantees that no other
operation will come between validating the checks, and writing the new version
of the object.



\paragraph{Definition:}
\begin{javacode}
public Deferred async_cond_set_remove(
        String spacename,
        Object key,
        Map<String, Object> predicates,
        Map<String, Object> attributes) throws HyperDexClientException
\end{javacode}

\paragraph{Parameters:}
\begin{itemize}[noitemsep]
\item \code{String spacename}\\
The name of the space as a string or symbol.

\item \code{Object key}\\
The key for the operation as a Python type.

\item \code{Map<String, Object> predicates}\\
A hash of predicates to check against.

\item \code{Map<String, Object> attributes}\\
The set of attributes to modify and their respective values.  \code{attrs} is a
map from the attributes' names to their values.

\end{itemize}

\paragraph{Returns:}
A Deferred object with a \code{wait} method that returns True if the operation
succeeded or False if any provided predicates failed.  Raises an exception on
error.


\paragraph{See also:}  This is the asynchronous form of \code{cond\_set\_remove}.

%%%%%%%%%%%%%%%%%%%% set_intersect %%%%%%%%%%%%%%%%%%%%
\pagebreak
\subsubsection{\code{set\_intersect}}
\label{api:java:set_intersect}
\index{set\_intersect!Java API}
Store the intersection of the specified set and the existing value for each
attribute.
This operation requires a pre-existing object in order to complete successfully.
If no object exists, the operation will fail with \code{NOTFOUND}.



\paragraph{Definition:}
\begin{javacode}
public Boolean set_intersect(
        String spacename,
        Object key,
        Map<String, Object> attributes) throws HyperDexClientException
\end{javacode}

\paragraph{Parameters:}
\begin{itemize}[noitemsep]
\item \code{String spacename}\\
The name of the space as a string or symbol.

\item \code{Object key}\\
The key for the operation as a Python type.

\item \code{Map<String, Object> attributes}\\
The set of attributes to modify and their respective values.  \code{attrs} is a
map from the attributes' names to their values.

\end{itemize}

\paragraph{Returns:}
This function returns an object indicating the success or failure of the
operation.  Valid values to be returned are:

\begin{itemize}[noitemsep]
\item \code{True} if the operation succeeded
\item \code{False} if any provided predicates failed.
\item \code{null} if the operation requires an existing value and none exists
\end{itemize}

On error, this function will raise a \code{HyperDexClientException} describing
the error.


\pagebreak
\subsubsection{\code{async\_set\_intersect}}
\label{api:java:async_set_intersect}
\index{async\_set\_intersect!Java API}
Store the intersection of the specified set and the existing value for each
attribute.
This operation requires a pre-existing object in order to complete successfully.
If no object exists, the operation will fail with \code{NOTFOUND}.



\paragraph{Definition:}
\begin{javacode}
public Deferred async_set_intersect(
        String spacename,
        Object key,
        Map<String, Object> attributes) throws HyperDexClientException
\end{javacode}

\paragraph{Parameters:}
\begin{itemize}[noitemsep]
\item \code{String spacename}\\
The name of the space as a string or symbol.

\item \code{Object key}\\
The key for the operation as a Python type.

\item \code{Map<String, Object> attributes}\\
The set of attributes to modify and their respective values.  \code{attrs} is a
map from the attributes' names to their values.

\end{itemize}

\paragraph{Returns:}
A Deferred object with a \code{wait} method that returns True if the operation
succeeded or False if any provided predicates failed.  Raises an exception on
error.


\paragraph{See also:}  This is the asynchronous form of \code{set\_intersect}.

%%%%%%%%%%%%%%%%%%%% cond_set_intersect %%%%%%%%%%%%%%%%%%%%
\pagebreak
\subsubsection{\code{cond\_set\_intersect}}
\label{api:java:cond_set_intersect}
\index{cond\_set\_intersect!Java API}
Store the intersection of the specified set and the existing value for each
attribute if and only if the \code{checks} hold on the object.
This operation requires a pre-existing object in order to complete successfully.
If no object exists, the operation will fail with \code{NOTFOUND}.


This operation will succeed if and only if the predicates specified by
\code{checks} hold on the pre-existing object.  If any of the predicates are not
true for the existing object, then the operation will have no effect and fail
with \code{CMPFAIL}.

All checks are atomic with the write.  HyperDex guarantees that no other
operation will come between validating the checks, and writing the new version
of the object.



\paragraph{Definition:}
\begin{javacode}
public Boolean cond_set_intersect(
        String spacename,
        Object key,
        Map<String, Object> predicates,
        Map<String, Object> attributes) throws HyperDexClientException
\end{javacode}

\paragraph{Parameters:}
\begin{itemize}[noitemsep]
\item \code{String spacename}\\
The name of the space as a string or symbol.

\item \code{Object key}\\
The key for the operation as a Python type.

\item \code{Map<String, Object> predicates}\\
A hash of predicates to check against.

\item \code{Map<String, Object> attributes}\\
The set of attributes to modify and their respective values.  \code{attrs} is a
map from the attributes' names to their values.

\end{itemize}

\paragraph{Returns:}
This function returns an object indicating the success or failure of the
operation.  Valid values to be returned are:

\begin{itemize}[noitemsep]
\item \code{True} if the operation succeeded
\item \code{False} if any provided predicates failed.
\item \code{null} if the operation requires an existing value and none exists
\end{itemize}

On error, this function will raise a \code{HyperDexClientException} describing
the error.


\pagebreak
\subsubsection{\code{async\_cond\_set\_intersect}}
\label{api:java:async_cond_set_intersect}
\index{async\_cond\_set\_intersect!Java API}
Store the intersection of the specified set and the existing value for each
attribute if and only if the \code{checks} hold on the object.
This operation requires a pre-existing object in order to complete successfully.
If no object exists, the operation will fail with \code{NOTFOUND}.


This operation will succeed if and only if the predicates specified by
\code{checks} hold on the pre-existing object.  If any of the predicates are not
true for the existing object, then the operation will have no effect and fail
with \code{CMPFAIL}.

All checks are atomic with the write.  HyperDex guarantees that no other
operation will come between validating the checks, and writing the new version
of the object.



\paragraph{Definition:}
\begin{javacode}
public Deferred async_cond_set_intersect(
        String spacename,
        Object key,
        Map<String, Object> predicates,
        Map<String, Object> attributes) throws HyperDexClientException
\end{javacode}

\paragraph{Parameters:}
\begin{itemize}[noitemsep]
\item \code{String spacename}\\
The name of the space as a string or symbol.

\item \code{Object key}\\
The key for the operation as a Python type.

\item \code{Map<String, Object> predicates}\\
A hash of predicates to check against.

\item \code{Map<String, Object> attributes}\\
The set of attributes to modify and their respective values.  \code{attrs} is a
map from the attributes' names to their values.

\end{itemize}

\paragraph{Returns:}
A Deferred object with a \code{wait} method that returns True if the operation
succeeded or False if any provided predicates failed.  Raises an exception on
error.


\paragraph{See also:}  This is the asynchronous form of \code{cond\_set\_intersect}.

%%%%%%%%%%%%%%%%%%%% set_union %%%%%%%%%%%%%%%%%%%%
\pagebreak
\subsubsection{\code{set\_union}}
\label{api:java:set_union}
\index{set\_union!Java API}
Store the union of the specified set and the existing value for each attribute.

%%% Generated below here
\paragraph{Behavior:}
\begin{itemize}[noitemsep]
This operation requires a pre-existing object in order to complete successfully.
If no object exists, the operation will fail with \code{NOTFOUND}.

\end{itemize}


\paragraph{Definition:}
\begin{javacode}
public Boolean set_union(
        String spacename,
        Object key,
        Map<String, Object> attributes) throws HyperDexClientException
\end{javacode}

\paragraph{Parameters:}
\begin{itemize}[noitemsep]
\item \code{String spacename}\\
The name of the space as a string or symbol.

\item \code{Object key}\\
The key for the operation as a Python type.

\item \code{Map<String, Object> attributes}\\
The set of attributes to modify and their respective values.  \code{attrs} is a
map from the attributes' names to their values.

\end{itemize}

\paragraph{Returns:}
This function returns an object indicating the success or failure of the
operation.  Valid values to be returned are:

\begin{itemize}[noitemsep]
\item \code{True} if the operation succeeded
\item \code{False} if any provided predicates failed.
\item \code{null} if the operation requires an existing value and none exists
\end{itemize}

On error, this function will raise a \code{HyperDexClientException} describing
the error.


\pagebreak
\subsubsection{\code{async\_set\_union}}
\label{api:java:async_set_union}
\index{async\_set\_union!Java API}
Store the union of the specified set and the existing value for each attribute.

%%% Generated below here
\paragraph{Behavior:}
\begin{itemize}[noitemsep]
This operation requires a pre-existing object in order to complete successfully.
If no object exists, the operation will fail with \code{NOTFOUND}.

\end{itemize}


\paragraph{Definition:}
\begin{javacode}
public Deferred async_set_union(
        String spacename,
        Object key,
        Map<String, Object> attributes) throws HyperDexClientException
\end{javacode}

\paragraph{Parameters:}
\begin{itemize}[noitemsep]
\item \code{String spacename}\\
The name of the space as a string or symbol.

\item \code{Object key}\\
The key for the operation as a Python type.

\item \code{Map<String, Object> attributes}\\
The set of attributes to modify and their respective values.  \code{attrs} is a
map from the attributes' names to their values.

\end{itemize}

\paragraph{Returns:}
A Deferred object with a \code{wait} method that returns True if the operation
succeeded or False if any provided predicates failed.  Raises an exception on
error.


\paragraph{See also:}  This is the asynchronous form of \code{set\_union}.

%%%%%%%%%%%%%%%%%%%% cond_set_union %%%%%%%%%%%%%%%%%%%%
\pagebreak
\subsubsection{\code{cond\_set\_union}}
\label{api:java:cond_set_union}
\index{cond\_set\_union!Java API}
Conditionally store the union of the specified set and the existing value for
each attribute.

%%% Generated below here
\paragraph{Behavior:}
\begin{itemize}[noitemsep]
This operation requires a pre-existing object in order to complete successfully.
If no object exists, the operation will fail with \code{NOTFOUND}.

This operation will succeed if and only if the predicates specified by
\code{checks} hold on the pre-existing object.  If any of the predicates are not
true for the existing object, then the operation will have no effect and fail
with \code{CMPFAIL}.

All checks are atomic with the write.  HyperDex guarantees that no other
operation will come between validating the checks, and writing the new version
of the object.

\end{itemize}


\paragraph{Definition:}
\begin{javacode}
public Boolean cond_set_union(
        String spacename,
        Object key,
        Map<String, Object> predicates,
        Map<String, Object> attributes) throws HyperDexClientException
\end{javacode}

\paragraph{Parameters:}
\begin{itemize}[noitemsep]
\item \code{String spacename}\\
The name of the space as a string or symbol.

\item \code{Object key}\\
The key for the operation as a Python type.

\item \code{Map<String, Object> predicates}\\
A hash of predicates to check against.

\item \code{Map<String, Object> attributes}\\
The set of attributes to modify and their respective values.  \code{attrs} is a
map from the attributes' names to their values.

\end{itemize}

\paragraph{Returns:}
This function returns an object indicating the success or failure of the
operation.  Valid values to be returned are:

\begin{itemize}[noitemsep]
\item \code{True} if the operation succeeded
\item \code{False} if any provided predicates failed.
\item \code{null} if the operation requires an existing value and none exists
\end{itemize}

On error, this function will raise a \code{HyperDexClientException} describing
the error.


\pagebreak
\subsubsection{\code{async\_cond\_set\_union}}
\label{api:java:async_cond_set_union}
\index{async\_cond\_set\_union!Java API}
Conditionally store the union of the specified set and the existing value for
each attribute.

%%% Generated below here
\paragraph{Behavior:}
\begin{itemize}[noitemsep]
This operation requires a pre-existing object in order to complete successfully.
If no object exists, the operation will fail with \code{NOTFOUND}.

This operation will succeed if and only if the predicates specified by
\code{checks} hold on the pre-existing object.  If any of the predicates are not
true for the existing object, then the operation will have no effect and fail
with \code{CMPFAIL}.

All checks are atomic with the write.  HyperDex guarantees that no other
operation will come between validating the checks, and writing the new version
of the object.

\end{itemize}


\paragraph{Definition:}
\begin{javacode}
public Deferred async_cond_set_union(
        String spacename,
        Object key,
        Map<String, Object> predicates,
        Map<String, Object> attributes) throws HyperDexClientException
\end{javacode}

\paragraph{Parameters:}
\begin{itemize}[noitemsep]
\item \code{String spacename}\\
The name of the space as a string or symbol.

\item \code{Object key}\\
The key for the operation as a Python type.

\item \code{Map<String, Object> predicates}\\
A hash of predicates to check against.

\item \code{Map<String, Object> attributes}\\
The set of attributes to modify and their respective values.  \code{attrs} is a
map from the attributes' names to their values.

\end{itemize}

\paragraph{Returns:}
A Deferred object with a \code{wait} method that returns True if the operation
succeeded or False if any provided predicates failed.  Raises an exception on
error.


\paragraph{See also:}  This is the asynchronous form of \code{cond\_set\_union}.

%%%%%%%%%%%%%%%%%%%% map_add %%%%%%%%%%%%%%%%%%%%
\pagebreak
\subsubsection{\code{map\_add}}
\label{api:java:map_add}
\index{map\_add!Java API}
Insert a key-value pair into the map specified by each map-attribute.
This operation requires a pre-existing object in order to complete successfully.
If no object exists, the operation will fail with \code{NOTFOUND}.



\paragraph{Definition:}
\begin{javacode}
public Boolean map_add(
        String spacename,
        Object key,
        Map<String, Map<Object, Object>> mapattributes) throws HyperDexClientException
\end{javacode}

\paragraph{Parameters:}
\begin{itemize}[noitemsep]
\item \code{String spacename}\\
The name of the space as a string or symbol.

\item \code{Object key}\\
The key for the operation as a Python type.

\item \code{Map<String, Map<Object, Object>> mapattributes}\\
A hash specifying map attributes to modify and their respective key/values.

\end{itemize}

\paragraph{Returns:}
This function returns an object indicating the success or failure of the
operation.  Valid values to be returned are:

\begin{itemize}[noitemsep]
\item \code{True} if the operation succeeded
\item \code{False} if any provided predicates failed.
\item \code{null} if the operation requires an existing value and none exists
\end{itemize}

On error, this function will raise a \code{HyperDexClientException} describing
the error.


\pagebreak
\subsubsection{\code{async\_map\_add}}
\label{api:java:async_map_add}
\index{async\_map\_add!Java API}
Insert a key-value pair into the map specified by each map-attribute.
This operation requires a pre-existing object in order to complete successfully.
If no object exists, the operation will fail with \code{NOTFOUND}.



\paragraph{Definition:}
\begin{javacode}
public Deferred async_map_add(
        String spacename,
        Object key,
        Map<String, Map<Object, Object>> mapattributes) throws HyperDexClientException
\end{javacode}

\paragraph{Parameters:}
\begin{itemize}[noitemsep]
\item \code{String spacename}\\
The name of the space as a string or symbol.

\item \code{Object key}\\
The key for the operation as a Python type.

\item \code{Map<String, Map<Object, Object>> mapattributes}\\
A hash specifying map attributes to modify and their respective key/values.

\end{itemize}

\paragraph{Returns:}
A Deferred object with a \code{wait} method that returns True if the operation
succeeded or False if any provided predicates failed.  Raises an exception on
error.


\paragraph{See also:}  This is the asynchronous form of \code{map\_add}.

%%%%%%%%%%%%%%%%%%%% cond_map_add %%%%%%%%%%%%%%%%%%%%
\pagebreak
\subsubsection{\code{cond\_map\_add}}
\label{api:java:cond_map_add}
\index{cond\_map\_add!Java API}
Insert a key-value pair into the map specified by each map-attribute if and only
if the \code{checks} hold on the object.
This operation requires a pre-existing object in order to complete successfully.
If no object exists, the operation will fail with \code{NOTFOUND}.


This operation will succeed if and only if the predicates specified by
\code{checks} hold on the pre-existing object.  If any of the predicates are not
true for the existing object, then the operation will have no effect and fail
with \code{CMPFAIL}.

All checks are atomic with the write.  HyperDex guarantees that no other
operation will come between validating the checks, and writing the new version
of the object.



\paragraph{Definition:}
\begin{javacode}
public Boolean cond_map_add(
        String spacename,
        Object key,
        Map<String, Object> predicates,
        Map<String, Map<Object, Object>> mapattributes) throws HyperDexClientException
\end{javacode}

\paragraph{Parameters:}
\begin{itemize}[noitemsep]
\item \code{String spacename}\\
The name of the space as a string or symbol.

\item \code{Object key}\\
The key for the operation as a Python type.

\item \code{Map<String, Object> predicates}\\
A hash of predicates to check against.

\item \code{Map<String, Map<Object, Object>> mapattributes}\\
A hash specifying map attributes to modify and their respective key/values.

\end{itemize}

\paragraph{Returns:}
This function returns an object indicating the success or failure of the
operation.  Valid values to be returned are:

\begin{itemize}[noitemsep]
\item \code{True} if the operation succeeded
\item \code{False} if any provided predicates failed.
\item \code{null} if the operation requires an existing value and none exists
\end{itemize}

On error, this function will raise a \code{HyperDexClientException} describing
the error.


\pagebreak
\subsubsection{\code{async\_cond\_map\_add}}
\label{api:java:async_cond_map_add}
\index{async\_cond\_map\_add!Java API}
Insert a key-value pair into the map specified by each map-attribute if and only
if the \code{checks} hold on the object.
This operation requires a pre-existing object in order to complete successfully.
If no object exists, the operation will fail with \code{NOTFOUND}.


This operation will succeed if and only if the predicates specified by
\code{checks} hold on the pre-existing object.  If any of the predicates are not
true for the existing object, then the operation will have no effect and fail
with \code{CMPFAIL}.

All checks are atomic with the write.  HyperDex guarantees that no other
operation will come between validating the checks, and writing the new version
of the object.



\paragraph{Definition:}
\begin{javacode}
public Deferred async_cond_map_add(
        String spacename,
        Object key,
        Map<String, Object> predicates,
        Map<String, Map<Object, Object>> mapattributes) throws HyperDexClientException
\end{javacode}

\paragraph{Parameters:}
\begin{itemize}[noitemsep]
\item \code{String spacename}\\
The name of the space as a string or symbol.

\item \code{Object key}\\
The key for the operation as a Python type.

\item \code{Map<String, Object> predicates}\\
A hash of predicates to check against.

\item \code{Map<String, Map<Object, Object>> mapattributes}\\
A hash specifying map attributes to modify and their respective key/values.

\end{itemize}

\paragraph{Returns:}
A Deferred object with a \code{wait} method that returns True if the operation
succeeded or False if any provided predicates failed.  Raises an exception on
error.


\paragraph{See also:}  This is the asynchronous form of \code{cond\_map\_add}.

%%%%%%%%%%%%%%%%%%%% map_remove %%%%%%%%%%%%%%%%%%%%
\pagebreak
\subsubsection{\code{map\_remove}}
\label{api:java:map_remove}
\index{map\_remove!Java API}
Remove a key-value pair from the map specified by each attribute.  If there is
no pair with the specified key within the map, this operation will do nothing.
This operation requires a pre-existing object in order to complete successfully.
If no object exists, the operation will fail with \code{NOTFOUND}.



\paragraph{Definition:}
\begin{javacode}
public Boolean map_remove(
        String spacename,
        Object key,
        Map<String, Object> attributes) throws HyperDexClientException
\end{javacode}

\paragraph{Parameters:}
\begin{itemize}[noitemsep]
\item \code{String spacename}\\
The name of the space as a string or symbol.

\item \code{Object key}\\
The key for the operation as a Python type.

\item \code{Map<String, Object> attributes}\\
The set of attributes to modify and their respective values.  \code{attrs} is a
map from the attributes' names to their values.

\end{itemize}

\paragraph{Returns:}
This function returns an object indicating the success or failure of the
operation.  Valid values to be returned are:

\begin{itemize}[noitemsep]
\item \code{True} if the operation succeeded
\item \code{False} if any provided predicates failed.
\item \code{null} if the operation requires an existing value and none exists
\end{itemize}

On error, this function will raise a \code{HyperDexClientException} describing
the error.


\pagebreak
\subsubsection{\code{async\_map\_remove}}
\label{api:java:async_map_remove}
\index{async\_map\_remove!Java API}
Remove a key-value pair from the map specified by each attribute.  If there is
no pair with the specified key within the map, this operation will do nothing.
This operation requires a pre-existing object in order to complete successfully.
If no object exists, the operation will fail with \code{NOTFOUND}.



\paragraph{Definition:}
\begin{javacode}
public Deferred async_map_remove(
        String spacename,
        Object key,
        Map<String, Object> attributes) throws HyperDexClientException
\end{javacode}

\paragraph{Parameters:}
\begin{itemize}[noitemsep]
\item \code{String spacename}\\
The name of the space as a string or symbol.

\item \code{Object key}\\
The key for the operation as a Python type.

\item \code{Map<String, Object> attributes}\\
The set of attributes to modify and their respective values.  \code{attrs} is a
map from the attributes' names to their values.

\end{itemize}

\paragraph{Returns:}
A Deferred object with a \code{wait} method that returns True if the operation
succeeded or False if any provided predicates failed.  Raises an exception on
error.


\paragraph{See also:}  This is the asynchronous form of \code{map\_remove}.

%%%%%%%%%%%%%%%%%%%% cond_map_remove %%%%%%%%%%%%%%%%%%%%
\pagebreak
\subsubsection{\code{cond\_map\_remove}}
\label{api:java:cond_map_remove}
\index{cond\_map\_remove!Java API}
Conditionally remove a key-value pair from the map specified by each attribute.

%%% Generated below here
\paragraph{Behavior:}
\begin{itemize}[noitemsep]
This operation requires a pre-existing object in order to complete successfully.
If no object exists, the operation will fail with \code{NOTFOUND}.

This operation will succeed if and only if the predicates specified by
\code{checks} hold on the pre-existing object.  If any of the predicates are not
true for the existing object, then the operation will have no effect and fail
with \code{CMPFAIL}.

All checks are atomic with the write.  HyperDex guarantees that no other
operation will come between validating the checks, and writing the new version
of the object.

\end{itemize}


\paragraph{Definition:}
\begin{javacode}
public Boolean cond_map_remove(
        String spacename,
        Object key,
        Map<String, Object> predicates,
        Map<String, Object> attributes) throws HyperDexClientException
\end{javacode}

\paragraph{Parameters:}
\begin{itemize}[noitemsep]
\item \code{String spacename}\\
The name of the space as a string or symbol.

\item \code{Object key}\\
The key for the operation as a Python type.

\item \code{Map<String, Object> predicates}\\
A hash of predicates to check against.

\item \code{Map<String, Object> attributes}\\
The set of attributes to modify and their respective values.  \code{attrs} is a
map from the attributes' names to their values.

\end{itemize}

\paragraph{Returns:}
This function returns an object indicating the success or failure of the
operation.  Valid values to be returned are:

\begin{itemize}[noitemsep]
\item \code{True} if the operation succeeded
\item \code{False} if any provided predicates failed.
\item \code{null} if the operation requires an existing value and none exists
\end{itemize}

On error, this function will raise a \code{HyperDexClientException} describing
the error.


\pagebreak
\subsubsection{\code{async\_cond\_map\_remove}}
\label{api:java:async_cond_map_remove}
\index{async\_cond\_map\_remove!Java API}
Conditionally remove a key-value pair from the map specified by each attribute.

%%% Generated below here
\paragraph{Behavior:}
\begin{itemize}[noitemsep]
This operation requires a pre-existing object in order to complete successfully.
If no object exists, the operation will fail with \code{NOTFOUND}.

This operation will succeed if and only if the predicates specified by
\code{checks} hold on the pre-existing object.  If any of the predicates are not
true for the existing object, then the operation will have no effect and fail
with \code{CMPFAIL}.

All checks are atomic with the write.  HyperDex guarantees that no other
operation will come between validating the checks, and writing the new version
of the object.

\end{itemize}


\paragraph{Definition:}
\begin{javacode}
public Deferred async_cond_map_remove(
        String spacename,
        Object key,
        Map<String, Object> predicates,
        Map<String, Object> attributes) throws HyperDexClientException
\end{javacode}

\paragraph{Parameters:}
\begin{itemize}[noitemsep]
\item \code{String spacename}\\
The name of the space as a string or symbol.

\item \code{Object key}\\
The key for the operation as a Python type.

\item \code{Map<String, Object> predicates}\\
A hash of predicates to check against.

\item \code{Map<String, Object> attributes}\\
The set of attributes to modify and their respective values.  \code{attrs} is a
map from the attributes' names to their values.

\end{itemize}

\paragraph{Returns:}
A Deferred object with a \code{wait} method that returns True if the operation
succeeded or False if any provided predicates failed.  Raises an exception on
error.


\paragraph{See also:}  This is the asynchronous form of \code{cond\_map\_remove}.

%%%%%%%%%%%%%%%%%%%% document_rename %%%%%%%%%%%%%%%%%%%%
\pagebreak
\subsubsection{\code{document\_rename}}
\label{api:java:document_rename}
\index{document\_rename!Java API}
Move a field within a document from one name to another.
This operation requires a pre-existing object in order to complete successfully.
If no object exists, the operation will fail with \code{NOTFOUND}.



\paragraph{Definition:}
\begin{javacode}
public Boolean document_rename(
        String spacename,
        Object key,
        Map<String, Object> attributes) throws HyperDexClientException
\end{javacode}

\paragraph{Parameters:}
\begin{itemize}[noitemsep]
\item \code{String spacename}\\
The name of the space as a string or symbol.

\item \code{Object key}\\
The key for the operation as a Python type.

\item \code{Map<String, Object> attributes}\\
The set of attributes to modify and their respective values.  \code{attrs} is a
map from the attributes' names to their values.

\end{itemize}

\paragraph{Returns:}
This function returns an object indicating the success or failure of the
operation.  Valid values to be returned are:

\begin{itemize}[noitemsep]
\item \code{True} if the operation succeeded
\item \code{False} if any provided predicates failed.
\item \code{null} if the operation requires an existing value and none exists
\end{itemize}

On error, this function will raise a \code{HyperDexClientException} describing
the error.


\pagebreak
\subsubsection{\code{async\_document\_rename}}
\label{api:java:async_document_rename}
\index{async\_document\_rename!Java API}
Move a field within a document from one name to another.
This operation requires a pre-existing object in order to complete successfully.
If no object exists, the operation will fail with \code{NOTFOUND}.



\paragraph{Definition:}
\begin{javacode}
public Deferred async_document_rename(
        String spacename,
        Object key,
        Map<String, Object> attributes) throws HyperDexClientException
\end{javacode}

\paragraph{Parameters:}
\begin{itemize}[noitemsep]
\item \code{String spacename}\\
The name of the space as a string or symbol.

\item \code{Object key}\\
The key for the operation as a Python type.

\item \code{Map<String, Object> attributes}\\
The set of attributes to modify and their respective values.  \code{attrs} is a
map from the attributes' names to their values.

\end{itemize}

\paragraph{Returns:}
A Deferred object with a \code{wait} method that returns True if the operation
succeeded or False if any provided predicates failed.  Raises an exception on
error.


\paragraph{See also:}  This is the asynchronous form of \code{document\_rename}.

%%%%%%%%%%%%%%%%%%%% document_unset %%%%%%%%%%%%%%%%%%%%
\pagebreak
\subsubsection{\code{document\_unset}}
\label{api:java:document_unset}
\index{document\_unset!Java API}
Remove a field or object from a document.
This operation requires a pre-existing object in order to complete successfully.
If no object exists, the operation will fail with \code{NOTFOUND}.



\paragraph{Definition:}
\begin{javacode}
public Boolean document_unset(
        String spacename,
        Object key,
        Map<String, Object> attributes) throws HyperDexClientException
\end{javacode}

\paragraph{Parameters:}
\begin{itemize}[noitemsep]
\item \code{String spacename}\\
The name of the space as a string or symbol.

\item \code{Object key}\\
The key for the operation as a Python type.

\item \code{Map<String, Object> attributes}\\
The set of attributes to modify and their respective values.  \code{attrs} is a
map from the attributes' names to their values.

\end{itemize}

\paragraph{Returns:}
This function returns an object indicating the success or failure of the
operation.  Valid values to be returned are:

\begin{itemize}[noitemsep]
\item \code{True} if the operation succeeded
\item \code{False} if any provided predicates failed.
\item \code{null} if the operation requires an existing value and none exists
\end{itemize}

On error, this function will raise a \code{HyperDexClientException} describing
the error.


\pagebreak
\subsubsection{\code{async\_document\_unset}}
\label{api:java:async_document_unset}
\index{async\_document\_unset!Java API}
Remove a field or object from a document.
This operation requires a pre-existing object in order to complete successfully.
If no object exists, the operation will fail with \code{NOTFOUND}.



\paragraph{Definition:}
\begin{javacode}
public Deferred async_document_unset(
        String spacename,
        Object key,
        Map<String, Object> attributes) throws HyperDexClientException
\end{javacode}

\paragraph{Parameters:}
\begin{itemize}[noitemsep]
\item \code{String spacename}\\
The name of the space as a string or symbol.

\item \code{Object key}\\
The key for the operation as a Python type.

\item \code{Map<String, Object> attributes}\\
The set of attributes to modify and their respective values.  \code{attrs} is a
map from the attributes' names to their values.

\end{itemize}

\paragraph{Returns:}
A Deferred object with a \code{wait} method that returns True if the operation
succeeded or False if any provided predicates failed.  Raises an exception on
error.


\paragraph{See also:}  This is the asynchronous form of \code{document\_unset}.

%%%%%%%%%%%%%%%%%%%% document_set %%%%%%%%%%%%%%%%%%%%
\pagebreak
\subsubsection{\code{document\_set}}
\label{api:java:document_set}
\index{document\_set!Java API}
\input{\topdir/api/desc/document_set}

\paragraph{Definition:}
\begin{javacode}
public Boolean document_set(
        String spacename,
        Object key,
        Map<String, Object> attributes) throws HyperDexClientException
\end{javacode}

\paragraph{Parameters:}
\begin{itemize}[noitemsep]
\item \code{String spacename}\\
The name of the space as a string or symbol.

\item \code{Object key}\\
The key for the operation as a Python type.

\item \code{Map<String, Object> attributes}\\
The set of attributes to modify and their respective values.  \code{attrs} is a
map from the attributes' names to their values.

\end{itemize}

\paragraph{Returns:}
This function returns an object indicating the success or failure of the
operation.  Valid values to be returned are:

\begin{itemize}[noitemsep]
\item \code{True} if the operation succeeded
\item \code{False} if any provided predicates failed.
\item \code{null} if the operation requires an existing value and none exists
\end{itemize}

On error, this function will raise a \code{HyperDexClientException} describing
the error.


\pagebreak
\subsubsection{\code{async\_document\_set}}
\label{api:java:async_document_set}
\index{async\_document\_set!Java API}
\input{\topdir/api/desc/document_set}

\paragraph{Definition:}
\begin{javacode}
public Deferred async_document_set(
        String spacename,
        Object key,
        Map<String, Object> attributes) throws HyperDexClientException
\end{javacode}

\paragraph{Parameters:}
\begin{itemize}[noitemsep]
\item \code{String spacename}\\
The name of the space as a string or symbol.

\item \code{Object key}\\
The key for the operation as a Python type.

\item \code{Map<String, Object> attributes}\\
The set of attributes to modify and their respective values.  \code{attrs} is a
map from the attributes' names to their values.

\end{itemize}

\paragraph{Returns:}
A Deferred object with a \code{wait} method that returns True if the operation
succeeded or False if any provided predicates failed.  Raises an exception on
error.


\paragraph{See also:}  This is the asynchronous form of \code{document\_set}.

%%%%%%%%%%%%%%%%%%%% map_atomic_add %%%%%%%%%%%%%%%%%%%%
\pagebreak
\subsubsection{\code{map\_atomic\_add}}
\label{api:java:map_atomic_add}
\index{map\_atomic\_add!Java API}
Add the specified number to the value of a key-value pair within each map.
This operation requires a pre-existing object in order to complete successfully.
If no object exists, the operation will fail with \code{NOTFOUND}.



\paragraph{Definition:}
\begin{javacode}
public Boolean map_atomic_add(
        String spacename,
        Object key,
        Map<String, Map<Object, Object>> mapattributes) throws HyperDexClientException
\end{javacode}

\paragraph{Parameters:}
\begin{itemize}[noitemsep]
\item \code{String spacename}\\
The name of the space as a string or symbol.

\item \code{Object key}\\
The key for the operation as a Python type.

\item \code{Map<String, Map<Object, Object>> mapattributes}\\
A hash specifying map attributes to modify and their respective key/values.

\end{itemize}

\paragraph{Returns:}
This function returns an object indicating the success or failure of the
operation.  Valid values to be returned are:

\begin{itemize}[noitemsep]
\item \code{True} if the operation succeeded
\item \code{False} if any provided predicates failed.
\item \code{null} if the operation requires an existing value and none exists
\end{itemize}

On error, this function will raise a \code{HyperDexClientException} describing
the error.


\pagebreak
\subsubsection{\code{async\_map\_atomic\_add}}
\label{api:java:async_map_atomic_add}
\index{async\_map\_atomic\_add!Java API}
Add the specified number to the value of a key-value pair within each map.
This operation requires a pre-existing object in order to complete successfully.
If no object exists, the operation will fail with \code{NOTFOUND}.



\paragraph{Definition:}
\begin{javacode}
public Deferred async_map_atomic_add(
        String spacename,
        Object key,
        Map<String, Map<Object, Object>> mapattributes) throws HyperDexClientException
\end{javacode}

\paragraph{Parameters:}
\begin{itemize}[noitemsep]
\item \code{String spacename}\\
The name of the space as a string or symbol.

\item \code{Object key}\\
The key for the operation as a Python type.

\item \code{Map<String, Map<Object, Object>> mapattributes}\\
A hash specifying map attributes to modify and their respective key/values.

\end{itemize}

\paragraph{Returns:}
A Deferred object with a \code{wait} method that returns True if the operation
succeeded or False if any provided predicates failed.  Raises an exception on
error.


\paragraph{See also:}  This is the asynchronous form of \code{map\_atomic\_add}.

%%%%%%%%%%%%%%%%%%%% cond_map_atomic_add %%%%%%%%%%%%%%%%%%%%
\pagebreak
\subsubsection{\code{cond\_map\_atomic\_add}}
\label{api:java:cond_map_atomic_add}
\index{cond\_map\_atomic\_add!Java API}
Add the specified number to the value of a key-value pair within each map if and
only if the \code{checks} hold on the object.
This operation requires a pre-existing object in order to complete successfully.
If no object exists, the operation will fail with \code{NOTFOUND}.


This operation will succeed if and only if the predicates specified by
\code{checks} hold on the pre-existing object.  If any of the predicates are not
true for the existing object, then the operation will have no effect and fail
with \code{CMPFAIL}.

All checks are atomic with the write.  HyperDex guarantees that no other
operation will come between validating the checks, and writing the new version
of the object.



\paragraph{Definition:}
\begin{javacode}
public Boolean cond_map_atomic_add(
        String spacename,
        Object key,
        Map<String, Object> predicates,
        Map<String, Map<Object, Object>> mapattributes) throws HyperDexClientException
\end{javacode}

\paragraph{Parameters:}
\begin{itemize}[noitemsep]
\item \code{String spacename}\\
The name of the space as a string or symbol.

\item \code{Object key}\\
The key for the operation as a Python type.

\item \code{Map<String, Object> predicates}\\
A hash of predicates to check against.

\item \code{Map<String, Map<Object, Object>> mapattributes}\\
A hash specifying map attributes to modify and their respective key/values.

\end{itemize}

\paragraph{Returns:}
This function returns an object indicating the success or failure of the
operation.  Valid values to be returned are:

\begin{itemize}[noitemsep]
\item \code{True} if the operation succeeded
\item \code{False} if any provided predicates failed.
\item \code{null} if the operation requires an existing value and none exists
\end{itemize}

On error, this function will raise a \code{HyperDexClientException} describing
the error.


\pagebreak
\subsubsection{\code{async\_cond\_map\_atomic\_add}}
\label{api:java:async_cond_map_atomic_add}
\index{async\_cond\_map\_atomic\_add!Java API}
Add the specified number to the value of a key-value pair within each map if and
only if the \code{checks} hold on the object.
This operation requires a pre-existing object in order to complete successfully.
If no object exists, the operation will fail with \code{NOTFOUND}.


This operation will succeed if and only if the predicates specified by
\code{checks} hold on the pre-existing object.  If any of the predicates are not
true for the existing object, then the operation will have no effect and fail
with \code{CMPFAIL}.

All checks are atomic with the write.  HyperDex guarantees that no other
operation will come between validating the checks, and writing the new version
of the object.



\paragraph{Definition:}
\begin{javacode}
public Deferred async_cond_map_atomic_add(
        String spacename,
        Object key,
        Map<String, Object> predicates,
        Map<String, Map<Object, Object>> mapattributes) throws HyperDexClientException
\end{javacode}

\paragraph{Parameters:}
\begin{itemize}[noitemsep]
\item \code{String spacename}\\
The name of the space as a string or symbol.

\item \code{Object key}\\
The key for the operation as a Python type.

\item \code{Map<String, Object> predicates}\\
A hash of predicates to check against.

\item \code{Map<String, Map<Object, Object>> mapattributes}\\
A hash specifying map attributes to modify and their respective key/values.

\end{itemize}

\paragraph{Returns:}
A Deferred object with a \code{wait} method that returns True if the operation
succeeded or False if any provided predicates failed.  Raises an exception on
error.


\paragraph{See also:}  This is the asynchronous form of \code{cond\_map\_atomic\_add}.

%%%%%%%%%%%%%%%%%%%% map_atomic_sub %%%%%%%%%%%%%%%%%%%%
\pagebreak
\subsubsection{\code{map\_atomic\_sub}}
\label{api:java:map_atomic_sub}
\index{map\_atomic\_sub!Java API}
Subtract the specified number from the value of a key-value pair within each
map.

%%% Generated below here
\paragraph{Behavior:}
\begin{itemize}[noitemsep]
This operation requires a pre-existing object in order to complete successfully.
If no object exists, the operation will fail with \code{NOTFOUND}.

\item This operation mutates the value of a key-value pair in a map.  This call
    is similar to the equivalent call without the \code{map\_} prefix, but
    operates on the value of a pair in a map, instead of on an attribute's
    value.  If there is no pair with the specified map key, a new pair will be
    created and initialized to its default value.  If this is undesirable, it
    may be avoided by using a conditional operation that requires that the map
    contain the key in question.

\end{itemize}


\paragraph{Definition:}
\begin{javacode}
public Boolean map_atomic_sub(
        String spacename,
        Object key,
        Map<String, Map<Object, Object>> mapattributes) throws HyperDexClientException
\end{javacode}

\paragraph{Parameters:}
\begin{itemize}[noitemsep]
\item \code{String spacename}\\
The name of the space as a string or symbol.

\item \code{Object key}\\
The key for the operation as a Python type.

\item \code{Map<String, Map<Object, Object>> mapattributes}\\
A hash specifying map attributes to modify and their respective key/values.

\end{itemize}

\paragraph{Returns:}
This function returns an object indicating the success or failure of the
operation.  Valid values to be returned are:

\begin{itemize}[noitemsep]
\item \code{True} if the operation succeeded
\item \code{False} if any provided predicates failed.
\item \code{null} if the operation requires an existing value and none exists
\end{itemize}

On error, this function will raise a \code{HyperDexClientException} describing
the error.


\pagebreak
\subsubsection{\code{async\_map\_atomic\_sub}}
\label{api:java:async_map_atomic_sub}
\index{async\_map\_atomic\_sub!Java API}
Subtract the specified number from the value of a key-value pair within each
map.

%%% Generated below here
\paragraph{Behavior:}
\begin{itemize}[noitemsep]
This operation requires a pre-existing object in order to complete successfully.
If no object exists, the operation will fail with \code{NOTFOUND}.

\item This operation mutates the value of a key-value pair in a map.  This call
    is similar to the equivalent call without the \code{map\_} prefix, but
    operates on the value of a pair in a map, instead of on an attribute's
    value.  If there is no pair with the specified map key, a new pair will be
    created and initialized to its default value.  If this is undesirable, it
    may be avoided by using a conditional operation that requires that the map
    contain the key in question.

\end{itemize}


\paragraph{Definition:}
\begin{javacode}
public Deferred async_map_atomic_sub(
        String spacename,
        Object key,
        Map<String, Map<Object, Object>> mapattributes) throws HyperDexClientException
\end{javacode}

\paragraph{Parameters:}
\begin{itemize}[noitemsep]
\item \code{String spacename}\\
The name of the space as a string or symbol.

\item \code{Object key}\\
The key for the operation as a Python type.

\item \code{Map<String, Map<Object, Object>> mapattributes}\\
A hash specifying map attributes to modify and their respective key/values.

\end{itemize}

\paragraph{Returns:}
A Deferred object with a \code{wait} method that returns True if the operation
succeeded or False if any provided predicates failed.  Raises an exception on
error.


\paragraph{See also:}  This is the asynchronous form of \code{map\_atomic\_sub}.

%%%%%%%%%%%%%%%%%%%% cond_map_atomic_sub %%%%%%%%%%%%%%%%%%%%
\pagebreak
\subsubsection{\code{cond\_map\_atomic\_sub}}
\label{api:java:cond_map_atomic_sub}
\index{cond\_map\_atomic\_sub!Java API}
Subtract the specified number from the value of a key-value pair within each
map.

%%% Generated below here
\paragraph{Behavior:}
\begin{itemize}[noitemsep]
This operation requires a pre-existing object in order to complete successfully.
If no object exists, the operation will fail with \code{NOTFOUND}.

This operation will succeed if and only if the predicates specified by
\code{checks} hold on the pre-existing object.  If any of the predicates are not
true for the existing object, then the operation will have no effect and fail
with \code{CMPFAIL}.

All checks are atomic with the write.  HyperDex guarantees that no other
operation will come between validating the checks, and writing the new version
of the object.

\item This operation mutates the value of a key-value pair in a map.  This call
    is similar to the equivalent call without the \code{map\_} prefix, but
    operates on the value of a pair in a map, instead of on an attribute's
    value.  If there is no pair with the specified map key, a new pair will be
    created and initialized to its default value.  If this is undesirable, it
    may be avoided by using a conditional operation that requires that the map
    contain the key in question.

\end{itemize}


\paragraph{Definition:}
\begin{javacode}
public Boolean cond_map_atomic_sub(
        String spacename,
        Object key,
        Map<String, Object> predicates,
        Map<String, Map<Object, Object>> mapattributes) throws HyperDexClientException
\end{javacode}

\paragraph{Parameters:}
\begin{itemize}[noitemsep]
\item \code{String spacename}\\
The name of the space as a string or symbol.

\item \code{Object key}\\
The key for the operation as a Python type.

\item \code{Map<String, Object> predicates}\\
A hash of predicates to check against.

\item \code{Map<String, Map<Object, Object>> mapattributes}\\
A hash specifying map attributes to modify and their respective key/values.

\end{itemize}

\paragraph{Returns:}
This function returns an object indicating the success or failure of the
operation.  Valid values to be returned are:

\begin{itemize}[noitemsep]
\item \code{True} if the operation succeeded
\item \code{False} if any provided predicates failed.
\item \code{null} if the operation requires an existing value and none exists
\end{itemize}

On error, this function will raise a \code{HyperDexClientException} describing
the error.


\pagebreak
\subsubsection{\code{async\_cond\_map\_atomic\_sub}}
\label{api:java:async_cond_map_atomic_sub}
\index{async\_cond\_map\_atomic\_sub!Java API}
Subtract the specified number from the value of a key-value pair within each
map.

%%% Generated below here
\paragraph{Behavior:}
\begin{itemize}[noitemsep]
This operation requires a pre-existing object in order to complete successfully.
If no object exists, the operation will fail with \code{NOTFOUND}.

This operation will succeed if and only if the predicates specified by
\code{checks} hold on the pre-existing object.  If any of the predicates are not
true for the existing object, then the operation will have no effect and fail
with \code{CMPFAIL}.

All checks are atomic with the write.  HyperDex guarantees that no other
operation will come between validating the checks, and writing the new version
of the object.

\item This operation mutates the value of a key-value pair in a map.  This call
    is similar to the equivalent call without the \code{map\_} prefix, but
    operates on the value of a pair in a map, instead of on an attribute's
    value.  If there is no pair with the specified map key, a new pair will be
    created and initialized to its default value.  If this is undesirable, it
    may be avoided by using a conditional operation that requires that the map
    contain the key in question.

\end{itemize}


\paragraph{Definition:}
\begin{javacode}
public Deferred async_cond_map_atomic_sub(
        String spacename,
        Object key,
        Map<String, Object> predicates,
        Map<String, Map<Object, Object>> mapattributes) throws HyperDexClientException
\end{javacode}

\paragraph{Parameters:}
\begin{itemize}[noitemsep]
\item \code{String spacename}\\
The name of the space as a string or symbol.

\item \code{Object key}\\
The key for the operation as a Python type.

\item \code{Map<String, Object> predicates}\\
A hash of predicates to check against.

\item \code{Map<String, Map<Object, Object>> mapattributes}\\
A hash specifying map attributes to modify and their respective key/values.

\end{itemize}

\paragraph{Returns:}
A Deferred object with a \code{wait} method that returns True if the operation
succeeded or False if any provided predicates failed.  Raises an exception on
error.


\paragraph{See also:}  This is the asynchronous form of \code{cond\_map\_atomic\_sub}.

%%%%%%%%%%%%%%%%%%%% map_atomic_mul %%%%%%%%%%%%%%%%%%%%
\pagebreak
\subsubsection{\code{map\_atomic\_mul}}
\label{api:java:map_atomic_mul}
\index{map\_atomic\_mul!Java API}
Multiply the value of each key-value pair by the specified number for each map.

%%% Generated below here
\paragraph{Behavior:}
\begin{itemize}[noitemsep]
This operation requires a pre-existing object in order to complete successfully.
If no object exists, the operation will fail with \code{NOTFOUND}.

\item This operation mutates the value of a key-value pair in a map.  This call
    is similar to the equivalent call without the \code{map\_} prefix, but
    operates on the value of a pair in a map, instead of on an attribute's
    value.  If there is no pair with the specified map key, a new pair will be
    created and initialized to its default value.  If this is undesirable, it
    may be avoided by using a conditional operation that requires that the map
    contain the key in question.

\end{itemize}


\paragraph{Definition:}
\begin{javacode}
public Boolean map_atomic_mul(
        String spacename,
        Object key,
        Map<String, Map<Object, Object>> mapattributes) throws HyperDexClientException
\end{javacode}

\paragraph{Parameters:}
\begin{itemize}[noitemsep]
\item \code{String spacename}\\
The name of the space as a string or symbol.

\item \code{Object key}\\
The key for the operation as a Python type.

\item \code{Map<String, Map<Object, Object>> mapattributes}\\
A hash specifying map attributes to modify and their respective key/values.

\end{itemize}

\paragraph{Returns:}
This function returns an object indicating the success or failure of the
operation.  Valid values to be returned are:

\begin{itemize}[noitemsep]
\item \code{True} if the operation succeeded
\item \code{False} if any provided predicates failed.
\item \code{null} if the operation requires an existing value and none exists
\end{itemize}

On error, this function will raise a \code{HyperDexClientException} describing
the error.


\pagebreak
\subsubsection{\code{async\_map\_atomic\_mul}}
\label{api:java:async_map_atomic_mul}
\index{async\_map\_atomic\_mul!Java API}
Multiply the value of each key-value pair by the specified number for each map.

%%% Generated below here
\paragraph{Behavior:}
\begin{itemize}[noitemsep]
This operation requires a pre-existing object in order to complete successfully.
If no object exists, the operation will fail with \code{NOTFOUND}.

\item This operation mutates the value of a key-value pair in a map.  This call
    is similar to the equivalent call without the \code{map\_} prefix, but
    operates on the value of a pair in a map, instead of on an attribute's
    value.  If there is no pair with the specified map key, a new pair will be
    created and initialized to its default value.  If this is undesirable, it
    may be avoided by using a conditional operation that requires that the map
    contain the key in question.

\end{itemize}


\paragraph{Definition:}
\begin{javacode}
public Deferred async_map_atomic_mul(
        String spacename,
        Object key,
        Map<String, Map<Object, Object>> mapattributes) throws HyperDexClientException
\end{javacode}

\paragraph{Parameters:}
\begin{itemize}[noitemsep]
\item \code{String spacename}\\
The name of the space as a string or symbol.

\item \code{Object key}\\
The key for the operation as a Python type.

\item \code{Map<String, Map<Object, Object>> mapattributes}\\
A hash specifying map attributes to modify and their respective key/values.

\end{itemize}

\paragraph{Returns:}
A Deferred object with a \code{wait} method that returns True if the operation
succeeded or False if any provided predicates failed.  Raises an exception on
error.


\paragraph{See also:}  This is the asynchronous form of \code{map\_atomic\_mul}.

%%%%%%%%%%%%%%%%%%%% cond_map_atomic_mul %%%%%%%%%%%%%%%%%%%%
\pagebreak
\subsubsection{\code{cond\_map\_atomic\_mul}}
\label{api:java:cond_map_atomic_mul}
\index{cond\_map\_atomic\_mul!Java API}
Conditionally multiply the value of each key-value pair by the specified number
for each map.

%%% Generated below here
\paragraph{Behavior:}
\begin{itemize}[noitemsep]
This operation requires a pre-existing object in order to complete successfully.
If no object exists, the operation will fail with \code{NOTFOUND}.

This operation will succeed if and only if the predicates specified by
\code{checks} hold on the pre-existing object.  If any of the predicates are not
true for the existing object, then the operation will have no effect and fail
with \code{CMPFAIL}.

All checks are atomic with the write.  HyperDex guarantees that no other
operation will come between validating the checks, and writing the new version
of the object.

\item This operation mutates the value of a key-value pair in a map.  This call
    is similar to the equivalent call without the \code{map\_} prefix, but
    operates on the value of a pair in a map, instead of on an attribute's
    value.  If there is no pair with the specified map key, a new pair will be
    created and initialized to its default value.  If this is undesirable, it
    may be avoided by using a conditional operation that requires that the map
    contain the key in question.

\end{itemize}


\paragraph{Definition:}
\begin{javacode}
public Boolean cond_map_atomic_mul(
        String spacename,
        Object key,
        Map<String, Object> predicates,
        Map<String, Map<Object, Object>> mapattributes) throws HyperDexClientException
\end{javacode}

\paragraph{Parameters:}
\begin{itemize}[noitemsep]
\item \code{String spacename}\\
The name of the space as a string or symbol.

\item \code{Object key}\\
The key for the operation as a Python type.

\item \code{Map<String, Object> predicates}\\
A hash of predicates to check against.

\item \code{Map<String, Map<Object, Object>> mapattributes}\\
A hash specifying map attributes to modify and their respective key/values.

\end{itemize}

\paragraph{Returns:}
This function returns an object indicating the success or failure of the
operation.  Valid values to be returned are:

\begin{itemize}[noitemsep]
\item \code{True} if the operation succeeded
\item \code{False} if any provided predicates failed.
\item \code{null} if the operation requires an existing value and none exists
\end{itemize}

On error, this function will raise a \code{HyperDexClientException} describing
the error.


\pagebreak
\subsubsection{\code{async\_cond\_map\_atomic\_mul}}
\label{api:java:async_cond_map_atomic_mul}
\index{async\_cond\_map\_atomic\_mul!Java API}
Conditionally multiply the value of each key-value pair by the specified number
for each map.

%%% Generated below here
\paragraph{Behavior:}
\begin{itemize}[noitemsep]
This operation requires a pre-existing object in order to complete successfully.
If no object exists, the operation will fail with \code{NOTFOUND}.

This operation will succeed if and only if the predicates specified by
\code{checks} hold on the pre-existing object.  If any of the predicates are not
true for the existing object, then the operation will have no effect and fail
with \code{CMPFAIL}.

All checks are atomic with the write.  HyperDex guarantees that no other
operation will come between validating the checks, and writing the new version
of the object.

\item This operation mutates the value of a key-value pair in a map.  This call
    is similar to the equivalent call without the \code{map\_} prefix, but
    operates on the value of a pair in a map, instead of on an attribute's
    value.  If there is no pair with the specified map key, a new pair will be
    created and initialized to its default value.  If this is undesirable, it
    may be avoided by using a conditional operation that requires that the map
    contain the key in question.

\end{itemize}


\paragraph{Definition:}
\begin{javacode}
public Deferred async_cond_map_atomic_mul(
        String spacename,
        Object key,
        Map<String, Object> predicates,
        Map<String, Map<Object, Object>> mapattributes) throws HyperDexClientException
\end{javacode}

\paragraph{Parameters:}
\begin{itemize}[noitemsep]
\item \code{String spacename}\\
The name of the space as a string or symbol.

\item \code{Object key}\\
The key for the operation as a Python type.

\item \code{Map<String, Object> predicates}\\
A hash of predicates to check against.

\item \code{Map<String, Map<Object, Object>> mapattributes}\\
A hash specifying map attributes to modify and their respective key/values.

\end{itemize}

\paragraph{Returns:}
A Deferred object with a \code{wait} method that returns True if the operation
succeeded or False if any provided predicates failed.  Raises an exception on
error.


\paragraph{See also:}  This is the asynchronous form of \code{cond\_map\_atomic\_mul}.

%%%%%%%%%%%%%%%%%%%% map_atomic_div %%%%%%%%%%%%%%%%%%%%
\pagebreak
\subsubsection{\code{map\_atomic\_div}}
\label{api:java:map_atomic_div}
\index{map\_atomic\_div!Java API}
Divide the value of each key-value pair by the specified number for each map.

%%% Generated below here
\paragraph{Behavior:}
\begin{itemize}[noitemsep]
This operation requires a pre-existing object in order to complete successfully.
If no object exists, the operation will fail with \code{NOTFOUND}.

\item This operation mutates the value of a key-value pair in a map.  This call
    is similar to the equivalent call without the \code{map\_} prefix, but
    operates on the value of a pair in a map, instead of on an attribute's
    value.  If there is no pair with the specified map key, a new pair will be
    created and initialized to its default value.  If this is undesirable, it
    may be avoided by using a conditional operation that requires that the map
    contain the key in question.

\end{itemize}


\paragraph{Definition:}
\begin{javacode}
public Boolean map_atomic_div(
        String spacename,
        Object key,
        Map<String, Map<Object, Object>> mapattributes) throws HyperDexClientException
\end{javacode}

\paragraph{Parameters:}
\begin{itemize}[noitemsep]
\item \code{String spacename}\\
The name of the space as a string or symbol.

\item \code{Object key}\\
The key for the operation as a Python type.

\item \code{Map<String, Map<Object, Object>> mapattributes}\\
A hash specifying map attributes to modify and their respective key/values.

\end{itemize}

\paragraph{Returns:}
This function returns an object indicating the success or failure of the
operation.  Valid values to be returned are:

\begin{itemize}[noitemsep]
\item \code{True} if the operation succeeded
\item \code{False} if any provided predicates failed.
\item \code{null} if the operation requires an existing value and none exists
\end{itemize}

On error, this function will raise a \code{HyperDexClientException} describing
the error.


\pagebreak
\subsubsection{\code{async\_map\_atomic\_div}}
\label{api:java:async_map_atomic_div}
\index{async\_map\_atomic\_div!Java API}
Divide the value of each key-value pair by the specified number for each map.

%%% Generated below here
\paragraph{Behavior:}
\begin{itemize}[noitemsep]
This operation requires a pre-existing object in order to complete successfully.
If no object exists, the operation will fail with \code{NOTFOUND}.

\item This operation mutates the value of a key-value pair in a map.  This call
    is similar to the equivalent call without the \code{map\_} prefix, but
    operates on the value of a pair in a map, instead of on an attribute's
    value.  If there is no pair with the specified map key, a new pair will be
    created and initialized to its default value.  If this is undesirable, it
    may be avoided by using a conditional operation that requires that the map
    contain the key in question.

\end{itemize}


\paragraph{Definition:}
\begin{javacode}
public Deferred async_map_atomic_div(
        String spacename,
        Object key,
        Map<String, Map<Object, Object>> mapattributes) throws HyperDexClientException
\end{javacode}

\paragraph{Parameters:}
\begin{itemize}[noitemsep]
\item \code{String spacename}\\
The name of the space as a string or symbol.

\item \code{Object key}\\
The key for the operation as a Python type.

\item \code{Map<String, Map<Object, Object>> mapattributes}\\
A hash specifying map attributes to modify and their respective key/values.

\end{itemize}

\paragraph{Returns:}
A Deferred object with a \code{wait} method that returns True if the operation
succeeded or False if any provided predicates failed.  Raises an exception on
error.


\paragraph{See also:}  This is the asynchronous form of \code{map\_atomic\_div}.

%%%%%%%%%%%%%%%%%%%% cond_map_atomic_div %%%%%%%%%%%%%%%%%%%%
\pagebreak
\subsubsection{\code{cond\_map\_atomic\_div}}
\label{api:java:cond_map_atomic_div}
\index{cond\_map\_atomic\_div!Java API}
Conditionally divide the value of each key-value pair by the specified number for each map.

%%% Generated below here
\paragraph{Behavior:}
\begin{itemize}[noitemsep]
This operation requires a pre-existing object in order to complete successfully.
If no object exists, the operation will fail with \code{NOTFOUND}.

This operation will succeed if and only if the predicates specified by
\code{checks} hold on the pre-existing object.  If any of the predicates are not
true for the existing object, then the operation will have no effect and fail
with \code{CMPFAIL}.

All checks are atomic with the write.  HyperDex guarantees that no other
operation will come between validating the checks, and writing the new version
of the object.

\item This operation mutates the value of a key-value pair in a map.  This call
    is similar to the equivalent call without the \code{map\_} prefix, but
    operates on the value of a pair in a map, instead of on an attribute's
    value.  If there is no pair with the specified map key, a new pair will be
    created and initialized to its default value.  If this is undesirable, it
    may be avoided by using a conditional operation that requires that the map
    contain the key in question.

\end{itemize}


\paragraph{Definition:}
\begin{javacode}
public Boolean cond_map_atomic_div(
        String spacename,
        Object key,
        Map<String, Object> predicates,
        Map<String, Map<Object, Object>> mapattributes) throws HyperDexClientException
\end{javacode}

\paragraph{Parameters:}
\begin{itemize}[noitemsep]
\item \code{String spacename}\\
The name of the space as a string or symbol.

\item \code{Object key}\\
The key for the operation as a Python type.

\item \code{Map<String, Object> predicates}\\
A hash of predicates to check against.

\item \code{Map<String, Map<Object, Object>> mapattributes}\\
A hash specifying map attributes to modify and their respective key/values.

\end{itemize}

\paragraph{Returns:}
This function returns an object indicating the success or failure of the
operation.  Valid values to be returned are:

\begin{itemize}[noitemsep]
\item \code{True} if the operation succeeded
\item \code{False} if any provided predicates failed.
\item \code{null} if the operation requires an existing value and none exists
\end{itemize}

On error, this function will raise a \code{HyperDexClientException} describing
the error.


\pagebreak
\subsubsection{\code{async\_cond\_map\_atomic\_div}}
\label{api:java:async_cond_map_atomic_div}
\index{async\_cond\_map\_atomic\_div!Java API}
Conditionally divide the value of each key-value pair by the specified number for each map.

%%% Generated below here
\paragraph{Behavior:}
\begin{itemize}[noitemsep]
This operation requires a pre-existing object in order to complete successfully.
If no object exists, the operation will fail with \code{NOTFOUND}.

This operation will succeed if and only if the predicates specified by
\code{checks} hold on the pre-existing object.  If any of the predicates are not
true for the existing object, then the operation will have no effect and fail
with \code{CMPFAIL}.

All checks are atomic with the write.  HyperDex guarantees that no other
operation will come between validating the checks, and writing the new version
of the object.

\item This operation mutates the value of a key-value pair in a map.  This call
    is similar to the equivalent call without the \code{map\_} prefix, but
    operates on the value of a pair in a map, instead of on an attribute's
    value.  If there is no pair with the specified map key, a new pair will be
    created and initialized to its default value.  If this is undesirable, it
    may be avoided by using a conditional operation that requires that the map
    contain the key in question.

\end{itemize}


\paragraph{Definition:}
\begin{javacode}
public Deferred async_cond_map_atomic_div(
        String spacename,
        Object key,
        Map<String, Object> predicates,
        Map<String, Map<Object, Object>> mapattributes) throws HyperDexClientException
\end{javacode}

\paragraph{Parameters:}
\begin{itemize}[noitemsep]
\item \code{String spacename}\\
The name of the space as a string or symbol.

\item \code{Object key}\\
The key for the operation as a Python type.

\item \code{Map<String, Object> predicates}\\
A hash of predicates to check against.

\item \code{Map<String, Map<Object, Object>> mapattributes}\\
A hash specifying map attributes to modify and their respective key/values.

\end{itemize}

\paragraph{Returns:}
A Deferred object with a \code{wait} method that returns True if the operation
succeeded or False if any provided predicates failed.  Raises an exception on
error.


\paragraph{See also:}  This is the asynchronous form of \code{cond\_map\_atomic\_div}.

%%%%%%%%%%%%%%%%%%%% map_atomic_mod %%%%%%%%%%%%%%%%%%%%
\pagebreak
\subsubsection{\code{map\_atomic\_mod}}
\label{api:java:map_atomic_mod}
\index{map\_atomic\_mod!Java API}
Store the value of the key-value pair modulo the specified number for each map.
This operation requires a pre-existing object in order to complete successfully.
If no object exists, the operation will fail with \code{NOTFOUND}.



\paragraph{Definition:}
\begin{javacode}
public Boolean map_atomic_mod(
        String spacename,
        Object key,
        Map<String, Map<Object, Object>> mapattributes) throws HyperDexClientException
\end{javacode}

\paragraph{Parameters:}
\begin{itemize}[noitemsep]
\item \code{String spacename}\\
The name of the space as a string or symbol.

\item \code{Object key}\\
The key for the operation as a Python type.

\item \code{Map<String, Map<Object, Object>> mapattributes}\\
A hash specifying map attributes to modify and their respective key/values.

\end{itemize}

\paragraph{Returns:}
This function returns an object indicating the success or failure of the
operation.  Valid values to be returned are:

\begin{itemize}[noitemsep]
\item \code{True} if the operation succeeded
\item \code{False} if any provided predicates failed.
\item \code{null} if the operation requires an existing value and none exists
\end{itemize}

On error, this function will raise a \code{HyperDexClientException} describing
the error.


\pagebreak
\subsubsection{\code{async\_map\_atomic\_mod}}
\label{api:java:async_map_atomic_mod}
\index{async\_map\_atomic\_mod!Java API}
Store the value of the key-value pair modulo the specified number for each map.
This operation requires a pre-existing object in order to complete successfully.
If no object exists, the operation will fail with \code{NOTFOUND}.



\paragraph{Definition:}
\begin{javacode}
public Deferred async_map_atomic_mod(
        String spacename,
        Object key,
        Map<String, Map<Object, Object>> mapattributes) throws HyperDexClientException
\end{javacode}

\paragraph{Parameters:}
\begin{itemize}[noitemsep]
\item \code{String spacename}\\
The name of the space as a string or symbol.

\item \code{Object key}\\
The key for the operation as a Python type.

\item \code{Map<String, Map<Object, Object>> mapattributes}\\
A hash specifying map attributes to modify and their respective key/values.

\end{itemize}

\paragraph{Returns:}
A Deferred object with a \code{wait} method that returns True if the operation
succeeded or False if any provided predicates failed.  Raises an exception on
error.


\paragraph{See also:}  This is the asynchronous form of \code{map\_atomic\_mod}.

%%%%%%%%%%%%%%%%%%%% cond_map_atomic_mod %%%%%%%%%%%%%%%%%%%%
\pagebreak
\subsubsection{\code{cond\_map\_atomic\_mod}}
\label{api:java:cond_map_atomic_mod}
\index{cond\_map\_atomic\_mod!Java API}
Conditionally store the value of the key-value pair modulo the specified number
for each map.

%%% Generated below here
\paragraph{Behavior:}
\begin{itemize}[noitemsep]
This operation requires a pre-existing object in order to complete successfully.
If no object exists, the operation will fail with \code{NOTFOUND}.

This operation will succeed if and only if the predicates specified by
\code{checks} hold on the pre-existing object.  If any of the predicates are not
true for the existing object, then the operation will have no effect and fail
with \code{CMPFAIL}.

All checks are atomic with the write.  HyperDex guarantees that no other
operation will come between validating the checks, and writing the new version
of the object.

\item This operation mutates the value of a key-value pair in a map.  This call
    is similar to the equivalent call without the \code{map\_} prefix, but
    operates on the value of a pair in a map, instead of on an attribute's
    value.  If there is no pair with the specified map key, a new pair will be
    created and initialized to its default value.  If this is undesirable, it
    may be avoided by using a conditional operation that requires that the map
    contain the key in question.

\end{itemize}


\paragraph{Definition:}
\begin{javacode}
public Boolean cond_map_atomic_mod(
        String spacename,
        Object key,
        Map<String, Object> predicates,
        Map<String, Map<Object, Object>> mapattributes) throws HyperDexClientException
\end{javacode}

\paragraph{Parameters:}
\begin{itemize}[noitemsep]
\item \code{String spacename}\\
The name of the space as a string or symbol.

\item \code{Object key}\\
The key for the operation as a Python type.

\item \code{Map<String, Object> predicates}\\
A hash of predicates to check against.

\item \code{Map<String, Map<Object, Object>> mapattributes}\\
A hash specifying map attributes to modify and their respective key/values.

\end{itemize}

\paragraph{Returns:}
This function returns an object indicating the success or failure of the
operation.  Valid values to be returned are:

\begin{itemize}[noitemsep]
\item \code{True} if the operation succeeded
\item \code{False} if any provided predicates failed.
\item \code{null} if the operation requires an existing value and none exists
\end{itemize}

On error, this function will raise a \code{HyperDexClientException} describing
the error.


\pagebreak
\subsubsection{\code{async\_cond\_map\_atomic\_mod}}
\label{api:java:async_cond_map_atomic_mod}
\index{async\_cond\_map\_atomic\_mod!Java API}
Conditionally store the value of the key-value pair modulo the specified number
for each map.

%%% Generated below here
\paragraph{Behavior:}
\begin{itemize}[noitemsep]
This operation requires a pre-existing object in order to complete successfully.
If no object exists, the operation will fail with \code{NOTFOUND}.

This operation will succeed if and only if the predicates specified by
\code{checks} hold on the pre-existing object.  If any of the predicates are not
true for the existing object, then the operation will have no effect and fail
with \code{CMPFAIL}.

All checks are atomic with the write.  HyperDex guarantees that no other
operation will come between validating the checks, and writing the new version
of the object.

\item This operation mutates the value of a key-value pair in a map.  This call
    is similar to the equivalent call without the \code{map\_} prefix, but
    operates on the value of a pair in a map, instead of on an attribute's
    value.  If there is no pair with the specified map key, a new pair will be
    created and initialized to its default value.  If this is undesirable, it
    may be avoided by using a conditional operation that requires that the map
    contain the key in question.

\end{itemize}


\paragraph{Definition:}
\begin{javacode}
public Deferred async_cond_map_atomic_mod(
        String spacename,
        Object key,
        Map<String, Object> predicates,
        Map<String, Map<Object, Object>> mapattributes) throws HyperDexClientException
\end{javacode}

\paragraph{Parameters:}
\begin{itemize}[noitemsep]
\item \code{String spacename}\\
The name of the space as a string or symbol.

\item \code{Object key}\\
The key for the operation as a Python type.

\item \code{Map<String, Object> predicates}\\
A hash of predicates to check against.

\item \code{Map<String, Map<Object, Object>> mapattributes}\\
A hash specifying map attributes to modify and their respective key/values.

\end{itemize}

\paragraph{Returns:}
A Deferred object with a \code{wait} method that returns True if the operation
succeeded or False if any provided predicates failed.  Raises an exception on
error.


\paragraph{See also:}  This is the asynchronous form of \code{cond\_map\_atomic\_mod}.

%%%%%%%%%%%%%%%%%%%% map_atomic_and %%%%%%%%%%%%%%%%%%%%
\pagebreak
\subsubsection{\code{map\_atomic\_and}}
\label{api:java:map_atomic_and}
\index{map\_atomic\_and!Java API}
Store the bitwise AND of the value of the key-value pair and the specified
number for each map.

%%% Generated below here
\paragraph{Behavior:}
\begin{itemize}[noitemsep]
This operation requires a pre-existing object in order to complete successfully.
If no object exists, the operation will fail with \code{NOTFOUND}.

\item This operation mutates the value of a key-value pair in a map.  This call
    is similar to the equivalent call without the \code{map\_} prefix, but
    operates on the value of a pair in a map, instead of on an attribute's
    value.  If there is no pair with the specified map key, a new pair will be
    created and initialized to its default value.  If this is undesirable, it
    may be avoided by using a conditional operation that requires that the map
    contain the key in question.

\end{itemize}


\paragraph{Definition:}
\begin{javacode}
public Boolean map_atomic_and(
        String spacename,
        Object key,
        Map<String, Map<Object, Object>> mapattributes) throws HyperDexClientException
\end{javacode}

\paragraph{Parameters:}
\begin{itemize}[noitemsep]
\item \code{String spacename}\\
The name of the space as a string or symbol.

\item \code{Object key}\\
The key for the operation as a Python type.

\item \code{Map<String, Map<Object, Object>> mapattributes}\\
A hash specifying map attributes to modify and their respective key/values.

\end{itemize}

\paragraph{Returns:}
This function returns an object indicating the success or failure of the
operation.  Valid values to be returned are:

\begin{itemize}[noitemsep]
\item \code{True} if the operation succeeded
\item \code{False} if any provided predicates failed.
\item \code{null} if the operation requires an existing value and none exists
\end{itemize}

On error, this function will raise a \code{HyperDexClientException} describing
the error.


\pagebreak
\subsubsection{\code{async\_map\_atomic\_and}}
\label{api:java:async_map_atomic_and}
\index{async\_map\_atomic\_and!Java API}
Store the bitwise AND of the value of the key-value pair and the specified
number for each map.

%%% Generated below here
\paragraph{Behavior:}
\begin{itemize}[noitemsep]
This operation requires a pre-existing object in order to complete successfully.
If no object exists, the operation will fail with \code{NOTFOUND}.

\item This operation mutates the value of a key-value pair in a map.  This call
    is similar to the equivalent call without the \code{map\_} prefix, but
    operates on the value of a pair in a map, instead of on an attribute's
    value.  If there is no pair with the specified map key, a new pair will be
    created and initialized to its default value.  If this is undesirable, it
    may be avoided by using a conditional operation that requires that the map
    contain the key in question.

\end{itemize}


\paragraph{Definition:}
\begin{javacode}
public Deferred async_map_atomic_and(
        String spacename,
        Object key,
        Map<String, Map<Object, Object>> mapattributes) throws HyperDexClientException
\end{javacode}

\paragraph{Parameters:}
\begin{itemize}[noitemsep]
\item \code{String spacename}\\
The name of the space as a string or symbol.

\item \code{Object key}\\
The key for the operation as a Python type.

\item \code{Map<String, Map<Object, Object>> mapattributes}\\
A hash specifying map attributes to modify and their respective key/values.

\end{itemize}

\paragraph{Returns:}
A Deferred object with a \code{wait} method that returns True if the operation
succeeded or False if any provided predicates failed.  Raises an exception on
error.


\paragraph{See also:}  This is the asynchronous form of \code{map\_atomic\_and}.

%%%%%%%%%%%%%%%%%%%% cond_map_atomic_and %%%%%%%%%%%%%%%%%%%%
\pagebreak
\subsubsection{\code{cond\_map\_atomic\_and}}
\label{api:java:cond_map_atomic_and}
\index{cond\_map\_atomic\_and!Java API}
Store the bitwise AND of the value of the key-value pair and the specified
number for each map attribute if and only if the \code{checks} hold on the
object.
This operation requires a pre-existing object in order to complete successfully.
If no object exists, the operation will fail with \code{NOTFOUND}.


This operation will succeed if and only if the predicates specified by
\code{checks} hold on the pre-existing object.  If any of the predicates are not
true for the existing object, then the operation will have no effect and fail
with \code{CMPFAIL}.

All checks are atomic with the write.  HyperDex guarantees that no other
operation will come between validating the checks, and writing the new version
of the object.



\paragraph{Definition:}
\begin{javacode}
public Boolean cond_map_atomic_and(
        String spacename,
        Object key,
        Map<String, Object> predicates,
        Map<String, Map<Object, Object>> mapattributes) throws HyperDexClientException
\end{javacode}

\paragraph{Parameters:}
\begin{itemize}[noitemsep]
\item \code{String spacename}\\
The name of the space as a string or symbol.

\item \code{Object key}\\
The key for the operation as a Python type.

\item \code{Map<String, Object> predicates}\\
A hash of predicates to check against.

\item \code{Map<String, Map<Object, Object>> mapattributes}\\
A hash specifying map attributes to modify and their respective key/values.

\end{itemize}

\paragraph{Returns:}
This function returns an object indicating the success or failure of the
operation.  Valid values to be returned are:

\begin{itemize}[noitemsep]
\item \code{True} if the operation succeeded
\item \code{False} if any provided predicates failed.
\item \code{null} if the operation requires an existing value and none exists
\end{itemize}

On error, this function will raise a \code{HyperDexClientException} describing
the error.


\pagebreak
\subsubsection{\code{async\_cond\_map\_atomic\_and}}
\label{api:java:async_cond_map_atomic_and}
\index{async\_cond\_map\_atomic\_and!Java API}
Store the bitwise AND of the value of the key-value pair and the specified
number for each map attribute if and only if the \code{checks} hold on the
object.
This operation requires a pre-existing object in order to complete successfully.
If no object exists, the operation will fail with \code{NOTFOUND}.


This operation will succeed if and only if the predicates specified by
\code{checks} hold on the pre-existing object.  If any of the predicates are not
true for the existing object, then the operation will have no effect and fail
with \code{CMPFAIL}.

All checks are atomic with the write.  HyperDex guarantees that no other
operation will come between validating the checks, and writing the new version
of the object.



\paragraph{Definition:}
\begin{javacode}
public Deferred async_cond_map_atomic_and(
        String spacename,
        Object key,
        Map<String, Object> predicates,
        Map<String, Map<Object, Object>> mapattributes) throws HyperDexClientException
\end{javacode}

\paragraph{Parameters:}
\begin{itemize}[noitemsep]
\item \code{String spacename}\\
The name of the space as a string or symbol.

\item \code{Object key}\\
The key for the operation as a Python type.

\item \code{Map<String, Object> predicates}\\
A hash of predicates to check against.

\item \code{Map<String, Map<Object, Object>> mapattributes}\\
A hash specifying map attributes to modify and their respective key/values.

\end{itemize}

\paragraph{Returns:}
A Deferred object with a \code{wait} method that returns True if the operation
succeeded or False if any provided predicates failed.  Raises an exception on
error.


\paragraph{See also:}  This is the asynchronous form of \code{cond\_map\_atomic\_and}.

%%%%%%%%%%%%%%%%%%%% map_atomic_or %%%%%%%%%%%%%%%%%%%%
\pagebreak
\subsubsection{\code{map\_atomic\_or}}
\label{api:java:map_atomic_or}
\index{map\_atomic\_or!Java API}
Store the bitwise OR of the value of the key-value pair and the specified number
for each map.

%%% Generated below here
\paragraph{Behavior:}
\begin{itemize}[noitemsep]
This operation requires a pre-existing object in order to complete successfully.
If no object exists, the operation will fail with \code{NOTFOUND}.

\item This operation mutates the value of a key-value pair in a map.  This call
    is similar to the equivalent call without the \code{map\_} prefix, but
    operates on the value of a pair in a map, instead of on an attribute's
    value.  If there is no pair with the specified map key, a new pair will be
    created and initialized to its default value.  If this is undesirable, it
    may be avoided by using a conditional operation that requires that the map
    contain the key in question.

\end{itemize}


\paragraph{Definition:}
\begin{javacode}
public Boolean map_atomic_or(
        String spacename,
        Object key,
        Map<String, Map<Object, Object>> mapattributes) throws HyperDexClientException
\end{javacode}

\paragraph{Parameters:}
\begin{itemize}[noitemsep]
\item \code{String spacename}\\
The name of the space as a string or symbol.

\item \code{Object key}\\
The key for the operation as a Python type.

\item \code{Map<String, Map<Object, Object>> mapattributes}\\
A hash specifying map attributes to modify and their respective key/values.

\end{itemize}

\paragraph{Returns:}
This function returns an object indicating the success or failure of the
operation.  Valid values to be returned are:

\begin{itemize}[noitemsep]
\item \code{True} if the operation succeeded
\item \code{False} if any provided predicates failed.
\item \code{null} if the operation requires an existing value and none exists
\end{itemize}

On error, this function will raise a \code{HyperDexClientException} describing
the error.


\pagebreak
\subsubsection{\code{async\_map\_atomic\_or}}
\label{api:java:async_map_atomic_or}
\index{async\_map\_atomic\_or!Java API}
Store the bitwise OR of the value of the key-value pair and the specified number
for each map.

%%% Generated below here
\paragraph{Behavior:}
\begin{itemize}[noitemsep]
This operation requires a pre-existing object in order to complete successfully.
If no object exists, the operation will fail with \code{NOTFOUND}.

\item This operation mutates the value of a key-value pair in a map.  This call
    is similar to the equivalent call without the \code{map\_} prefix, but
    operates on the value of a pair in a map, instead of on an attribute's
    value.  If there is no pair with the specified map key, a new pair will be
    created and initialized to its default value.  If this is undesirable, it
    may be avoided by using a conditional operation that requires that the map
    contain the key in question.

\end{itemize}


\paragraph{Definition:}
\begin{javacode}
public Deferred async_map_atomic_or(
        String spacename,
        Object key,
        Map<String, Map<Object, Object>> mapattributes) throws HyperDexClientException
\end{javacode}

\paragraph{Parameters:}
\begin{itemize}[noitemsep]
\item \code{String spacename}\\
The name of the space as a string or symbol.

\item \code{Object key}\\
The key for the operation as a Python type.

\item \code{Map<String, Map<Object, Object>> mapattributes}\\
A hash specifying map attributes to modify and their respective key/values.

\end{itemize}

\paragraph{Returns:}
A Deferred object with a \code{wait} method that returns True if the operation
succeeded or False if any provided predicates failed.  Raises an exception on
error.


\paragraph{See also:}  This is the asynchronous form of \code{map\_atomic\_or}.

%%%%%%%%%%%%%%%%%%%% cond_map_atomic_or %%%%%%%%%%%%%%%%%%%%
\pagebreak
\subsubsection{\code{cond\_map\_atomic\_or}}
\label{api:java:cond_map_atomic_or}
\index{cond\_map\_atomic\_or!Java API}
Conditionally store the bitwise OR of the value of the key-value pair and the
specified number for each map.

%%% Generated below here
\paragraph{Behavior:}
\begin{itemize}[noitemsep]
This operation requires a pre-existing object in order to complete successfully.
If no object exists, the operation will fail with \code{NOTFOUND}.

This operation will succeed if and only if the predicates specified by
\code{checks} hold on the pre-existing object.  If any of the predicates are not
true for the existing object, then the operation will have no effect and fail
with \code{CMPFAIL}.

All checks are atomic with the write.  HyperDex guarantees that no other
operation will come between validating the checks, and writing the new version
of the object.

\item This operation mutates the value of a key-value pair in a map.  This call
    is similar to the equivalent call without the \code{map\_} prefix, but
    operates on the value of a pair in a map, instead of on an attribute's
    value.  If there is no pair with the specified map key, a new pair will be
    created and initialized to its default value.  If this is undesirable, it
    may be avoided by using a conditional operation that requires that the map
    contain the key in question.

\end{itemize}


\paragraph{Definition:}
\begin{javacode}
public Boolean cond_map_atomic_or(
        String spacename,
        Object key,
        Map<String, Object> predicates,
        Map<String, Map<Object, Object>> mapattributes) throws HyperDexClientException
\end{javacode}

\paragraph{Parameters:}
\begin{itemize}[noitemsep]
\item \code{String spacename}\\
The name of the space as a string or symbol.

\item \code{Object key}\\
The key for the operation as a Python type.

\item \code{Map<String, Object> predicates}\\
A hash of predicates to check against.

\item \code{Map<String, Map<Object, Object>> mapattributes}\\
A hash specifying map attributes to modify and their respective key/values.

\end{itemize}

\paragraph{Returns:}
This function returns an object indicating the success or failure of the
operation.  Valid values to be returned are:

\begin{itemize}[noitemsep]
\item \code{True} if the operation succeeded
\item \code{False} if any provided predicates failed.
\item \code{null} if the operation requires an existing value and none exists
\end{itemize}

On error, this function will raise a \code{HyperDexClientException} describing
the error.


\pagebreak
\subsubsection{\code{async\_cond\_map\_atomic\_or}}
\label{api:java:async_cond_map_atomic_or}
\index{async\_cond\_map\_atomic\_or!Java API}
Conditionally store the bitwise OR of the value of the key-value pair and the
specified number for each map.

%%% Generated below here
\paragraph{Behavior:}
\begin{itemize}[noitemsep]
This operation requires a pre-existing object in order to complete successfully.
If no object exists, the operation will fail with \code{NOTFOUND}.

This operation will succeed if and only if the predicates specified by
\code{checks} hold on the pre-existing object.  If any of the predicates are not
true for the existing object, then the operation will have no effect and fail
with \code{CMPFAIL}.

All checks are atomic with the write.  HyperDex guarantees that no other
operation will come between validating the checks, and writing the new version
of the object.

\item This operation mutates the value of a key-value pair in a map.  This call
    is similar to the equivalent call without the \code{map\_} prefix, but
    operates on the value of a pair in a map, instead of on an attribute's
    value.  If there is no pair with the specified map key, a new pair will be
    created and initialized to its default value.  If this is undesirable, it
    may be avoided by using a conditional operation that requires that the map
    contain the key in question.

\end{itemize}


\paragraph{Definition:}
\begin{javacode}
public Deferred async_cond_map_atomic_or(
        String spacename,
        Object key,
        Map<String, Object> predicates,
        Map<String, Map<Object, Object>> mapattributes) throws HyperDexClientException
\end{javacode}

\paragraph{Parameters:}
\begin{itemize}[noitemsep]
\item \code{String spacename}\\
The name of the space as a string or symbol.

\item \code{Object key}\\
The key for the operation as a Python type.

\item \code{Map<String, Object> predicates}\\
A hash of predicates to check against.

\item \code{Map<String, Map<Object, Object>> mapattributes}\\
A hash specifying map attributes to modify and their respective key/values.

\end{itemize}

\paragraph{Returns:}
A Deferred object with a \code{wait} method that returns True if the operation
succeeded or False if any provided predicates failed.  Raises an exception on
error.


\paragraph{See also:}  This is the asynchronous form of \code{cond\_map\_atomic\_or}.

%%%%%%%%%%%%%%%%%%%% map_atomic_xor %%%%%%%%%%%%%%%%%%%%
\pagebreak
\subsubsection{\code{map\_atomic\_xor}}
\label{api:java:map_atomic_xor}
\index{map\_atomic\_xor!Java API}
Store the bitwise XOR of the value of the key-value pair and the specified
number for each map.

%%% Generated below here
\paragraph{Behavior:}
\begin{itemize}[noitemsep]
This operation requires a pre-existing object in order to complete successfully.
If no object exists, the operation will fail with \code{NOTFOUND}.

\item This operation mutates the value of a key-value pair in a map.  This call
    is similar to the equivalent call without the \code{map\_} prefix, but
    operates on the value of a pair in a map, instead of on an attribute's
    value.  If there is no pair with the specified map key, a new pair will be
    created and initialized to its default value.  If this is undesirable, it
    may be avoided by using a conditional operation that requires that the map
    contain the key in question.

\end{itemize}


\paragraph{Definition:}
\begin{javacode}
public Boolean map_atomic_xor(
        String spacename,
        Object key,
        Map<String, Map<Object, Object>> mapattributes) throws HyperDexClientException
\end{javacode}

\paragraph{Parameters:}
\begin{itemize}[noitemsep]
\item \code{String spacename}\\
The name of the space as a string or symbol.

\item \code{Object key}\\
The key for the operation as a Python type.

\item \code{Map<String, Map<Object, Object>> mapattributes}\\
A hash specifying map attributes to modify and their respective key/values.

\end{itemize}

\paragraph{Returns:}
This function returns an object indicating the success or failure of the
operation.  Valid values to be returned are:

\begin{itemize}[noitemsep]
\item \code{True} if the operation succeeded
\item \code{False} if any provided predicates failed.
\item \code{null} if the operation requires an existing value and none exists
\end{itemize}

On error, this function will raise a \code{HyperDexClientException} describing
the error.


\pagebreak
\subsubsection{\code{async\_map\_atomic\_xor}}
\label{api:java:async_map_atomic_xor}
\index{async\_map\_atomic\_xor!Java API}
Store the bitwise XOR of the value of the key-value pair and the specified
number for each map.

%%% Generated below here
\paragraph{Behavior:}
\begin{itemize}[noitemsep]
This operation requires a pre-existing object in order to complete successfully.
If no object exists, the operation will fail with \code{NOTFOUND}.

\item This operation mutates the value of a key-value pair in a map.  This call
    is similar to the equivalent call without the \code{map\_} prefix, but
    operates on the value of a pair in a map, instead of on an attribute's
    value.  If there is no pair with the specified map key, a new pair will be
    created and initialized to its default value.  If this is undesirable, it
    may be avoided by using a conditional operation that requires that the map
    contain the key in question.

\end{itemize}


\paragraph{Definition:}
\begin{javacode}
public Deferred async_map_atomic_xor(
        String spacename,
        Object key,
        Map<String, Map<Object, Object>> mapattributes) throws HyperDexClientException
\end{javacode}

\paragraph{Parameters:}
\begin{itemize}[noitemsep]
\item \code{String spacename}\\
The name of the space as a string or symbol.

\item \code{Object key}\\
The key for the operation as a Python type.

\item \code{Map<String, Map<Object, Object>> mapattributes}\\
A hash specifying map attributes to modify and their respective key/values.

\end{itemize}

\paragraph{Returns:}
A Deferred object with a \code{wait} method that returns True if the operation
succeeded or False if any provided predicates failed.  Raises an exception on
error.


\paragraph{See also:}  This is the asynchronous form of \code{map\_atomic\_xor}.

%%%%%%%%%%%%%%%%%%%% cond_map_atomic_xor %%%%%%%%%%%%%%%%%%%%
\pagebreak
\subsubsection{\code{cond\_map\_atomic\_xor}}
\label{api:java:cond_map_atomic_xor}
\index{cond\_map\_atomic\_xor!Java API}
Store the bitwise XOR of the value of the key-value pair and the specified
number for each map attribute if and only if the \code{checks} hold on the
object.
This operation requires a pre-existing object in order to complete successfully.
If no object exists, the operation will fail with \code{NOTFOUND}.


This operation will succeed if and only if the predicates specified by
\code{checks} hold on the pre-existing object.  If any of the predicates are not
true for the existing object, then the operation will have no effect and fail
with \code{CMPFAIL}.

All checks are atomic with the write.  HyperDex guarantees that no other
operation will come between validating the checks, and writing the new version
of the object.



\paragraph{Definition:}
\begin{javacode}
public Boolean cond_map_atomic_xor(
        String spacename,
        Object key,
        Map<String, Object> predicates,
        Map<String, Map<Object, Object>> mapattributes) throws HyperDexClientException
\end{javacode}

\paragraph{Parameters:}
\begin{itemize}[noitemsep]
\item \code{String spacename}\\
The name of the space as a string or symbol.

\item \code{Object key}\\
The key for the operation as a Python type.

\item \code{Map<String, Object> predicates}\\
A hash of predicates to check against.

\item \code{Map<String, Map<Object, Object>> mapattributes}\\
A hash specifying map attributes to modify and their respective key/values.

\end{itemize}

\paragraph{Returns:}
This function returns an object indicating the success or failure of the
operation.  Valid values to be returned are:

\begin{itemize}[noitemsep]
\item \code{True} if the operation succeeded
\item \code{False} if any provided predicates failed.
\item \code{null} if the operation requires an existing value and none exists
\end{itemize}

On error, this function will raise a \code{HyperDexClientException} describing
the error.


\pagebreak
\subsubsection{\code{async\_cond\_map\_atomic\_xor}}
\label{api:java:async_cond_map_atomic_xor}
\index{async\_cond\_map\_atomic\_xor!Java API}
Store the bitwise XOR of the value of the key-value pair and the specified
number for each map attribute if and only if the \code{checks} hold on the
object.
This operation requires a pre-existing object in order to complete successfully.
If no object exists, the operation will fail with \code{NOTFOUND}.


This operation will succeed if and only if the predicates specified by
\code{checks} hold on the pre-existing object.  If any of the predicates are not
true for the existing object, then the operation will have no effect and fail
with \code{CMPFAIL}.

All checks are atomic with the write.  HyperDex guarantees that no other
operation will come between validating the checks, and writing the new version
of the object.



\paragraph{Definition:}
\begin{javacode}
public Deferred async_cond_map_atomic_xor(
        String spacename,
        Object key,
        Map<String, Object> predicates,
        Map<String, Map<Object, Object>> mapattributes) throws HyperDexClientException
\end{javacode}

\paragraph{Parameters:}
\begin{itemize}[noitemsep]
\item \code{String spacename}\\
The name of the space as a string or symbol.

\item \code{Object key}\\
The key for the operation as a Python type.

\item \code{Map<String, Object> predicates}\\
A hash of predicates to check against.

\item \code{Map<String, Map<Object, Object>> mapattributes}\\
A hash specifying map attributes to modify and their respective key/values.

\end{itemize}

\paragraph{Returns:}
A Deferred object with a \code{wait} method that returns True if the operation
succeeded or False if any provided predicates failed.  Raises an exception on
error.


\paragraph{See also:}  This is the asynchronous form of \code{cond\_map\_atomic\_xor}.

%%%%%%%%%%%%%%%%%%%% map_string_prepend %%%%%%%%%%%%%%%%%%%%
\pagebreak
\subsubsection{\code{map\_string\_prepend}}
\label{api:java:map_string_prepend}
\index{map\_string\_prepend!Java API}
Prepend the specified string to the value of the key-value pair for each map.

%%% Generated below here
\paragraph{Behavior:}
\begin{itemize}[noitemsep]
This operation requires a pre-existing object in order to complete successfully.
If no object exists, the operation will fail with \code{NOTFOUND}.

\item This operation mutates the value of a key-value pair in a map.  This call
    is similar to the equivalent call without the \code{map\_} prefix, but
    operates on the value of a pair in a map, instead of on an attribute's
    value.  If there is no pair with the specified map key, a new pair will be
    created and initialized to its default value.  If this is undesirable, it
    may be avoided by using a conditional operation that requires that the map
    contain the key in question.

\end{itemize}


\paragraph{Definition:}
\begin{javacode}
public Boolean map_string_prepend(
        String spacename,
        Object key,
        Map<String, Map<Object, Object>> mapattributes) throws HyperDexClientException
\end{javacode}

\paragraph{Parameters:}
\begin{itemize}[noitemsep]
\item \code{String spacename}\\
The name of the space as a string or symbol.

\item \code{Object key}\\
The key for the operation as a Python type.

\item \code{Map<String, Map<Object, Object>> mapattributes}\\
A hash specifying map attributes to modify and their respective key/values.

\end{itemize}

\paragraph{Returns:}
This function returns an object indicating the success or failure of the
operation.  Valid values to be returned are:

\begin{itemize}[noitemsep]
\item \code{True} if the operation succeeded
\item \code{False} if any provided predicates failed.
\item \code{null} if the operation requires an existing value and none exists
\end{itemize}

On error, this function will raise a \code{HyperDexClientException} describing
the error.


\pagebreak
\subsubsection{\code{async\_map\_string\_prepend}}
\label{api:java:async_map_string_prepend}
\index{async\_map\_string\_prepend!Java API}
Prepend the specified string to the value of the key-value pair for each map.

%%% Generated below here
\paragraph{Behavior:}
\begin{itemize}[noitemsep]
This operation requires a pre-existing object in order to complete successfully.
If no object exists, the operation will fail with \code{NOTFOUND}.

\item This operation mutates the value of a key-value pair in a map.  This call
    is similar to the equivalent call without the \code{map\_} prefix, but
    operates on the value of a pair in a map, instead of on an attribute's
    value.  If there is no pair with the specified map key, a new pair will be
    created and initialized to its default value.  If this is undesirable, it
    may be avoided by using a conditional operation that requires that the map
    contain the key in question.

\end{itemize}


\paragraph{Definition:}
\begin{javacode}
public Deferred async_map_string_prepend(
        String spacename,
        Object key,
        Map<String, Map<Object, Object>> mapattributes) throws HyperDexClientException
\end{javacode}

\paragraph{Parameters:}
\begin{itemize}[noitemsep]
\item \code{String spacename}\\
The name of the space as a string or symbol.

\item \code{Object key}\\
The key for the operation as a Python type.

\item \code{Map<String, Map<Object, Object>> mapattributes}\\
A hash specifying map attributes to modify and their respective key/values.

\end{itemize}

\paragraph{Returns:}
A Deferred object with a \code{wait} method that returns True if the operation
succeeded or False if any provided predicates failed.  Raises an exception on
error.


\paragraph{See also:}  This is the asynchronous form of \code{map\_string\_prepend}.

%%%%%%%%%%%%%%%%%%%% cond_map_string_prepend %%%%%%%%%%%%%%%%%%%%
\pagebreak
\subsubsection{\code{cond\_map\_string\_prepend}}
\label{api:java:cond_map_string_prepend}
\index{cond\_map\_string\_prepend!Java API}
Conditionally prepend the specified string to the value of the key-value pair
for each map.

%%% Generated below here
\paragraph{Behavior:}
\begin{itemize}[noitemsep]
This operation requires a pre-existing object in order to complete successfully.
If no object exists, the operation will fail with \code{NOTFOUND}.

This operation will succeed if and only if the predicates specified by
\code{checks} hold on the pre-existing object.  If any of the predicates are not
true for the existing object, then the operation will have no effect and fail
with \code{CMPFAIL}.

All checks are atomic with the write.  HyperDex guarantees that no other
operation will come between validating the checks, and writing the new version
of the object.

\item This operation mutates the value of a key-value pair in a map.  This call
    is similar to the equivalent call without the \code{map\_} prefix, but
    operates on the value of a pair in a map, instead of on an attribute's
    value.  If there is no pair with the specified map key, a new pair will be
    created and initialized to its default value.  If this is undesirable, it
    may be avoided by using a conditional operation that requires that the map
    contain the key in question.

\end{itemize}


\paragraph{Definition:}
\begin{javacode}
public Boolean cond_map_string_prepend(
        String spacename,
        Object key,
        Map<String, Object> predicates,
        Map<String, Map<Object, Object>> mapattributes) throws HyperDexClientException
\end{javacode}

\paragraph{Parameters:}
\begin{itemize}[noitemsep]
\item \code{String spacename}\\
The name of the space as a string or symbol.

\item \code{Object key}\\
The key for the operation as a Python type.

\item \code{Map<String, Object> predicates}\\
A hash of predicates to check against.

\item \code{Map<String, Map<Object, Object>> mapattributes}\\
A hash specifying map attributes to modify and their respective key/values.

\end{itemize}

\paragraph{Returns:}
This function returns an object indicating the success or failure of the
operation.  Valid values to be returned are:

\begin{itemize}[noitemsep]
\item \code{True} if the operation succeeded
\item \code{False} if any provided predicates failed.
\item \code{null} if the operation requires an existing value and none exists
\end{itemize}

On error, this function will raise a \code{HyperDexClientException} describing
the error.


\pagebreak
\subsubsection{\code{async\_cond\_map\_string\_prepend}}
\label{api:java:async_cond_map_string_prepend}
\index{async\_cond\_map\_string\_prepend!Java API}
Conditionally prepend the specified string to the value of the key-value pair
for each map.

%%% Generated below here
\paragraph{Behavior:}
\begin{itemize}[noitemsep]
This operation requires a pre-existing object in order to complete successfully.
If no object exists, the operation will fail with \code{NOTFOUND}.

This operation will succeed if and only if the predicates specified by
\code{checks} hold on the pre-existing object.  If any of the predicates are not
true for the existing object, then the operation will have no effect and fail
with \code{CMPFAIL}.

All checks are atomic with the write.  HyperDex guarantees that no other
operation will come between validating the checks, and writing the new version
of the object.

\item This operation mutates the value of a key-value pair in a map.  This call
    is similar to the equivalent call without the \code{map\_} prefix, but
    operates on the value of a pair in a map, instead of on an attribute's
    value.  If there is no pair with the specified map key, a new pair will be
    created and initialized to its default value.  If this is undesirable, it
    may be avoided by using a conditional operation that requires that the map
    contain the key in question.

\end{itemize}


\paragraph{Definition:}
\begin{javacode}
public Deferred async_cond_map_string_prepend(
        String spacename,
        Object key,
        Map<String, Object> predicates,
        Map<String, Map<Object, Object>> mapattributes) throws HyperDexClientException
\end{javacode}

\paragraph{Parameters:}
\begin{itemize}[noitemsep]
\item \code{String spacename}\\
The name of the space as a string or symbol.

\item \code{Object key}\\
The key for the operation as a Python type.

\item \code{Map<String, Object> predicates}\\
A hash of predicates to check against.

\item \code{Map<String, Map<Object, Object>> mapattributes}\\
A hash specifying map attributes to modify and their respective key/values.

\end{itemize}

\paragraph{Returns:}
A Deferred object with a \code{wait} method that returns True if the operation
succeeded or False if any provided predicates failed.  Raises an exception on
error.


\paragraph{See also:}  This is the asynchronous form of \code{cond\_map\_string\_prepend}.

%%%%%%%%%%%%%%%%%%%% map_string_append %%%%%%%%%%%%%%%%%%%%
\pagebreak
\subsubsection{\code{map\_string\_append}}
\label{api:java:map_string_append}
\index{map\_string\_append!Java API}
Append the specified string to the value of the key-value pair for each map.

%%% Generated below here
\paragraph{Behavior:}
\begin{itemize}[noitemsep]
This operation requires a pre-existing object in order to complete successfully.
If no object exists, the operation will fail with \code{NOTFOUND}.

\item This operation mutates the value of a key-value pair in a map.  This call
    is similar to the equivalent call without the \code{map\_} prefix, but
    operates on the value of a pair in a map, instead of on an attribute's
    value.  If there is no pair with the specified map key, a new pair will be
    created and initialized to its default value.  If this is undesirable, it
    may be avoided by using a conditional operation that requires that the map
    contain the key in question.

\end{itemize}


\paragraph{Definition:}
\begin{javacode}
public Boolean map_string_append(
        String spacename,
        Object key,
        Map<String, Map<Object, Object>> mapattributes) throws HyperDexClientException
\end{javacode}

\paragraph{Parameters:}
\begin{itemize}[noitemsep]
\item \code{String spacename}\\
The name of the space as a string or symbol.

\item \code{Object key}\\
The key for the operation as a Python type.

\item \code{Map<String, Map<Object, Object>> mapattributes}\\
A hash specifying map attributes to modify and their respective key/values.

\end{itemize}

\paragraph{Returns:}
This function returns an object indicating the success or failure of the
operation.  Valid values to be returned are:

\begin{itemize}[noitemsep]
\item \code{True} if the operation succeeded
\item \code{False} if any provided predicates failed.
\item \code{null} if the operation requires an existing value and none exists
\end{itemize}

On error, this function will raise a \code{HyperDexClientException} describing
the error.


\pagebreak
\subsubsection{\code{async\_map\_string\_append}}
\label{api:java:async_map_string_append}
\index{async\_map\_string\_append!Java API}
Append the specified string to the value of the key-value pair for each map.

%%% Generated below here
\paragraph{Behavior:}
\begin{itemize}[noitemsep]
This operation requires a pre-existing object in order to complete successfully.
If no object exists, the operation will fail with \code{NOTFOUND}.

\item This operation mutates the value of a key-value pair in a map.  This call
    is similar to the equivalent call without the \code{map\_} prefix, but
    operates on the value of a pair in a map, instead of on an attribute's
    value.  If there is no pair with the specified map key, a new pair will be
    created and initialized to its default value.  If this is undesirable, it
    may be avoided by using a conditional operation that requires that the map
    contain the key in question.

\end{itemize}


\paragraph{Definition:}
\begin{javacode}
public Deferred async_map_string_append(
        String spacename,
        Object key,
        Map<String, Map<Object, Object>> mapattributes) throws HyperDexClientException
\end{javacode}

\paragraph{Parameters:}
\begin{itemize}[noitemsep]
\item \code{String spacename}\\
The name of the space as a string or symbol.

\item \code{Object key}\\
The key for the operation as a Python type.

\item \code{Map<String, Map<Object, Object>> mapattributes}\\
A hash specifying map attributes to modify and their respective key/values.

\end{itemize}

\paragraph{Returns:}
A Deferred object with a \code{wait} method that returns True if the operation
succeeded or False if any provided predicates failed.  Raises an exception on
error.


\paragraph{See also:}  This is the asynchronous form of \code{map\_string\_append}.

%%%%%%%%%%%%%%%%%%%% cond_map_string_append %%%%%%%%%%%%%%%%%%%%
\pagebreak
\subsubsection{\code{cond\_map\_string\_append}}
\label{api:java:cond_map_string_append}
\index{cond\_map\_string\_append!Java API}
Conditionally append the specified string to the value of the key-value pair for
each map.

%%% Generated below here
\paragraph{Behavior:}
\begin{itemize}[noitemsep]
This operation requires a pre-existing object in order to complete successfully.
If no object exists, the operation will fail with \code{NOTFOUND}.

This operation will succeed if and only if the predicates specified by
\code{checks} hold on the pre-existing object.  If any of the predicates are not
true for the existing object, then the operation will have no effect and fail
with \code{CMPFAIL}.

All checks are atomic with the write.  HyperDex guarantees that no other
operation will come between validating the checks, and writing the new version
of the object.

\item This operation mutates the value of a key-value pair in a map.  This call
    is similar to the equivalent call without the \code{map\_} prefix, but
    operates on the value of a pair in a map, instead of on an attribute's
    value.  If there is no pair with the specified map key, a new pair will be
    created and initialized to its default value.  If this is undesirable, it
    may be avoided by using a conditional operation that requires that the map
    contain the key in question.

\end{itemize}


\paragraph{Definition:}
\begin{javacode}
public Boolean cond_map_string_append(
        String spacename,
        Object key,
        Map<String, Object> predicates,
        Map<String, Map<Object, Object>> mapattributes) throws HyperDexClientException
\end{javacode}

\paragraph{Parameters:}
\begin{itemize}[noitemsep]
\item \code{String spacename}\\
The name of the space as a string or symbol.

\item \code{Object key}\\
The key for the operation as a Python type.

\item \code{Map<String, Object> predicates}\\
A hash of predicates to check against.

\item \code{Map<String, Map<Object, Object>> mapattributes}\\
A hash specifying map attributes to modify and their respective key/values.

\end{itemize}

\paragraph{Returns:}
This function returns an object indicating the success or failure of the
operation.  Valid values to be returned are:

\begin{itemize}[noitemsep]
\item \code{True} if the operation succeeded
\item \code{False} if any provided predicates failed.
\item \code{null} if the operation requires an existing value and none exists
\end{itemize}

On error, this function will raise a \code{HyperDexClientException} describing
the error.


\pagebreak
\subsubsection{\code{async\_cond\_map\_string\_append}}
\label{api:java:async_cond_map_string_append}
\index{async\_cond\_map\_string\_append!Java API}
Conditionally append the specified string to the value of the key-value pair for
each map.

%%% Generated below here
\paragraph{Behavior:}
\begin{itemize}[noitemsep]
This operation requires a pre-existing object in order to complete successfully.
If no object exists, the operation will fail with \code{NOTFOUND}.

This operation will succeed if and only if the predicates specified by
\code{checks} hold on the pre-existing object.  If any of the predicates are not
true for the existing object, then the operation will have no effect and fail
with \code{CMPFAIL}.

All checks are atomic with the write.  HyperDex guarantees that no other
operation will come between validating the checks, and writing the new version
of the object.

\item This operation mutates the value of a key-value pair in a map.  This call
    is similar to the equivalent call without the \code{map\_} prefix, but
    operates on the value of a pair in a map, instead of on an attribute's
    value.  If there is no pair with the specified map key, a new pair will be
    created and initialized to its default value.  If this is undesirable, it
    may be avoided by using a conditional operation that requires that the map
    contain the key in question.

\end{itemize}


\paragraph{Definition:}
\begin{javacode}
public Deferred async_cond_map_string_append(
        String spacename,
        Object key,
        Map<String, Object> predicates,
        Map<String, Map<Object, Object>> mapattributes) throws HyperDexClientException
\end{javacode}

\paragraph{Parameters:}
\begin{itemize}[noitemsep]
\item \code{String spacename}\\
The name of the space as a string or symbol.

\item \code{Object key}\\
The key for the operation as a Python type.

\item \code{Map<String, Object> predicates}\\
A hash of predicates to check against.

\item \code{Map<String, Map<Object, Object>> mapattributes}\\
A hash specifying map attributes to modify and their respective key/values.

\end{itemize}

\paragraph{Returns:}
A Deferred object with a \code{wait} method that returns True if the operation
succeeded or False if any provided predicates failed.  Raises an exception on
error.


\paragraph{See also:}  This is the asynchronous form of \code{cond\_map\_string\_append}.

%%%%%%%%%%%%%%%%%%%% map_atomic_min %%%%%%%%%%%%%%%%%%%%
\pagebreak
\subsubsection{\code{map\_atomic\_min}}
\label{api:java:map_atomic_min}
\index{map\_atomic\_min!Java API}
Take the minium of the specified value and existing value for each key-value
pair.
This operation requires a pre-existing object in order to complete successfully.
If no object exists, the operation will fail with \code{NOTFOUND}.



\paragraph{Definition:}
\begin{javacode}
public Boolean map_atomic_min(
        String spacename,
        Object key,
        Map<String, Map<Object, Object>> mapattributes) throws HyperDexClientException
\end{javacode}

\paragraph{Parameters:}
\begin{itemize}[noitemsep]
\item \code{String spacename}\\
The name of the space as a string or symbol.

\item \code{Object key}\\
The key for the operation as a Python type.

\item \code{Map<String, Map<Object, Object>> mapattributes}\\
A hash specifying map attributes to modify and their respective key/values.

\end{itemize}

\paragraph{Returns:}
This function returns an object indicating the success or failure of the
operation.  Valid values to be returned are:

\begin{itemize}[noitemsep]
\item \code{True} if the operation succeeded
\item \code{False} if any provided predicates failed.
\item \code{null} if the operation requires an existing value and none exists
\end{itemize}

On error, this function will raise a \code{HyperDexClientException} describing
the error.


\pagebreak
\subsubsection{\code{async\_map\_atomic\_min}}
\label{api:java:async_map_atomic_min}
\index{async\_map\_atomic\_min!Java API}
Take the minium of the specified value and existing value for each key-value
pair.
This operation requires a pre-existing object in order to complete successfully.
If no object exists, the operation will fail with \code{NOTFOUND}.



\paragraph{Definition:}
\begin{javacode}
public Deferred async_map_atomic_min(
        String spacename,
        Object key,
        Map<String, Map<Object, Object>> mapattributes) throws HyperDexClientException
\end{javacode}

\paragraph{Parameters:}
\begin{itemize}[noitemsep]
\item \code{String spacename}\\
The name of the space as a string or symbol.

\item \code{Object key}\\
The key for the operation as a Python type.

\item \code{Map<String, Map<Object, Object>> mapattributes}\\
A hash specifying map attributes to modify and their respective key/values.

\end{itemize}

\paragraph{Returns:}
A Deferred object with a \code{wait} method that returns True if the operation
succeeded or False if any provided predicates failed.  Raises an exception on
error.


\paragraph{See also:}  This is the asynchronous form of \code{map\_atomic\_min}.

%%%%%%%%%%%%%%%%%%%% cond_map_atomic_min %%%%%%%%%%%%%%%%%%%%
\pagebreak
\subsubsection{\code{cond\_map\_atomic\_min}}
\label{api:java:cond_map_atomic_min}
\index{cond\_map\_atomic\_min!Java API}
XXX


\paragraph{Definition:}
\begin{javacode}
public Boolean cond_map_atomic_min(
        String spacename,
        Object key,
        Map<String, Object> predicates,
        Map<String, Map<Object, Object>> mapattributes) throws HyperDexClientException
\end{javacode}

\paragraph{Parameters:}
\begin{itemize}[noitemsep]
\item \code{String spacename}\\
The name of the space as a string or symbol.

\item \code{Object key}\\
The key for the operation as a Python type.

\item \code{Map<String, Object> predicates}\\
A hash of predicates to check against.

\item \code{Map<String, Map<Object, Object>> mapattributes}\\
A hash specifying map attributes to modify and their respective key/values.

\end{itemize}

\paragraph{Returns:}
This function returns an object indicating the success or failure of the
operation.  Valid values to be returned are:

\begin{itemize}[noitemsep]
\item \code{True} if the operation succeeded
\item \code{False} if any provided predicates failed.
\item \code{null} if the operation requires an existing value and none exists
\end{itemize}

On error, this function will raise a \code{HyperDexClientException} describing
the error.


\pagebreak
\subsubsection{\code{async\_cond\_map\_atomic\_min}}
\label{api:java:async_cond_map_atomic_min}
\index{async\_cond\_map\_atomic\_min!Java API}
XXX


\paragraph{Definition:}
\begin{javacode}
public Deferred async_cond_map_atomic_min(
        String spacename,
        Object key,
        Map<String, Object> predicates,
        Map<String, Map<Object, Object>> mapattributes) throws HyperDexClientException
\end{javacode}

\paragraph{Parameters:}
\begin{itemize}[noitemsep]
\item \code{String spacename}\\
The name of the space as a string or symbol.

\item \code{Object key}\\
The key for the operation as a Python type.

\item \code{Map<String, Object> predicates}\\
A hash of predicates to check against.

\item \code{Map<String, Map<Object, Object>> mapattributes}\\
A hash specifying map attributes to modify and their respective key/values.

\end{itemize}

\paragraph{Returns:}
A Deferred object with a \code{wait} method that returns True if the operation
succeeded or False if any provided predicates failed.  Raises an exception on
error.


\paragraph{See also:}  This is the asynchronous form of \code{cond\_map\_atomic\_min}.

%%%%%%%%%%%%%%%%%%%% map_atomic_max %%%%%%%%%%%%%%%%%%%%
\pagebreak
\subsubsection{\code{map\_atomic\_max}}
\label{api:java:map_atomic_max}
\index{map\_atomic\_max!Java API}
Take the maximum of the specified value and existing value for each key-value
pair.
This operation requires a pre-existing object in order to complete successfully.
If no object exists, the operation will fail with \code{NOTFOUND}.



\paragraph{Definition:}
\begin{javacode}
public Boolean map_atomic_max(
        String spacename,
        Object key,
        Map<String, Map<Object, Object>> mapattributes) throws HyperDexClientException
\end{javacode}

\paragraph{Parameters:}
\begin{itemize}[noitemsep]
\item \code{String spacename}\\
The name of the space as a string or symbol.

\item \code{Object key}\\
The key for the operation as a Python type.

\item \code{Map<String, Map<Object, Object>> mapattributes}\\
A hash specifying map attributes to modify and their respective key/values.

\end{itemize}

\paragraph{Returns:}
This function returns an object indicating the success or failure of the
operation.  Valid values to be returned are:

\begin{itemize}[noitemsep]
\item \code{True} if the operation succeeded
\item \code{False} if any provided predicates failed.
\item \code{null} if the operation requires an existing value and none exists
\end{itemize}

On error, this function will raise a \code{HyperDexClientException} describing
the error.


\pagebreak
\subsubsection{\code{async\_map\_atomic\_max}}
\label{api:java:async_map_atomic_max}
\index{async\_map\_atomic\_max!Java API}
Take the maximum of the specified value and existing value for each key-value
pair.
This operation requires a pre-existing object in order to complete successfully.
If no object exists, the operation will fail with \code{NOTFOUND}.



\paragraph{Definition:}
\begin{javacode}
public Deferred async_map_atomic_max(
        String spacename,
        Object key,
        Map<String, Map<Object, Object>> mapattributes) throws HyperDexClientException
\end{javacode}

\paragraph{Parameters:}
\begin{itemize}[noitemsep]
\item \code{String spacename}\\
The name of the space as a string or symbol.

\item \code{Object key}\\
The key for the operation as a Python type.

\item \code{Map<String, Map<Object, Object>> mapattributes}\\
A hash specifying map attributes to modify and their respective key/values.

\end{itemize}

\paragraph{Returns:}
A Deferred object with a \code{wait} method that returns True if the operation
succeeded or False if any provided predicates failed.  Raises an exception on
error.


\paragraph{See also:}  This is the asynchronous form of \code{map\_atomic\_max}.

%%%%%%%%%%%%%%%%%%%% cond_map_atomic_max %%%%%%%%%%%%%%%%%%%%
\pagebreak
\subsubsection{\code{cond\_map\_atomic\_max}}
\label{api:java:cond_map_atomic_max}
\index{cond\_map\_atomic\_max!Java API}
XXX


\paragraph{Definition:}
\begin{javacode}
public Boolean cond_map_atomic_max(
        String spacename,
        Object key,
        Map<String, Object> predicates,
        Map<String, Map<Object, Object>> mapattributes) throws HyperDexClientException
\end{javacode}

\paragraph{Parameters:}
\begin{itemize}[noitemsep]
\item \code{String spacename}\\
The name of the space as a string or symbol.

\item \code{Object key}\\
The key for the operation as a Python type.

\item \code{Map<String, Object> predicates}\\
A hash of predicates to check against.

\item \code{Map<String, Map<Object, Object>> mapattributes}\\
A hash specifying map attributes to modify and their respective key/values.

\end{itemize}

\paragraph{Returns:}
This function returns an object indicating the success or failure of the
operation.  Valid values to be returned are:

\begin{itemize}[noitemsep]
\item \code{True} if the operation succeeded
\item \code{False} if any provided predicates failed.
\item \code{null} if the operation requires an existing value and none exists
\end{itemize}

On error, this function will raise a \code{HyperDexClientException} describing
the error.


\pagebreak
\subsubsection{\code{async\_cond\_map\_atomic\_max}}
\label{api:java:async_cond_map_atomic_max}
\index{async\_cond\_map\_atomic\_max!Java API}
XXX


\paragraph{Definition:}
\begin{javacode}
public Deferred async_cond_map_atomic_max(
        String spacename,
        Object key,
        Map<String, Object> predicates,
        Map<String, Map<Object, Object>> mapattributes) throws HyperDexClientException
\end{javacode}

\paragraph{Parameters:}
\begin{itemize}[noitemsep]
\item \code{String spacename}\\
The name of the space as a string or symbol.

\item \code{Object key}\\
The key for the operation as a Python type.

\item \code{Map<String, Object> predicates}\\
A hash of predicates to check against.

\item \code{Map<String, Map<Object, Object>> mapattributes}\\
A hash specifying map attributes to modify and their respective key/values.

\end{itemize}

\paragraph{Returns:}
A Deferred object with a \code{wait} method that returns True if the operation
succeeded or False if any provided predicates failed.  Raises an exception on
error.


\paragraph{See also:}  This is the asynchronous form of \code{cond\_map\_atomic\_max}.

%%%%%%%%%%%%%%%%%%%% search %%%%%%%%%%%%%%%%%%%%
\pagebreak
\subsubsection{\code{search}}
\label{api:java:search}
\index{search!Java API}
Return all objects that match the specified \code{checks}.

\paragraph{Behavior:}
\begin{itemize}[noitemsep]
\item XXX % XXX

\item XXX % XXX

\end{itemize}


\paragraph{Definition:}
\begin{javacode}
public Iterator search(String spacename, Map<String, Object> predicates)
\end{javacode}

\paragraph{Parameters:}
\begin{itemize}[noitemsep]
\item \code{String spacename}\\
The name of the space as a string or symbol.

\item \code{Map<String, Object> predicates}\\
A hash of predicates to check against.

\end{itemize}

\paragraph{Returns:}
Two channels, one for returning objects that match the search, and one for
returning errors encountered during the search.


%%%%%%%%%%%%%%%%%%%% search_describe %%%%%%%%%%%%%%%%%%%%
\pagebreak
\subsubsection{\code{search\_describe}}
\label{api:java:search_describe}
\index{search\_describe!Java API}
Return a human-readable string suitable for debugging search internals.  This
API is only really relevant for debugging the internals of \code{search}.


\paragraph{Definition:}
\begin{javacode}
public String search_describe(
        String spacename,
        Map<String, Object> predicates) throws HyperDexClientException
\end{javacode}

\paragraph{Parameters:}
\begin{itemize}[noitemsep]
\item \code{String spacename}\\
The name of the space as a string or symbol.

\item \code{Map<String, Object> predicates}\\
A hash of predicates to check against.

\end{itemize}

\paragraph{Returns:}
Description of search.  Raises exception on error.


\pagebreak
\subsubsection{\code{async\_search\_describe}}
\label{api:java:async_search_describe}
\index{async\_search\_describe!Java API}
Return a human-readable string suitable for debugging search internals.  This
API is only really relevant for debugging the internals of \code{search}.


\paragraph{Definition:}
\begin{javacode}
public Deferred async_search_describe(
        String spacename,
        Map<String, Object> predicates) throws HyperDexClientException
\end{javacode}

\paragraph{Parameters:}
\begin{itemize}[noitemsep]
\item \code{String spacename}\\
The name of the space as a string or symbol.

\item \code{Map<String, Object> predicates}\\
A hash of predicates to check against.

\end{itemize}

\paragraph{Returns:}
A Deferred object with a \code{wait} method that returns a description of a
search.  Raises an exception on error.


\paragraph{See also:}  This is the asynchronous form of \code{search\_describe}.

%%%%%%%%%%%%%%%%%%%% sorted_search %%%%%%%%%%%%%%%%%%%%
\pagebreak
\subsubsection{\code{sorted\_search}}
\label{api:java:sorted_search}
\index{sorted\_search!Java API}
Return all objects that match the specified \code{checks}, sorted according to
\code{attr}.
\item XXX % XXX



\paragraph{Definition:}
\begin{javacode}
public Iterator sorted_search(
        String spacename,
        Map<String, Object> predicates,
        String sortby,
        int limit,
        boolean maxmin)
\end{javacode}

\paragraph{Parameters:}
\begin{itemize}[noitemsep]
\item \code{String spacename}\\
The name of the space as a string or symbol.

\item \code{Map<String, Object> predicates}\\
A hash of predicates to check against.

\item \code{String sortby}\\
XXX

\item \code{int limit}\\
XXX

\item \code{boolean maxmin}\\
Maximize (!= 0) or minimize (== 0).

\end{itemize}

\paragraph{Returns:}
Two channels, one for returning objects that match the search, and one for
returning errors encountered during the search.


%%%%%%%%%%%%%%%%%%%% group_del %%%%%%%%%%%%%%%%%%%%
\pagebreak
\subsubsection{\code{group\_del}}
\label{api:java:group_del}
\index{group\_del!Java API}
Asynchronously delete all objects that match the specified \code{checks}.

\paragraph{Behavior:}
\begin{itemize}[noitemsep]
\item This operation is roughly equivalent to a client manually deleting every
    object returned from a search, but saves HyperDex from sending to the client
    objects that are soon to be deleted.
\end{itemize}


\paragraph{Definition:}
\begin{javacode}
public Boolean group_del(
        String spacename,
        Map<String, Object> predicates) throws HyperDexClientException
\end{javacode}

\paragraph{Parameters:}
\begin{itemize}[noitemsep]
\item \code{String spacename}\\
The name of the space as a string or symbol.

\item \code{Map<String, Object> predicates}\\
A hash of predicates to check against.

\end{itemize}

\paragraph{Returns:}
This function returns an object indicating the success or failure of the
operation.  Valid values to be returned are:

\begin{itemize}[noitemsep]
\item \code{True} if the operation succeeded
\item \code{False} if any provided predicates failed.
\item \code{null} if the operation requires an existing value and none exists
\end{itemize}

On error, this function will raise a \code{HyperDexClientException} describing
the error.


\pagebreak
\subsubsection{\code{async\_group\_del}}
\label{api:java:async_group_del}
\index{async\_group\_del!Java API}
Asynchronously delete all objects that match the specified \code{checks}.

\paragraph{Behavior:}
\begin{itemize}[noitemsep]
\item This operation is roughly equivalent to a client manually deleting every
    object returned from a search, but saves HyperDex from sending to the client
    objects that are soon to be deleted.
\end{itemize}


\paragraph{Definition:}
\begin{javacode}
public Deferred async_group_del(
        String spacename,
        Map<String, Object> predicates) throws HyperDexClientException
\end{javacode}

\paragraph{Parameters:}
\begin{itemize}[noitemsep]
\item \code{String spacename}\\
The name of the space as a string or symbol.

\item \code{Map<String, Object> predicates}\\
A hash of predicates to check against.

\end{itemize}

\paragraph{Returns:}
A Deferred object with a \code{wait} method that returns True if the operation
succeeded or False if any provided predicates failed.  Raises an exception on
error.


\paragraph{See also:}  This is the asynchronous form of \code{group\_del}.

%%%%%%%%%%%%%%%%%%%% count %%%%%%%%%%%%%%%%%%%%
\pagebreak
\subsubsection{\code{count}}
\label{api:java:count}
\index{count!Java API}
Count the number of objects that match the specified \code{checks}.

\paragraph{Behavior:}
\begin{itemize}[noitemsep]
\item This will return the number of objects counted by the search.  If an error
    occurs during the count, the count will reflect a partial count.  The real
    count will be higher than the returned value.  Some languages will throw an
    exception rather than return the partial count.
\end{itemize}


\paragraph{Definition:}
\begin{javacode}
public Long count(
        String spacename,
        Map<String, Object> predicates) throws HyperDexClientException
\end{javacode}

\paragraph{Parameters:}
\begin{itemize}[noitemsep]
\item \code{String spacename}\\
The name of the space as a string or symbol.

\item \code{Map<String, Object> predicates}\\
A hash of predicates to check against.

\end{itemize}

\paragraph{Returns:}
A count of the number of objects, and a \code{client.Error} object indicating
the status of the operation.


\pagebreak
\subsubsection{\code{async\_count}}
\label{api:java:async_count}
\index{async\_count!Java API}
Count the number of objects that match the specified \code{checks}.

\paragraph{Behavior:}
\begin{itemize}[noitemsep]
\item This will return the number of objects counted by the search.  If an error
    occurs during the count, the count will reflect a partial count.  The real
    count will be higher than the returned value.  Some languages will throw an
    exception rather than return the partial count.
\end{itemize}


\paragraph{Definition:}
\begin{javacode}
public Deferred async_count(
        String spacename,
        Map<String, Object> predicates) throws HyperDexClientException
\end{javacode}

\paragraph{Parameters:}
\begin{itemize}[noitemsep]
\item \code{String spacename}\\
The name of the space as a string or symbol.

\item \code{Map<String, Object> predicates}\\
A hash of predicates to check against.

\end{itemize}

\paragraph{Returns:}
A Deferred object with a \code{wait} method that returns the number of objects
found.  Raises exception on error.


\paragraph{See also:}  This is the asynchronous form of \code{count}.

\pagebreak

\subsection{Working with Signals}
\label{sec:api:java:signals}

The HyperDex client library is signal-safe.  Should a signal interrupt the
client during a blocking operation, it will raise a
\code{HyperDexClientException} with status \code{HYPERDEX\_CLIENT\_INTERRUPTED}.

\subsection{Working with Threads}
\label{sec:api:Java:threads}

The Java package is fully reentrant.  Instances of
\code{HyperDex::Client::Client} and their associated state may be accessed from
multiple threads, provided that the application employs its own synchronization
that provides mutual exclusion.

Put simply, a multi-threaded application should protect each \code{Client}
instance with a mutex or lock to ensure correct operation.
