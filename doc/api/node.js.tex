\chapter{Node API}
\label{chap:api:node}

\section{Client Library}
\label{sec:api:node:client}

HyperDex provides Node bindings under the module \code{HyperDex::Client}.  This
library wraps the HyperDex C Client library and enables use of native Javascript
data types.

This module was re-introduced in HyperDex 1.2.0.

\subsection{Building the HyperDex Node.js Binding}
\label{sec:api:node:building}

The HyperDex Node.js Binding must be built and installed after HyperDex is
built.  After installing HyperDex, you can build the Node.js bindings from
either source or git checkout with:

\begin{consolecode}
% cd bindings/node.js
% node rebuild
\end{consolecode}

\subsection{Using Node.js Within Your Application}
\label{sec:api:node:using}

All client operation are defined in the \code{hyperdex\_client} module.  You can
access this in your program with:

\begin{javascriptcode}
var hyperdex_client = require('hyperdex-client');
\end{javascriptcode}

\subsection{Hello World}
\label{sec:api:node:hello-world}

The following is a minimal application that stores the value "Hello World" and
then immediately retrieves the value:

\inputminted{javascript}{\topdir/api/node.js/hello-world.js}

You can run this example with:

\begin{consolecode}
% node hello-world.js
put: true
get: [object Object]
\end{consolecode}

Right away, there are several points worth noting in this example:

\begin{itemize}
\item Each operation takes a callback.  While the operation is outstanding, your
program is free to execute other code.

\item Javascript types are automatically converted to HyperDex types.  There's
no need to specify information such as the length of each string, as one would
do with the C API.

\item There's no need to manually enter the HyperDex event loop.  HyperDex will
add and remove itself from the event loop as operations start and finish.
\end{itemize}

\subsection{Asynchronous Operations}
\label{sec:api:node:async-ops}

HyperDex provides native integration with the asynchronous world of Node.js.
You can issue several operations concurrently, and Node.js and HyperDex will
work together to complete these operations quickly and efficiently.  It's easy
to work with data concurrently.  A common pattern is to keep a constant number
of operations outstanding concurrently:

\inputminted{javascript}{\topdir/api/node.js/window-pattern.js}

\subsection{Data Structures}
\label{sec:api:node:data-structures}

The Node bindings automatically manage conversion of data types from Javascript
to HyperDex types, enabling applications to be written in idiomatic Javascript.

\subsubsection{Examples}
\label{sec:api:node:examples}

This section shows examples of Java data structures that are recognized by
HyperDex.  The examples here are for illustration purposes and are not
exhaustive.

\paragraph{Strings}

The HyperDex client recognizes Nodes's strings and buffers and automatically
converts them to HyperDex strings.  For example, the following two calls
have the same effect:

\begin{javascriptcode}
c.put("kv", "somekey", {v: "somevalue"}, function (success, err) {});
c.put("kv", "somekey", {v: new Buffer("somevalue")}, function (success, err) {});
\end{javascriptcode}

\paragraph{Other Types}

At the current point in time, other data types are not supported within Node.js.
This is primarily because Javascript doesn't support the majority of HyperDex
types without third party libraries.  We are looking for feedback from the
Node.js communtiy about the best way to support all HyperDex types.

\subsection{Attributes}
\label{sec:api:node:attributes}

Attributes in Node are specified in the form of a Javascript object.  As you can
see in the examples above, attributes are specified in the form:

\begin{javascriptcode}
{name: "value"}
\end{javascriptcode}

\subsection{Operations}
\label{sec:api:node:ops}

% Copyright (c) 2014, Cornell University
% All rights reserved.
%
% Redistribution and use in source and binary forms, with or without
% modification, are permitted provided that the following conditions are met:
%
%     * Redistributions of source code must retain the above copyright notice,
%       this list of conditions and the following disclaimer.
%     * Redistributions in binary form must reproduce the above copyright
%       notice, this list of conditions and the following disclaimer in the
%       documentation and/or other materials provided with the distribution.
%     * Neither the name of HyperDex nor the names of its contributors may be
%       used to endorse or promote products derived from this software without
%       specific prior written permission.
%
% THIS SOFTWARE IS PROVIDED BY THE COPYRIGHT HOLDERS AND CONTRIBUTORS "AS IS"
% AND ANY EXPRESS OR IMPLIED WARRANTIES, INCLUDING, BUT NOT LIMITED TO, THE
% IMPLIED WARRANTIES OF MERCHANTABILITY AND FITNESS FOR A PARTICULAR PURPOSE ARE
% DISCLAIMED. IN NO EVENT SHALL THE COPYRIGHT OWNER OR CONTRIBUTORS BE LIABLE
% FOR ANY DIRECT, INDIRECT, INCIDENTAL, SPECIAL, EXEMPLARY, OR CONSEQUENTIAL
% DAMAGES (INCLUDING, BUT NOT LIMITED TO, PROCUREMENT OF SUBSTITUTE GOODS OR
% SERVICES; LOSS OF USE, DATA, OR PROFITS; OR BUSINESS INTERRUPTION) HOWEVER
% CAUSED AND ON ANY THEORY OF LIABILITY, WHETHER IN CONTRACT, STRICT LIABILITY,
% OR TORT (INCLUDING NEGLIGENCE OR OTHERWISE) ARISING IN ANY WAY OUT OF THE USE
% OF THIS SOFTWARE, EVEN IF ADVISED OF THE POSSIBILITY OF SUCH DAMAGE.

% This LaTeX file is generated by bindings/nodejs.py

%%%%%%%%%%%%%%%%%%%% get %%%%%%%%%%%%%%%%%%%%
\pagebreak
\subsubsection{\code{get}}
\label{api:nodejs:get}
\index{get!Node.js API}
Get an object by key.

\paragraph{Behavior:}
\begin{itemize}[noitemsep]
\item XXX % XXX

\end{itemize}


\paragraph{Definition:}
\begin{javascriptcode}
get(spacename, key, function (obj, done, err) {})
\end{javascriptcode}
\paragraph{Parameters:}
\begin{itemize}[noitemsep]
\item \code{spacename}\\
The name of the space as a string or symbol.

\item \code{key}\\
The key for the operation as a Python type.

\end{itemize}

\paragraph{Returns:}
This function returns via the provided callback.  In the normal case, the first
argument will indicate success or failure of the operation with one of the
following values:

\begin{itemize}[noitemsep]
\item A Javascript representation of the stored object.
\item \code{null} if the operation is a retrieval operation and no object was
    found.
\end{itemize}

If the operation encounters any error, the error argument will be provided and
will specify the error, in which case the first argument is undefined.


%%%%%%%%%%%%%%%%%%%% get_partial %%%%%%%%%%%%%%%%%%%%
\pagebreak
\subsubsection{\code{get\_partial}}
\label{api:nodejs:get_partial}
\index{get\_partial!Node.js API}
Get part of an object by key.  This will return only the listed attribute names.

\paragraph{Behavior:}
\begin{itemize}[noitemsep]
\item XXX % XXX

\end{itemize}


\paragraph{Definition:}
\begin{javascriptcode}
get_partial(spacename, key, attributenames, function (obj, done, err) {})
\end{javascriptcode}
\paragraph{Parameters:}
\begin{itemize}[noitemsep]
\item \code{spacename}\\
The name of the space as a string or symbol.

\item \code{key}\\
The key for the operation as a Python type.

\item \code{attributenames}\\
A list of attributes to return.  \code{attrnames} is a \code{List<String>}.

\end{itemize}

\paragraph{Returns:}
This function returns via the provided callback.  In the normal case, the first
argument will indicate success or failure of the operation with one of the
following values:

\begin{itemize}[noitemsep]
\item A Javascript representation of the stored object.
\item \code{null} if the operation is a retrieval operation and no object was
    found.
\end{itemize}

If the operation encounters any error, the error argument will be provided and
will specify the error, in which case the first argument is undefined.


%%%%%%%%%%%%%%%%%%%% put %%%%%%%%%%%%%%%%%%%%
\pagebreak
\subsubsection{\code{put}}
\label{api:nodejs:put}
\index{put!Node.js API}
Store or update an object by key.  The object's attributes will be set to the
values specified by \code{attrs}.
\item An existing object will be updated by the operation.  If no object does
    exists, a new object will be created, with attributes initialized to their
    default values.



\paragraph{Definition:}
\begin{javascriptcode}
put(spacename, key, attributes, function (success, err) {})
\end{javascriptcode}
\paragraph{Parameters:}
\begin{itemize}[noitemsep]
\item \code{spacename}\\
The name of the space as a string or symbol.

\item \code{key}\\
The key for the operation as a Python type.

\item \code{attributes}\\
The set of attributes to modify and their respective values.  \code{attrs} is a
map from the attributes' names to their values.

\end{itemize}

\paragraph{Returns:}
This function returns an object indicating the success or failure of the
operation.  Valid values to be returned are:

\begin{itemize}[noitemsep]
\item \code{True} if the operation succeeded
\item \code{False} if any provided predicates failed.
\item \code{null} if the operation requires an existing value and none exists
\end{itemize}

On error, this function will raise a \code{HyperDexClientException} describing
the error.


%%%%%%%%%%%%%%%%%%%% cond_put %%%%%%%%%%%%%%%%%%%%
\pagebreak
\subsubsection{\code{cond\_put}}
\label{api:nodejs:cond_put}
\index{cond\_put!Node.js API}
Conditionally write the specified attributes to the object in space "space" under key "key".

The operation will modify the object if and only if all \texttt{checks} are true
for the latest version of the object.  This test is atomic with the write.  If
the object does not exist, the checks will fail.

The attributes specified by \texttt{attrs} will be overwritten with their
respective values.  Any attributes not specified by \texttt{attrs} will be
preserved.

\noindent\textbf{Cost:}  Approximately one traversal of the value-dependent
chain.


\noindent\textbf{Consistency:}  Linearizable



\paragraph{Definition:}
\begin{javascriptcode}
cond_put(spacename, key, predicates, attributes, function (success, err) {})
\end{javascriptcode}
\paragraph{Parameters:}
\begin{itemize}[noitemsep]
\item \code{spacename}\\
The name of the space as a string or symbol.

\item \code{key}\\
The key for the operation as a Python type.

\item \code{predicates}\\
A hash of predicates to check against.

\item \code{attributes}\\
The set of attributes to modify and their respective values.  \code{attrs} is a
map from the attributes' names to their values.

\end{itemize}

\paragraph{Returns:}
This function returns an object indicating the success or failure of the
operation.  Valid values to be returned are:

\begin{itemize}[noitemsep]
\item \code{True} if the operation succeeded
\item \code{False} if any provided predicates failed.
\item \code{null} if the operation requires an existing value and none exists
\end{itemize}

On error, this function will raise a \code{HyperDexClientException} describing
the error.


%%%%%%%%%%%%%%%%%%%% group_put %%%%%%%%%%%%%%%%%%%%
\pagebreak
\subsubsection{\code{group\_put}}
\label{api:nodejs:group_put}
\index{group\_put!Node.js API}
Update all objects stored in \code{space} that match \code{checks}.  Existing
values will be overwitten with the values specified by \code{attrs}.  Values not
specified by \code{attrs} will remain unchanged.

This operation will only affect objects that match the provided \code{checks}.
Objects that do not match \code{checks} will be unaffected by the group call.
Each object that matches \code{checks} will be atomically updated with the check
on the object.  HyperDex guarantees that no object will be altered if the
\code{checks} do not pass at the time of the write.  Objects that are updated
concurrently with the group call may or may not be updated; however, regardless
of any other concurrent operations, the preceding guarantee will always hold.



\paragraph{Definition:}
\begin{javascriptcode}
group_put(spacename, predicates, attributes, function (count, err) {})
\end{javascriptcode}
\paragraph{Parameters:}
\begin{itemize}[noitemsep]
\item \code{spacename}\\
The name of the space as a string or symbol.

\item \code{predicates}\\
A hash of predicates to check against.

\item \code{attributes}\\
The set of attributes to modify and their respective values.  \code{attrs} is a
map from the attributes' names to their values.

\end{itemize}

\paragraph{Returns:}
A count of the number of objects, and a \code{client.Error} object indicating
the status of the operation.


%%%%%%%%%%%%%%%%%%%% put_if_not_exist %%%%%%%%%%%%%%%%%%%%
\pagebreak
\subsubsection{\code{put\_if\_not\_exist}}
\label{api:nodejs:put_if_not_exist}
\index{put\_if\_not\_exist!Node.js API}
Store or object under \code{key} in \code{space} if and only if the operation
creates a new object.  The object's attributes will be set to the values
specified by \code{attrs}; any attributes not specified by \code{attrs} will be
initialized to their defaults.  If the object exists, the operation will fail
with \code{CMPFAIL}.


\paragraph{Definition:}
\begin{javascriptcode}
put_if_not_exist(spacename, key, attributes, function (success, err) {})
\end{javascriptcode}
\paragraph{Parameters:}
\begin{itemize}[noitemsep]
\item \code{spacename}\\
The name of the space as a string or symbol.

\item \code{key}\\
The key for the operation as a Python type.

\item \code{attributes}\\
The set of attributes to modify and their respective values.  \code{attrs} is a
map from the attributes' names to their values.

\end{itemize}

\paragraph{Returns:}
This function returns an object indicating the success or failure of the
operation.  Valid values to be returned are:

\begin{itemize}[noitemsep]
\item \code{True} if the operation succeeded
\item \code{False} if any provided predicates failed.
\item \code{null} if the operation requires an existing value and none exists
\end{itemize}

On error, this function will raise a \code{HyperDexClientException} describing
the error.


%%%%%%%%%%%%%%%%%%%% del %%%%%%%%%%%%%%%%%%%%
\pagebreak
\subsubsection{\code{del}}
\label{api:nodejs:del}
\index{del!Node.js API}
Delete an object by key.

%%% Generated below here
\paragraph{Behavior:}
\begin{itemize}[noitemsep]
If no object exists, the operation will fail with \code{NOTFOUND}.

\end{itemize}


\paragraph{Definition:}
\begin{javascriptcode}
del(spacename, key, function (success, err) {})
\end{javascriptcode}
\paragraph{Parameters:}
\begin{itemize}[noitemsep]
\item \code{spacename}\\
The name of the space as a string or symbol.

\item \code{key}\\
The key for the operation as a Python type.

\end{itemize}

\paragraph{Returns:}
This function returns an object indicating the success or failure of the
operation.  Valid values to be returned are:

\begin{itemize}[noitemsep]
\item \code{True} if the operation succeeded
\item \code{False} if any provided predicates failed.
\item \code{null} if the operation requires an existing value and none exists
\end{itemize}

On error, this function will raise a \code{HyperDexClientException} describing
the error.


%%%%%%%%%%%%%%%%%%%% cond_del %%%%%%%%%%%%%%%%%%%%
\pagebreak
\subsubsection{\code{cond\_del}}
\label{api:nodejs:cond_del}
\index{cond\_del!Node.js API}
Conditionally delete the object stored under \code{key} from \code{space}.
If no object exists, the operation will fail with \code{NOTFOUND}.


This operation will succeed if and only if the predicates specified by
\code{checks} hold on the pre-existing object.  If any of the predicates are not
true for the existing object, then the operation will have no effect and fail
with \code{CMPFAIL}.

All checks are atomic with the write.  HyperDex guarantees that no other
operation will come between validating the checks, and writing the new version
of the object.



\paragraph{Definition:}
\begin{javascriptcode}
cond_del(spacename, key, predicates, function (success, err) {})
\end{javascriptcode}
\paragraph{Parameters:}
\begin{itemize}[noitemsep]
\item \code{spacename}\\
The name of the space as a string or symbol.

\item \code{key}\\
The key for the operation as a Python type.

\item \code{predicates}\\
A hash of predicates to check against.

\end{itemize}

\paragraph{Returns:}
This function returns an object indicating the success or failure of the
operation.  Valid values to be returned are:

\begin{itemize}[noitemsep]
\item \code{True} if the operation succeeded
\item \code{False} if any provided predicates failed.
\item \code{null} if the operation requires an existing value and none exists
\end{itemize}

On error, this function will raise a \code{HyperDexClientException} describing
the error.


%%%%%%%%%%%%%%%%%%%% group_del %%%%%%%%%%%%%%%%%%%%
\pagebreak
\subsubsection{\code{group\_del}}
\label{api:nodejs:group_del}
\index{group\_del!Node.js API}
Asynchronously delete all objects that match the specified \code{checks}.

\paragraph{Behavior:}
\begin{itemize}[noitemsep]
\item This operation is roughly equivalent to a client manually deleting every
    object returned from a search, but saves HyperDex from sending to the client
    objects that are soon to be deleted.
\end{itemize}


\paragraph{Definition:}
\begin{javascriptcode}
group_del(spacename, predicates, function (count, err) {})
\end{javascriptcode}
\paragraph{Parameters:}
\begin{itemize}[noitemsep]
\item \code{spacename}\\
The name of the space as a string or symbol.

\item \code{predicates}\\
A hash of predicates to check against.

\end{itemize}

\paragraph{Returns:}
A count of the number of objects, and a \code{client.Error} object indicating
the status of the operation.


%%%%%%%%%%%%%%%%%%%% atomic_add %%%%%%%%%%%%%%%%%%%%
\pagebreak
\subsubsection{\code{atomic\_add}}
\label{api:nodejs:atomic_add}
\index{atomic\_add!Node.js API}
Add the specified number to the existing value for each attribute.

%%% Generated below here
\paragraph{Behavior:}
\begin{itemize}[noitemsep]
This operation requires a pre-existing object in order to complete successfully.
If no object exists, the operation will fail with \code{NOTFOUND}.

\end{itemize}


\paragraph{Definition:}
\begin{javascriptcode}
atomic_add(spacename, key, attributes, function (success, err) {})
\end{javascriptcode}
\paragraph{Parameters:}
\begin{itemize}[noitemsep]
\item \code{spacename}\\
The name of the space as a string or symbol.

\item \code{key}\\
The key for the operation as a Python type.

\item \code{attributes}\\
The set of attributes to modify and their respective values.  \code{attrs} is a
map from the attributes' names to their values.

\end{itemize}

\paragraph{Returns:}
This function returns an object indicating the success or failure of the
operation.  Valid values to be returned are:

\begin{itemize}[noitemsep]
\item \code{True} if the operation succeeded
\item \code{False} if any provided predicates failed.
\item \code{null} if the operation requires an existing value and none exists
\end{itemize}

On error, this function will raise a \code{HyperDexClientException} describing
the error.


%%%%%%%%%%%%%%%%%%%% cond_atomic_add %%%%%%%%%%%%%%%%%%%%
\pagebreak
\subsubsection{\code{cond\_atomic\_add}}
\label{api:nodejs:cond_atomic_add}
\index{cond\_atomic\_add!Node.js API}
Conditionally add the specified number to the existing value for each attribute.

%%% Generated below here
\paragraph{Behavior:}
\begin{itemize}[noitemsep]
This operation requires a pre-existing object in order to complete successfully.
If no object exists, the operation will fail with \code{NOTFOUND}.

This operation will succeed if and only if the predicates specified by
\code{checks} hold on the pre-existing object.  If any of the predicates are not
true for the existing object, then the operation will have no effect and fail
with \code{CMPFAIL}.

All checks are atomic with the write.  HyperDex guarantees that no other
operation will come between validating the checks, and writing the new version
of the object.

\end{itemize}


\paragraph{Definition:}
\begin{javascriptcode}
cond_atomic_add(spacename, key, predicates, attributes, function (success, err) {})
\end{javascriptcode}
\paragraph{Parameters:}
\begin{itemize}[noitemsep]
\item \code{spacename}\\
The name of the space as a string or symbol.

\item \code{key}\\
The key for the operation as a Python type.

\item \code{predicates}\\
A hash of predicates to check against.

\item \code{attributes}\\
The set of attributes to modify and their respective values.  \code{attrs} is a
map from the attributes' names to their values.

\end{itemize}

\paragraph{Returns:}
This function returns an object indicating the success or failure of the
operation.  Valid values to be returned are:

\begin{itemize}[noitemsep]
\item \code{True} if the operation succeeded
\item \code{False} if any provided predicates failed.
\item \code{null} if the operation requires an existing value and none exists
\end{itemize}

On error, this function will raise a \code{HyperDexClientException} describing
the error.


%%%%%%%%%%%%%%%%%%%% group_atomic_add %%%%%%%%%%%%%%%%%%%%
\pagebreak
\subsubsection{\code{group\_atomic\_add}}
\label{api:nodejs:group_atomic_add}
\index{group\_atomic\_add!Node.js API}
Add the specified number to the existing value for each object in \code{space}
that matches \code{checks}.

This operation will only affect objects that match the provided \code{checks}.
Objects that do not match \code{checks} will be unaffected by the group call.
Each object that matches \code{checks} will be atomically updated with the check
on the object.  HyperDex guarantees that no object will be altered if the
\code{checks} do not pass at the time of the write.  Objects that are updated
concurrently with the group call may or may not be updated; however, regardless
of any other concurrent operations, the preceding guarantee will always hold.



\paragraph{Definition:}
\begin{javascriptcode}
group_atomic_add(spacename, predicates, attributes, function (count, err) {})
\end{javascriptcode}
\paragraph{Parameters:}
\begin{itemize}[noitemsep]
\item \code{spacename}\\
The name of the space as a string or symbol.

\item \code{predicates}\\
A hash of predicates to check against.

\item \code{attributes}\\
The set of attributes to modify and their respective values.  \code{attrs} is a
map from the attributes' names to their values.

\end{itemize}

\paragraph{Returns:}
A count of the number of objects, and a \code{client.Error} object indicating
the status of the operation.


%%%%%%%%%%%%%%%%%%%% atomic_sub %%%%%%%%%%%%%%%%%%%%
\pagebreak
\subsubsection{\code{atomic\_sub}}
\label{api:nodejs:atomic_sub}
\index{atomic\_sub!Node.js API}
Subtract the specified number from the existing value for each attribute.

%%% Generated below here
\paragraph{Behavior:}
\begin{itemize}[noitemsep]
This operation requires a pre-existing object in order to complete successfully.
If no object exists, the operation will fail with \code{NOTFOUND}.

\end{itemize}


\paragraph{Definition:}
\begin{javascriptcode}
atomic_sub(spacename, key, attributes, function (success, err) {})
\end{javascriptcode}
\paragraph{Parameters:}
\begin{itemize}[noitemsep]
\item \code{spacename}\\
The name of the space as a string or symbol.

\item \code{key}\\
The key for the operation as a Python type.

\item \code{attributes}\\
The set of attributes to modify and their respective values.  \code{attrs} is a
map from the attributes' names to their values.

\end{itemize}

\paragraph{Returns:}
This function returns an object indicating the success or failure of the
operation.  Valid values to be returned are:

\begin{itemize}[noitemsep]
\item \code{True} if the operation succeeded
\item \code{False} if any provided predicates failed.
\item \code{null} if the operation requires an existing value and none exists
\end{itemize}

On error, this function will raise a \code{HyperDexClientException} describing
the error.


%%%%%%%%%%%%%%%%%%%% cond_atomic_sub %%%%%%%%%%%%%%%%%%%%
\pagebreak
\subsubsection{\code{cond\_atomic\_sub}}
\label{api:nodejs:cond_atomic_sub}
\index{cond\_atomic\_sub!Node.js API}
Conditionally subtract the specified number from the existing value for each attribute.

%%% Generated below here
\paragraph{Behavior:}
\begin{itemize}[noitemsep]
This operation requires a pre-existing object in order to complete successfully.
If no object exists, the operation will fail with \code{NOTFOUND}.

This operation will succeed if and only if the predicates specified by
\code{checks} hold on the pre-existing object.  If any of the predicates are not
true for the existing object, then the operation will have no effect and fail
with \code{CMPFAIL}.

All checks are atomic with the write.  HyperDex guarantees that no other
operation will come between validating the checks, and writing the new version
of the object.

\end{itemize}


\paragraph{Definition:}
\begin{javascriptcode}
cond_atomic_sub(spacename, key, predicates, attributes, function (success, err) {})
\end{javascriptcode}
\paragraph{Parameters:}
\begin{itemize}[noitemsep]
\item \code{spacename}\\
The name of the space as a string or symbol.

\item \code{key}\\
The key for the operation as a Python type.

\item \code{predicates}\\
A hash of predicates to check against.

\item \code{attributes}\\
The set of attributes to modify and their respective values.  \code{attrs} is a
map from the attributes' names to their values.

\end{itemize}

\paragraph{Returns:}
This function returns an object indicating the success or failure of the
operation.  Valid values to be returned are:

\begin{itemize}[noitemsep]
\item \code{True} if the operation succeeded
\item \code{False} if any provided predicates failed.
\item \code{null} if the operation requires an existing value and none exists
\end{itemize}

On error, this function will raise a \code{HyperDexClientException} describing
the error.


%%%%%%%%%%%%%%%%%%%% group_atomic_sub %%%%%%%%%%%%%%%%%%%%
\pagebreak
\subsubsection{\code{group\_atomic\_sub}}
\label{api:nodejs:group_atomic_sub}
\index{group\_atomic\_sub!Node.js API}
Subtract the specified number from the existing value for each object in
\code{space} that matches \code{checks}.

This operation will only affect objects that match the provided \code{checks}.
Objects that do not match \code{checks} will be unaffected by the group call.
Each object that matches \code{checks} will be atomically updated with the check
on the object.  HyperDex guarantees that no object will be altered if the
\code{checks} do not pass at the time of the write.  Objects that are updated
concurrently with the group call may or may not be updated; however, regardless
of any other concurrent operations, the preceding guarantee will always hold.



\paragraph{Definition:}
\begin{javascriptcode}
group_atomic_sub(spacename, predicates, attributes, function (count, err) {})
\end{javascriptcode}
\paragraph{Parameters:}
\begin{itemize}[noitemsep]
\item \code{spacename}\\
The name of the space as a string or symbol.

\item \code{predicates}\\
A hash of predicates to check against.

\item \code{attributes}\\
The set of attributes to modify and their respective values.  \code{attrs} is a
map from the attributes' names to their values.

\end{itemize}

\paragraph{Returns:}
A count of the number of objects, and a \code{client.Error} object indicating
the status of the operation.


%%%%%%%%%%%%%%%%%%%% atomic_mul %%%%%%%%%%%%%%%%%%%%
\pagebreak
\subsubsection{\code{atomic\_mul}}
\label{api:nodejs:atomic_mul}
\index{atomic\_mul!Node.js API}
Multiply the existing value by the number specified for each attribute.

The multiplication is atomic with the write.  If the object does not exist, the
operation will fail.

\noindent\textbf{Cost:}  Approximately one traversal of the value-dependent
chain.


\noindent\textbf{Consistency:}  Linearizable



\paragraph{Definition:}
\begin{javascriptcode}
atomic_mul(spacename, key, attributes, function (success, err) {})
\end{javascriptcode}
\paragraph{Parameters:}
\begin{itemize}[noitemsep]
\item \code{spacename}\\
The name of the space as a string or symbol.

\item \code{key}\\
The key for the operation as a Python type.

\item \code{attributes}\\
The set of attributes to modify and their respective values.  \code{attrs} is a
map from the attributes' names to their values.

\end{itemize}

\paragraph{Returns:}
This function returns an object indicating the success or failure of the
operation.  Valid values to be returned are:

\begin{itemize}[noitemsep]
\item \code{True} if the operation succeeded
\item \code{False} if any provided predicates failed.
\item \code{null} if the operation requires an existing value and none exists
\end{itemize}

On error, this function will raise a \code{HyperDexClientException} describing
the error.


%%%%%%%%%%%%%%%%%%%% cond_atomic_mul %%%%%%%%%%%%%%%%%%%%
\pagebreak
\subsubsection{\code{cond\_atomic\_mul}}
\label{api:nodejs:cond_atomic_mul}
\index{cond\_atomic\_mul!Node.js API}
Conditionally multiply the existing value by the specified number for each
attribute.

%%% Generated below here
\paragraph{Behavior:}
\begin{itemize}[noitemsep]
This operation requires a pre-existing object in order to complete successfully.
If no object exists, the operation will fail with \code{NOTFOUND}.

This operation will succeed if and only if the predicates specified by
\code{checks} hold on the pre-existing object.  If any of the predicates are not
true for the existing object, then the operation will have no effect and fail
with \code{CMPFAIL}.

All checks are atomic with the write.  HyperDex guarantees that no other
operation will come between validating the checks, and writing the new version
of the object.

\end{itemize}


\paragraph{Definition:}
\begin{javascriptcode}
cond_atomic_mul(spacename, key, predicates, attributes, function (success, err) {})
\end{javascriptcode}
\paragraph{Parameters:}
\begin{itemize}[noitemsep]
\item \code{spacename}\\
The name of the space as a string or symbol.

\item \code{key}\\
The key for the operation as a Python type.

\item \code{predicates}\\
A hash of predicates to check against.

\item \code{attributes}\\
The set of attributes to modify and their respective values.  \code{attrs} is a
map from the attributes' names to their values.

\end{itemize}

\paragraph{Returns:}
This function returns an object indicating the success or failure of the
operation.  Valid values to be returned are:

\begin{itemize}[noitemsep]
\item \code{True} if the operation succeeded
\item \code{False} if any provided predicates failed.
\item \code{null} if the operation requires an existing value and none exists
\end{itemize}

On error, this function will raise a \code{HyperDexClientException} describing
the error.


%%%%%%%%%%%%%%%%%%%% group_atomic_mul %%%%%%%%%%%%%%%%%%%%
\pagebreak
\subsubsection{\code{group\_atomic\_mul}}
\label{api:nodejs:group_atomic_mul}
\index{group\_atomic\_mul!Node.js API}
Multiply the existing value by the specified number for each object in
\code{space} that matches \code{checks}.

This operation will only affect objects that match the provided \code{checks}.
Objects that do not match \code{checks} will be unaffected by the group call.
Each object that matches \code{checks} will be atomically updated with the check
on the object.  HyperDex guarantees that no object will be altered if the
\code{checks} do not pass at the time of the write.  Objects that are updated
concurrently with the group call may or may not be updated; however, regardless
of any other concurrent operations, the preceding guarantee will always hold.



\paragraph{Definition:}
\begin{javascriptcode}
group_atomic_mul(spacename, predicates, attributes, function (count, err) {})
\end{javascriptcode}
\paragraph{Parameters:}
\begin{itemize}[noitemsep]
\item \code{spacename}\\
The name of the space as a string or symbol.

\item \code{predicates}\\
A hash of predicates to check against.

\item \code{attributes}\\
The set of attributes to modify and their respective values.  \code{attrs} is a
map from the attributes' names to their values.

\end{itemize}

\paragraph{Returns:}
A count of the number of objects, and a \code{client.Error} object indicating
the status of the operation.


%%%%%%%%%%%%%%%%%%%% atomic_div %%%%%%%%%%%%%%%%%%%%
\pagebreak
\subsubsection{\code{atomic\_div}}
\label{api:nodejs:atomic_div}
\index{atomic\_div!Node.js API}
Divide the existing value by the specified number for each attribute.
This operation requires a pre-existing object in order to complete successfully.
If no object exists, the operation will fail with \code{NOTFOUND}.



\paragraph{Definition:}
\begin{javascriptcode}
atomic_div(spacename, key, attributes, function (success, err) {})
\end{javascriptcode}
\paragraph{Parameters:}
\begin{itemize}[noitemsep]
\item \code{spacename}\\
The name of the space as a string or symbol.

\item \code{key}\\
The key for the operation as a Python type.

\item \code{attributes}\\
The set of attributes to modify and their respective values.  \code{attrs} is a
map from the attributes' names to their values.

\end{itemize}

\paragraph{Returns:}
This function returns an object indicating the success or failure of the
operation.  Valid values to be returned are:

\begin{itemize}[noitemsep]
\item \code{True} if the operation succeeded
\item \code{False} if any provided predicates failed.
\item \code{null} if the operation requires an existing value and none exists
\end{itemize}

On error, this function will raise a \code{HyperDexClientException} describing
the error.


%%%%%%%%%%%%%%%%%%%% cond_atomic_div %%%%%%%%%%%%%%%%%%%%
\pagebreak
\subsubsection{\code{cond\_atomic\_div}}
\label{api:nodejs:cond_atomic_div}
\index{cond\_atomic\_div!Node.js API}
Conditionally divide the existing value by the specified number for each
attribute.

%%% Generated below here
\paragraph{Behavior:}
\begin{itemize}[noitemsep]
This operation requires a pre-existing object in order to complete successfully.
If no object exists, the operation will fail with \code{NOTFOUND}.

This operation will succeed if and only if the predicates specified by
\code{checks} hold on the pre-existing object.  If any of the predicates are not
true for the existing object, then the operation will have no effect and fail
with \code{CMPFAIL}.

All checks are atomic with the write.  HyperDex guarantees that no other
operation will come between validating the checks, and writing the new version
of the object.

\end{itemize}


\paragraph{Definition:}
\begin{javascriptcode}
cond_atomic_div(spacename, key, predicates, attributes, function (success, err) {})
\end{javascriptcode}
\paragraph{Parameters:}
\begin{itemize}[noitemsep]
\item \code{spacename}\\
The name of the space as a string or symbol.

\item \code{key}\\
The key for the operation as a Python type.

\item \code{predicates}\\
A hash of predicates to check against.

\item \code{attributes}\\
The set of attributes to modify and their respective values.  \code{attrs} is a
map from the attributes' names to their values.

\end{itemize}

\paragraph{Returns:}
This function returns an object indicating the success or failure of the
operation.  Valid values to be returned are:

\begin{itemize}[noitemsep]
\item \code{True} if the operation succeeded
\item \code{False} if any provided predicates failed.
\item \code{null} if the operation requires an existing value and none exists
\end{itemize}

On error, this function will raise a \code{HyperDexClientException} describing
the error.


%%%%%%%%%%%%%%%%%%%% group_atomic_div %%%%%%%%%%%%%%%%%%%%
\pagebreak
\subsubsection{\code{group\_atomic\_div}}
\label{api:nodejs:group_atomic_div}
\index{group\_atomic\_div!Node.js API}
Divide the existing value by the specified number for each object in
\code{space} that matches \code{checks}.

This operation will only affect objects that match the provided \code{checks}.
Objects that do not match \code{checks} will be unaffected by the group call.
Each object that matches \code{checks} will be atomically updated with the check
on the object.  HyperDex guarantees that no object will be altered if the
\code{checks} do not pass at the time of the write.  Objects that are updated
concurrently with the group call may or may not be updated; however, regardless
of any other concurrent operations, the preceding guarantee will always hold.



\paragraph{Definition:}
\begin{javascriptcode}
group_atomic_div(spacename, predicates, attributes, function (count, err) {})
\end{javascriptcode}
\paragraph{Parameters:}
\begin{itemize}[noitemsep]
\item \code{spacename}\\
The name of the space as a string or symbol.

\item \code{predicates}\\
A hash of predicates to check against.

\item \code{attributes}\\
The set of attributes to modify and their respective values.  \code{attrs} is a
map from the attributes' names to their values.

\end{itemize}

\paragraph{Returns:}
A count of the number of objects, and a \code{client.Error} object indicating
the status of the operation.


%%%%%%%%%%%%%%%%%%%% atomic_mod %%%%%%%%%%%%%%%%%%%%
\pagebreak
\subsubsection{\code{atomic\_mod}}
\label{api:nodejs:atomic_mod}
\index{atomic\_mod!Node.js API}
Store the existing value modulo the specified number for each attribute.

%%% Generated below here
\paragraph{Behavior:}
\begin{itemize}[noitemsep]
This operation requires a pre-existing object in order to complete successfully.
If no object exists, the operation will fail with \code{NOTFOUND}.

\end{itemize}


\paragraph{Definition:}
\begin{javascriptcode}
atomic_mod(spacename, key, attributes, function (success, err) {})
\end{javascriptcode}
\paragraph{Parameters:}
\begin{itemize}[noitemsep]
\item \code{spacename}\\
The name of the space as a string or symbol.

\item \code{key}\\
The key for the operation as a Python type.

\item \code{attributes}\\
The set of attributes to modify and their respective values.  \code{attrs} is a
map from the attributes' names to their values.

\end{itemize}

\paragraph{Returns:}
This function returns an object indicating the success or failure of the
operation.  Valid values to be returned are:

\begin{itemize}[noitemsep]
\item \code{True} if the operation succeeded
\item \code{False} if any provided predicates failed.
\item \code{null} if the operation requires an existing value and none exists
\end{itemize}

On error, this function will raise a \code{HyperDexClientException} describing
the error.


%%%%%%%%%%%%%%%%%%%% cond_atomic_mod %%%%%%%%%%%%%%%%%%%%
\pagebreak
\subsubsection{\code{cond\_atomic\_mod}}
\label{api:nodejs:cond_atomic_mod}
\index{cond\_atomic\_mod!Node.js API}
Conditionally store the existing value modulo the specified number for each
attribute.

%%% Generated below here
\paragraph{Behavior:}
\begin{itemize}[noitemsep]
This operation requires a pre-existing object in order to complete successfully.
If no object exists, the operation will fail with \code{NOTFOUND}.

This operation will succeed if and only if the predicates specified by
\code{checks} hold on the pre-existing object.  If any of the predicates are not
true for the existing object, then the operation will have no effect and fail
with \code{CMPFAIL}.

All checks are atomic with the write.  HyperDex guarantees that no other
operation will come between validating the checks, and writing the new version
of the object.

\end{itemize}


\paragraph{Definition:}
\begin{javascriptcode}
cond_atomic_mod(spacename, key, predicates, attributes, function (success, err) {})
\end{javascriptcode}
\paragraph{Parameters:}
\begin{itemize}[noitemsep]
\item \code{spacename}\\
The name of the space as a string or symbol.

\item \code{key}\\
The key for the operation as a Python type.

\item \code{predicates}\\
A hash of predicates to check against.

\item \code{attributes}\\
The set of attributes to modify and their respective values.  \code{attrs} is a
map from the attributes' names to their values.

\end{itemize}

\paragraph{Returns:}
This function returns an object indicating the success or failure of the
operation.  Valid values to be returned are:

\begin{itemize}[noitemsep]
\item \code{True} if the operation succeeded
\item \code{False} if any provided predicates failed.
\item \code{null} if the operation requires an existing value and none exists
\end{itemize}

On error, this function will raise a \code{HyperDexClientException} describing
the error.


%%%%%%%%%%%%%%%%%%%% group_atomic_mod %%%%%%%%%%%%%%%%%%%%
\pagebreak
\subsubsection{\code{group\_atomic\_mod}}
\label{api:nodejs:group_atomic_mod}
\index{group\_atomic\_mod!Node.js API}
Store the existing value modulo the specified number for each object in
\code{space} that matches \code{checks}.

This operation will only affect objects that match the provided \code{checks}.
Objects that do not match \code{checks} will be unaffected by the group call.
Each object that matches \code{checks} will be atomically updated with the check
on the object.  HyperDex guarantees that no object will be altered if the
\code{checks} do not pass at the time of the write.  Objects that are updated
concurrently with the group call may or may not be updated; however, regardless
of any other concurrent operations, the preceding guarantee will always hold.



\paragraph{Definition:}
\begin{javascriptcode}
group_atomic_mod(spacename, predicates, attributes, function (count, err) {})
\end{javascriptcode}
\paragraph{Parameters:}
\begin{itemize}[noitemsep]
\item \code{spacename}\\
The name of the space as a string or symbol.

\item \code{predicates}\\
A hash of predicates to check against.

\item \code{attributes}\\
The set of attributes to modify and their respective values.  \code{attrs} is a
map from the attributes' names to their values.

\end{itemize}

\paragraph{Returns:}
A count of the number of objects, and a \code{client.Error} object indicating
the status of the operation.


%%%%%%%%%%%%%%%%%%%% atomic_and %%%%%%%%%%%%%%%%%%%%
\pagebreak
\subsubsection{\code{atomic\_and}}
\label{api:nodejs:atomic_and}
\index{atomic\_and!Node.js API}
Store the bitwise AND of the existing value and the specified number for
each attribute.
This operation requires a pre-existing object in order to complete successfully.
If no object exists, the operation will fail with \code{NOTFOUND}.



\paragraph{Definition:}
\begin{javascriptcode}
atomic_and(spacename, key, attributes, function (success, err) {})
\end{javascriptcode}
\paragraph{Parameters:}
\begin{itemize}[noitemsep]
\item \code{spacename}\\
The name of the space as a string or symbol.

\item \code{key}\\
The key for the operation as a Python type.

\item \code{attributes}\\
The set of attributes to modify and their respective values.  \code{attrs} is a
map from the attributes' names to their values.

\end{itemize}

\paragraph{Returns:}
This function returns an object indicating the success or failure of the
operation.  Valid values to be returned are:

\begin{itemize}[noitemsep]
\item \code{True} if the operation succeeded
\item \code{False} if any provided predicates failed.
\item \code{null} if the operation requires an existing value and none exists
\end{itemize}

On error, this function will raise a \code{HyperDexClientException} describing
the error.


%%%%%%%%%%%%%%%%%%%% cond_atomic_and %%%%%%%%%%%%%%%%%%%%
\pagebreak
\subsubsection{\code{cond\_atomic\_and}}
\label{api:nodejs:cond_atomic_and}
\index{cond\_atomic\_and!Node.js API}
Conditionally store the bitwise AND of the existing value and the specified
number for each attribute.

%%% Generated below here
\paragraph{Behavior:}
\begin{itemize}[noitemsep]
This operation requires a pre-existing object in order to complete successfully.
If no object exists, the operation will fail with \code{NOTFOUND}.

This operation will succeed if and only if the predicates specified by
\code{checks} hold on the pre-existing object.  If any of the predicates are not
true for the existing object, then the operation will have no effect and fail
with \code{CMPFAIL}.

All checks are atomic with the write.  HyperDex guarantees that no other
operation will come between validating the checks, and writing the new version
of the object.

\end{itemize}


\paragraph{Definition:}
\begin{javascriptcode}
cond_atomic_and(spacename, key, predicates, attributes, function (success, err) {})
\end{javascriptcode}
\paragraph{Parameters:}
\begin{itemize}[noitemsep]
\item \code{spacename}\\
The name of the space as a string or symbol.

\item \code{key}\\
The key for the operation as a Python type.

\item \code{predicates}\\
A hash of predicates to check against.

\item \code{attributes}\\
The set of attributes to modify and their respective values.  \code{attrs} is a
map from the attributes' names to their values.

\end{itemize}

\paragraph{Returns:}
This function returns an object indicating the success or failure of the
operation.  Valid values to be returned are:

\begin{itemize}[noitemsep]
\item \code{True} if the operation succeeded
\item \code{False} if any provided predicates failed.
\item \code{null} if the operation requires an existing value and none exists
\end{itemize}

On error, this function will raise a \code{HyperDexClientException} describing
the error.


%%%%%%%%%%%%%%%%%%%% group_atomic_and %%%%%%%%%%%%%%%%%%%%
\pagebreak
\subsubsection{\code{group\_atomic\_and}}
\label{api:nodejs:group_atomic_and}
\index{group\_atomic\_and!Node.js API}
Store the bitwise AND of the existing value and the specified number for
each object in \code{space} that matches \code{checks}.

This operation will only affect objects that match the provided \code{checks}.
Objects that do not match \code{checks} will be unaffected by the group call.
Each object that matches \code{checks} will be atomically updated with the check
on the object.  HyperDex guarantees that no object will be altered if the
\code{checks} do not pass at the time of the write.  Objects that are updated
concurrently with the group call may or may not be updated; however, regardless
of any other concurrent operations, the preceding guarantee will always hold.



\paragraph{Definition:}
\begin{javascriptcode}
group_atomic_and(spacename, predicates, attributes, function (count, err) {})
\end{javascriptcode}
\paragraph{Parameters:}
\begin{itemize}[noitemsep]
\item \code{spacename}\\
The name of the space as a string or symbol.

\item \code{predicates}\\
A hash of predicates to check against.

\item \code{attributes}\\
The set of attributes to modify and their respective values.  \code{attrs} is a
map from the attributes' names to their values.

\end{itemize}

\paragraph{Returns:}
A count of the number of objects, and a \code{client.Error} object indicating
the status of the operation.


%%%%%%%%%%%%%%%%%%%% atomic_or %%%%%%%%%%%%%%%%%%%%
\pagebreak
\subsubsection{\code{atomic\_or}}
\label{api:nodejs:atomic_or}
\index{atomic\_or!Node.js API}
Store the bitwise OR of the existing value and the specified number for each
attribute.

%%% Generated below here
\paragraph{Behavior:}
\begin{itemize}[noitemsep]
This operation requires a pre-existing object in order to complete successfully.
If no object exists, the operation will fail with \code{NOTFOUND}.

\end{itemize}


\paragraph{Definition:}
\begin{javascriptcode}
atomic_or(spacename, key, attributes, function (success, err) {})
\end{javascriptcode}
\paragraph{Parameters:}
\begin{itemize}[noitemsep]
\item \code{spacename}\\
The name of the space as a string or symbol.

\item \code{key}\\
The key for the operation as a Python type.

\item \code{attributes}\\
The set of attributes to modify and their respective values.  \code{attrs} is a
map from the attributes' names to their values.

\end{itemize}

\paragraph{Returns:}
This function returns an object indicating the success or failure of the
operation.  Valid values to be returned are:

\begin{itemize}[noitemsep]
\item \code{True} if the operation succeeded
\item \code{False} if any provided predicates failed.
\item \code{null} if the operation requires an existing value and none exists
\end{itemize}

On error, this function will raise a \code{HyperDexClientException} describing
the error.


%%%%%%%%%%%%%%%%%%%% cond_atomic_or %%%%%%%%%%%%%%%%%%%%
\pagebreak
\subsubsection{\code{cond\_atomic\_or}}
\label{api:nodejs:cond_atomic_or}
\index{cond\_atomic\_or!Node.js API}
Conditionally store the bitwise OR of the existing value and the specified
number for each attribute.

%%% Generated below here
\paragraph{Behavior:}
\begin{itemize}[noitemsep]
This operation requires a pre-existing object in order to complete successfully.
If no object exists, the operation will fail with \code{NOTFOUND}.

This operation will succeed if and only if the predicates specified by
\code{checks} hold on the pre-existing object.  If any of the predicates are not
true for the existing object, then the operation will have no effect and fail
with \code{CMPFAIL}.

All checks are atomic with the write.  HyperDex guarantees that no other
operation will come between validating the checks, and writing the new version
of the object.

\end{itemize}


\paragraph{Definition:}
\begin{javascriptcode}
cond_atomic_or(spacename, key, predicates, attributes, function (success, err) {})
\end{javascriptcode}
\paragraph{Parameters:}
\begin{itemize}[noitemsep]
\item \code{spacename}\\
The name of the space as a string or symbol.

\item \code{key}\\
The key for the operation as a Python type.

\item \code{predicates}\\
A hash of predicates to check against.

\item \code{attributes}\\
The set of attributes to modify and their respective values.  \code{attrs} is a
map from the attributes' names to their values.

\end{itemize}

\paragraph{Returns:}
This function returns an object indicating the success or failure of the
operation.  Valid values to be returned are:

\begin{itemize}[noitemsep]
\item \code{True} if the operation succeeded
\item \code{False} if any provided predicates failed.
\item \code{null} if the operation requires an existing value and none exists
\end{itemize}

On error, this function will raise a \code{HyperDexClientException} describing
the error.


%%%%%%%%%%%%%%%%%%%% group_atomic_or %%%%%%%%%%%%%%%%%%%%
\pagebreak
\subsubsection{\code{group\_atomic\_or}}
\label{api:nodejs:group_atomic_or}
\index{group\_atomic\_or!Node.js API}
Store the bitwise OR of the existing value and the specified number for each
object in \code{space} that matches \code{checks}.

This operation will only affect objects that match the provided \code{checks}.
Objects that do not match \code{checks} will be unaffected by the group call.
Each object that matches \code{checks} will be atomically updated with the check
on the object.  HyperDex guarantees that no object will be altered if the
\code{checks} do not pass at the time of the write.  Objects that are updated
concurrently with the group call may or may not be updated; however, regardless
of any other concurrent operations, the preceding guarantee will always hold.



\paragraph{Definition:}
\begin{javascriptcode}
group_atomic_or(spacename, predicates, attributes, function (count, err) {})
\end{javascriptcode}
\paragraph{Parameters:}
\begin{itemize}[noitemsep]
\item \code{spacename}\\
The name of the space as a string or symbol.

\item \code{predicates}\\
A hash of predicates to check against.

\item \code{attributes}\\
The set of attributes to modify and their respective values.  \code{attrs} is a
map from the attributes' names to their values.

\end{itemize}

\paragraph{Returns:}
A count of the number of objects, and a \code{client.Error} object indicating
the status of the operation.


%%%%%%%%%%%%%%%%%%%% atomic_xor %%%%%%%%%%%%%%%%%%%%
\pagebreak
\subsubsection{\code{atomic\_xor}}
\label{api:nodejs:atomic_xor}
\index{atomic\_xor!Node.js API}
Store the bitwise XOR of the existing value and the specified number for each
attribute.

%%% Generated below here
\paragraph{Behavior:}
\begin{itemize}[noitemsep]
This operation requires a pre-existing object in order to complete successfully.
If no object exists, the operation will fail with \code{NOTFOUND}.

\end{itemize}


\paragraph{Definition:}
\begin{javascriptcode}
atomic_xor(spacename, key, attributes, function (success, err) {})
\end{javascriptcode}
\paragraph{Parameters:}
\begin{itemize}[noitemsep]
\item \code{spacename}\\
The name of the space as a string or symbol.

\item \code{key}\\
The key for the operation as a Python type.

\item \code{attributes}\\
The set of attributes to modify and their respective values.  \code{attrs} is a
map from the attributes' names to their values.

\end{itemize}

\paragraph{Returns:}
This function returns an object indicating the success or failure of the
operation.  Valid values to be returned are:

\begin{itemize}[noitemsep]
\item \code{True} if the operation succeeded
\item \code{False} if any provided predicates failed.
\item \code{null} if the operation requires an existing value and none exists
\end{itemize}

On error, this function will raise a \code{HyperDexClientException} describing
the error.


%%%%%%%%%%%%%%%%%%%% cond_atomic_xor %%%%%%%%%%%%%%%%%%%%
\pagebreak
\subsubsection{\code{cond\_atomic\_xor}}
\label{api:nodejs:cond_atomic_xor}
\index{cond\_atomic\_xor!Node.js API}
Conditionally store the bitwise XOR of the existing value and the specified
number for each attribute.

%%% Generated below here
\paragraph{Behavior:}
\begin{itemize}[noitemsep]
This operation requires a pre-existing object in order to complete successfully.
If no object exists, the operation will fail with \code{NOTFOUND}.

This operation will succeed if and only if the predicates specified by
\code{checks} hold on the pre-existing object.  If any of the predicates are not
true for the existing object, then the operation will have no effect and fail
with \code{CMPFAIL}.

All checks are atomic with the write.  HyperDex guarantees that no other
operation will come between validating the checks, and writing the new version
of the object.

\end{itemize}


\paragraph{Definition:}
\begin{javascriptcode}
cond_atomic_xor(spacename, key, predicates, attributes, function (success, err) {})
\end{javascriptcode}
\paragraph{Parameters:}
\begin{itemize}[noitemsep]
\item \code{spacename}\\
The name of the space as a string or symbol.

\item \code{key}\\
The key for the operation as a Python type.

\item \code{predicates}\\
A hash of predicates to check against.

\item \code{attributes}\\
The set of attributes to modify and their respective values.  \code{attrs} is a
map from the attributes' names to their values.

\end{itemize}

\paragraph{Returns:}
This function returns an object indicating the success or failure of the
operation.  Valid values to be returned are:

\begin{itemize}[noitemsep]
\item \code{True} if the operation succeeded
\item \code{False} if any provided predicates failed.
\item \code{null} if the operation requires an existing value and none exists
\end{itemize}

On error, this function will raise a \code{HyperDexClientException} describing
the error.


%%%%%%%%%%%%%%%%%%%% group_atomic_xor %%%%%%%%%%%%%%%%%%%%
\pagebreak
\subsubsection{\code{group\_atomic\_xor}}
\label{api:nodejs:group_atomic_xor}
\index{group\_atomic\_xor!Node.js API}
Store the bitwise XOR of the existing value and the specified number for each
object in \code{space} that matches \code{checks}.

This operation will only affect objects that match the provided \code{checks}.
Objects that do not match \code{checks} will be unaffected by the group call.
Each object that matches \code{checks} will be atomically updated with the check
on the object.  HyperDex guarantees that no object will be altered if the
\code{checks} do not pass at the time of the write.  Objects that are updated
concurrently with the group call may or may not be updated; however, regardless
of any other concurrent operations, the preceding guarantee will always hold.



\paragraph{Definition:}
\begin{javascriptcode}
group_atomic_xor(spacename, predicates, attributes, function (count, err) {})
\end{javascriptcode}
\paragraph{Parameters:}
\begin{itemize}[noitemsep]
\item \code{spacename}\\
The name of the space as a string or symbol.

\item \code{predicates}\\
A hash of predicates to check against.

\item \code{attributes}\\
The set of attributes to modify and their respective values.  \code{attrs} is a
map from the attributes' names to their values.

\end{itemize}

\paragraph{Returns:}
A count of the number of objects, and a \code{client.Error} object indicating
the status of the operation.


%%%%%%%%%%%%%%%%%%%% atomic_min %%%%%%%%%%%%%%%%%%%%
\pagebreak
\subsubsection{\code{atomic\_min}}
\label{api:nodejs:atomic_min}
\index{atomic\_min!Node.js API}
Store the minimum of the existing value and the provided value for each
attribute.
This operation requires a pre-existing object in order to complete successfully.
If no object exists, the operation will fail with \code{NOTFOUND}.



\paragraph{Definition:}
\begin{javascriptcode}
atomic_min(spacename, key, attributes, function (success, err) {})
\end{javascriptcode}
\paragraph{Parameters:}
\begin{itemize}[noitemsep]
\item \code{spacename}\\
The name of the space as a string or symbol.

\item \code{key}\\
The key for the operation as a Python type.

\item \code{attributes}\\
The set of attributes to modify and their respective values.  \code{attrs} is a
map from the attributes' names to their values.

\end{itemize}

\paragraph{Returns:}
This function returns an object indicating the success or failure of the
operation.  Valid values to be returned are:

\begin{itemize}[noitemsep]
\item \code{True} if the operation succeeded
\item \code{False} if any provided predicates failed.
\item \code{null} if the operation requires an existing value and none exists
\end{itemize}

On error, this function will raise a \code{HyperDexClientException} describing
the error.


%%%%%%%%%%%%%%%%%%%% cond_atomic_min %%%%%%%%%%%%%%%%%%%%
\pagebreak
\subsubsection{\code{cond\_atomic\_min}}
\label{api:nodejs:cond_atomic_min}
\index{cond\_atomic\_min!Node.js API}
Store the minimum of the existing value and the provided value for each
attribute if and only if \code{checks} hold on the object.
This operation requires a pre-existing object in order to complete successfully.
If no object exists, the operation will fail with \code{NOTFOUND}.


This operation will succeed if and only if the predicates specified by
\code{checks} hold on the pre-existing object.  If any of the predicates are not
true for the existing object, then the operation will have no effect and fail
with \code{CMPFAIL}.

All checks are atomic with the write.  HyperDex guarantees that no other
operation will come between validating the checks, and writing the new version
of the object.



\paragraph{Definition:}
\begin{javascriptcode}
cond_atomic_min(spacename, key, predicates, attributes, function (success, err) {})
\end{javascriptcode}
\paragraph{Parameters:}
\begin{itemize}[noitemsep]
\item \code{spacename}\\
The name of the space as a string or symbol.

\item \code{key}\\
The key for the operation as a Python type.

\item \code{predicates}\\
A hash of predicates to check against.

\item \code{attributes}\\
The set of attributes to modify and their respective values.  \code{attrs} is a
map from the attributes' names to their values.

\end{itemize}

\paragraph{Returns:}
This function returns an object indicating the success or failure of the
operation.  Valid values to be returned are:

\begin{itemize}[noitemsep]
\item \code{True} if the operation succeeded
\item \code{False} if any provided predicates failed.
\item \code{null} if the operation requires an existing value and none exists
\end{itemize}

On error, this function will raise a \code{HyperDexClientException} describing
the error.


%%%%%%%%%%%%%%%%%%%% group_atomic_min %%%%%%%%%%%%%%%%%%%%
\pagebreak
\subsubsection{\code{group\_atomic\_min}}
\label{api:nodejs:group_atomic_min}
\index{group\_atomic\_min!Node.js API}
Store the minimum of the existing value and the provided value for each
object in \code{space} that matches \code{checks}.

This operation will only affect objects that match the provided \code{checks}.
Objects that do not match \code{checks} will be unaffected by the group call.
Each object that matches \code{checks} will be atomically updated with the check
on the object.  HyperDex guarantees that no object will be altered if the
\code{checks} do not pass at the time of the write.  Objects that are updated
concurrently with the group call may or may not be updated; however, regardless
of any other concurrent operations, the preceding guarantee will always hold.



\paragraph{Definition:}
\begin{javascriptcode}
group_atomic_min(spacename, predicates, attributes, function (count, err) {})
\end{javascriptcode}
\paragraph{Parameters:}
\begin{itemize}[noitemsep]
\item \code{spacename}\\
The name of the space as a string or symbol.

\item \code{predicates}\\
A hash of predicates to check against.

\item \code{attributes}\\
The set of attributes to modify and their respective values.  \code{attrs} is a
map from the attributes' names to their values.

\end{itemize}

\paragraph{Returns:}
A count of the number of objects, and a \code{client.Error} object indicating
the status of the operation.


%%%%%%%%%%%%%%%%%%%% atomic_max %%%%%%%%%%%%%%%%%%%%
\pagebreak
\subsubsection{\code{atomic\_max}}
\label{api:nodejs:atomic_max}
\index{atomic\_max!Node.js API}
Store the maximum of the existing value and the provided value for each
attribute.
This operation requires a pre-existing object in order to complete successfully.
If no object exists, the operation will fail with \code{NOTFOUND}.



\paragraph{Definition:}
\begin{javascriptcode}
atomic_max(spacename, key, attributes, function (success, err) {})
\end{javascriptcode}
\paragraph{Parameters:}
\begin{itemize}[noitemsep]
\item \code{spacename}\\
The name of the space as a string or symbol.

\item \code{key}\\
The key for the operation as a Python type.

\item \code{attributes}\\
The set of attributes to modify and their respective values.  \code{attrs} is a
map from the attributes' names to their values.

\end{itemize}

\paragraph{Returns:}
This function returns an object indicating the success or failure of the
operation.  Valid values to be returned are:

\begin{itemize}[noitemsep]
\item \code{True} if the operation succeeded
\item \code{False} if any provided predicates failed.
\item \code{null} if the operation requires an existing value and none exists
\end{itemize}

On error, this function will raise a \code{HyperDexClientException} describing
the error.


%%%%%%%%%%%%%%%%%%%% cond_atomic_max %%%%%%%%%%%%%%%%%%%%
\pagebreak
\subsubsection{\code{cond\_atomic\_max}}
\label{api:nodejs:cond_atomic_max}
\index{cond\_atomic\_max!Node.js API}
Store the maximum of the existing value and the provided value for each
attribute if and only if \code{checks} hold on the object.
This operation requires a pre-existing object in order to complete successfully.
If no object exists, the operation will fail with \code{NOTFOUND}.


This operation will succeed if and only if the predicates specified by
\code{checks} hold on the pre-existing object.  If any of the predicates are not
true for the existing object, then the operation will have no effect and fail
with \code{CMPFAIL}.

All checks are atomic with the write.  HyperDex guarantees that no other
operation will come between validating the checks, and writing the new version
of the object.



\paragraph{Definition:}
\begin{javascriptcode}
cond_atomic_max(spacename, key, predicates, attributes, function (success, err) {})
\end{javascriptcode}
\paragraph{Parameters:}
\begin{itemize}[noitemsep]
\item \code{spacename}\\
The name of the space as a string or symbol.

\item \code{key}\\
The key for the operation as a Python type.

\item \code{predicates}\\
A hash of predicates to check against.

\item \code{attributes}\\
The set of attributes to modify and their respective values.  \code{attrs} is a
map from the attributes' names to their values.

\end{itemize}

\paragraph{Returns:}
This function returns an object indicating the success or failure of the
operation.  Valid values to be returned are:

\begin{itemize}[noitemsep]
\item \code{True} if the operation succeeded
\item \code{False} if any provided predicates failed.
\item \code{null} if the operation requires an existing value and none exists
\end{itemize}

On error, this function will raise a \code{HyperDexClientException} describing
the error.


%%%%%%%%%%%%%%%%%%%% group_atomic_max %%%%%%%%%%%%%%%%%%%%
\pagebreak
\subsubsection{\code{group\_atomic\_max}}
\label{api:nodejs:group_atomic_max}
\index{group\_atomic\_max!Node.js API}
Store the maximum of the existing value and the provided value for each
object in \code{space} that matches \code{checks}.

This operation will only affect objects that match the provided \code{checks}.
Objects that do not match \code{checks} will be unaffected by the group call.
Each object that matches \code{checks} will be atomically updated with the check
on the object.  HyperDex guarantees that no object will be altered if the
\code{checks} do not pass at the time of the write.  Objects that are updated
concurrently with the group call may or may not be updated; however, regardless
of any other concurrent operations, the preceding guarantee will always hold.



\paragraph{Definition:}
\begin{javascriptcode}
group_atomic_max(spacename, predicates, attributes, function (count, err) {})
\end{javascriptcode}
\paragraph{Parameters:}
\begin{itemize}[noitemsep]
\item \code{spacename}\\
The name of the space as a string or symbol.

\item \code{predicates}\\
A hash of predicates to check against.

\item \code{attributes}\\
The set of attributes to modify and their respective values.  \code{attrs} is a
map from the attributes' names to their values.

\end{itemize}

\paragraph{Returns:}
A count of the number of objects, and a \code{client.Error} object indicating
the status of the operation.


%%%%%%%%%%%%%%%%%%%% string_prepend %%%%%%%%%%%%%%%%%%%%
\pagebreak
\subsubsection{\code{string\_prepend}}
\label{api:nodejs:string_prepend}
\index{string\_prepend!Node.js API}
Prepend the specified string to the existing value for each attribute.

%%% Generated below here
\paragraph{Behavior:}
\begin{itemize}[noitemsep]
This operation requires a pre-existing object in order to complete successfully.
If no object exists, the operation will fail with \code{NOTFOUND}.

\end{itemize}


\paragraph{Definition:}
\begin{javascriptcode}
string_prepend(spacename, key, attributes, function (success, err) {})
\end{javascriptcode}
\paragraph{Parameters:}
\begin{itemize}[noitemsep]
\item \code{spacename}\\
The name of the space as a string or symbol.

\item \code{key}\\
The key for the operation as a Python type.

\item \code{attributes}\\
The set of attributes to modify and their respective values.  \code{attrs} is a
map from the attributes' names to their values.

\end{itemize}

\paragraph{Returns:}
This function returns an object indicating the success or failure of the
operation.  Valid values to be returned are:

\begin{itemize}[noitemsep]
\item \code{True} if the operation succeeded
\item \code{False} if any provided predicates failed.
\item \code{null} if the operation requires an existing value and none exists
\end{itemize}

On error, this function will raise a \code{HyperDexClientException} describing
the error.


%%%%%%%%%%%%%%%%%%%% cond_string_prepend %%%%%%%%%%%%%%%%%%%%
\pagebreak
\subsubsection{\code{cond\_string\_prepend}}
\label{api:nodejs:cond_string_prepend}
\index{cond\_string\_prepend!Node.js API}
Conditionally prepend the specified string to the existing value for each
attribute.

%%% Generated below here
\paragraph{Behavior:}
\begin{itemize}[noitemsep]
This operation requires a pre-existing object in order to complete successfully.
If no object exists, the operation will fail with \code{NOTFOUND}.

This operation will succeed if and only if the predicates specified by
\code{checks} hold on the pre-existing object.  If any of the predicates are not
true for the existing object, then the operation will have no effect and fail
with \code{CMPFAIL}.

All checks are atomic with the write.  HyperDex guarantees that no other
operation will come between validating the checks, and writing the new version
of the object.

\end{itemize}


\paragraph{Definition:}
\begin{javascriptcode}
cond_string_prepend(
        spacename, key, predicates, attributes, function (success, err) {})
\end{javascriptcode}
\paragraph{Parameters:}
\begin{itemize}[noitemsep]
\item \code{spacename}\\
The name of the space as a string or symbol.

\item \code{key}\\
The key for the operation as a Python type.

\item \code{predicates}\\
A hash of predicates to check against.

\item \code{attributes}\\
The set of attributes to modify and their respective values.  \code{attrs} is a
map from the attributes' names to their values.

\end{itemize}

\paragraph{Returns:}
This function returns an object indicating the success or failure of the
operation.  Valid values to be returned are:

\begin{itemize}[noitemsep]
\item \code{True} if the operation succeeded
\item \code{False} if any provided predicates failed.
\item \code{null} if the operation requires an existing value and none exists
\end{itemize}

On error, this function will raise a \code{HyperDexClientException} describing
the error.


%%%%%%%%%%%%%%%%%%%% group_string_prepend %%%%%%%%%%%%%%%%%%%%
\pagebreak
\subsubsection{\code{group\_string\_prepend}}
\label{api:nodejs:group_string_prepend}
\index{group\_string\_prepend!Node.js API}
Prepend the specified string to the existing value for each object in
\code{space} that matches \code{checks}.

This operation will only affect objects that match the provided \code{checks}.
Objects that do not match \code{checks} will be unaffected by the group call.
Each object that matches \code{checks} will be atomically updated with the check
on the object.  HyperDex guarantees that no object will be altered if the
\code{checks} do not pass at the time of the write.  Objects that are updated
concurrently with the group call may or may not be updated; however, regardless
of any other concurrent operations, the preceding guarantee will always hold.



\paragraph{Definition:}
\begin{javascriptcode}
group_string_prepend(spacename, predicates, attributes, function (count, err) {})
\end{javascriptcode}
\paragraph{Parameters:}
\begin{itemize}[noitemsep]
\item \code{spacename}\\
The name of the space as a string or symbol.

\item \code{predicates}\\
A hash of predicates to check against.

\item \code{attributes}\\
The set of attributes to modify and their respective values.  \code{attrs} is a
map from the attributes' names to their values.

\end{itemize}

\paragraph{Returns:}
A count of the number of objects, and a \code{client.Error} object indicating
the status of the operation.


%%%%%%%%%%%%%%%%%%%% string_append %%%%%%%%%%%%%%%%%%%%
\pagebreak
\subsubsection{\code{string\_append}}
\label{api:nodejs:string_append}
\index{string\_append!Node.js API}
Append the specified string to the existing value for each attribute.

%%% Generated below here
\paragraph{Behavior:}
\begin{itemize}[noitemsep]
This operation requires a pre-existing object in order to complete successfully.
If no object exists, the operation will fail with \code{NOTFOUND}.

\end{itemize}


\paragraph{Definition:}
\begin{javascriptcode}
string_append(spacename, key, attributes, function (success, err) {})
\end{javascriptcode}
\paragraph{Parameters:}
\begin{itemize}[noitemsep]
\item \code{spacename}\\
The name of the space as a string or symbol.

\item \code{key}\\
The key for the operation as a Python type.

\item \code{attributes}\\
The set of attributes to modify and their respective values.  \code{attrs} is a
map from the attributes' names to their values.

\end{itemize}

\paragraph{Returns:}
This function returns an object indicating the success or failure of the
operation.  Valid values to be returned are:

\begin{itemize}[noitemsep]
\item \code{True} if the operation succeeded
\item \code{False} if any provided predicates failed.
\item \code{null} if the operation requires an existing value and none exists
\end{itemize}

On error, this function will raise a \code{HyperDexClientException} describing
the error.


%%%%%%%%%%%%%%%%%%%% cond_string_append %%%%%%%%%%%%%%%%%%%%
\pagebreak
\subsubsection{\code{cond\_string\_append}}
\label{api:nodejs:cond_string_append}
\index{cond\_string\_append!Node.js API}
Conditionally append the specified string to the existing value for each
attribute.

%%% Generated below here
\paragraph{Behavior:}
\begin{itemize}[noitemsep]
This operation requires a pre-existing object in order to complete successfully.
If no object exists, the operation will fail with \code{NOTFOUND}.

This operation will succeed if and only if the predicates specified by
\code{checks} hold on the pre-existing object.  If any of the predicates are not
true for the existing object, then the operation will have no effect and fail
with \code{CMPFAIL}.

All checks are atomic with the write.  HyperDex guarantees that no other
operation will come between validating the checks, and writing the new version
of the object.

\end{itemize}


\paragraph{Definition:}
\begin{javascriptcode}
cond_string_append(
        spacename, key, predicates, attributes, function (success, err) {})
\end{javascriptcode}
\paragraph{Parameters:}
\begin{itemize}[noitemsep]
\item \code{spacename}\\
The name of the space as a string or symbol.

\item \code{key}\\
The key for the operation as a Python type.

\item \code{predicates}\\
A hash of predicates to check against.

\item \code{attributes}\\
The set of attributes to modify and their respective values.  \code{attrs} is a
map from the attributes' names to their values.

\end{itemize}

\paragraph{Returns:}
This function returns an object indicating the success or failure of the
operation.  Valid values to be returned are:

\begin{itemize}[noitemsep]
\item \code{True} if the operation succeeded
\item \code{False} if any provided predicates failed.
\item \code{null} if the operation requires an existing value and none exists
\end{itemize}

On error, this function will raise a \code{HyperDexClientException} describing
the error.


%%%%%%%%%%%%%%%%%%%% group_string_append %%%%%%%%%%%%%%%%%%%%
\pagebreak
\subsubsection{\code{group\_string\_append}}
\label{api:nodejs:group_string_append}
\index{group\_string\_append!Node.js API}
Append the specified string to the existing value for each object in
\code{space} that matches \code{checks}.

This operation will only affect objects that match the provided \code{checks}.
Objects that do not match \code{checks} will be unaffected by the group call.
Each object that matches \code{checks} will be atomically updated with the check
on the object.  HyperDex guarantees that no object will be altered if the
\code{checks} do not pass at the time of the write.  Objects that are updated
concurrently with the group call may or may not be updated; however, regardless
of any other concurrent operations, the preceding guarantee will always hold.



\paragraph{Definition:}
\begin{javascriptcode}
group_string_append(spacename, predicates, attributes, function (count, err) {})
\end{javascriptcode}
\paragraph{Parameters:}
\begin{itemize}[noitemsep]
\item \code{spacename}\\
The name of the space as a string or symbol.

\item \code{predicates}\\
A hash of predicates to check against.

\item \code{attributes}\\
The set of attributes to modify and their respective values.  \code{attrs} is a
map from the attributes' names to their values.

\end{itemize}

\paragraph{Returns:}
A count of the number of objects, and a \code{client.Error} object indicating
the status of the operation.


%%%%%%%%%%%%%%%%%%%% list_lpush %%%%%%%%%%%%%%%%%%%%
\pagebreak
\subsubsection{\code{list\_lpush}}
\label{api:nodejs:list_lpush}
\index{list\_lpush!Node.js API}
Push the specified value onto the front of the list for each attribute.

%%% Generated below here
\paragraph{Behavior:}
\begin{itemize}[noitemsep]
This operation requires a pre-existing object in order to complete successfully.
If no object exists, the operation will fail with \code{NOTFOUND}.

\end{itemize}


\paragraph{Definition:}
\begin{javascriptcode}
list_lpush(spacename, key, attributes, function (success, err) {})
\end{javascriptcode}
\paragraph{Parameters:}
\begin{itemize}[noitemsep]
\item \code{spacename}\\
The name of the space as a string or symbol.

\item \code{key}\\
The key for the operation as a Python type.

\item \code{attributes}\\
The set of attributes to modify and their respective values.  \code{attrs} is a
map from the attributes' names to their values.

\end{itemize}

\paragraph{Returns:}
This function returns an object indicating the success or failure of the
operation.  Valid values to be returned are:

\begin{itemize}[noitemsep]
\item \code{True} if the operation succeeded
\item \code{False} if any provided predicates failed.
\item \code{null} if the operation requires an existing value and none exists
\end{itemize}

On error, this function will raise a \code{HyperDexClientException} describing
the error.


%%%%%%%%%%%%%%%%%%%% cond_list_lpush %%%%%%%%%%%%%%%%%%%%
\pagebreak
\subsubsection{\code{cond\_list\_lpush}}
\label{api:nodejs:cond_list_lpush}
\index{cond\_list\_lpush!Node.js API}
Condtitionally push the specified value onto the front of the list for each
attribute.

%%% Generated below here
\paragraph{Behavior:}
\begin{itemize}[noitemsep]
This operation requires a pre-existing object in order to complete successfully.
If no object exists, the operation will fail with \code{NOTFOUND}.

This operation will succeed if and only if the predicates specified by
\code{checks} hold on the pre-existing object.  If any of the predicates are not
true for the existing object, then the operation will have no effect and fail
with \code{CMPFAIL}.

All checks are atomic with the write.  HyperDex guarantees that no other
operation will come between validating the checks, and writing the new version
of the object.

\end{itemize}


\paragraph{Definition:}
\begin{javascriptcode}
cond_list_lpush(spacename, key, predicates, attributes, function (success, err) {})
\end{javascriptcode}
\paragraph{Parameters:}
\begin{itemize}[noitemsep]
\item \code{spacename}\\
The name of the space as a string or symbol.

\item \code{key}\\
The key for the operation as a Python type.

\item \code{predicates}\\
A hash of predicates to check against.

\item \code{attributes}\\
The set of attributes to modify and their respective values.  \code{attrs} is a
map from the attributes' names to their values.

\end{itemize}

\paragraph{Returns:}
This function returns an object indicating the success or failure of the
operation.  Valid values to be returned are:

\begin{itemize}[noitemsep]
\item \code{True} if the operation succeeded
\item \code{False} if any provided predicates failed.
\item \code{null} if the operation requires an existing value and none exists
\end{itemize}

On error, this function will raise a \code{HyperDexClientException} describing
the error.


%%%%%%%%%%%%%%%%%%%% group_list_lpush %%%%%%%%%%%%%%%%%%%%
\pagebreak
\subsubsection{\code{group\_list\_lpush}}
\label{api:nodejs:group_list_lpush}
\index{group\_list\_lpush!Node.js API}
Push the specified value onto the front of the list for each object in
\code{space} that matches \code{checks}.

This operation will only affect objects that match the provided \code{checks}.
Objects that do not match \code{checks} will be unaffected by the group call.
Each object that matches \code{checks} will be atomically updated with the check
on the object.  HyperDex guarantees that no object will be altered if the
\code{checks} do not pass at the time of the write.  Objects that are updated
concurrently with the group call may or may not be updated; however, regardless
of any other concurrent operations, the preceding guarantee will always hold.



\paragraph{Definition:}
\begin{javascriptcode}
group_list_lpush(spacename, predicates, attributes, function (count, err) {})
\end{javascriptcode}
\paragraph{Parameters:}
\begin{itemize}[noitemsep]
\item \code{spacename}\\
The name of the space as a string or symbol.

\item \code{predicates}\\
A hash of predicates to check against.

\item \code{attributes}\\
The set of attributes to modify and their respective values.  \code{attrs} is a
map from the attributes' names to their values.

\end{itemize}

\paragraph{Returns:}
A count of the number of objects, and a \code{client.Error} object indicating
the status of the operation.


%%%%%%%%%%%%%%%%%%%% list_rpush %%%%%%%%%%%%%%%%%%%%
\pagebreak
\subsubsection{\code{list\_rpush}}
\label{api:nodejs:list_rpush}
\index{list\_rpush!Node.js API}
Push the specified value onto the back of the list for each attribute.
This operation requires a pre-existing object in order to complete successfully.
If no object exists, the operation will fail with \code{NOTFOUND}.



\paragraph{Definition:}
\begin{javascriptcode}
list_rpush(spacename, key, attributes, function (success, err) {})
\end{javascriptcode}
\paragraph{Parameters:}
\begin{itemize}[noitemsep]
\item \code{spacename}\\
The name of the space as a string or symbol.

\item \code{key}\\
The key for the operation as a Python type.

\item \code{attributes}\\
The set of attributes to modify and their respective values.  \code{attrs} is a
map from the attributes' names to their values.

\end{itemize}

\paragraph{Returns:}
This function returns an object indicating the success or failure of the
operation.  Valid values to be returned are:

\begin{itemize}[noitemsep]
\item \code{True} if the operation succeeded
\item \code{False} if any provided predicates failed.
\item \code{null} if the operation requires an existing value and none exists
\end{itemize}

On error, this function will raise a \code{HyperDexClientException} describing
the error.


%%%%%%%%%%%%%%%%%%%% cond_list_rpush %%%%%%%%%%%%%%%%%%%%
\pagebreak
\subsubsection{\code{cond\_list\_rpush}}
\label{api:nodejs:cond_list_rpush}
\index{cond\_list\_rpush!Node.js API}
Push the specified value onto the back of the list for each attribute if and
only if the \code{checks} hold on the object.
This operation requires a pre-existing object in order to complete successfully.
If no object exists, the operation will fail with \code{NOTFOUND}.


This operation will succeed if and only if the predicates specified by
\code{checks} hold on the pre-existing object.  If any of the predicates are not
true for the existing object, then the operation will have no effect and fail
with \code{CMPFAIL}.

All checks are atomic with the write.  HyperDex guarantees that no other
operation will come between validating the checks, and writing the new version
of the object.



\paragraph{Definition:}
\begin{javascriptcode}
cond_list_rpush(spacename, key, predicates, attributes, function (success, err) {})
\end{javascriptcode}
\paragraph{Parameters:}
\begin{itemize}[noitemsep]
\item \code{spacename}\\
The name of the space as a string or symbol.

\item \code{key}\\
The key for the operation as a Python type.

\item \code{predicates}\\
A hash of predicates to check against.

\item \code{attributes}\\
The set of attributes to modify and their respective values.  \code{attrs} is a
map from the attributes' names to their values.

\end{itemize}

\paragraph{Returns:}
This function returns an object indicating the success or failure of the
operation.  Valid values to be returned are:

\begin{itemize}[noitemsep]
\item \code{True} if the operation succeeded
\item \code{False} if any provided predicates failed.
\item \code{null} if the operation requires an existing value and none exists
\end{itemize}

On error, this function will raise a \code{HyperDexClientException} describing
the error.


%%%%%%%%%%%%%%%%%%%% group_list_rpush %%%%%%%%%%%%%%%%%%%%
\pagebreak
\subsubsection{\code{group\_list\_rpush}}
\label{api:nodejs:group_list_rpush}
\index{group\_list\_rpush!Node.js API}
Push the specified value onto the back of the list for each object in
\code{space} that matches \code{checks}.

This operation will only affect objects that match the provided \code{checks}.
Objects that do not match \code{checks} will be unaffected by the group call.
Each object that matches \code{checks} will be atomically updated with the check
on the object.  HyperDex guarantees that no object will be altered if the
\code{checks} do not pass at the time of the write.  Objects that are updated
concurrently with the group call may or may not be updated; however, regardless
of any other concurrent operations, the preceding guarantee will always hold.



\paragraph{Definition:}
\begin{javascriptcode}
group_list_rpush(spacename, predicates, attributes, function (count, err) {})
\end{javascriptcode}
\paragraph{Parameters:}
\begin{itemize}[noitemsep]
\item \code{spacename}\\
The name of the space as a string or symbol.

\item \code{predicates}\\
A hash of predicates to check against.

\item \code{attributes}\\
The set of attributes to modify and their respective values.  \code{attrs} is a
map from the attributes' names to their values.

\end{itemize}

\paragraph{Returns:}
A count of the number of objects, and a \code{client.Error} object indicating
the status of the operation.


%%%%%%%%%%%%%%%%%%%% set_add %%%%%%%%%%%%%%%%%%%%
\pagebreak
\subsubsection{\code{set\_add}}
\label{api:nodejs:set_add}
\index{set\_add!Node.js API}
Add the specified value to the set for each attribute.

%%% Generated below here
\paragraph{Behavior:}
\begin{itemize}[noitemsep]
This operation requires a pre-existing object in order to complete successfully.
If no object exists, the operation will fail with \code{NOTFOUND}.

\end{itemize}


\paragraph{Definition:}
\begin{javascriptcode}
set_add(spacename, key, attributes, function (success, err) {})
\end{javascriptcode}
\paragraph{Parameters:}
\begin{itemize}[noitemsep]
\item \code{spacename}\\
The name of the space as a string or symbol.

\item \code{key}\\
The key for the operation as a Python type.

\item \code{attributes}\\
The set of attributes to modify and their respective values.  \code{attrs} is a
map from the attributes' names to their values.

\end{itemize}

\paragraph{Returns:}
This function returns an object indicating the success or failure of the
operation.  Valid values to be returned are:

\begin{itemize}[noitemsep]
\item \code{True} if the operation succeeded
\item \code{False} if any provided predicates failed.
\item \code{null} if the operation requires an existing value and none exists
\end{itemize}

On error, this function will raise a \code{HyperDexClientException} describing
the error.


%%%%%%%%%%%%%%%%%%%% cond_set_add %%%%%%%%%%%%%%%%%%%%
\pagebreak
\subsubsection{\code{cond\_set\_add}}
\label{api:nodejs:cond_set_add}
\index{cond\_set\_add!Node.js API}
Add the specified value to the set for each attribute if and only if the
\code{checks} hold on the object.
This operation requires a pre-existing object in order to complete successfully.
If no object exists, the operation will fail with \code{NOTFOUND}.


This operation will succeed if and only if the predicates specified by
\code{checks} hold on the pre-existing object.  If any of the predicates are not
true for the existing object, then the operation will have no effect and fail
with \code{CMPFAIL}.

All checks are atomic with the write.  HyperDex guarantees that no other
operation will come between validating the checks, and writing the new version
of the object.



\paragraph{Definition:}
\begin{javascriptcode}
cond_set_add(spacename, key, predicates, attributes, function (success, err) {})
\end{javascriptcode}
\paragraph{Parameters:}
\begin{itemize}[noitemsep]
\item \code{spacename}\\
The name of the space as a string or symbol.

\item \code{key}\\
The key for the operation as a Python type.

\item \code{predicates}\\
A hash of predicates to check against.

\item \code{attributes}\\
The set of attributes to modify and their respective values.  \code{attrs} is a
map from the attributes' names to their values.

\end{itemize}

\paragraph{Returns:}
This function returns an object indicating the success or failure of the
operation.  Valid values to be returned are:

\begin{itemize}[noitemsep]
\item \code{True} if the operation succeeded
\item \code{False} if any provided predicates failed.
\item \code{null} if the operation requires an existing value and none exists
\end{itemize}

On error, this function will raise a \code{HyperDexClientException} describing
the error.


%%%%%%%%%%%%%%%%%%%% set_remove %%%%%%%%%%%%%%%%%%%%
\pagebreak
\subsubsection{\code{set\_remove}}
\label{api:nodejs:set_remove}
\index{set\_remove!Node.js API}
Remove the specified value from the set.  If the value is not contained within
the set, this operation will do nothing.

%%% Generated below here
\paragraph{Behavior:}
\begin{itemize}[noitemsep]
This operation requires a pre-existing object in order to complete successfully.
If no object exists, the operation will fail with \code{NOTFOUND}.

\end{itemize}


\paragraph{Definition:}
\begin{javascriptcode}
set_remove(spacename, key, attributes, function (success, err) {})
\end{javascriptcode}
\paragraph{Parameters:}
\begin{itemize}[noitemsep]
\item \code{spacename}\\
The name of the space as a string or symbol.

\item \code{key}\\
The key for the operation as a Python type.

\item \code{attributes}\\
The set of attributes to modify and their respective values.  \code{attrs} is a
map from the attributes' names to their values.

\end{itemize}

\paragraph{Returns:}
This function returns an object indicating the success or failure of the
operation.  Valid values to be returned are:

\begin{itemize}[noitemsep]
\item \code{True} if the operation succeeded
\item \code{False} if any provided predicates failed.
\item \code{null} if the operation requires an existing value and none exists
\end{itemize}

On error, this function will raise a \code{HyperDexClientException} describing
the error.


%%%%%%%%%%%%%%%%%%%% cond_set_remove %%%%%%%%%%%%%%%%%%%%
\pagebreak
\subsubsection{\code{cond\_set\_remove}}
\label{api:nodejs:cond_set_remove}
\index{cond\_set\_remove!Node.js API}
Remove the specified value from the set if and only if the \code{checks} hold on
the object.  If the value is not contained within the set, this operation will
do nothing.
This operation requires a pre-existing object in order to complete successfully.
If no object exists, the operation will fail with \code{NOTFOUND}.


This operation will succeed if and only if the predicates specified by
\code{checks} hold on the pre-existing object.  If any of the predicates are not
true for the existing object, then the operation will have no effect and fail
with \code{CMPFAIL}.

All checks are atomic with the write.  HyperDex guarantees that no other
operation will come between validating the checks, and writing the new version
of the object.



\paragraph{Definition:}
\begin{javascriptcode}
cond_set_remove(spacename, key, predicates, attributes, function (success, err) {})
\end{javascriptcode}
\paragraph{Parameters:}
\begin{itemize}[noitemsep]
\item \code{spacename}\\
The name of the space as a string or symbol.

\item \code{key}\\
The key for the operation as a Python type.

\item \code{predicates}\\
A hash of predicates to check against.

\item \code{attributes}\\
The set of attributes to modify and their respective values.  \code{attrs} is a
map from the attributes' names to their values.

\end{itemize}

\paragraph{Returns:}
This function returns an object indicating the success or failure of the
operation.  Valid values to be returned are:

\begin{itemize}[noitemsep]
\item \code{True} if the operation succeeded
\item \code{False} if any provided predicates failed.
\item \code{null} if the operation requires an existing value and none exists
\end{itemize}

On error, this function will raise a \code{HyperDexClientException} describing
the error.


%%%%%%%%%%%%%%%%%%%% set_intersect %%%%%%%%%%%%%%%%%%%%
\pagebreak
\subsubsection{\code{set\_intersect}}
\label{api:nodejs:set_intersect}
\index{set\_intersect!Node.js API}
Store the intersection of the specified set and the existing value for each
attribute.
This operation requires a pre-existing object in order to complete successfully.
If no object exists, the operation will fail with \code{NOTFOUND}.



\paragraph{Definition:}
\begin{javascriptcode}
set_intersect(spacename, key, attributes, function (success, err) {})
\end{javascriptcode}
\paragraph{Parameters:}
\begin{itemize}[noitemsep]
\item \code{spacename}\\
The name of the space as a string or symbol.

\item \code{key}\\
The key for the operation as a Python type.

\item \code{attributes}\\
The set of attributes to modify and their respective values.  \code{attrs} is a
map from the attributes' names to their values.

\end{itemize}

\paragraph{Returns:}
This function returns an object indicating the success or failure of the
operation.  Valid values to be returned are:

\begin{itemize}[noitemsep]
\item \code{True} if the operation succeeded
\item \code{False} if any provided predicates failed.
\item \code{null} if the operation requires an existing value and none exists
\end{itemize}

On error, this function will raise a \code{HyperDexClientException} describing
the error.


%%%%%%%%%%%%%%%%%%%% cond_set_intersect %%%%%%%%%%%%%%%%%%%%
\pagebreak
\subsubsection{\code{cond\_set\_intersect}}
\label{api:nodejs:cond_set_intersect}
\index{cond\_set\_intersect!Node.js API}
Store the intersection of the specified set and the existing value for each
attribute if and only if the \code{checks} hold on the object.
This operation requires a pre-existing object in order to complete successfully.
If no object exists, the operation will fail with \code{NOTFOUND}.


This operation will succeed if and only if the predicates specified by
\code{checks} hold on the pre-existing object.  If any of the predicates are not
true for the existing object, then the operation will have no effect and fail
with \code{CMPFAIL}.

All checks are atomic with the write.  HyperDex guarantees that no other
operation will come between validating the checks, and writing the new version
of the object.



\paragraph{Definition:}
\begin{javascriptcode}
cond_set_intersect(
        spacename, key, predicates, attributes, function (success, err) {})
\end{javascriptcode}
\paragraph{Parameters:}
\begin{itemize}[noitemsep]
\item \code{spacename}\\
The name of the space as a string or symbol.

\item \code{key}\\
The key for the operation as a Python type.

\item \code{predicates}\\
A hash of predicates to check against.

\item \code{attributes}\\
The set of attributes to modify and their respective values.  \code{attrs} is a
map from the attributes' names to their values.

\end{itemize}

\paragraph{Returns:}
This function returns an object indicating the success or failure of the
operation.  Valid values to be returned are:

\begin{itemize}[noitemsep]
\item \code{True} if the operation succeeded
\item \code{False} if any provided predicates failed.
\item \code{null} if the operation requires an existing value and none exists
\end{itemize}

On error, this function will raise a \code{HyperDexClientException} describing
the error.


%%%%%%%%%%%%%%%%%%%% set_union %%%%%%%%%%%%%%%%%%%%
\pagebreak
\subsubsection{\code{set\_union}}
\label{api:nodejs:set_union}
\index{set\_union!Node.js API}
Store the union of the specified set and the existing value for each attribute.

%%% Generated below here
\paragraph{Behavior:}
\begin{itemize}[noitemsep]
This operation requires a pre-existing object in order to complete successfully.
If no object exists, the operation will fail with \code{NOTFOUND}.

\end{itemize}


\paragraph{Definition:}
\begin{javascriptcode}
set_union(spacename, key, attributes, function (success, err) {})
\end{javascriptcode}
\paragraph{Parameters:}
\begin{itemize}[noitemsep]
\item \code{spacename}\\
The name of the space as a string or symbol.

\item \code{key}\\
The key for the operation as a Python type.

\item \code{attributes}\\
The set of attributes to modify and their respective values.  \code{attrs} is a
map from the attributes' names to their values.

\end{itemize}

\paragraph{Returns:}
This function returns an object indicating the success or failure of the
operation.  Valid values to be returned are:

\begin{itemize}[noitemsep]
\item \code{True} if the operation succeeded
\item \code{False} if any provided predicates failed.
\item \code{null} if the operation requires an existing value and none exists
\end{itemize}

On error, this function will raise a \code{HyperDexClientException} describing
the error.


%%%%%%%%%%%%%%%%%%%% cond_set_union %%%%%%%%%%%%%%%%%%%%
\pagebreak
\subsubsection{\code{cond\_set\_union}}
\label{api:nodejs:cond_set_union}
\index{cond\_set\_union!Node.js API}
Conditionally store the union of the specified set and the existing value for
each attribute.

%%% Generated below here
\paragraph{Behavior:}
\begin{itemize}[noitemsep]
This operation requires a pre-existing object in order to complete successfully.
If no object exists, the operation will fail with \code{NOTFOUND}.

This operation will succeed if and only if the predicates specified by
\code{checks} hold on the pre-existing object.  If any of the predicates are not
true for the existing object, then the operation will have no effect and fail
with \code{CMPFAIL}.

All checks are atomic with the write.  HyperDex guarantees that no other
operation will come between validating the checks, and writing the new version
of the object.

\end{itemize}


\paragraph{Definition:}
\begin{javascriptcode}
cond_set_union(spacename, key, predicates, attributes, function (success, err) {})
\end{javascriptcode}
\paragraph{Parameters:}
\begin{itemize}[noitemsep]
\item \code{spacename}\\
The name of the space as a string or symbol.

\item \code{key}\\
The key for the operation as a Python type.

\item \code{predicates}\\
A hash of predicates to check against.

\item \code{attributes}\\
The set of attributes to modify and their respective values.  \code{attrs} is a
map from the attributes' names to their values.

\end{itemize}

\paragraph{Returns:}
This function returns an object indicating the success or failure of the
operation.  Valid values to be returned are:

\begin{itemize}[noitemsep]
\item \code{True} if the operation succeeded
\item \code{False} if any provided predicates failed.
\item \code{null} if the operation requires an existing value and none exists
\end{itemize}

On error, this function will raise a \code{HyperDexClientException} describing
the error.


%%%%%%%%%%%%%%%%%%%% map_add %%%%%%%%%%%%%%%%%%%%
\pagebreak
\subsubsection{\code{map\_add}}
\label{api:nodejs:map_add}
\index{map\_add!Node.js API}
Insert a key-value pair into the map specified by each map-attribute.
This operation requires a pre-existing object in order to complete successfully.
If no object exists, the operation will fail with \code{NOTFOUND}.



\paragraph{Definition:}
\begin{javascriptcode}
map_add(spacename, key, mapattributes, function (success, err) {})
\end{javascriptcode}
\paragraph{Parameters:}
\begin{itemize}[noitemsep]
\item \code{spacename}\\
The name of the space as a string or symbol.

\item \code{key}\\
The key for the operation as a Python type.

\item \code{mapattributes}\\
A hash specifying map attributes to modify and their respective key/values.

\end{itemize}

\paragraph{Returns:}
This function returns an object indicating the success or failure of the
operation.  Valid values to be returned are:

\begin{itemize}[noitemsep]
\item \code{True} if the operation succeeded
\item \code{False} if any provided predicates failed.
\item \code{null} if the operation requires an existing value and none exists
\end{itemize}

On error, this function will raise a \code{HyperDexClientException} describing
the error.


%%%%%%%%%%%%%%%%%%%% cond_map_add %%%%%%%%%%%%%%%%%%%%
\pagebreak
\subsubsection{\code{cond\_map\_add}}
\label{api:nodejs:cond_map_add}
\index{cond\_map\_add!Node.js API}
Insert a key-value pair into the map specified by each map-attribute if and only
if the \code{checks} hold on the object.
This operation requires a pre-existing object in order to complete successfully.
If no object exists, the operation will fail with \code{NOTFOUND}.


This operation will succeed if and only if the predicates specified by
\code{checks} hold on the pre-existing object.  If any of the predicates are not
true for the existing object, then the operation will have no effect and fail
with \code{CMPFAIL}.

All checks are atomic with the write.  HyperDex guarantees that no other
operation will come between validating the checks, and writing the new version
of the object.



\paragraph{Definition:}
\begin{javascriptcode}
cond_map_add(spacename, key, predicates, mapattributes, function (success, err) {})
\end{javascriptcode}
\paragraph{Parameters:}
\begin{itemize}[noitemsep]
\item \code{spacename}\\
The name of the space as a string or symbol.

\item \code{key}\\
The key for the operation as a Python type.

\item \code{predicates}\\
A hash of predicates to check against.

\item \code{mapattributes}\\
A hash specifying map attributes to modify and their respective key/values.

\end{itemize}

\paragraph{Returns:}
This function returns an object indicating the success or failure of the
operation.  Valid values to be returned are:

\begin{itemize}[noitemsep]
\item \code{True} if the operation succeeded
\item \code{False} if any provided predicates failed.
\item \code{null} if the operation requires an existing value and none exists
\end{itemize}

On error, this function will raise a \code{HyperDexClientException} describing
the error.


%%%%%%%%%%%%%%%%%%%% map_remove %%%%%%%%%%%%%%%%%%%%
\pagebreak
\subsubsection{\code{map\_remove}}
\label{api:nodejs:map_remove}
\index{map\_remove!Node.js API}
Remove a key-value pair from the map specified by each attribute.  If there is
no pair with the specified key within the map, this operation will do nothing.
This operation requires a pre-existing object in order to complete successfully.
If no object exists, the operation will fail with \code{NOTFOUND}.



\paragraph{Definition:}
\begin{javascriptcode}
map_remove(spacename, key, attributes, function (success, err) {})
\end{javascriptcode}
\paragraph{Parameters:}
\begin{itemize}[noitemsep]
\item \code{spacename}\\
The name of the space as a string or symbol.

\item \code{key}\\
The key for the operation as a Python type.

\item \code{attributes}\\
The set of attributes to modify and their respective values.  \code{attrs} is a
map from the attributes' names to their values.

\end{itemize}

\paragraph{Returns:}
This function returns an object indicating the success or failure of the
operation.  Valid values to be returned are:

\begin{itemize}[noitemsep]
\item \code{True} if the operation succeeded
\item \code{False} if any provided predicates failed.
\item \code{null} if the operation requires an existing value and none exists
\end{itemize}

On error, this function will raise a \code{HyperDexClientException} describing
the error.


%%%%%%%%%%%%%%%%%%%% cond_map_remove %%%%%%%%%%%%%%%%%%%%
\pagebreak
\subsubsection{\code{cond\_map\_remove}}
\label{api:nodejs:cond_map_remove}
\index{cond\_map\_remove!Node.js API}
Conditionally remove a key-value pair from the map specified by each attribute.

%%% Generated below here
\paragraph{Behavior:}
\begin{itemize}[noitemsep]
This operation requires a pre-existing object in order to complete successfully.
If no object exists, the operation will fail with \code{NOTFOUND}.

This operation will succeed if and only if the predicates specified by
\code{checks} hold on the pre-existing object.  If any of the predicates are not
true for the existing object, then the operation will have no effect and fail
with \code{CMPFAIL}.

All checks are atomic with the write.  HyperDex guarantees that no other
operation will come between validating the checks, and writing the new version
of the object.

\end{itemize}


\paragraph{Definition:}
\begin{javascriptcode}
cond_map_remove(spacename, key, predicates, attributes, function (success, err) {})
\end{javascriptcode}
\paragraph{Parameters:}
\begin{itemize}[noitemsep]
\item \code{spacename}\\
The name of the space as a string or symbol.

\item \code{key}\\
The key for the operation as a Python type.

\item \code{predicates}\\
A hash of predicates to check against.

\item \code{attributes}\\
The set of attributes to modify and their respective values.  \code{attrs} is a
map from the attributes' names to their values.

\end{itemize}

\paragraph{Returns:}
This function returns an object indicating the success or failure of the
operation.  Valid values to be returned are:

\begin{itemize}[noitemsep]
\item \code{True} if the operation succeeded
\item \code{False} if any provided predicates failed.
\item \code{null} if the operation requires an existing value and none exists
\end{itemize}

On error, this function will raise a \code{HyperDexClientException} describing
the error.


%%%%%%%%%%%%%%%%%%%% document_rename %%%%%%%%%%%%%%%%%%%%
\pagebreak
\subsubsection{\code{document\_rename}}
\label{api:nodejs:document_rename}
\index{document\_rename!Node.js API}
Move a field within a document from one name to another.
This operation requires a pre-existing object in order to complete successfully.
If no object exists, the operation will fail with \code{NOTFOUND}.



\paragraph{Definition:}
\begin{javascriptcode}
document_rename(spacename, key, attributes, function (success, err) {})
\end{javascriptcode}
\paragraph{Parameters:}
\begin{itemize}[noitemsep]
\item \code{spacename}\\
The name of the space as a string or symbol.

\item \code{key}\\
The key for the operation as a Python type.

\item \code{attributes}\\
The set of attributes to modify and their respective values.  \code{attrs} is a
map from the attributes' names to their values.

\end{itemize}

\paragraph{Returns:}
This function returns an object indicating the success or failure of the
operation.  Valid values to be returned are:

\begin{itemize}[noitemsep]
\item \code{True} if the operation succeeded
\item \code{False} if any provided predicates failed.
\item \code{null} if the operation requires an existing value and none exists
\end{itemize}

On error, this function will raise a \code{HyperDexClientException} describing
the error.


%%%%%%%%%%%%%%%%%%%% group_document_rename %%%%%%%%%%%%%%%%%%%%
\pagebreak
\subsubsection{\code{group\_document\_rename}}
\label{api:nodejs:group_document_rename}
\index{group\_document\_rename!Node.js API}
Move a field within a document from one name to another for each object in
\code{space} that matches \code{checks}.

This operation will only affect objects that match the provided \code{checks}.
Objects that do not match \code{checks} will be unaffected by the group call.
Each object that matches \code{checks} will be atomically updated with the check
on the object.  HyperDex guarantees that no object will be altered if the
\code{checks} do not pass at the time of the write.  Objects that are updated
concurrently with the group call may or may not be updated; however, regardless
of any other concurrent operations, the preceding guarantee will always hold.



\paragraph{Definition:}
\begin{javascriptcode}
group_document_rename(spacename, predicates, attributes, function (count, err) {})
\end{javascriptcode}
\paragraph{Parameters:}
\begin{itemize}[noitemsep]
\item \code{spacename}\\
The name of the space as a string or symbol.

\item \code{predicates}\\
A hash of predicates to check against.

\item \code{attributes}\\
The set of attributes to modify and their respective values.  \code{attrs} is a
map from the attributes' names to their values.

\end{itemize}

\paragraph{Returns:}
A count of the number of objects, and a \code{client.Error} object indicating
the status of the operation.


%%%%%%%%%%%%%%%%%%%% document_unset %%%%%%%%%%%%%%%%%%%%
\pagebreak
\subsubsection{\code{document\_unset}}
\label{api:nodejs:document_unset}
\index{document\_unset!Node.js API}
Remove a field or object from a document.
This operation requires a pre-existing object in order to complete successfully.
If no object exists, the operation will fail with \code{NOTFOUND}.



\paragraph{Definition:}
\begin{javascriptcode}
document_unset(spacename, key, attributes, function (success, err) {})
\end{javascriptcode}
\paragraph{Parameters:}
\begin{itemize}[noitemsep]
\item \code{spacename}\\
The name of the space as a string or symbol.

\item \code{key}\\
The key for the operation as a Python type.

\item \code{attributes}\\
The set of attributes to modify and their respective values.  \code{attrs} is a
map from the attributes' names to their values.

\end{itemize}

\paragraph{Returns:}
This function returns an object indicating the success or failure of the
operation.  Valid values to be returned are:

\begin{itemize}[noitemsep]
\item \code{True} if the operation succeeded
\item \code{False} if any provided predicates failed.
\item \code{null} if the operation requires an existing value and none exists
\end{itemize}

On error, this function will raise a \code{HyperDexClientException} describing
the error.


%%%%%%%%%%%%%%%%%%%% group_document_unset %%%%%%%%%%%%%%%%%%%%
\pagebreak
\subsubsection{\code{group\_document\_unset}}
\label{api:nodejs:group_document_unset}
\index{group\_document\_unset!Node.js API}
Remove a field or object from a document for each object in \code{space} that
matches \code{checks}.

This operation will only affect objects that match the provided \code{checks}.
Objects that do not match \code{checks} will be unaffected by the group call.
Each object that matches \code{checks} will be atomically updated with the check
on the object.  HyperDex guarantees that no object will be altered if the
\code{checks} do not pass at the time of the write.  Objects that are updated
concurrently with the group call may or may not be updated; however, regardless
of any other concurrent operations, the preceding guarantee will always hold.



\paragraph{Definition:}
\begin{javascriptcode}
group_document_unset(spacename, predicates, attributes, function (count, err) {})
\end{javascriptcode}
\paragraph{Parameters:}
\begin{itemize}[noitemsep]
\item \code{spacename}\\
The name of the space as a string or symbol.

\item \code{predicates}\\
A hash of predicates to check against.

\item \code{attributes}\\
The set of attributes to modify and their respective values.  \code{attrs} is a
map from the attributes' names to their values.

\end{itemize}

\paragraph{Returns:}
A count of the number of objects, and a \code{client.Error} object indicating
the status of the operation.


%%%%%%%%%%%%%%%%%%%% map_atomic_add %%%%%%%%%%%%%%%%%%%%
\pagebreak
\subsubsection{\code{map\_atomic\_add}}
\label{api:nodejs:map_atomic_add}
\index{map\_atomic\_add!Node.js API}
Add the specified number to the value of a key-value pair within each map.
This operation requires a pre-existing object in order to complete successfully.
If no object exists, the operation will fail with \code{NOTFOUND}.



\paragraph{Definition:}
\begin{javascriptcode}
map_atomic_add(spacename, key, mapattributes, function (success, err) {})
\end{javascriptcode}
\paragraph{Parameters:}
\begin{itemize}[noitemsep]
\item \code{spacename}\\
The name of the space as a string or symbol.

\item \code{key}\\
The key for the operation as a Python type.

\item \code{mapattributes}\\
A hash specifying map attributes to modify and their respective key/values.

\end{itemize}

\paragraph{Returns:}
This function returns an object indicating the success or failure of the
operation.  Valid values to be returned are:

\begin{itemize}[noitemsep]
\item \code{True} if the operation succeeded
\item \code{False} if any provided predicates failed.
\item \code{null} if the operation requires an existing value and none exists
\end{itemize}

On error, this function will raise a \code{HyperDexClientException} describing
the error.


%%%%%%%%%%%%%%%%%%%% cond_map_atomic_add %%%%%%%%%%%%%%%%%%%%
\pagebreak
\subsubsection{\code{cond\_map\_atomic\_add}}
\label{api:nodejs:cond_map_atomic_add}
\index{cond\_map\_atomic\_add!Node.js API}
Add the specified number to the value of a key-value pair within each map if and
only if the \code{checks} hold on the object.
This operation requires a pre-existing object in order to complete successfully.
If no object exists, the operation will fail with \code{NOTFOUND}.


This operation will succeed if and only if the predicates specified by
\code{checks} hold on the pre-existing object.  If any of the predicates are not
true for the existing object, then the operation will have no effect and fail
with \code{CMPFAIL}.

All checks are atomic with the write.  HyperDex guarantees that no other
operation will come between validating the checks, and writing the new version
of the object.



\paragraph{Definition:}
\begin{javascriptcode}
cond_map_atomic_add(
        spacename, key, predicates, mapattributes, function (success, err) {})
\end{javascriptcode}
\paragraph{Parameters:}
\begin{itemize}[noitemsep]
\item \code{spacename}\\
The name of the space as a string or symbol.

\item \code{key}\\
The key for the operation as a Python type.

\item \code{predicates}\\
A hash of predicates to check against.

\item \code{mapattributes}\\
A hash specifying map attributes to modify and their respective key/values.

\end{itemize}

\paragraph{Returns:}
This function returns an object indicating the success or failure of the
operation.  Valid values to be returned are:

\begin{itemize}[noitemsep]
\item \code{True} if the operation succeeded
\item \code{False} if any provided predicates failed.
\item \code{null} if the operation requires an existing value and none exists
\end{itemize}

On error, this function will raise a \code{HyperDexClientException} describing
the error.


%%%%%%%%%%%%%%%%%%%% map_atomic_sub %%%%%%%%%%%%%%%%%%%%
\pagebreak
\subsubsection{\code{map\_atomic\_sub}}
\label{api:nodejs:map_atomic_sub}
\index{map\_atomic\_sub!Node.js API}
Subtract the specified number from the value of a key-value pair within each
map.

%%% Generated below here
\paragraph{Behavior:}
\begin{itemize}[noitemsep]
This operation requires a pre-existing object in order to complete successfully.
If no object exists, the operation will fail with \code{NOTFOUND}.

\item This operation mutates the value of a key-value pair in a map.  This call
    is similar to the equivalent call without the \code{map\_} prefix, but
    operates on the value of a pair in a map, instead of on an attribute's
    value.  If there is no pair with the specified map key, a new pair will be
    created and initialized to its default value.  If this is undesirable, it
    may be avoided by using a conditional operation that requires that the map
    contain the key in question.

\end{itemize}


\paragraph{Definition:}
\begin{javascriptcode}
map_atomic_sub(spacename, key, mapattributes, function (success, err) {})
\end{javascriptcode}
\paragraph{Parameters:}
\begin{itemize}[noitemsep]
\item \code{spacename}\\
The name of the space as a string or symbol.

\item \code{key}\\
The key for the operation as a Python type.

\item \code{mapattributes}\\
A hash specifying map attributes to modify and their respective key/values.

\end{itemize}

\paragraph{Returns:}
This function returns an object indicating the success or failure of the
operation.  Valid values to be returned are:

\begin{itemize}[noitemsep]
\item \code{True} if the operation succeeded
\item \code{False} if any provided predicates failed.
\item \code{null} if the operation requires an existing value and none exists
\end{itemize}

On error, this function will raise a \code{HyperDexClientException} describing
the error.


%%%%%%%%%%%%%%%%%%%% cond_map_atomic_sub %%%%%%%%%%%%%%%%%%%%
\pagebreak
\subsubsection{\code{cond\_map\_atomic\_sub}}
\label{api:nodejs:cond_map_atomic_sub}
\index{cond\_map\_atomic\_sub!Node.js API}
Subtract the specified number from the value of a key-value pair within each
map.

%%% Generated below here
\paragraph{Behavior:}
\begin{itemize}[noitemsep]
This operation requires a pre-existing object in order to complete successfully.
If no object exists, the operation will fail with \code{NOTFOUND}.

This operation will succeed if and only if the predicates specified by
\code{checks} hold on the pre-existing object.  If any of the predicates are not
true for the existing object, then the operation will have no effect and fail
with \code{CMPFAIL}.

All checks are atomic with the write.  HyperDex guarantees that no other
operation will come between validating the checks, and writing the new version
of the object.

\item This operation mutates the value of a key-value pair in a map.  This call
    is similar to the equivalent call without the \code{map\_} prefix, but
    operates on the value of a pair in a map, instead of on an attribute's
    value.  If there is no pair with the specified map key, a new pair will be
    created and initialized to its default value.  If this is undesirable, it
    may be avoided by using a conditional operation that requires that the map
    contain the key in question.

\end{itemize}


\paragraph{Definition:}
\begin{javascriptcode}
cond_map_atomic_sub(
        spacename, key, predicates, mapattributes, function (success, err) {})
\end{javascriptcode}
\paragraph{Parameters:}
\begin{itemize}[noitemsep]
\item \code{spacename}\\
The name of the space as a string or symbol.

\item \code{key}\\
The key for the operation as a Python type.

\item \code{predicates}\\
A hash of predicates to check against.

\item \code{mapattributes}\\
A hash specifying map attributes to modify and their respective key/values.

\end{itemize}

\paragraph{Returns:}
This function returns an object indicating the success or failure of the
operation.  Valid values to be returned are:

\begin{itemize}[noitemsep]
\item \code{True} if the operation succeeded
\item \code{False} if any provided predicates failed.
\item \code{null} if the operation requires an existing value and none exists
\end{itemize}

On error, this function will raise a \code{HyperDexClientException} describing
the error.


%%%%%%%%%%%%%%%%%%%% map_atomic_mul %%%%%%%%%%%%%%%%%%%%
\pagebreak
\subsubsection{\code{map\_atomic\_mul}}
\label{api:nodejs:map_atomic_mul}
\index{map\_atomic\_mul!Node.js API}
Multiply the value of each key-value pair by the specified number for each map.

%%% Generated below here
\paragraph{Behavior:}
\begin{itemize}[noitemsep]
This operation requires a pre-existing object in order to complete successfully.
If no object exists, the operation will fail with \code{NOTFOUND}.

\item This operation mutates the value of a key-value pair in a map.  This call
    is similar to the equivalent call without the \code{map\_} prefix, but
    operates on the value of a pair in a map, instead of on an attribute's
    value.  If there is no pair with the specified map key, a new pair will be
    created and initialized to its default value.  If this is undesirable, it
    may be avoided by using a conditional operation that requires that the map
    contain the key in question.

\end{itemize}


\paragraph{Definition:}
\begin{javascriptcode}
map_atomic_mul(spacename, key, mapattributes, function (success, err) {})
\end{javascriptcode}
\paragraph{Parameters:}
\begin{itemize}[noitemsep]
\item \code{spacename}\\
The name of the space as a string or symbol.

\item \code{key}\\
The key for the operation as a Python type.

\item \code{mapattributes}\\
A hash specifying map attributes to modify and their respective key/values.

\end{itemize}

\paragraph{Returns:}
This function returns an object indicating the success or failure of the
operation.  Valid values to be returned are:

\begin{itemize}[noitemsep]
\item \code{True} if the operation succeeded
\item \code{False} if any provided predicates failed.
\item \code{null} if the operation requires an existing value and none exists
\end{itemize}

On error, this function will raise a \code{HyperDexClientException} describing
the error.


%%%%%%%%%%%%%%%%%%%% cond_map_atomic_mul %%%%%%%%%%%%%%%%%%%%
\pagebreak
\subsubsection{\code{cond\_map\_atomic\_mul}}
\label{api:nodejs:cond_map_atomic_mul}
\index{cond\_map\_atomic\_mul!Node.js API}
Conditionally multiply the value of each key-value pair by the specified number
for each map.

%%% Generated below here
\paragraph{Behavior:}
\begin{itemize}[noitemsep]
This operation requires a pre-existing object in order to complete successfully.
If no object exists, the operation will fail with \code{NOTFOUND}.

This operation will succeed if and only if the predicates specified by
\code{checks} hold on the pre-existing object.  If any of the predicates are not
true for the existing object, then the operation will have no effect and fail
with \code{CMPFAIL}.

All checks are atomic with the write.  HyperDex guarantees that no other
operation will come between validating the checks, and writing the new version
of the object.

\item This operation mutates the value of a key-value pair in a map.  This call
    is similar to the equivalent call without the \code{map\_} prefix, but
    operates on the value of a pair in a map, instead of on an attribute's
    value.  If there is no pair with the specified map key, a new pair will be
    created and initialized to its default value.  If this is undesirable, it
    may be avoided by using a conditional operation that requires that the map
    contain the key in question.

\end{itemize}


\paragraph{Definition:}
\begin{javascriptcode}
cond_map_atomic_mul(
        spacename, key, predicates, mapattributes, function (success, err) {})
\end{javascriptcode}
\paragraph{Parameters:}
\begin{itemize}[noitemsep]
\item \code{spacename}\\
The name of the space as a string or symbol.

\item \code{key}\\
The key for the operation as a Python type.

\item \code{predicates}\\
A hash of predicates to check against.

\item \code{mapattributes}\\
A hash specifying map attributes to modify and their respective key/values.

\end{itemize}

\paragraph{Returns:}
This function returns an object indicating the success or failure of the
operation.  Valid values to be returned are:

\begin{itemize}[noitemsep]
\item \code{True} if the operation succeeded
\item \code{False} if any provided predicates failed.
\item \code{null} if the operation requires an existing value and none exists
\end{itemize}

On error, this function will raise a \code{HyperDexClientException} describing
the error.


%%%%%%%%%%%%%%%%%%%% map_atomic_div %%%%%%%%%%%%%%%%%%%%
\pagebreak
\subsubsection{\code{map\_atomic\_div}}
\label{api:nodejs:map_atomic_div}
\index{map\_atomic\_div!Node.js API}
Divide the value of each key-value pair by the specified number for each map.

%%% Generated below here
\paragraph{Behavior:}
\begin{itemize}[noitemsep]
This operation requires a pre-existing object in order to complete successfully.
If no object exists, the operation will fail with \code{NOTFOUND}.

\item This operation mutates the value of a key-value pair in a map.  This call
    is similar to the equivalent call without the \code{map\_} prefix, but
    operates on the value of a pair in a map, instead of on an attribute's
    value.  If there is no pair with the specified map key, a new pair will be
    created and initialized to its default value.  If this is undesirable, it
    may be avoided by using a conditional operation that requires that the map
    contain the key in question.

\end{itemize}


\paragraph{Definition:}
\begin{javascriptcode}
map_atomic_div(spacename, key, mapattributes, function (success, err) {})
\end{javascriptcode}
\paragraph{Parameters:}
\begin{itemize}[noitemsep]
\item \code{spacename}\\
The name of the space as a string or symbol.

\item \code{key}\\
The key for the operation as a Python type.

\item \code{mapattributes}\\
A hash specifying map attributes to modify and their respective key/values.

\end{itemize}

\paragraph{Returns:}
This function returns an object indicating the success or failure of the
operation.  Valid values to be returned are:

\begin{itemize}[noitemsep]
\item \code{True} if the operation succeeded
\item \code{False} if any provided predicates failed.
\item \code{null} if the operation requires an existing value and none exists
\end{itemize}

On error, this function will raise a \code{HyperDexClientException} describing
the error.


%%%%%%%%%%%%%%%%%%%% cond_map_atomic_div %%%%%%%%%%%%%%%%%%%%
\pagebreak
\subsubsection{\code{cond\_map\_atomic\_div}}
\label{api:nodejs:cond_map_atomic_div}
\index{cond\_map\_atomic\_div!Node.js API}
Conditionally divide the value of each key-value pair by the specified number for each map.

%%% Generated below here
\paragraph{Behavior:}
\begin{itemize}[noitemsep]
This operation requires a pre-existing object in order to complete successfully.
If no object exists, the operation will fail with \code{NOTFOUND}.

This operation will succeed if and only if the predicates specified by
\code{checks} hold on the pre-existing object.  If any of the predicates are not
true for the existing object, then the operation will have no effect and fail
with \code{CMPFAIL}.

All checks are atomic with the write.  HyperDex guarantees that no other
operation will come between validating the checks, and writing the new version
of the object.

\item This operation mutates the value of a key-value pair in a map.  This call
    is similar to the equivalent call without the \code{map\_} prefix, but
    operates on the value of a pair in a map, instead of on an attribute's
    value.  If there is no pair with the specified map key, a new pair will be
    created and initialized to its default value.  If this is undesirable, it
    may be avoided by using a conditional operation that requires that the map
    contain the key in question.

\end{itemize}


\paragraph{Definition:}
\begin{javascriptcode}
cond_map_atomic_div(
        spacename, key, predicates, mapattributes, function (success, err) {})
\end{javascriptcode}
\paragraph{Parameters:}
\begin{itemize}[noitemsep]
\item \code{spacename}\\
The name of the space as a string or symbol.

\item \code{key}\\
The key for the operation as a Python type.

\item \code{predicates}\\
A hash of predicates to check against.

\item \code{mapattributes}\\
A hash specifying map attributes to modify and their respective key/values.

\end{itemize}

\paragraph{Returns:}
This function returns an object indicating the success or failure of the
operation.  Valid values to be returned are:

\begin{itemize}[noitemsep]
\item \code{True} if the operation succeeded
\item \code{False} if any provided predicates failed.
\item \code{null} if the operation requires an existing value and none exists
\end{itemize}

On error, this function will raise a \code{HyperDexClientException} describing
the error.


%%%%%%%%%%%%%%%%%%%% map_atomic_mod %%%%%%%%%%%%%%%%%%%%
\pagebreak
\subsubsection{\code{map\_atomic\_mod}}
\label{api:nodejs:map_atomic_mod}
\index{map\_atomic\_mod!Node.js API}
Store the value of the key-value pair modulo the specified number for each map.
This operation requires a pre-existing object in order to complete successfully.
If no object exists, the operation will fail with \code{NOTFOUND}.



\paragraph{Definition:}
\begin{javascriptcode}
map_atomic_mod(spacename, key, mapattributes, function (success, err) {})
\end{javascriptcode}
\paragraph{Parameters:}
\begin{itemize}[noitemsep]
\item \code{spacename}\\
The name of the space as a string or symbol.

\item \code{key}\\
The key for the operation as a Python type.

\item \code{mapattributes}\\
A hash specifying map attributes to modify and their respective key/values.

\end{itemize}

\paragraph{Returns:}
This function returns an object indicating the success or failure of the
operation.  Valid values to be returned are:

\begin{itemize}[noitemsep]
\item \code{True} if the operation succeeded
\item \code{False} if any provided predicates failed.
\item \code{null} if the operation requires an existing value and none exists
\end{itemize}

On error, this function will raise a \code{HyperDexClientException} describing
the error.


%%%%%%%%%%%%%%%%%%%% cond_map_atomic_mod %%%%%%%%%%%%%%%%%%%%
\pagebreak
\subsubsection{\code{cond\_map\_atomic\_mod}}
\label{api:nodejs:cond_map_atomic_mod}
\index{cond\_map\_atomic\_mod!Node.js API}
Conditionally store the value of the key-value pair modulo the specified number
for each map.

%%% Generated below here
\paragraph{Behavior:}
\begin{itemize}[noitemsep]
This operation requires a pre-existing object in order to complete successfully.
If no object exists, the operation will fail with \code{NOTFOUND}.

This operation will succeed if and only if the predicates specified by
\code{checks} hold on the pre-existing object.  If any of the predicates are not
true for the existing object, then the operation will have no effect and fail
with \code{CMPFAIL}.

All checks are atomic with the write.  HyperDex guarantees that no other
operation will come between validating the checks, and writing the new version
of the object.

\item This operation mutates the value of a key-value pair in a map.  This call
    is similar to the equivalent call without the \code{map\_} prefix, but
    operates on the value of a pair in a map, instead of on an attribute's
    value.  If there is no pair with the specified map key, a new pair will be
    created and initialized to its default value.  If this is undesirable, it
    may be avoided by using a conditional operation that requires that the map
    contain the key in question.

\end{itemize}


\paragraph{Definition:}
\begin{javascriptcode}
cond_map_atomic_mod(
        spacename, key, predicates, mapattributes, function (success, err) {})
\end{javascriptcode}
\paragraph{Parameters:}
\begin{itemize}[noitemsep]
\item \code{spacename}\\
The name of the space as a string or symbol.

\item \code{key}\\
The key for the operation as a Python type.

\item \code{predicates}\\
A hash of predicates to check against.

\item \code{mapattributes}\\
A hash specifying map attributes to modify and their respective key/values.

\end{itemize}

\paragraph{Returns:}
This function returns an object indicating the success or failure of the
operation.  Valid values to be returned are:

\begin{itemize}[noitemsep]
\item \code{True} if the operation succeeded
\item \code{False} if any provided predicates failed.
\item \code{null} if the operation requires an existing value and none exists
\end{itemize}

On error, this function will raise a \code{HyperDexClientException} describing
the error.


%%%%%%%%%%%%%%%%%%%% map_atomic_and %%%%%%%%%%%%%%%%%%%%
\pagebreak
\subsubsection{\code{map\_atomic\_and}}
\label{api:nodejs:map_atomic_and}
\index{map\_atomic\_and!Node.js API}
Store the bitwise AND of the value of the key-value pair and the specified
number for each map.

%%% Generated below here
\paragraph{Behavior:}
\begin{itemize}[noitemsep]
This operation requires a pre-existing object in order to complete successfully.
If no object exists, the operation will fail with \code{NOTFOUND}.

\item This operation mutates the value of a key-value pair in a map.  This call
    is similar to the equivalent call without the \code{map\_} prefix, but
    operates on the value of a pair in a map, instead of on an attribute's
    value.  If there is no pair with the specified map key, a new pair will be
    created and initialized to its default value.  If this is undesirable, it
    may be avoided by using a conditional operation that requires that the map
    contain the key in question.

\end{itemize}


\paragraph{Definition:}
\begin{javascriptcode}
map_atomic_and(spacename, key, mapattributes, function (success, err) {})
\end{javascriptcode}
\paragraph{Parameters:}
\begin{itemize}[noitemsep]
\item \code{spacename}\\
The name of the space as a string or symbol.

\item \code{key}\\
The key for the operation as a Python type.

\item \code{mapattributes}\\
A hash specifying map attributes to modify and their respective key/values.

\end{itemize}

\paragraph{Returns:}
This function returns an object indicating the success or failure of the
operation.  Valid values to be returned are:

\begin{itemize}[noitemsep]
\item \code{True} if the operation succeeded
\item \code{False} if any provided predicates failed.
\item \code{null} if the operation requires an existing value and none exists
\end{itemize}

On error, this function will raise a \code{HyperDexClientException} describing
the error.


%%%%%%%%%%%%%%%%%%%% cond_map_atomic_and %%%%%%%%%%%%%%%%%%%%
\pagebreak
\subsubsection{\code{cond\_map\_atomic\_and}}
\label{api:nodejs:cond_map_atomic_and}
\index{cond\_map\_atomic\_and!Node.js API}
Store the bitwise AND of the value of the key-value pair and the specified
number for each map attribute if and only if the \code{checks} hold on the
object.
This operation requires a pre-existing object in order to complete successfully.
If no object exists, the operation will fail with \code{NOTFOUND}.


This operation will succeed if and only if the predicates specified by
\code{checks} hold on the pre-existing object.  If any of the predicates are not
true for the existing object, then the operation will have no effect and fail
with \code{CMPFAIL}.

All checks are atomic with the write.  HyperDex guarantees that no other
operation will come between validating the checks, and writing the new version
of the object.



\paragraph{Definition:}
\begin{javascriptcode}
cond_map_atomic_and(
        spacename, key, predicates, mapattributes, function (success, err) {})
\end{javascriptcode}
\paragraph{Parameters:}
\begin{itemize}[noitemsep]
\item \code{spacename}\\
The name of the space as a string or symbol.

\item \code{key}\\
The key for the operation as a Python type.

\item \code{predicates}\\
A hash of predicates to check against.

\item \code{mapattributes}\\
A hash specifying map attributes to modify and their respective key/values.

\end{itemize}

\paragraph{Returns:}
This function returns an object indicating the success or failure of the
operation.  Valid values to be returned are:

\begin{itemize}[noitemsep]
\item \code{True} if the operation succeeded
\item \code{False} if any provided predicates failed.
\item \code{null} if the operation requires an existing value and none exists
\end{itemize}

On error, this function will raise a \code{HyperDexClientException} describing
the error.


%%%%%%%%%%%%%%%%%%%% map_atomic_or %%%%%%%%%%%%%%%%%%%%
\pagebreak
\subsubsection{\code{map\_atomic\_or}}
\label{api:nodejs:map_atomic_or}
\index{map\_atomic\_or!Node.js API}
Store the bitwise OR of the value of the key-value pair and the specified number
for each map.

%%% Generated below here
\paragraph{Behavior:}
\begin{itemize}[noitemsep]
This operation requires a pre-existing object in order to complete successfully.
If no object exists, the operation will fail with \code{NOTFOUND}.

\item This operation mutates the value of a key-value pair in a map.  This call
    is similar to the equivalent call without the \code{map\_} prefix, but
    operates on the value of a pair in a map, instead of on an attribute's
    value.  If there is no pair with the specified map key, a new pair will be
    created and initialized to its default value.  If this is undesirable, it
    may be avoided by using a conditional operation that requires that the map
    contain the key in question.

\end{itemize}


\paragraph{Definition:}
\begin{javascriptcode}
map_atomic_or(spacename, key, mapattributes, function (success, err) {})
\end{javascriptcode}
\paragraph{Parameters:}
\begin{itemize}[noitemsep]
\item \code{spacename}\\
The name of the space as a string or symbol.

\item \code{key}\\
The key for the operation as a Python type.

\item \code{mapattributes}\\
A hash specifying map attributes to modify and their respective key/values.

\end{itemize}

\paragraph{Returns:}
This function returns an object indicating the success or failure of the
operation.  Valid values to be returned are:

\begin{itemize}[noitemsep]
\item \code{True} if the operation succeeded
\item \code{False} if any provided predicates failed.
\item \code{null} if the operation requires an existing value and none exists
\end{itemize}

On error, this function will raise a \code{HyperDexClientException} describing
the error.


%%%%%%%%%%%%%%%%%%%% cond_map_atomic_or %%%%%%%%%%%%%%%%%%%%
\pagebreak
\subsubsection{\code{cond\_map\_atomic\_or}}
\label{api:nodejs:cond_map_atomic_or}
\index{cond\_map\_atomic\_or!Node.js API}
Conditionally store the bitwise OR of the value of the key-value pair and the
specified number for each map.

%%% Generated below here
\paragraph{Behavior:}
\begin{itemize}[noitemsep]
This operation requires a pre-existing object in order to complete successfully.
If no object exists, the operation will fail with \code{NOTFOUND}.

This operation will succeed if and only if the predicates specified by
\code{checks} hold on the pre-existing object.  If any of the predicates are not
true for the existing object, then the operation will have no effect and fail
with \code{CMPFAIL}.

All checks are atomic with the write.  HyperDex guarantees that no other
operation will come between validating the checks, and writing the new version
of the object.

\item This operation mutates the value of a key-value pair in a map.  This call
    is similar to the equivalent call without the \code{map\_} prefix, but
    operates on the value of a pair in a map, instead of on an attribute's
    value.  If there is no pair with the specified map key, a new pair will be
    created and initialized to its default value.  If this is undesirable, it
    may be avoided by using a conditional operation that requires that the map
    contain the key in question.

\end{itemize}


\paragraph{Definition:}
\begin{javascriptcode}
cond_map_atomic_or(
        spacename, key, predicates, mapattributes, function (success, err) {})
\end{javascriptcode}
\paragraph{Parameters:}
\begin{itemize}[noitemsep]
\item \code{spacename}\\
The name of the space as a string or symbol.

\item \code{key}\\
The key for the operation as a Python type.

\item \code{predicates}\\
A hash of predicates to check against.

\item \code{mapattributes}\\
A hash specifying map attributes to modify and their respective key/values.

\end{itemize}

\paragraph{Returns:}
This function returns an object indicating the success or failure of the
operation.  Valid values to be returned are:

\begin{itemize}[noitemsep]
\item \code{True} if the operation succeeded
\item \code{False} if any provided predicates failed.
\item \code{null} if the operation requires an existing value and none exists
\end{itemize}

On error, this function will raise a \code{HyperDexClientException} describing
the error.


%%%%%%%%%%%%%%%%%%%% map_atomic_xor %%%%%%%%%%%%%%%%%%%%
\pagebreak
\subsubsection{\code{map\_atomic\_xor}}
\label{api:nodejs:map_atomic_xor}
\index{map\_atomic\_xor!Node.js API}
Store the bitwise XOR of the value of the key-value pair and the specified
number for each map.

%%% Generated below here
\paragraph{Behavior:}
\begin{itemize}[noitemsep]
This operation requires a pre-existing object in order to complete successfully.
If no object exists, the operation will fail with \code{NOTFOUND}.

\item This operation mutates the value of a key-value pair in a map.  This call
    is similar to the equivalent call without the \code{map\_} prefix, but
    operates on the value of a pair in a map, instead of on an attribute's
    value.  If there is no pair with the specified map key, a new pair will be
    created and initialized to its default value.  If this is undesirable, it
    may be avoided by using a conditional operation that requires that the map
    contain the key in question.

\end{itemize}


\paragraph{Definition:}
\begin{javascriptcode}
map_atomic_xor(spacename, key, mapattributes, function (success, err) {})
\end{javascriptcode}
\paragraph{Parameters:}
\begin{itemize}[noitemsep]
\item \code{spacename}\\
The name of the space as a string or symbol.

\item \code{key}\\
The key for the operation as a Python type.

\item \code{mapattributes}\\
A hash specifying map attributes to modify and their respective key/values.

\end{itemize}

\paragraph{Returns:}
This function returns an object indicating the success or failure of the
operation.  Valid values to be returned are:

\begin{itemize}[noitemsep]
\item \code{True} if the operation succeeded
\item \code{False} if any provided predicates failed.
\item \code{null} if the operation requires an existing value and none exists
\end{itemize}

On error, this function will raise a \code{HyperDexClientException} describing
the error.


%%%%%%%%%%%%%%%%%%%% cond_map_atomic_xor %%%%%%%%%%%%%%%%%%%%
\pagebreak
\subsubsection{\code{cond\_map\_atomic\_xor}}
\label{api:nodejs:cond_map_atomic_xor}
\index{cond\_map\_atomic\_xor!Node.js API}
Store the bitwise XOR of the value of the key-value pair and the specified
number for each map attribute if and only if the \code{checks} hold on the
object.
This operation requires a pre-existing object in order to complete successfully.
If no object exists, the operation will fail with \code{NOTFOUND}.


This operation will succeed if and only if the predicates specified by
\code{checks} hold on the pre-existing object.  If any of the predicates are not
true for the existing object, then the operation will have no effect and fail
with \code{CMPFAIL}.

All checks are atomic with the write.  HyperDex guarantees that no other
operation will come between validating the checks, and writing the new version
of the object.



\paragraph{Definition:}
\begin{javascriptcode}
cond_map_atomic_xor(
        spacename, key, predicates, mapattributes, function (success, err) {})
\end{javascriptcode}
\paragraph{Parameters:}
\begin{itemize}[noitemsep]
\item \code{spacename}\\
The name of the space as a string or symbol.

\item \code{key}\\
The key for the operation as a Python type.

\item \code{predicates}\\
A hash of predicates to check against.

\item \code{mapattributes}\\
A hash specifying map attributes to modify and their respective key/values.

\end{itemize}

\paragraph{Returns:}
This function returns an object indicating the success or failure of the
operation.  Valid values to be returned are:

\begin{itemize}[noitemsep]
\item \code{True} if the operation succeeded
\item \code{False} if any provided predicates failed.
\item \code{null} if the operation requires an existing value and none exists
\end{itemize}

On error, this function will raise a \code{HyperDexClientException} describing
the error.


%%%%%%%%%%%%%%%%%%%% map_string_prepend %%%%%%%%%%%%%%%%%%%%
\pagebreak
\subsubsection{\code{map\_string\_prepend}}
\label{api:nodejs:map_string_prepend}
\index{map\_string\_prepend!Node.js API}
Prepend the specified string to the value of the key-value pair for each map.

%%% Generated below here
\paragraph{Behavior:}
\begin{itemize}[noitemsep]
This operation requires a pre-existing object in order to complete successfully.
If no object exists, the operation will fail with \code{NOTFOUND}.

\item This operation mutates the value of a key-value pair in a map.  This call
    is similar to the equivalent call without the \code{map\_} prefix, but
    operates on the value of a pair in a map, instead of on an attribute's
    value.  If there is no pair with the specified map key, a new pair will be
    created and initialized to its default value.  If this is undesirable, it
    may be avoided by using a conditional operation that requires that the map
    contain the key in question.

\end{itemize}


\paragraph{Definition:}
\begin{javascriptcode}
map_string_prepend(spacename, key, mapattributes, function (success, err) {})
\end{javascriptcode}
\paragraph{Parameters:}
\begin{itemize}[noitemsep]
\item \code{spacename}\\
The name of the space as a string or symbol.

\item \code{key}\\
The key for the operation as a Python type.

\item \code{mapattributes}\\
A hash specifying map attributes to modify and their respective key/values.

\end{itemize}

\paragraph{Returns:}
This function returns an object indicating the success or failure of the
operation.  Valid values to be returned are:

\begin{itemize}[noitemsep]
\item \code{True} if the operation succeeded
\item \code{False} if any provided predicates failed.
\item \code{null} if the operation requires an existing value and none exists
\end{itemize}

On error, this function will raise a \code{HyperDexClientException} describing
the error.


%%%%%%%%%%%%%%%%%%%% cond_map_string_prepend %%%%%%%%%%%%%%%%%%%%
\pagebreak
\subsubsection{\code{cond\_map\_string\_prepend}}
\label{api:nodejs:cond_map_string_prepend}
\index{cond\_map\_string\_prepend!Node.js API}
Conditionally prepend the specified string to the value of the key-value pair
for each map.

%%% Generated below here
\paragraph{Behavior:}
\begin{itemize}[noitemsep]
This operation requires a pre-existing object in order to complete successfully.
If no object exists, the operation will fail with \code{NOTFOUND}.

This operation will succeed if and only if the predicates specified by
\code{checks} hold on the pre-existing object.  If any of the predicates are not
true for the existing object, then the operation will have no effect and fail
with \code{CMPFAIL}.

All checks are atomic with the write.  HyperDex guarantees that no other
operation will come between validating the checks, and writing the new version
of the object.

\item This operation mutates the value of a key-value pair in a map.  This call
    is similar to the equivalent call without the \code{map\_} prefix, but
    operates on the value of a pair in a map, instead of on an attribute's
    value.  If there is no pair with the specified map key, a new pair will be
    created and initialized to its default value.  If this is undesirable, it
    may be avoided by using a conditional operation that requires that the map
    contain the key in question.

\end{itemize}


\paragraph{Definition:}
\begin{javascriptcode}
cond_map_string_prepend(
        spacename, key, predicates, mapattributes, function (success, err) {})
\end{javascriptcode}
\paragraph{Parameters:}
\begin{itemize}[noitemsep]
\item \code{spacename}\\
The name of the space as a string or symbol.

\item \code{key}\\
The key for the operation as a Python type.

\item \code{predicates}\\
A hash of predicates to check against.

\item \code{mapattributes}\\
A hash specifying map attributes to modify and their respective key/values.

\end{itemize}

\paragraph{Returns:}
This function returns an object indicating the success or failure of the
operation.  Valid values to be returned are:

\begin{itemize}[noitemsep]
\item \code{True} if the operation succeeded
\item \code{False} if any provided predicates failed.
\item \code{null} if the operation requires an existing value and none exists
\end{itemize}

On error, this function will raise a \code{HyperDexClientException} describing
the error.


%%%%%%%%%%%%%%%%%%%% map_string_append %%%%%%%%%%%%%%%%%%%%
\pagebreak
\subsubsection{\code{map\_string\_append}}
\label{api:nodejs:map_string_append}
\index{map\_string\_append!Node.js API}
Append the specified string to the value of the key-value pair for each map.

%%% Generated below here
\paragraph{Behavior:}
\begin{itemize}[noitemsep]
This operation requires a pre-existing object in order to complete successfully.
If no object exists, the operation will fail with \code{NOTFOUND}.

\item This operation mutates the value of a key-value pair in a map.  This call
    is similar to the equivalent call without the \code{map\_} prefix, but
    operates on the value of a pair in a map, instead of on an attribute's
    value.  If there is no pair with the specified map key, a new pair will be
    created and initialized to its default value.  If this is undesirable, it
    may be avoided by using a conditional operation that requires that the map
    contain the key in question.

\end{itemize}


\paragraph{Definition:}
\begin{javascriptcode}
map_string_append(spacename, key, mapattributes, function (success, err) {})
\end{javascriptcode}
\paragraph{Parameters:}
\begin{itemize}[noitemsep]
\item \code{spacename}\\
The name of the space as a string or symbol.

\item \code{key}\\
The key for the operation as a Python type.

\item \code{mapattributes}\\
A hash specifying map attributes to modify and their respective key/values.

\end{itemize}

\paragraph{Returns:}
This function returns an object indicating the success or failure of the
operation.  Valid values to be returned are:

\begin{itemize}[noitemsep]
\item \code{True} if the operation succeeded
\item \code{False} if any provided predicates failed.
\item \code{null} if the operation requires an existing value and none exists
\end{itemize}

On error, this function will raise a \code{HyperDexClientException} describing
the error.


%%%%%%%%%%%%%%%%%%%% cond_map_string_append %%%%%%%%%%%%%%%%%%%%
\pagebreak
\subsubsection{\code{cond\_map\_string\_append}}
\label{api:nodejs:cond_map_string_append}
\index{cond\_map\_string\_append!Node.js API}
Conditionally append the specified string to the value of the key-value pair for
each map.

%%% Generated below here
\paragraph{Behavior:}
\begin{itemize}[noitemsep]
This operation requires a pre-existing object in order to complete successfully.
If no object exists, the operation will fail with \code{NOTFOUND}.

This operation will succeed if and only if the predicates specified by
\code{checks} hold on the pre-existing object.  If any of the predicates are not
true for the existing object, then the operation will have no effect and fail
with \code{CMPFAIL}.

All checks are atomic with the write.  HyperDex guarantees that no other
operation will come between validating the checks, and writing the new version
of the object.

\item This operation mutates the value of a key-value pair in a map.  This call
    is similar to the equivalent call without the \code{map\_} prefix, but
    operates on the value of a pair in a map, instead of on an attribute's
    value.  If there is no pair with the specified map key, a new pair will be
    created and initialized to its default value.  If this is undesirable, it
    may be avoided by using a conditional operation that requires that the map
    contain the key in question.

\end{itemize}


\paragraph{Definition:}
\begin{javascriptcode}
cond_map_string_append(
        spacename, key, predicates, mapattributes, function (success, err) {})
\end{javascriptcode}
\paragraph{Parameters:}
\begin{itemize}[noitemsep]
\item \code{spacename}\\
The name of the space as a string or symbol.

\item \code{key}\\
The key for the operation as a Python type.

\item \code{predicates}\\
A hash of predicates to check against.

\item \code{mapattributes}\\
A hash specifying map attributes to modify and their respective key/values.

\end{itemize}

\paragraph{Returns:}
This function returns an object indicating the success or failure of the
operation.  Valid values to be returned are:

\begin{itemize}[noitemsep]
\item \code{True} if the operation succeeded
\item \code{False} if any provided predicates failed.
\item \code{null} if the operation requires an existing value and none exists
\end{itemize}

On error, this function will raise a \code{HyperDexClientException} describing
the error.


%%%%%%%%%%%%%%%%%%%% map_atomic_min %%%%%%%%%%%%%%%%%%%%
\pagebreak
\subsubsection{\code{map\_atomic\_min}}
\label{api:nodejs:map_atomic_min}
\index{map\_atomic\_min!Node.js API}
Take the minium of the specified value and existing value for each key-value
pair.
This operation requires a pre-existing object in order to complete successfully.
If no object exists, the operation will fail with \code{NOTFOUND}.



\paragraph{Definition:}
\begin{javascriptcode}
map_atomic_min(spacename, key, mapattributes, function (success, err) {})
\end{javascriptcode}
\paragraph{Parameters:}
\begin{itemize}[noitemsep]
\item \code{spacename}\\
The name of the space as a string or symbol.

\item \code{key}\\
The key for the operation as a Python type.

\item \code{mapattributes}\\
A hash specifying map attributes to modify and their respective key/values.

\end{itemize}

\paragraph{Returns:}
This function returns an object indicating the success or failure of the
operation.  Valid values to be returned are:

\begin{itemize}[noitemsep]
\item \code{True} if the operation succeeded
\item \code{False} if any provided predicates failed.
\item \code{null} if the operation requires an existing value and none exists
\end{itemize}

On error, this function will raise a \code{HyperDexClientException} describing
the error.


%%%%%%%%%%%%%%%%%%%% cond_map_atomic_min %%%%%%%%%%%%%%%%%%%%
\pagebreak
\subsubsection{\code{cond\_map\_atomic\_min}}
\label{api:nodejs:cond_map_atomic_min}
\index{cond\_map\_atomic\_min!Node.js API}
XXX


\paragraph{Definition:}
\begin{javascriptcode}
cond_map_atomic_min(
        spacename, key, predicates, mapattributes, function (success, err) {})
\end{javascriptcode}
\paragraph{Parameters:}
\begin{itemize}[noitemsep]
\item \code{spacename}\\
The name of the space as a string or symbol.

\item \code{key}\\
The key for the operation as a Python type.

\item \code{predicates}\\
A hash of predicates to check against.

\item \code{mapattributes}\\
A hash specifying map attributes to modify and their respective key/values.

\end{itemize}

\paragraph{Returns:}
This function returns an object indicating the success or failure of the
operation.  Valid values to be returned are:

\begin{itemize}[noitemsep]
\item \code{True} if the operation succeeded
\item \code{False} if any provided predicates failed.
\item \code{null} if the operation requires an existing value and none exists
\end{itemize}

On error, this function will raise a \code{HyperDexClientException} describing
the error.


%%%%%%%%%%%%%%%%%%%% map_atomic_max %%%%%%%%%%%%%%%%%%%%
\pagebreak
\subsubsection{\code{map\_atomic\_max}}
\label{api:nodejs:map_atomic_max}
\index{map\_atomic\_max!Node.js API}
Take the maximum of the specified value and existing value for each key-value
pair.
This operation requires a pre-existing object in order to complete successfully.
If no object exists, the operation will fail with \code{NOTFOUND}.



\paragraph{Definition:}
\begin{javascriptcode}
map_atomic_max(spacename, key, mapattributes, function (success, err) {})
\end{javascriptcode}
\paragraph{Parameters:}
\begin{itemize}[noitemsep]
\item \code{spacename}\\
The name of the space as a string or symbol.

\item \code{key}\\
The key for the operation as a Python type.

\item \code{mapattributes}\\
A hash specifying map attributes to modify and their respective key/values.

\end{itemize}

\paragraph{Returns:}
This function returns an object indicating the success or failure of the
operation.  Valid values to be returned are:

\begin{itemize}[noitemsep]
\item \code{True} if the operation succeeded
\item \code{False} if any provided predicates failed.
\item \code{null} if the operation requires an existing value and none exists
\end{itemize}

On error, this function will raise a \code{HyperDexClientException} describing
the error.


%%%%%%%%%%%%%%%%%%%% cond_map_atomic_max %%%%%%%%%%%%%%%%%%%%
\pagebreak
\subsubsection{\code{cond\_map\_atomic\_max}}
\label{api:nodejs:cond_map_atomic_max}
\index{cond\_map\_atomic\_max!Node.js API}
XXX


\paragraph{Definition:}
\begin{javascriptcode}
cond_map_atomic_max(
        spacename, key, predicates, mapattributes, function (success, err) {})
\end{javascriptcode}
\paragraph{Parameters:}
\begin{itemize}[noitemsep]
\item \code{spacename}\\
The name of the space as a string or symbol.

\item \code{key}\\
The key for the operation as a Python type.

\item \code{predicates}\\
A hash of predicates to check against.

\item \code{mapattributes}\\
A hash specifying map attributes to modify and their respective key/values.

\end{itemize}

\paragraph{Returns:}
This function returns an object indicating the success or failure of the
operation.  Valid values to be returned are:

\begin{itemize}[noitemsep]
\item \code{True} if the operation succeeded
\item \code{False} if any provided predicates failed.
\item \code{null} if the operation requires an existing value and none exists
\end{itemize}

On error, this function will raise a \code{HyperDexClientException} describing
the error.


%%%%%%%%%%%%%%%%%%%% search %%%%%%%%%%%%%%%%%%%%
\pagebreak
\subsubsection{\code{search}}
\label{api:nodejs:search}
\index{search!Node.js API}
Return all objects that match the specified \code{checks}.

\paragraph{Behavior:}
\begin{itemize}[noitemsep]
\item XXX % XXX

\item XXX % XXX

\end{itemize}


\paragraph{Definition:}
\begin{javascriptcode}
search(spacename, predicates, function (obj, err) {})
\end{javascriptcode}
\paragraph{Parameters:}
\begin{itemize}[noitemsep]
\item \code{spacename}\\
The name of the space as a string or symbol.

\item \code{predicates}\\
A hash of predicates to check against.

\end{itemize}

\paragraph{Returns:}
Two channels, one for returning objects that match the search, and one for
returning errors encountered during the search.


%%%%%%%%%%%%%%%%%%%% search_describe %%%%%%%%%%%%%%%%%%%%
\pagebreak
\subsubsection{\code{search\_describe}}
\label{api:nodejs:search_describe}
\index{search\_describe!Node.js API}
Return a human-readable string suitable for debugging search internals.  This
API is only really relevant for debugging the internals of \code{search}.


\paragraph{Definition:}
\begin{javascriptcode}
search_describe(spacename, predicates, function (desc, err) {})
\end{javascriptcode}
\paragraph{Parameters:}
\begin{itemize}[noitemsep]
\item \code{spacename}\\
The name of the space as a string or symbol.

\item \code{predicates}\\
A hash of predicates to check against.

\end{itemize}

\paragraph{Returns:}
Description of search.  Raises exception on error.


%%%%%%%%%%%%%%%%%%%% sorted_search %%%%%%%%%%%%%%%%%%%%
\pagebreak
\subsubsection{\code{sorted\_search}}
\label{api:nodejs:sorted_search}
\index{sorted\_search!Node.js API}
Return all objects that match the specified \code{checks}, sorted according to
\code{attr}.
\item XXX % XXX



\paragraph{Definition:}
\begin{javascriptcode}
sorted_search(spacename, predicates, sortby, limit, maxmin, function (obj, err) {})
\end{javascriptcode}
\paragraph{Parameters:}
\begin{itemize}[noitemsep]
\item \code{spacename}\\
The name of the space as a string or symbol.

\item \code{predicates}\\
A hash of predicates to check against.

\item \code{sortby}\\
XXX

\item \code{limit}\\
XXX

\item \code{maxmin}\\
Maximize (!= 0) or minimize (== 0).

\end{itemize}

\paragraph{Returns:}
Two channels, one for returning objects that match the search, and one for
returning errors encountered during the search.


%%%%%%%%%%%%%%%%%%%% count %%%%%%%%%%%%%%%%%%%%
\pagebreak
\subsubsection{\code{count}}
\label{api:nodejs:count}
\index{count!Node.js API}
Count the number of objects that match the specified \code{checks}.

\paragraph{Behavior:}
\begin{itemize}[noitemsep]
\item This will return the number of objects counted by the search.  If an error
    occurs during the count, the count will reflect a partial count.  The real
    count will be higher than the returned value.  Some languages will throw an
    exception rather than return the partial count.
\end{itemize}


\paragraph{Definition:}
\begin{javascriptcode}
count(spacename, predicates, function (count, err) {})
\end{javascriptcode}
\paragraph{Parameters:}
\begin{itemize}[noitemsep]
\item \code{spacename}\\
The name of the space as a string or symbol.

\item \code{predicates}\\
A hash of predicates to check against.

\end{itemize}

\paragraph{Returns:}
A count of the number of objects, and a \code{client.Error} object indicating
the status of the operation.



\subsection{Working with Signals}
\label{sec:api:node:signals}

Your application must mask all signals prior to making any calls into the Node
bindings.  The Node bindings will unmask the signals during blocking operations
and raise a \code{HyperDexClientException} with status
\code{'HYPERDEX\_CLIENT\_INTERRUPTED'} should any signals be received.

\subsection{Working with Events}
\label{sec:api:node:threads}

The Node module naturally integrates with the Node.js event loop.  Each instance
of \code{Client} registers itself with the Node event loop and makes callbacks
as soon as events complete on the HyperDex side.

Put simply, a Node.js application can use \code{Client} instances in a
straight-forward fashion without worrying about threading or manual integration.
