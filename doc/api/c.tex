\chapter{C API}

\section{Client Library}
% Copyright (c) 2013-2014, Cornell University
% All rights reserved.
%
% Redistribution and use in source and binary forms, with or without
% modification, are permitted provided that the following conditions are met:
%
%     * Redistributions of source code must retain the above copyright notice,
%       this list of conditions and the following disclaimer.
%     * Redistributions in binary form must reproduce the above copyright
%       notice, this list of conditions and the following disclaimer in the
%       documentation and/or other materials provided with the distribution.
%     * Neither the name of HyperDex nor the names of its contributors may be
%       used to endorse or promote products derived from this software without
%       specific prior written permission.
%
% THIS SOFTWARE IS PROVIDED BY THE COPYRIGHT HOLDERS AND CONTRIBUTORS "AS IS"
% AND ANY EXPRESS OR IMPLIED WARRANTIES, INCLUDING, BUT NOT LIMITED TO, THE
% IMPLIED WARRANTIES OF MERCHANTABILITY AND FITNESS FOR A PARTICULAR PURPOSE ARE
% DISCLAIMED. IN NO EVENT SHALL THE COPYRIGHT OWNER OR CONTRIBUTORS BE LIABLE
% FOR ANY DIRECT, INDIRECT, INCIDENTAL, SPECIAL, EXEMPLARY, OR CONSEQUENTIAL
% DAMAGES (INCLUDING, BUT NOT LIMITED TO, PROCUREMENT OF SUBSTITUTE GOODS OR
% SERVICES; LOSS OF USE, DATA, OR PROFITS; OR BUSINESS INTERRUPTION) HOWEVER
% CAUSED AND ON ANY THEORY OF LIABILITY, WHETHER IN CONTRACT, STRICT LIABILITY,
% OR TORT (INCLUDING NEGLIGENCE OR OTHERWISE) ARISING IN ANY WAY OUT OF THE USE
% OF THIS SOFTWARE, EVEN IF ADVISED OF THE POSSIBILITY OF SUCH DAMAGE.

% This LaTeX file is generated by bindings/c.py

%%%%%%%%%%%%%%%%%%%% get %%%%%%%%%%%%%%%%%%%%
\pagebreak
\subsubsection{\code{get}}
\label{api:c:get}
\index{get!C API}
Get an object by key.

\paragraph{Behavior:}
\begin{itemize}[noitemsep]
\item XXX % XXX

\end{itemize}


\paragraph{Definition:}
\begin{ccode}
int64_t hyperdex_client_get(struct hyperdex_client* client,
        const char* space,
        const char* key, size_t key_sz,
        enum hyperdex_client_returncode* status,
        const struct hyperdex_client_attribute** attrs, size_t* attrs_sz);
\end{ccode}

\paragraph{Parameters:}
\begin{itemize}[noitemsep]
\item \code{struct hyperdex\_client* client}\\
The HyperDex client connection to use for the operation.

\item \code{const char* space}\\
The name of the space as a string or symbol.

\item \code{const char* key, size\_t key\_sz}\\
The key for the operation as a Python type.

\end{itemize}

\paragraph{Returns:}
\begin{itemize}[noitemsep]
\item \code{enum hyperdex\_client\_returncode* status}\\
The status of the operation.  The client library will fill in this variable
before returning this operation's request id from \code{hyperdex\_client\_loop}.
The pointer must remain valid until the operation completes, and the pointer
should not be aliased to the status for any other outstanding operation.

\item \code{const struct hyperdex\_client\_attribute** attrs, size\_t* attrs\_sz}\\
XXX

\end{itemize}

%%%%%%%%%%%%%%%%%%%% get_partial %%%%%%%%%%%%%%%%%%%%
\pagebreak
\subsubsection{\code{get\_partial}}
\label{api:c:get_partial}
\index{get\_partial!C API}
Get part of an object by key.  This will return only the listed attribute names.

\paragraph{Behavior:}
\begin{itemize}[noitemsep]
\item XXX % XXX

\end{itemize}


\paragraph{Definition:}
\begin{ccode}
int64_t hyperdex_client_get_partial(struct hyperdex_client* client,
        const char* space,
        const char* key, size_t key_sz,
        const char** attrnames, size_t attrnames_sz,
        enum hyperdex_client_returncode* status,
        const struct hyperdex_client_attribute** attrs, size_t* attrs_sz);
\end{ccode}

\paragraph{Parameters:}
\begin{itemize}[noitemsep]
\item \code{struct hyperdex\_client* client}\\
The HyperDex client connection to use for the operation.

\item \code{const char* space}\\
The name of the space as a string or symbol.

\item \code{const char* key, size\_t key\_sz}\\
The key for the operation as a Python type.

\item \code{const char** attrnames, size\_t attrnames\_sz}\\
A list of attributes to return.  \code{attrnames} is a \code{List<String>}.

\end{itemize}

\paragraph{Returns:}
\begin{itemize}[noitemsep]
\item \code{enum hyperdex\_client\_returncode* status}\\
The status of the operation.  The client library will fill in this variable
before returning this operation's request id from \code{hyperdex\_client\_loop}.
The pointer must remain valid until the operation completes, and the pointer
should not be aliased to the status for any other outstanding operation.

\item \code{const struct hyperdex\_client\_attribute** attrs, size\_t* attrs\_sz}\\
XXX

\end{itemize}

%%%%%%%%%%%%%%%%%%%% put %%%%%%%%%%%%%%%%%%%%
\pagebreak
\subsubsection{\code{put}}
\label{api:c:put}
\index{put!C API}
Store or update an object by key.  The object's attributes will be set to the
values specified by \code{attrs}.
\item An existing object will be updated by the operation.  If no object does
    exists, a new object will be created, with attributes initialized to their
    default values.



\paragraph{Definition:}
\begin{ccode}
int64_t hyperdex_client_put(struct hyperdex_client* client,
        const char* space,
        const char* key, size_t key_sz,
        const struct hyperdex_client_attribute* attrs, size_t attrs_sz,
        enum hyperdex_client_returncode* status);
\end{ccode}

\paragraph{Parameters:}
\begin{itemize}[noitemsep]
\item \code{struct hyperdex\_client* client}\\
The HyperDex client connection to use for the operation.

\item \code{const char* space}\\
The name of the space as a string or symbol.

\item \code{const char* key, size\_t key\_sz}\\
The key for the operation as a Python type.

\item \code{const struct hyperdex\_client\_attribute* attrs, size\_t attrs\_sz}\\
The set of attributes to modify and their respective values.  \code{attrs} is a
map from the attributes' names to their values.

\end{itemize}

\paragraph{Returns:}
\begin{itemize}[noitemsep]
\item \code{enum hyperdex\_client\_returncode* status}\\
The status of the operation.  The client library will fill in this variable
before returning this operation's request id from \code{hyperdex\_client\_loop}.
The pointer must remain valid until the operation completes, and the pointer
should not be aliased to the status for any other outstanding operation.

\end{itemize}

%%%%%%%%%%%%%%%%%%%% cond_put %%%%%%%%%%%%%%%%%%%%
\pagebreak
\subsubsection{\code{cond\_put}}
\label{api:c:cond_put}
\index{cond\_put!C API}
Conditionally write the specified attributes to the object in space "space" under key "key".

The operation will modify the object if and only if all \texttt{checks} are true
for the latest version of the object.  This test is atomic with the write.  If
the object does not exist, the checks will fail.

The attributes specified by \texttt{attrs} will be overwritten with their
respective values.  Any attributes not specified by \texttt{attrs} will be
preserved.

\noindent\textbf{Cost:}  Approximately one traversal of the value-dependent
chain.


\noindent\textbf{Consistency:}  Linearizable



\paragraph{Definition:}
\begin{ccode}
int64_t hyperdex_client_cond_put(struct hyperdex_client* client,
        const char* space,
        const char* key, size_t key_sz,
        const struct hyperdex_client_attribute_check* checks, size_t checks_sz,
        const struct hyperdex_client_attribute* attrs, size_t attrs_sz,
        enum hyperdex_client_returncode* status);
\end{ccode}

\paragraph{Parameters:}
\begin{itemize}[noitemsep]
\item \code{struct hyperdex\_client* client}\\
The HyperDex client connection to use for the operation.

\item \code{const char* space}\\
The name of the space as a string or symbol.

\item \code{const char* key, size\_t key\_sz}\\
The key for the operation as a Python type.

\item \code{const struct hyperdex\_client\_attribute\_check* checks, size\_t checks\_sz}\\
A hash of predicates to check against.

\item \code{const struct hyperdex\_client\_attribute* attrs, size\_t attrs\_sz}\\
The set of attributes to modify and their respective values.  \code{attrs} is a
map from the attributes' names to their values.

\end{itemize}

\paragraph{Returns:}
\begin{itemize}[noitemsep]
\item \code{enum hyperdex\_client\_returncode* status}\\
The status of the operation.  The client library will fill in this variable
before returning this operation's request id from \code{hyperdex\_client\_loop}.
The pointer must remain valid until the operation completes, and the pointer
should not be aliased to the status for any other outstanding operation.

\end{itemize}

%%%%%%%%%%%%%%%%%%%% put_if_not_exist %%%%%%%%%%%%%%%%%%%%
\pagebreak
\subsubsection{\code{put\_if\_not\_exist}}
\label{api:c:put_if_not_exist}
\index{put\_if\_not\_exist!C API}
Store or object under \code{key} in \code{space} if and only if the operation
creates a new object.  The object's attributes will be set to the values
specified by \code{attrs}; any attributes not specified by \code{attrs} will be
initialized to their defaults.  If the object exists, the operation will fail
with \code{CMPFAIL}.


\paragraph{Definition:}
\begin{ccode}
int64_t hyperdex_client_put_if_not_exist(struct hyperdex_client* client,
        const char* space,
        const char* key, size_t key_sz,
        const struct hyperdex_client_attribute* attrs, size_t attrs_sz,
        enum hyperdex_client_returncode* status);
\end{ccode}

\paragraph{Parameters:}
\begin{itemize}[noitemsep]
\item \code{struct hyperdex\_client* client}\\
The HyperDex client connection to use for the operation.

\item \code{const char* space}\\
The name of the space as a string or symbol.

\item \code{const char* key, size\_t key\_sz}\\
The key for the operation as a Python type.

\item \code{const struct hyperdex\_client\_attribute* attrs, size\_t attrs\_sz}\\
The set of attributes to modify and their respective values.  \code{attrs} is a
map from the attributes' names to their values.

\end{itemize}

\paragraph{Returns:}
\begin{itemize}[noitemsep]
\item \code{enum hyperdex\_client\_returncode* status}\\
The status of the operation.  The client library will fill in this variable
before returning this operation's request id from \code{hyperdex\_client\_loop}.
The pointer must remain valid until the operation completes, and the pointer
should not be aliased to the status for any other outstanding operation.

\end{itemize}

%%%%%%%%%%%%%%%%%%%% del %%%%%%%%%%%%%%%%%%%%
\pagebreak
\subsubsection{\code{del}}
\label{api:c:del}
\index{del!C API}
Delete an object by key.

%%% Generated below here
\paragraph{Behavior:}
\begin{itemize}[noitemsep]
If no object exists, the operation will fail with \code{NOTFOUND}.

\end{itemize}


\paragraph{Definition:}
\begin{ccode}
int64_t hyperdex_client_del(struct hyperdex_client* client,
        const char* space,
        const char* key, size_t key_sz,
        enum hyperdex_client_returncode* status);
\end{ccode}

\paragraph{Parameters:}
\begin{itemize}[noitemsep]
\item \code{struct hyperdex\_client* client}\\
The HyperDex client connection to use for the operation.

\item \code{const char* space}\\
The name of the space as a string or symbol.

\item \code{const char* key, size\_t key\_sz}\\
The key for the operation as a Python type.

\end{itemize}

\paragraph{Returns:}
\begin{itemize}[noitemsep]
\item \code{enum hyperdex\_client\_returncode* status}\\
The status of the operation.  The client library will fill in this variable
before returning this operation's request id from \code{hyperdex\_client\_loop}.
The pointer must remain valid until the operation completes, and the pointer
should not be aliased to the status for any other outstanding operation.

\end{itemize}

%%%%%%%%%%%%%%%%%%%% cond_del %%%%%%%%%%%%%%%%%%%%
\pagebreak
\subsubsection{\code{cond\_del}}
\label{api:c:cond_del}
\index{cond\_del!C API}
Conditionally delete the object stored under \code{key} from \code{space}.
If no object exists, the operation will fail with \code{NOTFOUND}.


This operation will succeed if and only if the predicates specified by
\code{checks} hold on the pre-existing object.  If any of the predicates are not
true for the existing object, then the operation will have no effect and fail
with \code{CMPFAIL}.

All checks are atomic with the write.  HyperDex guarantees that no other
operation will come between validating the checks, and writing the new version
of the object.



\paragraph{Definition:}
\begin{ccode}
int64_t hyperdex_client_cond_del(struct hyperdex_client* client,
        const char* space,
        const char* key, size_t key_sz,
        const struct hyperdex_client_attribute_check* checks, size_t checks_sz,
        enum hyperdex_client_returncode* status);
\end{ccode}

\paragraph{Parameters:}
\begin{itemize}[noitemsep]
\item \code{struct hyperdex\_client* client}\\
The HyperDex client connection to use for the operation.

\item \code{const char* space}\\
The name of the space as a string or symbol.

\item \code{const char* key, size\_t key\_sz}\\
The key for the operation as a Python type.

\item \code{const struct hyperdex\_client\_attribute\_check* checks, size\_t checks\_sz}\\
A hash of predicates to check against.

\end{itemize}

\paragraph{Returns:}
\begin{itemize}[noitemsep]
\item \code{enum hyperdex\_client\_returncode* status}\\
The status of the operation.  The client library will fill in this variable
before returning this operation's request id from \code{hyperdex\_client\_loop}.
The pointer must remain valid until the operation completes, and the pointer
should not be aliased to the status for any other outstanding operation.

\end{itemize}

%%%%%%%%%%%%%%%%%%%% atomic_add %%%%%%%%%%%%%%%%%%%%
\pagebreak
\subsubsection{\code{atomic\_add}}
\label{api:c:atomic_add}
\index{atomic\_add!C API}
Add the specified number to the existing value for each attribute.

%%% Generated below here
\paragraph{Behavior:}
\begin{itemize}[noitemsep]
This operation requires a pre-existing object in order to complete successfully.
If no object exists, the operation will fail with \code{NOTFOUND}.

\end{itemize}


\paragraph{Definition:}
\begin{ccode}
int64_t hyperdex_client_atomic_add(struct hyperdex_client* client,
        const char* space,
        const char* key, size_t key_sz,
        const struct hyperdex_client_attribute* attrs, size_t attrs_sz,
        enum hyperdex_client_returncode* status);
\end{ccode}

\paragraph{Parameters:}
\begin{itemize}[noitemsep]
\item \code{struct hyperdex\_client* client}\\
The HyperDex client connection to use for the operation.

\item \code{const char* space}\\
The name of the space as a string or symbol.

\item \code{const char* key, size\_t key\_sz}\\
The key for the operation as a Python type.

\item \code{const struct hyperdex\_client\_attribute* attrs, size\_t attrs\_sz}\\
The set of attributes to modify and their respective values.  \code{attrs} is a
map from the attributes' names to their values.

\end{itemize}

\paragraph{Returns:}
\begin{itemize}[noitemsep]
\item \code{enum hyperdex\_client\_returncode* status}\\
The status of the operation.  The client library will fill in this variable
before returning this operation's request id from \code{hyperdex\_client\_loop}.
The pointer must remain valid until the operation completes, and the pointer
should not be aliased to the status for any other outstanding operation.

\end{itemize}

%%%%%%%%%%%%%%%%%%%% group_atomic_add %%%%%%%%%%%%%%%%%%%%
\pagebreak
\subsubsection{\code{group\_atomic\_add}}
\label{api:c:group_atomic_add}
\index{group\_atomic\_add!C API}
Add the specified number to the existing value for each object in \code{space}
that matches \code{checks}.

This operation will only affect objects that match the provided \code{checks}.
Objects that do not match \code{checks} will be unaffected by the group call.
Each object that matches \code{checks} will be atomically updated with the check
on the object.  HyperDex guarantees that no object will be altered if the
\code{checks} do not pass at the time of the write.  Objects that are updated
concurrently with the group call may or may not be updated; however, regardless
of any other concurrent operations, the preceding guarantee will always hold.



\paragraph{Definition:}
\begin{ccode}
int64_t hyperdex_client_group_atomic_add(struct hyperdex_client* client,
        const char* space,
        const struct hyperdex_client_attribute_check* checks, size_t checks_sz,
        const struct hyperdex_client_attribute* attrs, size_t attrs_sz,
        enum hyperdex_client_returncode* status);
\end{ccode}

\paragraph{Parameters:}
\begin{itemize}[noitemsep]
\item \code{struct hyperdex\_client* client}\\
The HyperDex client connection to use for the operation.

\item \code{const char* space}\\
The name of the space as a string or symbol.

\item \code{const struct hyperdex\_client\_attribute\_check* checks, size\_t checks\_sz}\\
A hash of predicates to check against.

\item \code{const struct hyperdex\_client\_attribute* attrs, size\_t attrs\_sz}\\
The set of attributes to modify and their respective values.  \code{attrs} is a
map from the attributes' names to their values.

\end{itemize}

\paragraph{Returns:}
\begin{itemize}[noitemsep]
\item \code{enum hyperdex\_client\_returncode* status}\\
The status of the operation.  The client library will fill in this variable
before returning this operation's request id from \code{hyperdex\_client\_loop}.
The pointer must remain valid until the operation completes, and the pointer
should not be aliased to the status for any other outstanding operation.

\end{itemize}

%%%%%%%%%%%%%%%%%%%% cond_atomic_add %%%%%%%%%%%%%%%%%%%%
\pagebreak
\subsubsection{\code{cond\_atomic\_add}}
\label{api:c:cond_atomic_add}
\index{cond\_atomic\_add!C API}
Conditionally add the specified number to the existing value for each attribute.

%%% Generated below here
\paragraph{Behavior:}
\begin{itemize}[noitemsep]
This operation requires a pre-existing object in order to complete successfully.
If no object exists, the operation will fail with \code{NOTFOUND}.

This operation will succeed if and only if the predicates specified by
\code{checks} hold on the pre-existing object.  If any of the predicates are not
true for the existing object, then the operation will have no effect and fail
with \code{CMPFAIL}.

All checks are atomic with the write.  HyperDex guarantees that no other
operation will come between validating the checks, and writing the new version
of the object.

\end{itemize}


\paragraph{Definition:}
\begin{ccode}
int64_t hyperdex_client_cond_atomic_add(struct hyperdex_client* client,
        const char* space,
        const char* key, size_t key_sz,
        const struct hyperdex_client_attribute_check* checks, size_t checks_sz,
        const struct hyperdex_client_attribute* attrs, size_t attrs_sz,
        enum hyperdex_client_returncode* status);
\end{ccode}

\paragraph{Parameters:}
\begin{itemize}[noitemsep]
\item \code{struct hyperdex\_client* client}\\
The HyperDex client connection to use for the operation.

\item \code{const char* space}\\
The name of the space as a string or symbol.

\item \code{const char* key, size\_t key\_sz}\\
The key for the operation as a Python type.

\item \code{const struct hyperdex\_client\_attribute\_check* checks, size\_t checks\_sz}\\
A hash of predicates to check against.

\item \code{const struct hyperdex\_client\_attribute* attrs, size\_t attrs\_sz}\\
The set of attributes to modify and their respective values.  \code{attrs} is a
map from the attributes' names to their values.

\end{itemize}

\paragraph{Returns:}
\begin{itemize}[noitemsep]
\item \code{enum hyperdex\_client\_returncode* status}\\
The status of the operation.  The client library will fill in this variable
before returning this operation's request id from \code{hyperdex\_client\_loop}.
The pointer must remain valid until the operation completes, and the pointer
should not be aliased to the status for any other outstanding operation.

\end{itemize}

%%%%%%%%%%%%%%%%%%%% atomic_sub %%%%%%%%%%%%%%%%%%%%
\pagebreak
\subsubsection{\code{atomic\_sub}}
\label{api:c:atomic_sub}
\index{atomic\_sub!C API}
Subtract the specified number from the existing value for each attribute.

%%% Generated below here
\paragraph{Behavior:}
\begin{itemize}[noitemsep]
This operation requires a pre-existing object in order to complete successfully.
If no object exists, the operation will fail with \code{NOTFOUND}.

\end{itemize}


\paragraph{Definition:}
\begin{ccode}
int64_t hyperdex_client_atomic_sub(struct hyperdex_client* client,
        const char* space,
        const char* key, size_t key_sz,
        const struct hyperdex_client_attribute* attrs, size_t attrs_sz,
        enum hyperdex_client_returncode* status);
\end{ccode}

\paragraph{Parameters:}
\begin{itemize}[noitemsep]
\item \code{struct hyperdex\_client* client}\\
The HyperDex client connection to use for the operation.

\item \code{const char* space}\\
The name of the space as a string or symbol.

\item \code{const char* key, size\_t key\_sz}\\
The key for the operation as a Python type.

\item \code{const struct hyperdex\_client\_attribute* attrs, size\_t attrs\_sz}\\
The set of attributes to modify and their respective values.  \code{attrs} is a
map from the attributes' names to their values.

\end{itemize}

\paragraph{Returns:}
\begin{itemize}[noitemsep]
\item \code{enum hyperdex\_client\_returncode* status}\\
The status of the operation.  The client library will fill in this variable
before returning this operation's request id from \code{hyperdex\_client\_loop}.
The pointer must remain valid until the operation completes, and the pointer
should not be aliased to the status for any other outstanding operation.

\end{itemize}

%%%%%%%%%%%%%%%%%%%% cond_atomic_sub %%%%%%%%%%%%%%%%%%%%
\pagebreak
\subsubsection{\code{cond\_atomic\_sub}}
\label{api:c:cond_atomic_sub}
\index{cond\_atomic\_sub!C API}
Conditionally subtract the specified number from the existing value for each attribute.

%%% Generated below here
\paragraph{Behavior:}
\begin{itemize}[noitemsep]
This operation requires a pre-existing object in order to complete successfully.
If no object exists, the operation will fail with \code{NOTFOUND}.

This operation will succeed if and only if the predicates specified by
\code{checks} hold on the pre-existing object.  If any of the predicates are not
true for the existing object, then the operation will have no effect and fail
with \code{CMPFAIL}.

All checks are atomic with the write.  HyperDex guarantees that no other
operation will come between validating the checks, and writing the new version
of the object.

\end{itemize}


\paragraph{Definition:}
\begin{ccode}
int64_t hyperdex_client_cond_atomic_sub(struct hyperdex_client* client,
        const char* space,
        const char* key, size_t key_sz,
        const struct hyperdex_client_attribute_check* checks, size_t checks_sz,
        const struct hyperdex_client_attribute* attrs, size_t attrs_sz,
        enum hyperdex_client_returncode* status);
\end{ccode}

\paragraph{Parameters:}
\begin{itemize}[noitemsep]
\item \code{struct hyperdex\_client* client}\\
The HyperDex client connection to use for the operation.

\item \code{const char* space}\\
The name of the space as a string or symbol.

\item \code{const char* key, size\_t key\_sz}\\
The key for the operation as a Python type.

\item \code{const struct hyperdex\_client\_attribute\_check* checks, size\_t checks\_sz}\\
A hash of predicates to check against.

\item \code{const struct hyperdex\_client\_attribute* attrs, size\_t attrs\_sz}\\
The set of attributes to modify and their respective values.  \code{attrs} is a
map from the attributes' names to their values.

\end{itemize}

\paragraph{Returns:}
\begin{itemize}[noitemsep]
\item \code{enum hyperdex\_client\_returncode* status}\\
The status of the operation.  The client library will fill in this variable
before returning this operation's request id from \code{hyperdex\_client\_loop}.
The pointer must remain valid until the operation completes, and the pointer
should not be aliased to the status for any other outstanding operation.

\end{itemize}

%%%%%%%%%%%%%%%%%%%% atomic_mul %%%%%%%%%%%%%%%%%%%%
\pagebreak
\subsubsection{\code{atomic\_mul}}
\label{api:c:atomic_mul}
\index{atomic\_mul!C API}
Multiply the existing value by the number specified for each attribute.

The multiplication is atomic with the write.  If the object does not exist, the
operation will fail.

\noindent\textbf{Cost:}  Approximately one traversal of the value-dependent
chain.


\noindent\textbf{Consistency:}  Linearizable



\paragraph{Definition:}
\begin{ccode}
int64_t hyperdex_client_atomic_mul(struct hyperdex_client* client,
        const char* space,
        const char* key, size_t key_sz,
        const struct hyperdex_client_attribute* attrs, size_t attrs_sz,
        enum hyperdex_client_returncode* status);
\end{ccode}

\paragraph{Parameters:}
\begin{itemize}[noitemsep]
\item \code{struct hyperdex\_client* client}\\
The HyperDex client connection to use for the operation.

\item \code{const char* space}\\
The name of the space as a string or symbol.

\item \code{const char* key, size\_t key\_sz}\\
The key for the operation as a Python type.

\item \code{const struct hyperdex\_client\_attribute* attrs, size\_t attrs\_sz}\\
The set of attributes to modify and their respective values.  \code{attrs} is a
map from the attributes' names to their values.

\end{itemize}

\paragraph{Returns:}
\begin{itemize}[noitemsep]
\item \code{enum hyperdex\_client\_returncode* status}\\
The status of the operation.  The client library will fill in this variable
before returning this operation's request id from \code{hyperdex\_client\_loop}.
The pointer must remain valid until the operation completes, and the pointer
should not be aliased to the status for any other outstanding operation.

\end{itemize}

%%%%%%%%%%%%%%%%%%%% cond_atomic_mul %%%%%%%%%%%%%%%%%%%%
\pagebreak
\subsubsection{\code{cond\_atomic\_mul}}
\label{api:c:cond_atomic_mul}
\index{cond\_atomic\_mul!C API}
Conditionally multiply the existing value by the specified number for each
attribute.

%%% Generated below here
\paragraph{Behavior:}
\begin{itemize}[noitemsep]
This operation requires a pre-existing object in order to complete successfully.
If no object exists, the operation will fail with \code{NOTFOUND}.

This operation will succeed if and only if the predicates specified by
\code{checks} hold on the pre-existing object.  If any of the predicates are not
true for the existing object, then the operation will have no effect and fail
with \code{CMPFAIL}.

All checks are atomic with the write.  HyperDex guarantees that no other
operation will come between validating the checks, and writing the new version
of the object.

\end{itemize}


\paragraph{Definition:}
\begin{ccode}
int64_t hyperdex_client_cond_atomic_mul(struct hyperdex_client* client,
        const char* space,
        const char* key, size_t key_sz,
        const struct hyperdex_client_attribute_check* checks, size_t checks_sz,
        const struct hyperdex_client_attribute* attrs, size_t attrs_sz,
        enum hyperdex_client_returncode* status);
\end{ccode}

\paragraph{Parameters:}
\begin{itemize}[noitemsep]
\item \code{struct hyperdex\_client* client}\\
The HyperDex client connection to use for the operation.

\item \code{const char* space}\\
The name of the space as a string or symbol.

\item \code{const char* key, size\_t key\_sz}\\
The key for the operation as a Python type.

\item \code{const struct hyperdex\_client\_attribute\_check* checks, size\_t checks\_sz}\\
A hash of predicates to check against.

\item \code{const struct hyperdex\_client\_attribute* attrs, size\_t attrs\_sz}\\
The set of attributes to modify and their respective values.  \code{attrs} is a
map from the attributes' names to their values.

\end{itemize}

\paragraph{Returns:}
\begin{itemize}[noitemsep]
\item \code{enum hyperdex\_client\_returncode* status}\\
The status of the operation.  The client library will fill in this variable
before returning this operation's request id from \code{hyperdex\_client\_loop}.
The pointer must remain valid until the operation completes, and the pointer
should not be aliased to the status for any other outstanding operation.

\end{itemize}

%%%%%%%%%%%%%%%%%%%% atomic_div %%%%%%%%%%%%%%%%%%%%
\pagebreak
\subsubsection{\code{atomic\_div}}
\label{api:c:atomic_div}
\index{atomic\_div!C API}
Divide the existing value by the specified number for each attribute.
This operation requires a pre-existing object in order to complete successfully.
If no object exists, the operation will fail with \code{NOTFOUND}.



\paragraph{Definition:}
\begin{ccode}
int64_t hyperdex_client_atomic_div(struct hyperdex_client* client,
        const char* space,
        const char* key, size_t key_sz,
        const struct hyperdex_client_attribute* attrs, size_t attrs_sz,
        enum hyperdex_client_returncode* status);
\end{ccode}

\paragraph{Parameters:}
\begin{itemize}[noitemsep]
\item \code{struct hyperdex\_client* client}\\
The HyperDex client connection to use for the operation.

\item \code{const char* space}\\
The name of the space as a string or symbol.

\item \code{const char* key, size\_t key\_sz}\\
The key for the operation as a Python type.

\item \code{const struct hyperdex\_client\_attribute* attrs, size\_t attrs\_sz}\\
The set of attributes to modify and their respective values.  \code{attrs} is a
map from the attributes' names to their values.

\end{itemize}

\paragraph{Returns:}
\begin{itemize}[noitemsep]
\item \code{enum hyperdex\_client\_returncode* status}\\
The status of the operation.  The client library will fill in this variable
before returning this operation's request id from \code{hyperdex\_client\_loop}.
The pointer must remain valid until the operation completes, and the pointer
should not be aliased to the status for any other outstanding operation.

\end{itemize}

%%%%%%%%%%%%%%%%%%%% cond_atomic_div %%%%%%%%%%%%%%%%%%%%
\pagebreak
\subsubsection{\code{cond\_atomic\_div}}
\label{api:c:cond_atomic_div}
\index{cond\_atomic\_div!C API}
Conditionally divide the existing value by the specified number for each
attribute.

%%% Generated below here
\paragraph{Behavior:}
\begin{itemize}[noitemsep]
This operation requires a pre-existing object in order to complete successfully.
If no object exists, the operation will fail with \code{NOTFOUND}.

This operation will succeed if and only if the predicates specified by
\code{checks} hold on the pre-existing object.  If any of the predicates are not
true for the existing object, then the operation will have no effect and fail
with \code{CMPFAIL}.

All checks are atomic with the write.  HyperDex guarantees that no other
operation will come between validating the checks, and writing the new version
of the object.

\end{itemize}


\paragraph{Definition:}
\begin{ccode}
int64_t hyperdex_client_cond_atomic_div(struct hyperdex_client* client,
        const char* space,
        const char* key, size_t key_sz,
        const struct hyperdex_client_attribute_check* checks, size_t checks_sz,
        const struct hyperdex_client_attribute* attrs, size_t attrs_sz,
        enum hyperdex_client_returncode* status);
\end{ccode}

\paragraph{Parameters:}
\begin{itemize}[noitemsep]
\item \code{struct hyperdex\_client* client}\\
The HyperDex client connection to use for the operation.

\item \code{const char* space}\\
The name of the space as a string or symbol.

\item \code{const char* key, size\_t key\_sz}\\
The key for the operation as a Python type.

\item \code{const struct hyperdex\_client\_attribute\_check* checks, size\_t checks\_sz}\\
A hash of predicates to check against.

\item \code{const struct hyperdex\_client\_attribute* attrs, size\_t attrs\_sz}\\
The set of attributes to modify and their respective values.  \code{attrs} is a
map from the attributes' names to their values.

\end{itemize}

\paragraph{Returns:}
\begin{itemize}[noitemsep]
\item \code{enum hyperdex\_client\_returncode* status}\\
The status of the operation.  The client library will fill in this variable
before returning this operation's request id from \code{hyperdex\_client\_loop}.
The pointer must remain valid until the operation completes, and the pointer
should not be aliased to the status for any other outstanding operation.

\end{itemize}

%%%%%%%%%%%%%%%%%%%% atomic_mod %%%%%%%%%%%%%%%%%%%%
\pagebreak
\subsubsection{\code{atomic\_mod}}
\label{api:c:atomic_mod}
\index{atomic\_mod!C API}
Store the existing value modulo the specified number for each attribute.

%%% Generated below here
\paragraph{Behavior:}
\begin{itemize}[noitemsep]
This operation requires a pre-existing object in order to complete successfully.
If no object exists, the operation will fail with \code{NOTFOUND}.

\end{itemize}


\paragraph{Definition:}
\begin{ccode}
int64_t hyperdex_client_atomic_mod(struct hyperdex_client* client,
        const char* space,
        const char* key, size_t key_sz,
        const struct hyperdex_client_attribute* attrs, size_t attrs_sz,
        enum hyperdex_client_returncode* status);
\end{ccode}

\paragraph{Parameters:}
\begin{itemize}[noitemsep]
\item \code{struct hyperdex\_client* client}\\
The HyperDex client connection to use for the operation.

\item \code{const char* space}\\
The name of the space as a string or symbol.

\item \code{const char* key, size\_t key\_sz}\\
The key for the operation as a Python type.

\item \code{const struct hyperdex\_client\_attribute* attrs, size\_t attrs\_sz}\\
The set of attributes to modify and their respective values.  \code{attrs} is a
map from the attributes' names to their values.

\end{itemize}

\paragraph{Returns:}
\begin{itemize}[noitemsep]
\item \code{enum hyperdex\_client\_returncode* status}\\
The status of the operation.  The client library will fill in this variable
before returning this operation's request id from \code{hyperdex\_client\_loop}.
The pointer must remain valid until the operation completes, and the pointer
should not be aliased to the status for any other outstanding operation.

\end{itemize}

%%%%%%%%%%%%%%%%%%%% cond_atomic_mod %%%%%%%%%%%%%%%%%%%%
\pagebreak
\subsubsection{\code{cond\_atomic\_mod}}
\label{api:c:cond_atomic_mod}
\index{cond\_atomic\_mod!C API}
Conditionally store the existing value modulo the specified number for each
attribute.

%%% Generated below here
\paragraph{Behavior:}
\begin{itemize}[noitemsep]
This operation requires a pre-existing object in order to complete successfully.
If no object exists, the operation will fail with \code{NOTFOUND}.

This operation will succeed if and only if the predicates specified by
\code{checks} hold on the pre-existing object.  If any of the predicates are not
true for the existing object, then the operation will have no effect and fail
with \code{CMPFAIL}.

All checks are atomic with the write.  HyperDex guarantees that no other
operation will come between validating the checks, and writing the new version
of the object.

\end{itemize}


\paragraph{Definition:}
\begin{ccode}
int64_t hyperdex_client_cond_atomic_mod(struct hyperdex_client* client,
        const char* space,
        const char* key, size_t key_sz,
        const struct hyperdex_client_attribute_check* checks, size_t checks_sz,
        const struct hyperdex_client_attribute* attrs, size_t attrs_sz,
        enum hyperdex_client_returncode* status);
\end{ccode}

\paragraph{Parameters:}
\begin{itemize}[noitemsep]
\item \code{struct hyperdex\_client* client}\\
The HyperDex client connection to use for the operation.

\item \code{const char* space}\\
The name of the space as a string or symbol.

\item \code{const char* key, size\_t key\_sz}\\
The key for the operation as a Python type.

\item \code{const struct hyperdex\_client\_attribute\_check* checks, size\_t checks\_sz}\\
A hash of predicates to check against.

\item \code{const struct hyperdex\_client\_attribute* attrs, size\_t attrs\_sz}\\
The set of attributes to modify and their respective values.  \code{attrs} is a
map from the attributes' names to their values.

\end{itemize}

\paragraph{Returns:}
\begin{itemize}[noitemsep]
\item \code{enum hyperdex\_client\_returncode* status}\\
The status of the operation.  The client library will fill in this variable
before returning this operation's request id from \code{hyperdex\_client\_loop}.
The pointer must remain valid until the operation completes, and the pointer
should not be aliased to the status for any other outstanding operation.

\end{itemize}

%%%%%%%%%%%%%%%%%%%% atomic_and %%%%%%%%%%%%%%%%%%%%
\pagebreak
\subsubsection{\code{atomic\_and}}
\label{api:c:atomic_and}
\index{atomic\_and!C API}
Store the bitwise AND of the existing value and the specified number for
each attribute.
This operation requires a pre-existing object in order to complete successfully.
If no object exists, the operation will fail with \code{NOTFOUND}.



\paragraph{Definition:}
\begin{ccode}
int64_t hyperdex_client_atomic_and(struct hyperdex_client* client,
        const char* space,
        const char* key, size_t key_sz,
        const struct hyperdex_client_attribute* attrs, size_t attrs_sz,
        enum hyperdex_client_returncode* status);
\end{ccode}

\paragraph{Parameters:}
\begin{itemize}[noitemsep]
\item \code{struct hyperdex\_client* client}\\
The HyperDex client connection to use for the operation.

\item \code{const char* space}\\
The name of the space as a string or symbol.

\item \code{const char* key, size\_t key\_sz}\\
The key for the operation as a Python type.

\item \code{const struct hyperdex\_client\_attribute* attrs, size\_t attrs\_sz}\\
The set of attributes to modify and their respective values.  \code{attrs} is a
map from the attributes' names to their values.

\end{itemize}

\paragraph{Returns:}
\begin{itemize}[noitemsep]
\item \code{enum hyperdex\_client\_returncode* status}\\
The status of the operation.  The client library will fill in this variable
before returning this operation's request id from \code{hyperdex\_client\_loop}.
The pointer must remain valid until the operation completes, and the pointer
should not be aliased to the status for any other outstanding operation.

\end{itemize}

%%%%%%%%%%%%%%%%%%%% cond_atomic_and %%%%%%%%%%%%%%%%%%%%
\pagebreak
\subsubsection{\code{cond\_atomic\_and}}
\label{api:c:cond_atomic_and}
\index{cond\_atomic\_and!C API}
Conditionally store the bitwise AND of the existing value and the specified
number for each attribute.

%%% Generated below here
\paragraph{Behavior:}
\begin{itemize}[noitemsep]
This operation requires a pre-existing object in order to complete successfully.
If no object exists, the operation will fail with \code{NOTFOUND}.

This operation will succeed if and only if the predicates specified by
\code{checks} hold on the pre-existing object.  If any of the predicates are not
true for the existing object, then the operation will have no effect and fail
with \code{CMPFAIL}.

All checks are atomic with the write.  HyperDex guarantees that no other
operation will come between validating the checks, and writing the new version
of the object.

\end{itemize}


\paragraph{Definition:}
\begin{ccode}
int64_t hyperdex_client_cond_atomic_and(struct hyperdex_client* client,
        const char* space,
        const char* key, size_t key_sz,
        const struct hyperdex_client_attribute_check* checks, size_t checks_sz,
        const struct hyperdex_client_attribute* attrs, size_t attrs_sz,
        enum hyperdex_client_returncode* status);
\end{ccode}

\paragraph{Parameters:}
\begin{itemize}[noitemsep]
\item \code{struct hyperdex\_client* client}\\
The HyperDex client connection to use for the operation.

\item \code{const char* space}\\
The name of the space as a string or symbol.

\item \code{const char* key, size\_t key\_sz}\\
The key for the operation as a Python type.

\item \code{const struct hyperdex\_client\_attribute\_check* checks, size\_t checks\_sz}\\
A hash of predicates to check against.

\item \code{const struct hyperdex\_client\_attribute* attrs, size\_t attrs\_sz}\\
The set of attributes to modify and their respective values.  \code{attrs} is a
map from the attributes' names to their values.

\end{itemize}

\paragraph{Returns:}
\begin{itemize}[noitemsep]
\item \code{enum hyperdex\_client\_returncode* status}\\
The status of the operation.  The client library will fill in this variable
before returning this operation's request id from \code{hyperdex\_client\_loop}.
The pointer must remain valid until the operation completes, and the pointer
should not be aliased to the status for any other outstanding operation.

\end{itemize}

%%%%%%%%%%%%%%%%%%%% atomic_or %%%%%%%%%%%%%%%%%%%%
\pagebreak
\subsubsection{\code{atomic\_or}}
\label{api:c:atomic_or}
\index{atomic\_or!C API}
Store the bitwise OR of the existing value and the specified number for each
attribute.

%%% Generated below here
\paragraph{Behavior:}
\begin{itemize}[noitemsep]
This operation requires a pre-existing object in order to complete successfully.
If no object exists, the operation will fail with \code{NOTFOUND}.

\end{itemize}


\paragraph{Definition:}
\begin{ccode}
int64_t hyperdex_client_atomic_or(struct hyperdex_client* client,
        const char* space,
        const char* key, size_t key_sz,
        const struct hyperdex_client_attribute* attrs, size_t attrs_sz,
        enum hyperdex_client_returncode* status);
\end{ccode}

\paragraph{Parameters:}
\begin{itemize}[noitemsep]
\item \code{struct hyperdex\_client* client}\\
The HyperDex client connection to use for the operation.

\item \code{const char* space}\\
The name of the space as a string or symbol.

\item \code{const char* key, size\_t key\_sz}\\
The key for the operation as a Python type.

\item \code{const struct hyperdex\_client\_attribute* attrs, size\_t attrs\_sz}\\
The set of attributes to modify and their respective values.  \code{attrs} is a
map from the attributes' names to their values.

\end{itemize}

\paragraph{Returns:}
\begin{itemize}[noitemsep]
\item \code{enum hyperdex\_client\_returncode* status}\\
The status of the operation.  The client library will fill in this variable
before returning this operation's request id from \code{hyperdex\_client\_loop}.
The pointer must remain valid until the operation completes, and the pointer
should not be aliased to the status for any other outstanding operation.

\end{itemize}

%%%%%%%%%%%%%%%%%%%% cond_atomic_or %%%%%%%%%%%%%%%%%%%%
\pagebreak
\subsubsection{\code{cond\_atomic\_or}}
\label{api:c:cond_atomic_or}
\index{cond\_atomic\_or!C API}
Conditionally store the bitwise OR of the existing value and the specified
number for each attribute.

%%% Generated below here
\paragraph{Behavior:}
\begin{itemize}[noitemsep]
This operation requires a pre-existing object in order to complete successfully.
If no object exists, the operation will fail with \code{NOTFOUND}.

This operation will succeed if and only if the predicates specified by
\code{checks} hold on the pre-existing object.  If any of the predicates are not
true for the existing object, then the operation will have no effect and fail
with \code{CMPFAIL}.

All checks are atomic with the write.  HyperDex guarantees that no other
operation will come between validating the checks, and writing the new version
of the object.

\end{itemize}


\paragraph{Definition:}
\begin{ccode}
int64_t hyperdex_client_cond_atomic_or(struct hyperdex_client* client,
        const char* space,
        const char* key, size_t key_sz,
        const struct hyperdex_client_attribute_check* checks, size_t checks_sz,
        const struct hyperdex_client_attribute* attrs, size_t attrs_sz,
        enum hyperdex_client_returncode* status);
\end{ccode}

\paragraph{Parameters:}
\begin{itemize}[noitemsep]
\item \code{struct hyperdex\_client* client}\\
The HyperDex client connection to use for the operation.

\item \code{const char* space}\\
The name of the space as a string or symbol.

\item \code{const char* key, size\_t key\_sz}\\
The key for the operation as a Python type.

\item \code{const struct hyperdex\_client\_attribute\_check* checks, size\_t checks\_sz}\\
A hash of predicates to check against.

\item \code{const struct hyperdex\_client\_attribute* attrs, size\_t attrs\_sz}\\
The set of attributes to modify and their respective values.  \code{attrs} is a
map from the attributes' names to their values.

\end{itemize}

\paragraph{Returns:}
\begin{itemize}[noitemsep]
\item \code{enum hyperdex\_client\_returncode* status}\\
The status of the operation.  The client library will fill in this variable
before returning this operation's request id from \code{hyperdex\_client\_loop}.
The pointer must remain valid until the operation completes, and the pointer
should not be aliased to the status for any other outstanding operation.

\end{itemize}

%%%%%%%%%%%%%%%%%%%% atomic_xor %%%%%%%%%%%%%%%%%%%%
\pagebreak
\subsubsection{\code{atomic\_xor}}
\label{api:c:atomic_xor}
\index{atomic\_xor!C API}
Store the bitwise XOR of the existing value and the specified number for each
attribute.

%%% Generated below here
\paragraph{Behavior:}
\begin{itemize}[noitemsep]
This operation requires a pre-existing object in order to complete successfully.
If no object exists, the operation will fail with \code{NOTFOUND}.

\end{itemize}


\paragraph{Definition:}
\begin{ccode}
int64_t hyperdex_client_atomic_xor(struct hyperdex_client* client,
        const char* space,
        const char* key, size_t key_sz,
        const struct hyperdex_client_attribute* attrs, size_t attrs_sz,
        enum hyperdex_client_returncode* status);
\end{ccode}

\paragraph{Parameters:}
\begin{itemize}[noitemsep]
\item \code{struct hyperdex\_client* client}\\
The HyperDex client connection to use for the operation.

\item \code{const char* space}\\
The name of the space as a string or symbol.

\item \code{const char* key, size\_t key\_sz}\\
The key for the operation as a Python type.

\item \code{const struct hyperdex\_client\_attribute* attrs, size\_t attrs\_sz}\\
The set of attributes to modify and their respective values.  \code{attrs} is a
map from the attributes' names to their values.

\end{itemize}

\paragraph{Returns:}
\begin{itemize}[noitemsep]
\item \code{enum hyperdex\_client\_returncode* status}\\
The status of the operation.  The client library will fill in this variable
before returning this operation's request id from \code{hyperdex\_client\_loop}.
The pointer must remain valid until the operation completes, and the pointer
should not be aliased to the status for any other outstanding operation.

\end{itemize}

%%%%%%%%%%%%%%%%%%%% cond_atomic_xor %%%%%%%%%%%%%%%%%%%%
\pagebreak
\subsubsection{\code{cond\_atomic\_xor}}
\label{api:c:cond_atomic_xor}
\index{cond\_atomic\_xor!C API}
Conditionally store the bitwise XOR of the existing value and the specified
number for each attribute.

%%% Generated below here
\paragraph{Behavior:}
\begin{itemize}[noitemsep]
This operation requires a pre-existing object in order to complete successfully.
If no object exists, the operation will fail with \code{NOTFOUND}.

This operation will succeed if and only if the predicates specified by
\code{checks} hold on the pre-existing object.  If any of the predicates are not
true for the existing object, then the operation will have no effect and fail
with \code{CMPFAIL}.

All checks are atomic with the write.  HyperDex guarantees that no other
operation will come between validating the checks, and writing the new version
of the object.

\end{itemize}


\paragraph{Definition:}
\begin{ccode}
int64_t hyperdex_client_cond_atomic_xor(struct hyperdex_client* client,
        const char* space,
        const char* key, size_t key_sz,
        const struct hyperdex_client_attribute_check* checks, size_t checks_sz,
        const struct hyperdex_client_attribute* attrs, size_t attrs_sz,
        enum hyperdex_client_returncode* status);
\end{ccode}

\paragraph{Parameters:}
\begin{itemize}[noitemsep]
\item \code{struct hyperdex\_client* client}\\
The HyperDex client connection to use for the operation.

\item \code{const char* space}\\
The name of the space as a string or symbol.

\item \code{const char* key, size\_t key\_sz}\\
The key for the operation as a Python type.

\item \code{const struct hyperdex\_client\_attribute\_check* checks, size\_t checks\_sz}\\
A hash of predicates to check against.

\item \code{const struct hyperdex\_client\_attribute* attrs, size\_t attrs\_sz}\\
The set of attributes to modify and their respective values.  \code{attrs} is a
map from the attributes' names to their values.

\end{itemize}

\paragraph{Returns:}
\begin{itemize}[noitemsep]
\item \code{enum hyperdex\_client\_returncode* status}\\
The status of the operation.  The client library will fill in this variable
before returning this operation's request id from \code{hyperdex\_client\_loop}.
The pointer must remain valid until the operation completes, and the pointer
should not be aliased to the status for any other outstanding operation.

\end{itemize}

%%%%%%%%%%%%%%%%%%%% atomic_min %%%%%%%%%%%%%%%%%%%%
\pagebreak
\subsubsection{\code{atomic\_min}}
\label{api:c:atomic_min}
\index{atomic\_min!C API}
Store the minimum of the existing value and the provided value for each
attribute.
This operation requires a pre-existing object in order to complete successfully.
If no object exists, the operation will fail with \code{NOTFOUND}.



\paragraph{Definition:}
\begin{ccode}
int64_t hyperdex_client_atomic_min(struct hyperdex_client* client,
        const char* space,
        const char* key, size_t key_sz,
        const struct hyperdex_client_attribute* attrs, size_t attrs_sz,
        enum hyperdex_client_returncode* status);
\end{ccode}

\paragraph{Parameters:}
\begin{itemize}[noitemsep]
\item \code{struct hyperdex\_client* client}\\
The HyperDex client connection to use for the operation.

\item \code{const char* space}\\
The name of the space as a string or symbol.

\item \code{const char* key, size\_t key\_sz}\\
The key for the operation as a Python type.

\item \code{const struct hyperdex\_client\_attribute* attrs, size\_t attrs\_sz}\\
The set of attributes to modify and their respective values.  \code{attrs} is a
map from the attributes' names to their values.

\end{itemize}

\paragraph{Returns:}
\begin{itemize}[noitemsep]
\item \code{enum hyperdex\_client\_returncode* status}\\
The status of the operation.  The client library will fill in this variable
before returning this operation's request id from \code{hyperdex\_client\_loop}.
The pointer must remain valid until the operation completes, and the pointer
should not be aliased to the status for any other outstanding operation.

\end{itemize}

%%%%%%%%%%%%%%%%%%%% cond_atomic_min %%%%%%%%%%%%%%%%%%%%
\pagebreak
\subsubsection{\code{cond\_atomic\_min}}
\label{api:c:cond_atomic_min}
\index{cond\_atomic\_min!C API}
Store the minimum of the existing value and the provided value for each
attribute if and only if \code{checks} hold on the object.
This operation requires a pre-existing object in order to complete successfully.
If no object exists, the operation will fail with \code{NOTFOUND}.


This operation will succeed if and only if the predicates specified by
\code{checks} hold on the pre-existing object.  If any of the predicates are not
true for the existing object, then the operation will have no effect and fail
with \code{CMPFAIL}.

All checks are atomic with the write.  HyperDex guarantees that no other
operation will come between validating the checks, and writing the new version
of the object.



\paragraph{Definition:}
\begin{ccode}
int64_t hyperdex_client_cond_atomic_min(struct hyperdex_client* client,
        const char* space,
        const char* key, size_t key_sz,
        const struct hyperdex_client_attribute_check* checks, size_t checks_sz,
        const struct hyperdex_client_attribute* attrs, size_t attrs_sz,
        enum hyperdex_client_returncode* status);
\end{ccode}

\paragraph{Parameters:}
\begin{itemize}[noitemsep]
\item \code{struct hyperdex\_client* client}\\
The HyperDex client connection to use for the operation.

\item \code{const char* space}\\
The name of the space as a string or symbol.

\item \code{const char* key, size\_t key\_sz}\\
The key for the operation as a Python type.

\item \code{const struct hyperdex\_client\_attribute\_check* checks, size\_t checks\_sz}\\
A hash of predicates to check against.

\item \code{const struct hyperdex\_client\_attribute* attrs, size\_t attrs\_sz}\\
The set of attributes to modify and their respective values.  \code{attrs} is a
map from the attributes' names to their values.

\end{itemize}

\paragraph{Returns:}
\begin{itemize}[noitemsep]
\item \code{enum hyperdex\_client\_returncode* status}\\
The status of the operation.  The client library will fill in this variable
before returning this operation's request id from \code{hyperdex\_client\_loop}.
The pointer must remain valid until the operation completes, and the pointer
should not be aliased to the status for any other outstanding operation.

\end{itemize}

%%%%%%%%%%%%%%%%%%%% atomic_max %%%%%%%%%%%%%%%%%%%%
\pagebreak
\subsubsection{\code{atomic\_max}}
\label{api:c:atomic_max}
\index{atomic\_max!C API}
Store the maximum of the existing value and the provided value for each
attribute.
This operation requires a pre-existing object in order to complete successfully.
If no object exists, the operation will fail with \code{NOTFOUND}.



\paragraph{Definition:}
\begin{ccode}
int64_t hyperdex_client_atomic_max(struct hyperdex_client* client,
        const char* space,
        const char* key, size_t key_sz,
        const struct hyperdex_client_attribute* attrs, size_t attrs_sz,
        enum hyperdex_client_returncode* status);
\end{ccode}

\paragraph{Parameters:}
\begin{itemize}[noitemsep]
\item \code{struct hyperdex\_client* client}\\
The HyperDex client connection to use for the operation.

\item \code{const char* space}\\
The name of the space as a string or symbol.

\item \code{const char* key, size\_t key\_sz}\\
The key for the operation as a Python type.

\item \code{const struct hyperdex\_client\_attribute* attrs, size\_t attrs\_sz}\\
The set of attributes to modify and their respective values.  \code{attrs} is a
map from the attributes' names to their values.

\end{itemize}

\paragraph{Returns:}
\begin{itemize}[noitemsep]
\item \code{enum hyperdex\_client\_returncode* status}\\
The status of the operation.  The client library will fill in this variable
before returning this operation's request id from \code{hyperdex\_client\_loop}.
The pointer must remain valid until the operation completes, and the pointer
should not be aliased to the status for any other outstanding operation.

\end{itemize}

%%%%%%%%%%%%%%%%%%%% cond_atomic_max %%%%%%%%%%%%%%%%%%%%
\pagebreak
\subsubsection{\code{cond\_atomic\_max}}
\label{api:c:cond_atomic_max}
\index{cond\_atomic\_max!C API}
Store the maximum of the existing value and the provided value for each
attribute if and only if \code{checks} hold on the object.
This operation requires a pre-existing object in order to complete successfully.
If no object exists, the operation will fail with \code{NOTFOUND}.


This operation will succeed if and only if the predicates specified by
\code{checks} hold on the pre-existing object.  If any of the predicates are not
true for the existing object, then the operation will have no effect and fail
with \code{CMPFAIL}.

All checks are atomic with the write.  HyperDex guarantees that no other
operation will come between validating the checks, and writing the new version
of the object.



\paragraph{Definition:}
\begin{ccode}
int64_t hyperdex_client_cond_atomic_max(struct hyperdex_client* client,
        const char* space,
        const char* key, size_t key_sz,
        const struct hyperdex_client_attribute_check* checks, size_t checks_sz,
        const struct hyperdex_client_attribute* attrs, size_t attrs_sz,
        enum hyperdex_client_returncode* status);
\end{ccode}

\paragraph{Parameters:}
\begin{itemize}[noitemsep]
\item \code{struct hyperdex\_client* client}\\
The HyperDex client connection to use for the operation.

\item \code{const char* space}\\
The name of the space as a string or symbol.

\item \code{const char* key, size\_t key\_sz}\\
The key for the operation as a Python type.

\item \code{const struct hyperdex\_client\_attribute\_check* checks, size\_t checks\_sz}\\
A hash of predicates to check against.

\item \code{const struct hyperdex\_client\_attribute* attrs, size\_t attrs\_sz}\\
The set of attributes to modify and their respective values.  \code{attrs} is a
map from the attributes' names to their values.

\end{itemize}

\paragraph{Returns:}
\begin{itemize}[noitemsep]
\item \code{enum hyperdex\_client\_returncode* status}\\
The status of the operation.  The client library will fill in this variable
before returning this operation's request id from \code{hyperdex\_client\_loop}.
The pointer must remain valid until the operation completes, and the pointer
should not be aliased to the status for any other outstanding operation.

\end{itemize}

%%%%%%%%%%%%%%%%%%%% string_prepend %%%%%%%%%%%%%%%%%%%%
\pagebreak
\subsubsection{\code{string\_prepend}}
\label{api:c:string_prepend}
\index{string\_prepend!C API}
Prepend the specified string to the existing value for each attribute.

%%% Generated below here
\paragraph{Behavior:}
\begin{itemize}[noitemsep]
This operation requires a pre-existing object in order to complete successfully.
If no object exists, the operation will fail with \code{NOTFOUND}.

\end{itemize}


\paragraph{Definition:}
\begin{ccode}
int64_t hyperdex_client_string_prepend(struct hyperdex_client* client,
        const char* space,
        const char* key, size_t key_sz,
        const struct hyperdex_client_attribute* attrs, size_t attrs_sz,
        enum hyperdex_client_returncode* status);
\end{ccode}

\paragraph{Parameters:}
\begin{itemize}[noitemsep]
\item \code{struct hyperdex\_client* client}\\
The HyperDex client connection to use for the operation.

\item \code{const char* space}\\
The name of the space as a string or symbol.

\item \code{const char* key, size\_t key\_sz}\\
The key for the operation as a Python type.

\item \code{const struct hyperdex\_client\_attribute* attrs, size\_t attrs\_sz}\\
The set of attributes to modify and their respective values.  \code{attrs} is a
map from the attributes' names to their values.

\end{itemize}

\paragraph{Returns:}
\begin{itemize}[noitemsep]
\item \code{enum hyperdex\_client\_returncode* status}\\
The status of the operation.  The client library will fill in this variable
before returning this operation's request id from \code{hyperdex\_client\_loop}.
The pointer must remain valid until the operation completes, and the pointer
should not be aliased to the status for any other outstanding operation.

\end{itemize}

%%%%%%%%%%%%%%%%%%%% cond_string_prepend %%%%%%%%%%%%%%%%%%%%
\pagebreak
\subsubsection{\code{cond\_string\_prepend}}
\label{api:c:cond_string_prepend}
\index{cond\_string\_prepend!C API}
Conditionally prepend the specified string to the existing value for each
attribute.

%%% Generated below here
\paragraph{Behavior:}
\begin{itemize}[noitemsep]
This operation requires a pre-existing object in order to complete successfully.
If no object exists, the operation will fail with \code{NOTFOUND}.

This operation will succeed if and only if the predicates specified by
\code{checks} hold on the pre-existing object.  If any of the predicates are not
true for the existing object, then the operation will have no effect and fail
with \code{CMPFAIL}.

All checks are atomic with the write.  HyperDex guarantees that no other
operation will come between validating the checks, and writing the new version
of the object.

\end{itemize}


\paragraph{Definition:}
\begin{ccode}
int64_t hyperdex_client_cond_string_prepend(struct hyperdex_client* client,
        const char* space,
        const char* key, size_t key_sz,
        const struct hyperdex_client_attribute_check* checks, size_t checks_sz,
        const struct hyperdex_client_attribute* attrs, size_t attrs_sz,
        enum hyperdex_client_returncode* status);
\end{ccode}

\paragraph{Parameters:}
\begin{itemize}[noitemsep]
\item \code{struct hyperdex\_client* client}\\
The HyperDex client connection to use for the operation.

\item \code{const char* space}\\
The name of the space as a string or symbol.

\item \code{const char* key, size\_t key\_sz}\\
The key for the operation as a Python type.

\item \code{const struct hyperdex\_client\_attribute\_check* checks, size\_t checks\_sz}\\
A hash of predicates to check against.

\item \code{const struct hyperdex\_client\_attribute* attrs, size\_t attrs\_sz}\\
The set of attributes to modify and their respective values.  \code{attrs} is a
map from the attributes' names to their values.

\end{itemize}

\paragraph{Returns:}
\begin{itemize}[noitemsep]
\item \code{enum hyperdex\_client\_returncode* status}\\
The status of the operation.  The client library will fill in this variable
before returning this operation's request id from \code{hyperdex\_client\_loop}.
The pointer must remain valid until the operation completes, and the pointer
should not be aliased to the status for any other outstanding operation.

\end{itemize}

%%%%%%%%%%%%%%%%%%%% string_append %%%%%%%%%%%%%%%%%%%%
\pagebreak
\subsubsection{\code{string\_append}}
\label{api:c:string_append}
\index{string\_append!C API}
Append the specified string to the existing value for each attribute.

%%% Generated below here
\paragraph{Behavior:}
\begin{itemize}[noitemsep]
This operation requires a pre-existing object in order to complete successfully.
If no object exists, the operation will fail with \code{NOTFOUND}.

\end{itemize}


\paragraph{Definition:}
\begin{ccode}
int64_t hyperdex_client_string_append(struct hyperdex_client* client,
        const char* space,
        const char* key, size_t key_sz,
        const struct hyperdex_client_attribute* attrs, size_t attrs_sz,
        enum hyperdex_client_returncode* status);
\end{ccode}

\paragraph{Parameters:}
\begin{itemize}[noitemsep]
\item \code{struct hyperdex\_client* client}\\
The HyperDex client connection to use for the operation.

\item \code{const char* space}\\
The name of the space as a string or symbol.

\item \code{const char* key, size\_t key\_sz}\\
The key for the operation as a Python type.

\item \code{const struct hyperdex\_client\_attribute* attrs, size\_t attrs\_sz}\\
The set of attributes to modify and their respective values.  \code{attrs} is a
map from the attributes' names to their values.

\end{itemize}

\paragraph{Returns:}
\begin{itemize}[noitemsep]
\item \code{enum hyperdex\_client\_returncode* status}\\
The status of the operation.  The client library will fill in this variable
before returning this operation's request id from \code{hyperdex\_client\_loop}.
The pointer must remain valid until the operation completes, and the pointer
should not be aliased to the status for any other outstanding operation.

\end{itemize}

%%%%%%%%%%%%%%%%%%%% cond_string_append %%%%%%%%%%%%%%%%%%%%
\pagebreak
\subsubsection{\code{cond\_string\_append}}
\label{api:c:cond_string_append}
\index{cond\_string\_append!C API}
Conditionally append the specified string to the existing value for each
attribute.

%%% Generated below here
\paragraph{Behavior:}
\begin{itemize}[noitemsep]
This operation requires a pre-existing object in order to complete successfully.
If no object exists, the operation will fail with \code{NOTFOUND}.

This operation will succeed if and only if the predicates specified by
\code{checks} hold on the pre-existing object.  If any of the predicates are not
true for the existing object, then the operation will have no effect and fail
with \code{CMPFAIL}.

All checks are atomic with the write.  HyperDex guarantees that no other
operation will come between validating the checks, and writing the new version
of the object.

\end{itemize}


\paragraph{Definition:}
\begin{ccode}
int64_t hyperdex_client_cond_string_append(struct hyperdex_client* client,
        const char* space,
        const char* key, size_t key_sz,
        const struct hyperdex_client_attribute_check* checks, size_t checks_sz,
        const struct hyperdex_client_attribute* attrs, size_t attrs_sz,
        enum hyperdex_client_returncode* status);
\end{ccode}

\paragraph{Parameters:}
\begin{itemize}[noitemsep]
\item \code{struct hyperdex\_client* client}\\
The HyperDex client connection to use for the operation.

\item \code{const char* space}\\
The name of the space as a string or symbol.

\item \code{const char* key, size\_t key\_sz}\\
The key for the operation as a Python type.

\item \code{const struct hyperdex\_client\_attribute\_check* checks, size\_t checks\_sz}\\
A hash of predicates to check against.

\item \code{const struct hyperdex\_client\_attribute* attrs, size\_t attrs\_sz}\\
The set of attributes to modify and their respective values.  \code{attrs} is a
map from the attributes' names to their values.

\end{itemize}

\paragraph{Returns:}
\begin{itemize}[noitemsep]
\item \code{enum hyperdex\_client\_returncode* status}\\
The status of the operation.  The client library will fill in this variable
before returning this operation's request id from \code{hyperdex\_client\_loop}.
The pointer must remain valid until the operation completes, and the pointer
should not be aliased to the status for any other outstanding operation.

\end{itemize}

%%%%%%%%%%%%%%%%%%%% list_lpush %%%%%%%%%%%%%%%%%%%%
\pagebreak
\subsubsection{\code{list\_lpush}}
\label{api:c:list_lpush}
\index{list\_lpush!C API}
Push the specified value onto the front of the list for each attribute.

%%% Generated below here
\paragraph{Behavior:}
\begin{itemize}[noitemsep]
This operation requires a pre-existing object in order to complete successfully.
If no object exists, the operation will fail with \code{NOTFOUND}.

\end{itemize}


\paragraph{Definition:}
\begin{ccode}
int64_t hyperdex_client_list_lpush(struct hyperdex_client* client,
        const char* space,
        const char* key, size_t key_sz,
        const struct hyperdex_client_attribute* attrs, size_t attrs_sz,
        enum hyperdex_client_returncode* status);
\end{ccode}

\paragraph{Parameters:}
\begin{itemize}[noitemsep]
\item \code{struct hyperdex\_client* client}\\
The HyperDex client connection to use for the operation.

\item \code{const char* space}\\
The name of the space as a string or symbol.

\item \code{const char* key, size\_t key\_sz}\\
The key for the operation as a Python type.

\item \code{const struct hyperdex\_client\_attribute* attrs, size\_t attrs\_sz}\\
The set of attributes to modify and their respective values.  \code{attrs} is a
map from the attributes' names to their values.

\end{itemize}

\paragraph{Returns:}
\begin{itemize}[noitemsep]
\item \code{enum hyperdex\_client\_returncode* status}\\
The status of the operation.  The client library will fill in this variable
before returning this operation's request id from \code{hyperdex\_client\_loop}.
The pointer must remain valid until the operation completes, and the pointer
should not be aliased to the status for any other outstanding operation.

\end{itemize}

%%%%%%%%%%%%%%%%%%%% cond_list_lpush %%%%%%%%%%%%%%%%%%%%
\pagebreak
\subsubsection{\code{cond\_list\_lpush}}
\label{api:c:cond_list_lpush}
\index{cond\_list\_lpush!C API}
Condtitionally push the specified value onto the front of the list for each
attribute.

%%% Generated below here
\paragraph{Behavior:}
\begin{itemize}[noitemsep]
This operation requires a pre-existing object in order to complete successfully.
If no object exists, the operation will fail with \code{NOTFOUND}.

This operation will succeed if and only if the predicates specified by
\code{checks} hold on the pre-existing object.  If any of the predicates are not
true for the existing object, then the operation will have no effect and fail
with \code{CMPFAIL}.

All checks are atomic with the write.  HyperDex guarantees that no other
operation will come between validating the checks, and writing the new version
of the object.

\end{itemize}


\paragraph{Definition:}
\begin{ccode}
int64_t hyperdex_client_cond_list_lpush(struct hyperdex_client* client,
        const char* space,
        const char* key, size_t key_sz,
        const struct hyperdex_client_attribute_check* checks, size_t checks_sz,
        const struct hyperdex_client_attribute* attrs, size_t attrs_sz,
        enum hyperdex_client_returncode* status);
\end{ccode}

\paragraph{Parameters:}
\begin{itemize}[noitemsep]
\item \code{struct hyperdex\_client* client}\\
The HyperDex client connection to use for the operation.

\item \code{const char* space}\\
The name of the space as a string or symbol.

\item \code{const char* key, size\_t key\_sz}\\
The key for the operation as a Python type.

\item \code{const struct hyperdex\_client\_attribute\_check* checks, size\_t checks\_sz}\\
A hash of predicates to check against.

\item \code{const struct hyperdex\_client\_attribute* attrs, size\_t attrs\_sz}\\
The set of attributes to modify and their respective values.  \code{attrs} is a
map from the attributes' names to their values.

\end{itemize}

\paragraph{Returns:}
\begin{itemize}[noitemsep]
\item \code{enum hyperdex\_client\_returncode* status}\\
The status of the operation.  The client library will fill in this variable
before returning this operation's request id from \code{hyperdex\_client\_loop}.
The pointer must remain valid until the operation completes, and the pointer
should not be aliased to the status for any other outstanding operation.

\end{itemize}

%%%%%%%%%%%%%%%%%%%% list_rpush %%%%%%%%%%%%%%%%%%%%
\pagebreak
\subsubsection{\code{list\_rpush}}
\label{api:c:list_rpush}
\index{list\_rpush!C API}
Push the specified value onto the back of the list for each attribute.
This operation requires a pre-existing object in order to complete successfully.
If no object exists, the operation will fail with \code{NOTFOUND}.



\paragraph{Definition:}
\begin{ccode}
int64_t hyperdex_client_list_rpush(struct hyperdex_client* client,
        const char* space,
        const char* key, size_t key_sz,
        const struct hyperdex_client_attribute* attrs, size_t attrs_sz,
        enum hyperdex_client_returncode* status);
\end{ccode}

\paragraph{Parameters:}
\begin{itemize}[noitemsep]
\item \code{struct hyperdex\_client* client}\\
The HyperDex client connection to use for the operation.

\item \code{const char* space}\\
The name of the space as a string or symbol.

\item \code{const char* key, size\_t key\_sz}\\
The key for the operation as a Python type.

\item \code{const struct hyperdex\_client\_attribute* attrs, size\_t attrs\_sz}\\
The set of attributes to modify and their respective values.  \code{attrs} is a
map from the attributes' names to their values.

\end{itemize}

\paragraph{Returns:}
\begin{itemize}[noitemsep]
\item \code{enum hyperdex\_client\_returncode* status}\\
The status of the operation.  The client library will fill in this variable
before returning this operation's request id from \code{hyperdex\_client\_loop}.
The pointer must remain valid until the operation completes, and the pointer
should not be aliased to the status for any other outstanding operation.

\end{itemize}

%%%%%%%%%%%%%%%%%%%% cond_list_rpush %%%%%%%%%%%%%%%%%%%%
\pagebreak
\subsubsection{\code{cond\_list\_rpush}}
\label{api:c:cond_list_rpush}
\index{cond\_list\_rpush!C API}
Push the specified value onto the back of the list for each attribute if and
only if the \code{checks} hold on the object.
This operation requires a pre-existing object in order to complete successfully.
If no object exists, the operation will fail with \code{NOTFOUND}.


This operation will succeed if and only if the predicates specified by
\code{checks} hold on the pre-existing object.  If any of the predicates are not
true for the existing object, then the operation will have no effect and fail
with \code{CMPFAIL}.

All checks are atomic with the write.  HyperDex guarantees that no other
operation will come between validating the checks, and writing the new version
of the object.



\paragraph{Definition:}
\begin{ccode}
int64_t hyperdex_client_cond_list_rpush(struct hyperdex_client* client,
        const char* space,
        const char* key, size_t key_sz,
        const struct hyperdex_client_attribute_check* checks, size_t checks_sz,
        const struct hyperdex_client_attribute* attrs, size_t attrs_sz,
        enum hyperdex_client_returncode* status);
\end{ccode}

\paragraph{Parameters:}
\begin{itemize}[noitemsep]
\item \code{struct hyperdex\_client* client}\\
The HyperDex client connection to use for the operation.

\item \code{const char* space}\\
The name of the space as a string or symbol.

\item \code{const char* key, size\_t key\_sz}\\
The key for the operation as a Python type.

\item \code{const struct hyperdex\_client\_attribute\_check* checks, size\_t checks\_sz}\\
A hash of predicates to check against.

\item \code{const struct hyperdex\_client\_attribute* attrs, size\_t attrs\_sz}\\
The set of attributes to modify and their respective values.  \code{attrs} is a
map from the attributes' names to their values.

\end{itemize}

\paragraph{Returns:}
\begin{itemize}[noitemsep]
\item \code{enum hyperdex\_client\_returncode* status}\\
The status of the operation.  The client library will fill in this variable
before returning this operation's request id from \code{hyperdex\_client\_loop}.
The pointer must remain valid until the operation completes, and the pointer
should not be aliased to the status for any other outstanding operation.

\end{itemize}

%%%%%%%%%%%%%%%%%%%% set_add %%%%%%%%%%%%%%%%%%%%
\pagebreak
\subsubsection{\code{set\_add}}
\label{api:c:set_add}
\index{set\_add!C API}
Add the specified value to the set for each attribute.

%%% Generated below here
\paragraph{Behavior:}
\begin{itemize}[noitemsep]
This operation requires a pre-existing object in order to complete successfully.
If no object exists, the operation will fail with \code{NOTFOUND}.

\end{itemize}


\paragraph{Definition:}
\begin{ccode}
int64_t hyperdex_client_set_add(struct hyperdex_client* client,
        const char* space,
        const char* key, size_t key_sz,
        const struct hyperdex_client_attribute* attrs, size_t attrs_sz,
        enum hyperdex_client_returncode* status);
\end{ccode}

\paragraph{Parameters:}
\begin{itemize}[noitemsep]
\item \code{struct hyperdex\_client* client}\\
The HyperDex client connection to use for the operation.

\item \code{const char* space}\\
The name of the space as a string or symbol.

\item \code{const char* key, size\_t key\_sz}\\
The key for the operation as a Python type.

\item \code{const struct hyperdex\_client\_attribute* attrs, size\_t attrs\_sz}\\
The set of attributes to modify and their respective values.  \code{attrs} is a
map from the attributes' names to their values.

\end{itemize}

\paragraph{Returns:}
\begin{itemize}[noitemsep]
\item \code{enum hyperdex\_client\_returncode* status}\\
The status of the operation.  The client library will fill in this variable
before returning this operation's request id from \code{hyperdex\_client\_loop}.
The pointer must remain valid until the operation completes, and the pointer
should not be aliased to the status for any other outstanding operation.

\end{itemize}

%%%%%%%%%%%%%%%%%%%% cond_set_add %%%%%%%%%%%%%%%%%%%%
\pagebreak
\subsubsection{\code{cond\_set\_add}}
\label{api:c:cond_set_add}
\index{cond\_set\_add!C API}
Add the specified value to the set for each attribute if and only if the
\code{checks} hold on the object.
This operation requires a pre-existing object in order to complete successfully.
If no object exists, the operation will fail with \code{NOTFOUND}.


This operation will succeed if and only if the predicates specified by
\code{checks} hold on the pre-existing object.  If any of the predicates are not
true for the existing object, then the operation will have no effect and fail
with \code{CMPFAIL}.

All checks are atomic with the write.  HyperDex guarantees that no other
operation will come between validating the checks, and writing the new version
of the object.



\paragraph{Definition:}
\begin{ccode}
int64_t hyperdex_client_cond_set_add(struct hyperdex_client* client,
        const char* space,
        const char* key, size_t key_sz,
        const struct hyperdex_client_attribute_check* checks, size_t checks_sz,
        const struct hyperdex_client_attribute* attrs, size_t attrs_sz,
        enum hyperdex_client_returncode* status);
\end{ccode}

\paragraph{Parameters:}
\begin{itemize}[noitemsep]
\item \code{struct hyperdex\_client* client}\\
The HyperDex client connection to use for the operation.

\item \code{const char* space}\\
The name of the space as a string or symbol.

\item \code{const char* key, size\_t key\_sz}\\
The key for the operation as a Python type.

\item \code{const struct hyperdex\_client\_attribute\_check* checks, size\_t checks\_sz}\\
A hash of predicates to check against.

\item \code{const struct hyperdex\_client\_attribute* attrs, size\_t attrs\_sz}\\
The set of attributes to modify and their respective values.  \code{attrs} is a
map from the attributes' names to their values.

\end{itemize}

\paragraph{Returns:}
\begin{itemize}[noitemsep]
\item \code{enum hyperdex\_client\_returncode* status}\\
The status of the operation.  The client library will fill in this variable
before returning this operation's request id from \code{hyperdex\_client\_loop}.
The pointer must remain valid until the operation completes, and the pointer
should not be aliased to the status for any other outstanding operation.

\end{itemize}

%%%%%%%%%%%%%%%%%%%% set_remove %%%%%%%%%%%%%%%%%%%%
\pagebreak
\subsubsection{\code{set\_remove}}
\label{api:c:set_remove}
\index{set\_remove!C API}
Remove the specified value from the set.  If the value is not contained within
the set, this operation will do nothing.

%%% Generated below here
\paragraph{Behavior:}
\begin{itemize}[noitemsep]
This operation requires a pre-existing object in order to complete successfully.
If no object exists, the operation will fail with \code{NOTFOUND}.

\end{itemize}


\paragraph{Definition:}
\begin{ccode}
int64_t hyperdex_client_set_remove(struct hyperdex_client* client,
        const char* space,
        const char* key, size_t key_sz,
        const struct hyperdex_client_attribute* attrs, size_t attrs_sz,
        enum hyperdex_client_returncode* status);
\end{ccode}

\paragraph{Parameters:}
\begin{itemize}[noitemsep]
\item \code{struct hyperdex\_client* client}\\
The HyperDex client connection to use for the operation.

\item \code{const char* space}\\
The name of the space as a string or symbol.

\item \code{const char* key, size\_t key\_sz}\\
The key for the operation as a Python type.

\item \code{const struct hyperdex\_client\_attribute* attrs, size\_t attrs\_sz}\\
The set of attributes to modify and their respective values.  \code{attrs} is a
map from the attributes' names to their values.

\end{itemize}

\paragraph{Returns:}
\begin{itemize}[noitemsep]
\item \code{enum hyperdex\_client\_returncode* status}\\
The status of the operation.  The client library will fill in this variable
before returning this operation's request id from \code{hyperdex\_client\_loop}.
The pointer must remain valid until the operation completes, and the pointer
should not be aliased to the status for any other outstanding operation.

\end{itemize}

%%%%%%%%%%%%%%%%%%%% cond_set_remove %%%%%%%%%%%%%%%%%%%%
\pagebreak
\subsubsection{\code{cond\_set\_remove}}
\label{api:c:cond_set_remove}
\index{cond\_set\_remove!C API}
Remove the specified value from the set if and only if the \code{checks} hold on
the object.  If the value is not contained within the set, this operation will
do nothing.
This operation requires a pre-existing object in order to complete successfully.
If no object exists, the operation will fail with \code{NOTFOUND}.


This operation will succeed if and only if the predicates specified by
\code{checks} hold on the pre-existing object.  If any of the predicates are not
true for the existing object, then the operation will have no effect and fail
with \code{CMPFAIL}.

All checks are atomic with the write.  HyperDex guarantees that no other
operation will come between validating the checks, and writing the new version
of the object.



\paragraph{Definition:}
\begin{ccode}
int64_t hyperdex_client_cond_set_remove(struct hyperdex_client* client,
        const char* space,
        const char* key, size_t key_sz,
        const struct hyperdex_client_attribute_check* checks, size_t checks_sz,
        const struct hyperdex_client_attribute* attrs, size_t attrs_sz,
        enum hyperdex_client_returncode* status);
\end{ccode}

\paragraph{Parameters:}
\begin{itemize}[noitemsep]
\item \code{struct hyperdex\_client* client}\\
The HyperDex client connection to use for the operation.

\item \code{const char* space}\\
The name of the space as a string or symbol.

\item \code{const char* key, size\_t key\_sz}\\
The key for the operation as a Python type.

\item \code{const struct hyperdex\_client\_attribute\_check* checks, size\_t checks\_sz}\\
A hash of predicates to check against.

\item \code{const struct hyperdex\_client\_attribute* attrs, size\_t attrs\_sz}\\
The set of attributes to modify and their respective values.  \code{attrs} is a
map from the attributes' names to their values.

\end{itemize}

\paragraph{Returns:}
\begin{itemize}[noitemsep]
\item \code{enum hyperdex\_client\_returncode* status}\\
The status of the operation.  The client library will fill in this variable
before returning this operation's request id from \code{hyperdex\_client\_loop}.
The pointer must remain valid until the operation completes, and the pointer
should not be aliased to the status for any other outstanding operation.

\end{itemize}

%%%%%%%%%%%%%%%%%%%% set_intersect %%%%%%%%%%%%%%%%%%%%
\pagebreak
\subsubsection{\code{set\_intersect}}
\label{api:c:set_intersect}
\index{set\_intersect!C API}
Store the intersection of the specified set and the existing value for each
attribute.
This operation requires a pre-existing object in order to complete successfully.
If no object exists, the operation will fail with \code{NOTFOUND}.



\paragraph{Definition:}
\begin{ccode}
int64_t hyperdex_client_set_intersect(struct hyperdex_client* client,
        const char* space,
        const char* key, size_t key_sz,
        const struct hyperdex_client_attribute* attrs, size_t attrs_sz,
        enum hyperdex_client_returncode* status);
\end{ccode}

\paragraph{Parameters:}
\begin{itemize}[noitemsep]
\item \code{struct hyperdex\_client* client}\\
The HyperDex client connection to use for the operation.

\item \code{const char* space}\\
The name of the space as a string or symbol.

\item \code{const char* key, size\_t key\_sz}\\
The key for the operation as a Python type.

\item \code{const struct hyperdex\_client\_attribute* attrs, size\_t attrs\_sz}\\
The set of attributes to modify and their respective values.  \code{attrs} is a
map from the attributes' names to their values.

\end{itemize}

\paragraph{Returns:}
\begin{itemize}[noitemsep]
\item \code{enum hyperdex\_client\_returncode* status}\\
The status of the operation.  The client library will fill in this variable
before returning this operation's request id from \code{hyperdex\_client\_loop}.
The pointer must remain valid until the operation completes, and the pointer
should not be aliased to the status for any other outstanding operation.

\end{itemize}

%%%%%%%%%%%%%%%%%%%% cond_set_intersect %%%%%%%%%%%%%%%%%%%%
\pagebreak
\subsubsection{\code{cond\_set\_intersect}}
\label{api:c:cond_set_intersect}
\index{cond\_set\_intersect!C API}
Store the intersection of the specified set and the existing value for each
attribute if and only if the \code{checks} hold on the object.
This operation requires a pre-existing object in order to complete successfully.
If no object exists, the operation will fail with \code{NOTFOUND}.


This operation will succeed if and only if the predicates specified by
\code{checks} hold on the pre-existing object.  If any of the predicates are not
true for the existing object, then the operation will have no effect and fail
with \code{CMPFAIL}.

All checks are atomic with the write.  HyperDex guarantees that no other
operation will come between validating the checks, and writing the new version
of the object.



\paragraph{Definition:}
\begin{ccode}
int64_t hyperdex_client_cond_set_intersect(struct hyperdex_client* client,
        const char* space,
        const char* key, size_t key_sz,
        const struct hyperdex_client_attribute_check* checks, size_t checks_sz,
        const struct hyperdex_client_attribute* attrs, size_t attrs_sz,
        enum hyperdex_client_returncode* status);
\end{ccode}

\paragraph{Parameters:}
\begin{itemize}[noitemsep]
\item \code{struct hyperdex\_client* client}\\
The HyperDex client connection to use for the operation.

\item \code{const char* space}\\
The name of the space as a string or symbol.

\item \code{const char* key, size\_t key\_sz}\\
The key for the operation as a Python type.

\item \code{const struct hyperdex\_client\_attribute\_check* checks, size\_t checks\_sz}\\
A hash of predicates to check against.

\item \code{const struct hyperdex\_client\_attribute* attrs, size\_t attrs\_sz}\\
The set of attributes to modify and their respective values.  \code{attrs} is a
map from the attributes' names to their values.

\end{itemize}

\paragraph{Returns:}
\begin{itemize}[noitemsep]
\item \code{enum hyperdex\_client\_returncode* status}\\
The status of the operation.  The client library will fill in this variable
before returning this operation's request id from \code{hyperdex\_client\_loop}.
The pointer must remain valid until the operation completes, and the pointer
should not be aliased to the status for any other outstanding operation.

\end{itemize}

%%%%%%%%%%%%%%%%%%%% set_union %%%%%%%%%%%%%%%%%%%%
\pagebreak
\subsubsection{\code{set\_union}}
\label{api:c:set_union}
\index{set\_union!C API}
Store the union of the specified set and the existing value for each attribute.

%%% Generated below here
\paragraph{Behavior:}
\begin{itemize}[noitemsep]
This operation requires a pre-existing object in order to complete successfully.
If no object exists, the operation will fail with \code{NOTFOUND}.

\end{itemize}


\paragraph{Definition:}
\begin{ccode}
int64_t hyperdex_client_set_union(struct hyperdex_client* client,
        const char* space,
        const char* key, size_t key_sz,
        const struct hyperdex_client_attribute* attrs, size_t attrs_sz,
        enum hyperdex_client_returncode* status);
\end{ccode}

\paragraph{Parameters:}
\begin{itemize}[noitemsep]
\item \code{struct hyperdex\_client* client}\\
The HyperDex client connection to use for the operation.

\item \code{const char* space}\\
The name of the space as a string or symbol.

\item \code{const char* key, size\_t key\_sz}\\
The key for the operation as a Python type.

\item \code{const struct hyperdex\_client\_attribute* attrs, size\_t attrs\_sz}\\
The set of attributes to modify and their respective values.  \code{attrs} is a
map from the attributes' names to their values.

\end{itemize}

\paragraph{Returns:}
\begin{itemize}[noitemsep]
\item \code{enum hyperdex\_client\_returncode* status}\\
The status of the operation.  The client library will fill in this variable
before returning this operation's request id from \code{hyperdex\_client\_loop}.
The pointer must remain valid until the operation completes, and the pointer
should not be aliased to the status for any other outstanding operation.

\end{itemize}

%%%%%%%%%%%%%%%%%%%% cond_set_union %%%%%%%%%%%%%%%%%%%%
\pagebreak
\subsubsection{\code{cond\_set\_union}}
\label{api:c:cond_set_union}
\index{cond\_set\_union!C API}
Conditionally store the union of the specified set and the existing value for
each attribute.

%%% Generated below here
\paragraph{Behavior:}
\begin{itemize}[noitemsep]
This operation requires a pre-existing object in order to complete successfully.
If no object exists, the operation will fail with \code{NOTFOUND}.

This operation will succeed if and only if the predicates specified by
\code{checks} hold on the pre-existing object.  If any of the predicates are not
true for the existing object, then the operation will have no effect and fail
with \code{CMPFAIL}.

All checks are atomic with the write.  HyperDex guarantees that no other
operation will come between validating the checks, and writing the new version
of the object.

\end{itemize}


\paragraph{Definition:}
\begin{ccode}
int64_t hyperdex_client_cond_set_union(struct hyperdex_client* client,
        const char* space,
        const char* key, size_t key_sz,
        const struct hyperdex_client_attribute_check* checks, size_t checks_sz,
        const struct hyperdex_client_attribute* attrs, size_t attrs_sz,
        enum hyperdex_client_returncode* status);
\end{ccode}

\paragraph{Parameters:}
\begin{itemize}[noitemsep]
\item \code{struct hyperdex\_client* client}\\
The HyperDex client connection to use for the operation.

\item \code{const char* space}\\
The name of the space as a string or symbol.

\item \code{const char* key, size\_t key\_sz}\\
The key for the operation as a Python type.

\item \code{const struct hyperdex\_client\_attribute\_check* checks, size\_t checks\_sz}\\
A hash of predicates to check against.

\item \code{const struct hyperdex\_client\_attribute* attrs, size\_t attrs\_sz}\\
The set of attributes to modify and their respective values.  \code{attrs} is a
map from the attributes' names to their values.

\end{itemize}

\paragraph{Returns:}
\begin{itemize}[noitemsep]
\item \code{enum hyperdex\_client\_returncode* status}\\
The status of the operation.  The client library will fill in this variable
before returning this operation's request id from \code{hyperdex\_client\_loop}.
The pointer must remain valid until the operation completes, and the pointer
should not be aliased to the status for any other outstanding operation.

\end{itemize}

%%%%%%%%%%%%%%%%%%%% map_add %%%%%%%%%%%%%%%%%%%%
\pagebreak
\subsubsection{\code{map\_add}}
\label{api:c:map_add}
\index{map\_add!C API}
Insert a key-value pair into the map specified by each map-attribute.
This operation requires a pre-existing object in order to complete successfully.
If no object exists, the operation will fail with \code{NOTFOUND}.



\paragraph{Definition:}
\begin{ccode}
int64_t hyperdex_client_map_add(struct hyperdex_client* client,
        const char* space,
        const char* key, size_t key_sz,
        const struct hyperdex_client_map_attribute* mapattrs, size_t mapattrs_sz,
        enum hyperdex_client_returncode* status);
\end{ccode}

\paragraph{Parameters:}
\begin{itemize}[noitemsep]
\item \code{struct hyperdex\_client* client}\\
The HyperDex client connection to use for the operation.

\item \code{const char* space}\\
The name of the space as a string or symbol.

\item \code{const char* key, size\_t key\_sz}\\
The key for the operation as a Python type.

\item \code{const struct hyperdex\_client\_map\_attribute* mapattrs, size\_t mapattrs\_sz}\\
A hash specifying map attributes to modify and their respective key/values.

\end{itemize}

\paragraph{Returns:}
\begin{itemize}[noitemsep]
\item \code{enum hyperdex\_client\_returncode* status}\\
The status of the operation.  The client library will fill in this variable
before returning this operation's request id from \code{hyperdex\_client\_loop}.
The pointer must remain valid until the operation completes, and the pointer
should not be aliased to the status for any other outstanding operation.

\end{itemize}

%%%%%%%%%%%%%%%%%%%% cond_map_add %%%%%%%%%%%%%%%%%%%%
\pagebreak
\subsubsection{\code{cond\_map\_add}}
\label{api:c:cond_map_add}
\index{cond\_map\_add!C API}
Insert a key-value pair into the map specified by each map-attribute if and only
if the \code{checks} hold on the object.
This operation requires a pre-existing object in order to complete successfully.
If no object exists, the operation will fail with \code{NOTFOUND}.


This operation will succeed if and only if the predicates specified by
\code{checks} hold on the pre-existing object.  If any of the predicates are not
true for the existing object, then the operation will have no effect and fail
with \code{CMPFAIL}.

All checks are atomic with the write.  HyperDex guarantees that no other
operation will come between validating the checks, and writing the new version
of the object.



\paragraph{Definition:}
\begin{ccode}
int64_t hyperdex_client_cond_map_add(struct hyperdex_client* client,
        const char* space,
        const char* key, size_t key_sz,
        const struct hyperdex_client_attribute_check* checks, size_t checks_sz,
        const struct hyperdex_client_map_attribute* mapattrs, size_t mapattrs_sz,
        enum hyperdex_client_returncode* status);
\end{ccode}

\paragraph{Parameters:}
\begin{itemize}[noitemsep]
\item \code{struct hyperdex\_client* client}\\
The HyperDex client connection to use for the operation.

\item \code{const char* space}\\
The name of the space as a string or symbol.

\item \code{const char* key, size\_t key\_sz}\\
The key for the operation as a Python type.

\item \code{const struct hyperdex\_client\_attribute\_check* checks, size\_t checks\_sz}\\
A hash of predicates to check against.

\item \code{const struct hyperdex\_client\_map\_attribute* mapattrs, size\_t mapattrs\_sz}\\
A hash specifying map attributes to modify and their respective key/values.

\end{itemize}

\paragraph{Returns:}
\begin{itemize}[noitemsep]
\item \code{enum hyperdex\_client\_returncode* status}\\
The status of the operation.  The client library will fill in this variable
before returning this operation's request id from \code{hyperdex\_client\_loop}.
The pointer must remain valid until the operation completes, and the pointer
should not be aliased to the status for any other outstanding operation.

\end{itemize}

%%%%%%%%%%%%%%%%%%%% map_remove %%%%%%%%%%%%%%%%%%%%
\pagebreak
\subsubsection{\code{map\_remove}}
\label{api:c:map_remove}
\index{map\_remove!C API}
Remove a key-value pair from the map specified by each attribute.  If there is
no pair with the specified key within the map, this operation will do nothing.
This operation requires a pre-existing object in order to complete successfully.
If no object exists, the operation will fail with \code{NOTFOUND}.



\paragraph{Definition:}
\begin{ccode}
int64_t hyperdex_client_map_remove(struct hyperdex_client* client,
        const char* space,
        const char* key, size_t key_sz,
        const struct hyperdex_client_attribute* attrs, size_t attrs_sz,
        enum hyperdex_client_returncode* status);
\end{ccode}

\paragraph{Parameters:}
\begin{itemize}[noitemsep]
\item \code{struct hyperdex\_client* client}\\
The HyperDex client connection to use for the operation.

\item \code{const char* space}\\
The name of the space as a string or symbol.

\item \code{const char* key, size\_t key\_sz}\\
The key for the operation as a Python type.

\item \code{const struct hyperdex\_client\_attribute* attrs, size\_t attrs\_sz}\\
The set of attributes to modify and their respective values.  \code{attrs} is a
map from the attributes' names to their values.

\end{itemize}

\paragraph{Returns:}
\begin{itemize}[noitemsep]
\item \code{enum hyperdex\_client\_returncode* status}\\
The status of the operation.  The client library will fill in this variable
before returning this operation's request id from \code{hyperdex\_client\_loop}.
The pointer must remain valid until the operation completes, and the pointer
should not be aliased to the status for any other outstanding operation.

\end{itemize}

%%%%%%%%%%%%%%%%%%%% cond_map_remove %%%%%%%%%%%%%%%%%%%%
\pagebreak
\subsubsection{\code{cond\_map\_remove}}
\label{api:c:cond_map_remove}
\index{cond\_map\_remove!C API}
Conditionally remove a key-value pair from the map specified by each attribute.

%%% Generated below here
\paragraph{Behavior:}
\begin{itemize}[noitemsep]
This operation requires a pre-existing object in order to complete successfully.
If no object exists, the operation will fail with \code{NOTFOUND}.

This operation will succeed if and only if the predicates specified by
\code{checks} hold on the pre-existing object.  If any of the predicates are not
true for the existing object, then the operation will have no effect and fail
with \code{CMPFAIL}.

All checks are atomic with the write.  HyperDex guarantees that no other
operation will come between validating the checks, and writing the new version
of the object.

\end{itemize}


\paragraph{Definition:}
\begin{ccode}
int64_t hyperdex_client_cond_map_remove(struct hyperdex_client* client,
        const char* space,
        const char* key, size_t key_sz,
        const struct hyperdex_client_attribute_check* checks, size_t checks_sz,
        const struct hyperdex_client_attribute* attrs, size_t attrs_sz,
        enum hyperdex_client_returncode* status);
\end{ccode}

\paragraph{Parameters:}
\begin{itemize}[noitemsep]
\item \code{struct hyperdex\_client* client}\\
The HyperDex client connection to use for the operation.

\item \code{const char* space}\\
The name of the space as a string or symbol.

\item \code{const char* key, size\_t key\_sz}\\
The key for the operation as a Python type.

\item \code{const struct hyperdex\_client\_attribute\_check* checks, size\_t checks\_sz}\\
A hash of predicates to check against.

\item \code{const struct hyperdex\_client\_attribute* attrs, size\_t attrs\_sz}\\
The set of attributes to modify and their respective values.  \code{attrs} is a
map from the attributes' names to their values.

\end{itemize}

\paragraph{Returns:}
\begin{itemize}[noitemsep]
\item \code{enum hyperdex\_client\_returncode* status}\\
The status of the operation.  The client library will fill in this variable
before returning this operation's request id from \code{hyperdex\_client\_loop}.
The pointer must remain valid until the operation completes, and the pointer
should not be aliased to the status for any other outstanding operation.

\end{itemize}

%%%%%%%%%%%%%%%%%%%% document_rename %%%%%%%%%%%%%%%%%%%%
\pagebreak
\subsubsection{\code{document\_rename}}
\label{api:c:document_rename}
\index{document\_rename!C API}
Move a field within a document from one name to another.
This operation requires a pre-existing object in order to complete successfully.
If no object exists, the operation will fail with \code{NOTFOUND}.



\paragraph{Definition:}
\begin{ccode}
int64_t hyperdex_client_document_rename(struct hyperdex_client* client,
        const char* space,
        const char* key, size_t key_sz,
        const struct hyperdex_client_attribute* attrs, size_t attrs_sz,
        enum hyperdex_client_returncode* status);
\end{ccode}

\paragraph{Parameters:}
\begin{itemize}[noitemsep]
\item \code{struct hyperdex\_client* client}\\
The HyperDex client connection to use for the operation.

\item \code{const char* space}\\
The name of the space as a string or symbol.

\item \code{const char* key, size\_t key\_sz}\\
The key for the operation as a Python type.

\item \code{const struct hyperdex\_client\_attribute* attrs, size\_t attrs\_sz}\\
The set of attributes to modify and their respective values.  \code{attrs} is a
map from the attributes' names to their values.

\end{itemize}

\paragraph{Returns:}
\begin{itemize}[noitemsep]
\item \code{enum hyperdex\_client\_returncode* status}\\
The status of the operation.  The client library will fill in this variable
before returning this operation's request id from \code{hyperdex\_client\_loop}.
The pointer must remain valid until the operation completes, and the pointer
should not be aliased to the status for any other outstanding operation.

\end{itemize}

%%%%%%%%%%%%%%%%%%%% document_unset %%%%%%%%%%%%%%%%%%%%
\pagebreak
\subsubsection{\code{document\_unset}}
\label{api:c:document_unset}
\index{document\_unset!C API}
Remove a field or object from a document.
This operation requires a pre-existing object in order to complete successfully.
If no object exists, the operation will fail with \code{NOTFOUND}.



\paragraph{Definition:}
\begin{ccode}
int64_t hyperdex_client_document_unset(struct hyperdex_client* client,
        const char* space,
        const char* key, size_t key_sz,
        const struct hyperdex_client_attribute* attrs, size_t attrs_sz,
        enum hyperdex_client_returncode* status);
\end{ccode}

\paragraph{Parameters:}
\begin{itemize}[noitemsep]
\item \code{struct hyperdex\_client* client}\\
The HyperDex client connection to use for the operation.

\item \code{const char* space}\\
The name of the space as a string or symbol.

\item \code{const char* key, size\_t key\_sz}\\
The key for the operation as a Python type.

\item \code{const struct hyperdex\_client\_attribute* attrs, size\_t attrs\_sz}\\
The set of attributes to modify and their respective values.  \code{attrs} is a
map from the attributes' names to their values.

\end{itemize}

\paragraph{Returns:}
\begin{itemize}[noitemsep]
\item \code{enum hyperdex\_client\_returncode* status}\\
The status of the operation.  The client library will fill in this variable
before returning this operation's request id from \code{hyperdex\_client\_loop}.
The pointer must remain valid until the operation completes, and the pointer
should not be aliased to the status for any other outstanding operation.

\end{itemize}

%%%%%%%%%%%%%%%%%%%% document_set %%%%%%%%%%%%%%%%%%%%
\pagebreak
\subsubsection{\code{document\_set}}
\label{api:c:document_set}
\index{document\_set!C API}
\input{\topdir/api/desc/document_set}

\paragraph{Definition:}
\begin{ccode}
int64_t hyperdex_client_document_set(struct hyperdex_client* client,
        const char* space,
        const char* key, size_t key_sz,
        const struct hyperdex_client_attribute* attrs, size_t attrs_sz,
        enum hyperdex_client_returncode* status);
\end{ccode}

\paragraph{Parameters:}
\begin{itemize}[noitemsep]
\item \code{struct hyperdex\_client* client}\\
The HyperDex client connection to use for the operation.

\item \code{const char* space}\\
The name of the space as a string or symbol.

\item \code{const char* key, size\_t key\_sz}\\
The key for the operation as a Python type.

\item \code{const struct hyperdex\_client\_attribute* attrs, size\_t attrs\_sz}\\
The set of attributes to modify and their respective values.  \code{attrs} is a
map from the attributes' names to their values.

\end{itemize}

\paragraph{Returns:}
\begin{itemize}[noitemsep]
\item \code{enum hyperdex\_client\_returncode* status}\\
The status of the operation.  The client library will fill in this variable
before returning this operation's request id from \code{hyperdex\_client\_loop}.
The pointer must remain valid until the operation completes, and the pointer
should not be aliased to the status for any other outstanding operation.

\end{itemize}

%%%%%%%%%%%%%%%%%%%% map_atomic_add %%%%%%%%%%%%%%%%%%%%
\pagebreak
\subsubsection{\code{map\_atomic\_add}}
\label{api:c:map_atomic_add}
\index{map\_atomic\_add!C API}
Add the specified number to the value of a key-value pair within each map.
This operation requires a pre-existing object in order to complete successfully.
If no object exists, the operation will fail with \code{NOTFOUND}.



\paragraph{Definition:}
\begin{ccode}
int64_t hyperdex_client_map_atomic_add(struct hyperdex_client* client,
        const char* space,
        const char* key, size_t key_sz,
        const struct hyperdex_client_map_attribute* mapattrs, size_t mapattrs_sz,
        enum hyperdex_client_returncode* status);
\end{ccode}

\paragraph{Parameters:}
\begin{itemize}[noitemsep]
\item \code{struct hyperdex\_client* client}\\
The HyperDex client connection to use for the operation.

\item \code{const char* space}\\
The name of the space as a string or symbol.

\item \code{const char* key, size\_t key\_sz}\\
The key for the operation as a Python type.

\item \code{const struct hyperdex\_client\_map\_attribute* mapattrs, size\_t mapattrs\_sz}\\
A hash specifying map attributes to modify and their respective key/values.

\end{itemize}

\paragraph{Returns:}
\begin{itemize}[noitemsep]
\item \code{enum hyperdex\_client\_returncode* status}\\
The status of the operation.  The client library will fill in this variable
before returning this operation's request id from \code{hyperdex\_client\_loop}.
The pointer must remain valid until the operation completes, and the pointer
should not be aliased to the status for any other outstanding operation.

\end{itemize}

%%%%%%%%%%%%%%%%%%%% cond_map_atomic_add %%%%%%%%%%%%%%%%%%%%
\pagebreak
\subsubsection{\code{cond\_map\_atomic\_add}}
\label{api:c:cond_map_atomic_add}
\index{cond\_map\_atomic\_add!C API}
Add the specified number to the value of a key-value pair within each map if and
only if the \code{checks} hold on the object.
This operation requires a pre-existing object in order to complete successfully.
If no object exists, the operation will fail with \code{NOTFOUND}.


This operation will succeed if and only if the predicates specified by
\code{checks} hold on the pre-existing object.  If any of the predicates are not
true for the existing object, then the operation will have no effect and fail
with \code{CMPFAIL}.

All checks are atomic with the write.  HyperDex guarantees that no other
operation will come between validating the checks, and writing the new version
of the object.



\paragraph{Definition:}
\begin{ccode}
int64_t hyperdex_client_cond_map_atomic_add(struct hyperdex_client* client,
        const char* space,
        const char* key, size_t key_sz,
        const struct hyperdex_client_attribute_check* checks, size_t checks_sz,
        const struct hyperdex_client_map_attribute* mapattrs, size_t mapattrs_sz,
        enum hyperdex_client_returncode* status);
\end{ccode}

\paragraph{Parameters:}
\begin{itemize}[noitemsep]
\item \code{struct hyperdex\_client* client}\\
The HyperDex client connection to use for the operation.

\item \code{const char* space}\\
The name of the space as a string or symbol.

\item \code{const char* key, size\_t key\_sz}\\
The key for the operation as a Python type.

\item \code{const struct hyperdex\_client\_attribute\_check* checks, size\_t checks\_sz}\\
A hash of predicates to check against.

\item \code{const struct hyperdex\_client\_map\_attribute* mapattrs, size\_t mapattrs\_sz}\\
A hash specifying map attributes to modify and their respective key/values.

\end{itemize}

\paragraph{Returns:}
\begin{itemize}[noitemsep]
\item \code{enum hyperdex\_client\_returncode* status}\\
The status of the operation.  The client library will fill in this variable
before returning this operation's request id from \code{hyperdex\_client\_loop}.
The pointer must remain valid until the operation completes, and the pointer
should not be aliased to the status for any other outstanding operation.

\end{itemize}

%%%%%%%%%%%%%%%%%%%% map_atomic_sub %%%%%%%%%%%%%%%%%%%%
\pagebreak
\subsubsection{\code{map\_atomic\_sub}}
\label{api:c:map_atomic_sub}
\index{map\_atomic\_sub!C API}
Subtract the specified number from the value of a key-value pair within each
map.

%%% Generated below here
\paragraph{Behavior:}
\begin{itemize}[noitemsep]
This operation requires a pre-existing object in order to complete successfully.
If no object exists, the operation will fail with \code{NOTFOUND}.

\item This operation mutates the value of a key-value pair in a map.  This call
    is similar to the equivalent call without the \code{map\_} prefix, but
    operates on the value of a pair in a map, instead of on an attribute's
    value.  If there is no pair with the specified map key, a new pair will be
    created and initialized to its default value.  If this is undesirable, it
    may be avoided by using a conditional operation that requires that the map
    contain the key in question.

\end{itemize}


\paragraph{Definition:}
\begin{ccode}
int64_t hyperdex_client_map_atomic_sub(struct hyperdex_client* client,
        const char* space,
        const char* key, size_t key_sz,
        const struct hyperdex_client_map_attribute* mapattrs, size_t mapattrs_sz,
        enum hyperdex_client_returncode* status);
\end{ccode}

\paragraph{Parameters:}
\begin{itemize}[noitemsep]
\item \code{struct hyperdex\_client* client}\\
The HyperDex client connection to use for the operation.

\item \code{const char* space}\\
The name of the space as a string or symbol.

\item \code{const char* key, size\_t key\_sz}\\
The key for the operation as a Python type.

\item \code{const struct hyperdex\_client\_map\_attribute* mapattrs, size\_t mapattrs\_sz}\\
A hash specifying map attributes to modify and their respective key/values.

\end{itemize}

\paragraph{Returns:}
\begin{itemize}[noitemsep]
\item \code{enum hyperdex\_client\_returncode* status}\\
The status of the operation.  The client library will fill in this variable
before returning this operation's request id from \code{hyperdex\_client\_loop}.
The pointer must remain valid until the operation completes, and the pointer
should not be aliased to the status for any other outstanding operation.

\end{itemize}

%%%%%%%%%%%%%%%%%%%% cond_map_atomic_sub %%%%%%%%%%%%%%%%%%%%
\pagebreak
\subsubsection{\code{cond\_map\_atomic\_sub}}
\label{api:c:cond_map_atomic_sub}
\index{cond\_map\_atomic\_sub!C API}
Subtract the specified number from the value of a key-value pair within each
map.

%%% Generated below here
\paragraph{Behavior:}
\begin{itemize}[noitemsep]
This operation requires a pre-existing object in order to complete successfully.
If no object exists, the operation will fail with \code{NOTFOUND}.

This operation will succeed if and only if the predicates specified by
\code{checks} hold on the pre-existing object.  If any of the predicates are not
true for the existing object, then the operation will have no effect and fail
with \code{CMPFAIL}.

All checks are atomic with the write.  HyperDex guarantees that no other
operation will come between validating the checks, and writing the new version
of the object.

\item This operation mutates the value of a key-value pair in a map.  This call
    is similar to the equivalent call without the \code{map\_} prefix, but
    operates on the value of a pair in a map, instead of on an attribute's
    value.  If there is no pair with the specified map key, a new pair will be
    created and initialized to its default value.  If this is undesirable, it
    may be avoided by using a conditional operation that requires that the map
    contain the key in question.

\end{itemize}


\paragraph{Definition:}
\begin{ccode}
int64_t hyperdex_client_cond_map_atomic_sub(struct hyperdex_client* client,
        const char* space,
        const char* key, size_t key_sz,
        const struct hyperdex_client_attribute_check* checks, size_t checks_sz,
        const struct hyperdex_client_map_attribute* mapattrs, size_t mapattrs_sz,
        enum hyperdex_client_returncode* status);
\end{ccode}

\paragraph{Parameters:}
\begin{itemize}[noitemsep]
\item \code{struct hyperdex\_client* client}\\
The HyperDex client connection to use for the operation.

\item \code{const char* space}\\
The name of the space as a string or symbol.

\item \code{const char* key, size\_t key\_sz}\\
The key for the operation as a Python type.

\item \code{const struct hyperdex\_client\_attribute\_check* checks, size\_t checks\_sz}\\
A hash of predicates to check against.

\item \code{const struct hyperdex\_client\_map\_attribute* mapattrs, size\_t mapattrs\_sz}\\
A hash specifying map attributes to modify and their respective key/values.

\end{itemize}

\paragraph{Returns:}
\begin{itemize}[noitemsep]
\item \code{enum hyperdex\_client\_returncode* status}\\
The status of the operation.  The client library will fill in this variable
before returning this operation's request id from \code{hyperdex\_client\_loop}.
The pointer must remain valid until the operation completes, and the pointer
should not be aliased to the status for any other outstanding operation.

\end{itemize}

%%%%%%%%%%%%%%%%%%%% map_atomic_mul %%%%%%%%%%%%%%%%%%%%
\pagebreak
\subsubsection{\code{map\_atomic\_mul}}
\label{api:c:map_atomic_mul}
\index{map\_atomic\_mul!C API}
Multiply the value of each key-value pair by the specified number for each map.

%%% Generated below here
\paragraph{Behavior:}
\begin{itemize}[noitemsep]
This operation requires a pre-existing object in order to complete successfully.
If no object exists, the operation will fail with \code{NOTFOUND}.

\item This operation mutates the value of a key-value pair in a map.  This call
    is similar to the equivalent call without the \code{map\_} prefix, but
    operates on the value of a pair in a map, instead of on an attribute's
    value.  If there is no pair with the specified map key, a new pair will be
    created and initialized to its default value.  If this is undesirable, it
    may be avoided by using a conditional operation that requires that the map
    contain the key in question.

\end{itemize}


\paragraph{Definition:}
\begin{ccode}
int64_t hyperdex_client_map_atomic_mul(struct hyperdex_client* client,
        const char* space,
        const char* key, size_t key_sz,
        const struct hyperdex_client_map_attribute* mapattrs, size_t mapattrs_sz,
        enum hyperdex_client_returncode* status);
\end{ccode}

\paragraph{Parameters:}
\begin{itemize}[noitemsep]
\item \code{struct hyperdex\_client* client}\\
The HyperDex client connection to use for the operation.

\item \code{const char* space}\\
The name of the space as a string or symbol.

\item \code{const char* key, size\_t key\_sz}\\
The key for the operation as a Python type.

\item \code{const struct hyperdex\_client\_map\_attribute* mapattrs, size\_t mapattrs\_sz}\\
A hash specifying map attributes to modify and their respective key/values.

\end{itemize}

\paragraph{Returns:}
\begin{itemize}[noitemsep]
\item \code{enum hyperdex\_client\_returncode* status}\\
The status of the operation.  The client library will fill in this variable
before returning this operation's request id from \code{hyperdex\_client\_loop}.
The pointer must remain valid until the operation completes, and the pointer
should not be aliased to the status for any other outstanding operation.

\end{itemize}

%%%%%%%%%%%%%%%%%%%% cond_map_atomic_mul %%%%%%%%%%%%%%%%%%%%
\pagebreak
\subsubsection{\code{cond\_map\_atomic\_mul}}
\label{api:c:cond_map_atomic_mul}
\index{cond\_map\_atomic\_mul!C API}
Conditionally multiply the value of each key-value pair by the specified number
for each map.

%%% Generated below here
\paragraph{Behavior:}
\begin{itemize}[noitemsep]
This operation requires a pre-existing object in order to complete successfully.
If no object exists, the operation will fail with \code{NOTFOUND}.

This operation will succeed if and only if the predicates specified by
\code{checks} hold on the pre-existing object.  If any of the predicates are not
true for the existing object, then the operation will have no effect and fail
with \code{CMPFAIL}.

All checks are atomic with the write.  HyperDex guarantees that no other
operation will come between validating the checks, and writing the new version
of the object.

\item This operation mutates the value of a key-value pair in a map.  This call
    is similar to the equivalent call without the \code{map\_} prefix, but
    operates on the value of a pair in a map, instead of on an attribute's
    value.  If there is no pair with the specified map key, a new pair will be
    created and initialized to its default value.  If this is undesirable, it
    may be avoided by using a conditional operation that requires that the map
    contain the key in question.

\end{itemize}


\paragraph{Definition:}
\begin{ccode}
int64_t hyperdex_client_cond_map_atomic_mul(struct hyperdex_client* client,
        const char* space,
        const char* key, size_t key_sz,
        const struct hyperdex_client_attribute_check* checks, size_t checks_sz,
        const struct hyperdex_client_map_attribute* mapattrs, size_t mapattrs_sz,
        enum hyperdex_client_returncode* status);
\end{ccode}

\paragraph{Parameters:}
\begin{itemize}[noitemsep]
\item \code{struct hyperdex\_client* client}\\
The HyperDex client connection to use for the operation.

\item \code{const char* space}\\
The name of the space as a string or symbol.

\item \code{const char* key, size\_t key\_sz}\\
The key for the operation as a Python type.

\item \code{const struct hyperdex\_client\_attribute\_check* checks, size\_t checks\_sz}\\
A hash of predicates to check against.

\item \code{const struct hyperdex\_client\_map\_attribute* mapattrs, size\_t mapattrs\_sz}\\
A hash specifying map attributes to modify and their respective key/values.

\end{itemize}

\paragraph{Returns:}
\begin{itemize}[noitemsep]
\item \code{enum hyperdex\_client\_returncode* status}\\
The status of the operation.  The client library will fill in this variable
before returning this operation's request id from \code{hyperdex\_client\_loop}.
The pointer must remain valid until the operation completes, and the pointer
should not be aliased to the status for any other outstanding operation.

\end{itemize}

%%%%%%%%%%%%%%%%%%%% map_atomic_div %%%%%%%%%%%%%%%%%%%%
\pagebreak
\subsubsection{\code{map\_atomic\_div}}
\label{api:c:map_atomic_div}
\index{map\_atomic\_div!C API}
Divide the value of each key-value pair by the specified number for each map.

%%% Generated below here
\paragraph{Behavior:}
\begin{itemize}[noitemsep]
This operation requires a pre-existing object in order to complete successfully.
If no object exists, the operation will fail with \code{NOTFOUND}.

\item This operation mutates the value of a key-value pair in a map.  This call
    is similar to the equivalent call without the \code{map\_} prefix, but
    operates on the value of a pair in a map, instead of on an attribute's
    value.  If there is no pair with the specified map key, a new pair will be
    created and initialized to its default value.  If this is undesirable, it
    may be avoided by using a conditional operation that requires that the map
    contain the key in question.

\end{itemize}


\paragraph{Definition:}
\begin{ccode}
int64_t hyperdex_client_map_atomic_div(struct hyperdex_client* client,
        const char* space,
        const char* key, size_t key_sz,
        const struct hyperdex_client_map_attribute* mapattrs, size_t mapattrs_sz,
        enum hyperdex_client_returncode* status);
\end{ccode}

\paragraph{Parameters:}
\begin{itemize}[noitemsep]
\item \code{struct hyperdex\_client* client}\\
The HyperDex client connection to use for the operation.

\item \code{const char* space}\\
The name of the space as a string or symbol.

\item \code{const char* key, size\_t key\_sz}\\
The key for the operation as a Python type.

\item \code{const struct hyperdex\_client\_map\_attribute* mapattrs, size\_t mapattrs\_sz}\\
A hash specifying map attributes to modify and their respective key/values.

\end{itemize}

\paragraph{Returns:}
\begin{itemize}[noitemsep]
\item \code{enum hyperdex\_client\_returncode* status}\\
The status of the operation.  The client library will fill in this variable
before returning this operation's request id from \code{hyperdex\_client\_loop}.
The pointer must remain valid until the operation completes, and the pointer
should not be aliased to the status for any other outstanding operation.

\end{itemize}

%%%%%%%%%%%%%%%%%%%% cond_map_atomic_div %%%%%%%%%%%%%%%%%%%%
\pagebreak
\subsubsection{\code{cond\_map\_atomic\_div}}
\label{api:c:cond_map_atomic_div}
\index{cond\_map\_atomic\_div!C API}
Conditionally divide the value of each key-value pair by the specified number for each map.

%%% Generated below here
\paragraph{Behavior:}
\begin{itemize}[noitemsep]
This operation requires a pre-existing object in order to complete successfully.
If no object exists, the operation will fail with \code{NOTFOUND}.

This operation will succeed if and only if the predicates specified by
\code{checks} hold on the pre-existing object.  If any of the predicates are not
true for the existing object, then the operation will have no effect and fail
with \code{CMPFAIL}.

All checks are atomic with the write.  HyperDex guarantees that no other
operation will come between validating the checks, and writing the new version
of the object.

\item This operation mutates the value of a key-value pair in a map.  This call
    is similar to the equivalent call without the \code{map\_} prefix, but
    operates on the value of a pair in a map, instead of on an attribute's
    value.  If there is no pair with the specified map key, a new pair will be
    created and initialized to its default value.  If this is undesirable, it
    may be avoided by using a conditional operation that requires that the map
    contain the key in question.

\end{itemize}


\paragraph{Definition:}
\begin{ccode}
int64_t hyperdex_client_cond_map_atomic_div(struct hyperdex_client* client,
        const char* space,
        const char* key, size_t key_sz,
        const struct hyperdex_client_attribute_check* checks, size_t checks_sz,
        const struct hyperdex_client_map_attribute* mapattrs, size_t mapattrs_sz,
        enum hyperdex_client_returncode* status);
\end{ccode}

\paragraph{Parameters:}
\begin{itemize}[noitemsep]
\item \code{struct hyperdex\_client* client}\\
The HyperDex client connection to use for the operation.

\item \code{const char* space}\\
The name of the space as a string or symbol.

\item \code{const char* key, size\_t key\_sz}\\
The key for the operation as a Python type.

\item \code{const struct hyperdex\_client\_attribute\_check* checks, size\_t checks\_sz}\\
A hash of predicates to check against.

\item \code{const struct hyperdex\_client\_map\_attribute* mapattrs, size\_t mapattrs\_sz}\\
A hash specifying map attributes to modify and their respective key/values.

\end{itemize}

\paragraph{Returns:}
\begin{itemize}[noitemsep]
\item \code{enum hyperdex\_client\_returncode* status}\\
The status of the operation.  The client library will fill in this variable
before returning this operation's request id from \code{hyperdex\_client\_loop}.
The pointer must remain valid until the operation completes, and the pointer
should not be aliased to the status for any other outstanding operation.

\end{itemize}

%%%%%%%%%%%%%%%%%%%% map_atomic_mod %%%%%%%%%%%%%%%%%%%%
\pagebreak
\subsubsection{\code{map\_atomic\_mod}}
\label{api:c:map_atomic_mod}
\index{map\_atomic\_mod!C API}
Store the value of the key-value pair modulo the specified number for each map.
This operation requires a pre-existing object in order to complete successfully.
If no object exists, the operation will fail with \code{NOTFOUND}.



\paragraph{Definition:}
\begin{ccode}
int64_t hyperdex_client_map_atomic_mod(struct hyperdex_client* client,
        const char* space,
        const char* key, size_t key_sz,
        const struct hyperdex_client_map_attribute* mapattrs, size_t mapattrs_sz,
        enum hyperdex_client_returncode* status);
\end{ccode}

\paragraph{Parameters:}
\begin{itemize}[noitemsep]
\item \code{struct hyperdex\_client* client}\\
The HyperDex client connection to use for the operation.

\item \code{const char* space}\\
The name of the space as a string or symbol.

\item \code{const char* key, size\_t key\_sz}\\
The key for the operation as a Python type.

\item \code{const struct hyperdex\_client\_map\_attribute* mapattrs, size\_t mapattrs\_sz}\\
A hash specifying map attributes to modify and their respective key/values.

\end{itemize}

\paragraph{Returns:}
\begin{itemize}[noitemsep]
\item \code{enum hyperdex\_client\_returncode* status}\\
The status of the operation.  The client library will fill in this variable
before returning this operation's request id from \code{hyperdex\_client\_loop}.
The pointer must remain valid until the operation completes, and the pointer
should not be aliased to the status for any other outstanding operation.

\end{itemize}

%%%%%%%%%%%%%%%%%%%% cond_map_atomic_mod %%%%%%%%%%%%%%%%%%%%
\pagebreak
\subsubsection{\code{cond\_map\_atomic\_mod}}
\label{api:c:cond_map_atomic_mod}
\index{cond\_map\_atomic\_mod!C API}
Conditionally store the value of the key-value pair modulo the specified number
for each map.

%%% Generated below here
\paragraph{Behavior:}
\begin{itemize}[noitemsep]
This operation requires a pre-existing object in order to complete successfully.
If no object exists, the operation will fail with \code{NOTFOUND}.

This operation will succeed if and only if the predicates specified by
\code{checks} hold on the pre-existing object.  If any of the predicates are not
true for the existing object, then the operation will have no effect and fail
with \code{CMPFAIL}.

All checks are atomic with the write.  HyperDex guarantees that no other
operation will come between validating the checks, and writing the new version
of the object.

\item This operation mutates the value of a key-value pair in a map.  This call
    is similar to the equivalent call without the \code{map\_} prefix, but
    operates on the value of a pair in a map, instead of on an attribute's
    value.  If there is no pair with the specified map key, a new pair will be
    created and initialized to its default value.  If this is undesirable, it
    may be avoided by using a conditional operation that requires that the map
    contain the key in question.

\end{itemize}


\paragraph{Definition:}
\begin{ccode}
int64_t hyperdex_client_cond_map_atomic_mod(struct hyperdex_client* client,
        const char* space,
        const char* key, size_t key_sz,
        const struct hyperdex_client_attribute_check* checks, size_t checks_sz,
        const struct hyperdex_client_map_attribute* mapattrs, size_t mapattrs_sz,
        enum hyperdex_client_returncode* status);
\end{ccode}

\paragraph{Parameters:}
\begin{itemize}[noitemsep]
\item \code{struct hyperdex\_client* client}\\
The HyperDex client connection to use for the operation.

\item \code{const char* space}\\
The name of the space as a string or symbol.

\item \code{const char* key, size\_t key\_sz}\\
The key for the operation as a Python type.

\item \code{const struct hyperdex\_client\_attribute\_check* checks, size\_t checks\_sz}\\
A hash of predicates to check against.

\item \code{const struct hyperdex\_client\_map\_attribute* mapattrs, size\_t mapattrs\_sz}\\
A hash specifying map attributes to modify and their respective key/values.

\end{itemize}

\paragraph{Returns:}
\begin{itemize}[noitemsep]
\item \code{enum hyperdex\_client\_returncode* status}\\
The status of the operation.  The client library will fill in this variable
before returning this operation's request id from \code{hyperdex\_client\_loop}.
The pointer must remain valid until the operation completes, and the pointer
should not be aliased to the status for any other outstanding operation.

\end{itemize}

%%%%%%%%%%%%%%%%%%%% map_atomic_and %%%%%%%%%%%%%%%%%%%%
\pagebreak
\subsubsection{\code{map\_atomic\_and}}
\label{api:c:map_atomic_and}
\index{map\_atomic\_and!C API}
Store the bitwise AND of the value of the key-value pair and the specified
number for each map.

%%% Generated below here
\paragraph{Behavior:}
\begin{itemize}[noitemsep]
This operation requires a pre-existing object in order to complete successfully.
If no object exists, the operation will fail with \code{NOTFOUND}.

\item This operation mutates the value of a key-value pair in a map.  This call
    is similar to the equivalent call without the \code{map\_} prefix, but
    operates on the value of a pair in a map, instead of on an attribute's
    value.  If there is no pair with the specified map key, a new pair will be
    created and initialized to its default value.  If this is undesirable, it
    may be avoided by using a conditional operation that requires that the map
    contain the key in question.

\end{itemize}


\paragraph{Definition:}
\begin{ccode}
int64_t hyperdex_client_map_atomic_and(struct hyperdex_client* client,
        const char* space,
        const char* key, size_t key_sz,
        const struct hyperdex_client_map_attribute* mapattrs, size_t mapattrs_sz,
        enum hyperdex_client_returncode* status);
\end{ccode}

\paragraph{Parameters:}
\begin{itemize}[noitemsep]
\item \code{struct hyperdex\_client* client}\\
The HyperDex client connection to use for the operation.

\item \code{const char* space}\\
The name of the space as a string or symbol.

\item \code{const char* key, size\_t key\_sz}\\
The key for the operation as a Python type.

\item \code{const struct hyperdex\_client\_map\_attribute* mapattrs, size\_t mapattrs\_sz}\\
A hash specifying map attributes to modify and their respective key/values.

\end{itemize}

\paragraph{Returns:}
\begin{itemize}[noitemsep]
\item \code{enum hyperdex\_client\_returncode* status}\\
The status of the operation.  The client library will fill in this variable
before returning this operation's request id from \code{hyperdex\_client\_loop}.
The pointer must remain valid until the operation completes, and the pointer
should not be aliased to the status for any other outstanding operation.

\end{itemize}

%%%%%%%%%%%%%%%%%%%% cond_map_atomic_and %%%%%%%%%%%%%%%%%%%%
\pagebreak
\subsubsection{\code{cond\_map\_atomic\_and}}
\label{api:c:cond_map_atomic_and}
\index{cond\_map\_atomic\_and!C API}
Store the bitwise AND of the value of the key-value pair and the specified
number for each map attribute if and only if the \code{checks} hold on the
object.
This operation requires a pre-existing object in order to complete successfully.
If no object exists, the operation will fail with \code{NOTFOUND}.


This operation will succeed if and only if the predicates specified by
\code{checks} hold on the pre-existing object.  If any of the predicates are not
true for the existing object, then the operation will have no effect and fail
with \code{CMPFAIL}.

All checks are atomic with the write.  HyperDex guarantees that no other
operation will come between validating the checks, and writing the new version
of the object.



\paragraph{Definition:}
\begin{ccode}
int64_t hyperdex_client_cond_map_atomic_and(struct hyperdex_client* client,
        const char* space,
        const char* key, size_t key_sz,
        const struct hyperdex_client_attribute_check* checks, size_t checks_sz,
        const struct hyperdex_client_map_attribute* mapattrs, size_t mapattrs_sz,
        enum hyperdex_client_returncode* status);
\end{ccode}

\paragraph{Parameters:}
\begin{itemize}[noitemsep]
\item \code{struct hyperdex\_client* client}\\
The HyperDex client connection to use for the operation.

\item \code{const char* space}\\
The name of the space as a string or symbol.

\item \code{const char* key, size\_t key\_sz}\\
The key for the operation as a Python type.

\item \code{const struct hyperdex\_client\_attribute\_check* checks, size\_t checks\_sz}\\
A hash of predicates to check against.

\item \code{const struct hyperdex\_client\_map\_attribute* mapattrs, size\_t mapattrs\_sz}\\
A hash specifying map attributes to modify and their respective key/values.

\end{itemize}

\paragraph{Returns:}
\begin{itemize}[noitemsep]
\item \code{enum hyperdex\_client\_returncode* status}\\
The status of the operation.  The client library will fill in this variable
before returning this operation's request id from \code{hyperdex\_client\_loop}.
The pointer must remain valid until the operation completes, and the pointer
should not be aliased to the status for any other outstanding operation.

\end{itemize}

%%%%%%%%%%%%%%%%%%%% map_atomic_or %%%%%%%%%%%%%%%%%%%%
\pagebreak
\subsubsection{\code{map\_atomic\_or}}
\label{api:c:map_atomic_or}
\index{map\_atomic\_or!C API}
Store the bitwise OR of the value of the key-value pair and the specified number
for each map.

%%% Generated below here
\paragraph{Behavior:}
\begin{itemize}[noitemsep]
This operation requires a pre-existing object in order to complete successfully.
If no object exists, the operation will fail with \code{NOTFOUND}.

\item This operation mutates the value of a key-value pair in a map.  This call
    is similar to the equivalent call without the \code{map\_} prefix, but
    operates on the value of a pair in a map, instead of on an attribute's
    value.  If there is no pair with the specified map key, a new pair will be
    created and initialized to its default value.  If this is undesirable, it
    may be avoided by using a conditional operation that requires that the map
    contain the key in question.

\end{itemize}


\paragraph{Definition:}
\begin{ccode}
int64_t hyperdex_client_map_atomic_or(struct hyperdex_client* client,
        const char* space,
        const char* key, size_t key_sz,
        const struct hyperdex_client_map_attribute* mapattrs, size_t mapattrs_sz,
        enum hyperdex_client_returncode* status);
\end{ccode}

\paragraph{Parameters:}
\begin{itemize}[noitemsep]
\item \code{struct hyperdex\_client* client}\\
The HyperDex client connection to use for the operation.

\item \code{const char* space}\\
The name of the space as a string or symbol.

\item \code{const char* key, size\_t key\_sz}\\
The key for the operation as a Python type.

\item \code{const struct hyperdex\_client\_map\_attribute* mapattrs, size\_t mapattrs\_sz}\\
A hash specifying map attributes to modify and their respective key/values.

\end{itemize}

\paragraph{Returns:}
\begin{itemize}[noitemsep]
\item \code{enum hyperdex\_client\_returncode* status}\\
The status of the operation.  The client library will fill in this variable
before returning this operation's request id from \code{hyperdex\_client\_loop}.
The pointer must remain valid until the operation completes, and the pointer
should not be aliased to the status for any other outstanding operation.

\end{itemize}

%%%%%%%%%%%%%%%%%%%% cond_map_atomic_or %%%%%%%%%%%%%%%%%%%%
\pagebreak
\subsubsection{\code{cond\_map\_atomic\_or}}
\label{api:c:cond_map_atomic_or}
\index{cond\_map\_atomic\_or!C API}
Conditionally store the bitwise OR of the value of the key-value pair and the
specified number for each map.

%%% Generated below here
\paragraph{Behavior:}
\begin{itemize}[noitemsep]
This operation requires a pre-existing object in order to complete successfully.
If no object exists, the operation will fail with \code{NOTFOUND}.

This operation will succeed if and only if the predicates specified by
\code{checks} hold on the pre-existing object.  If any of the predicates are not
true for the existing object, then the operation will have no effect and fail
with \code{CMPFAIL}.

All checks are atomic with the write.  HyperDex guarantees that no other
operation will come between validating the checks, and writing the new version
of the object.

\item This operation mutates the value of a key-value pair in a map.  This call
    is similar to the equivalent call without the \code{map\_} prefix, but
    operates on the value of a pair in a map, instead of on an attribute's
    value.  If there is no pair with the specified map key, a new pair will be
    created and initialized to its default value.  If this is undesirable, it
    may be avoided by using a conditional operation that requires that the map
    contain the key in question.

\end{itemize}


\paragraph{Definition:}
\begin{ccode}
int64_t hyperdex_client_cond_map_atomic_or(struct hyperdex_client* client,
        const char* space,
        const char* key, size_t key_sz,
        const struct hyperdex_client_attribute_check* checks, size_t checks_sz,
        const struct hyperdex_client_map_attribute* mapattrs, size_t mapattrs_sz,
        enum hyperdex_client_returncode* status);
\end{ccode}

\paragraph{Parameters:}
\begin{itemize}[noitemsep]
\item \code{struct hyperdex\_client* client}\\
The HyperDex client connection to use for the operation.

\item \code{const char* space}\\
The name of the space as a string or symbol.

\item \code{const char* key, size\_t key\_sz}\\
The key for the operation as a Python type.

\item \code{const struct hyperdex\_client\_attribute\_check* checks, size\_t checks\_sz}\\
A hash of predicates to check against.

\item \code{const struct hyperdex\_client\_map\_attribute* mapattrs, size\_t mapattrs\_sz}\\
A hash specifying map attributes to modify and their respective key/values.

\end{itemize}

\paragraph{Returns:}
\begin{itemize}[noitemsep]
\item \code{enum hyperdex\_client\_returncode* status}\\
The status of the operation.  The client library will fill in this variable
before returning this operation's request id from \code{hyperdex\_client\_loop}.
The pointer must remain valid until the operation completes, and the pointer
should not be aliased to the status for any other outstanding operation.

\end{itemize}

%%%%%%%%%%%%%%%%%%%% map_atomic_xor %%%%%%%%%%%%%%%%%%%%
\pagebreak
\subsubsection{\code{map\_atomic\_xor}}
\label{api:c:map_atomic_xor}
\index{map\_atomic\_xor!C API}
Store the bitwise XOR of the value of the key-value pair and the specified
number for each map.

%%% Generated below here
\paragraph{Behavior:}
\begin{itemize}[noitemsep]
This operation requires a pre-existing object in order to complete successfully.
If no object exists, the operation will fail with \code{NOTFOUND}.

\item This operation mutates the value of a key-value pair in a map.  This call
    is similar to the equivalent call without the \code{map\_} prefix, but
    operates on the value of a pair in a map, instead of on an attribute's
    value.  If there is no pair with the specified map key, a new pair will be
    created and initialized to its default value.  If this is undesirable, it
    may be avoided by using a conditional operation that requires that the map
    contain the key in question.

\end{itemize}


\paragraph{Definition:}
\begin{ccode}
int64_t hyperdex_client_map_atomic_xor(struct hyperdex_client* client,
        const char* space,
        const char* key, size_t key_sz,
        const struct hyperdex_client_map_attribute* mapattrs, size_t mapattrs_sz,
        enum hyperdex_client_returncode* status);
\end{ccode}

\paragraph{Parameters:}
\begin{itemize}[noitemsep]
\item \code{struct hyperdex\_client* client}\\
The HyperDex client connection to use for the operation.

\item \code{const char* space}\\
The name of the space as a string or symbol.

\item \code{const char* key, size\_t key\_sz}\\
The key for the operation as a Python type.

\item \code{const struct hyperdex\_client\_map\_attribute* mapattrs, size\_t mapattrs\_sz}\\
A hash specifying map attributes to modify and their respective key/values.

\end{itemize}

\paragraph{Returns:}
\begin{itemize}[noitemsep]
\item \code{enum hyperdex\_client\_returncode* status}\\
The status of the operation.  The client library will fill in this variable
before returning this operation's request id from \code{hyperdex\_client\_loop}.
The pointer must remain valid until the operation completes, and the pointer
should not be aliased to the status for any other outstanding operation.

\end{itemize}

%%%%%%%%%%%%%%%%%%%% cond_map_atomic_xor %%%%%%%%%%%%%%%%%%%%
\pagebreak
\subsubsection{\code{cond\_map\_atomic\_xor}}
\label{api:c:cond_map_atomic_xor}
\index{cond\_map\_atomic\_xor!C API}
Store the bitwise XOR of the value of the key-value pair and the specified
number for each map attribute if and only if the \code{checks} hold on the
object.
This operation requires a pre-existing object in order to complete successfully.
If no object exists, the operation will fail with \code{NOTFOUND}.


This operation will succeed if and only if the predicates specified by
\code{checks} hold on the pre-existing object.  If any of the predicates are not
true for the existing object, then the operation will have no effect and fail
with \code{CMPFAIL}.

All checks are atomic with the write.  HyperDex guarantees that no other
operation will come between validating the checks, and writing the new version
of the object.



\paragraph{Definition:}
\begin{ccode}
int64_t hyperdex_client_cond_map_atomic_xor(struct hyperdex_client* client,
        const char* space,
        const char* key, size_t key_sz,
        const struct hyperdex_client_attribute_check* checks, size_t checks_sz,
        const struct hyperdex_client_map_attribute* mapattrs, size_t mapattrs_sz,
        enum hyperdex_client_returncode* status);
\end{ccode}

\paragraph{Parameters:}
\begin{itemize}[noitemsep]
\item \code{struct hyperdex\_client* client}\\
The HyperDex client connection to use for the operation.

\item \code{const char* space}\\
The name of the space as a string or symbol.

\item \code{const char* key, size\_t key\_sz}\\
The key for the operation as a Python type.

\item \code{const struct hyperdex\_client\_attribute\_check* checks, size\_t checks\_sz}\\
A hash of predicates to check against.

\item \code{const struct hyperdex\_client\_map\_attribute* mapattrs, size\_t mapattrs\_sz}\\
A hash specifying map attributes to modify and their respective key/values.

\end{itemize}

\paragraph{Returns:}
\begin{itemize}[noitemsep]
\item \code{enum hyperdex\_client\_returncode* status}\\
The status of the operation.  The client library will fill in this variable
before returning this operation's request id from \code{hyperdex\_client\_loop}.
The pointer must remain valid until the operation completes, and the pointer
should not be aliased to the status for any other outstanding operation.

\end{itemize}

%%%%%%%%%%%%%%%%%%%% map_string_prepend %%%%%%%%%%%%%%%%%%%%
\pagebreak
\subsubsection{\code{map\_string\_prepend}}
\label{api:c:map_string_prepend}
\index{map\_string\_prepend!C API}
Prepend the specified string to the value of the key-value pair for each map.

%%% Generated below here
\paragraph{Behavior:}
\begin{itemize}[noitemsep]
This operation requires a pre-existing object in order to complete successfully.
If no object exists, the operation will fail with \code{NOTFOUND}.

\item This operation mutates the value of a key-value pair in a map.  This call
    is similar to the equivalent call without the \code{map\_} prefix, but
    operates on the value of a pair in a map, instead of on an attribute's
    value.  If there is no pair with the specified map key, a new pair will be
    created and initialized to its default value.  If this is undesirable, it
    may be avoided by using a conditional operation that requires that the map
    contain the key in question.

\end{itemize}


\paragraph{Definition:}
\begin{ccode}
int64_t hyperdex_client_map_string_prepend(struct hyperdex_client* client,
        const char* space,
        const char* key, size_t key_sz,
        const struct hyperdex_client_map_attribute* mapattrs, size_t mapattrs_sz,
        enum hyperdex_client_returncode* status);
\end{ccode}

\paragraph{Parameters:}
\begin{itemize}[noitemsep]
\item \code{struct hyperdex\_client* client}\\
The HyperDex client connection to use for the operation.

\item \code{const char* space}\\
The name of the space as a string or symbol.

\item \code{const char* key, size\_t key\_sz}\\
The key for the operation as a Python type.

\item \code{const struct hyperdex\_client\_map\_attribute* mapattrs, size\_t mapattrs\_sz}\\
A hash specifying map attributes to modify and their respective key/values.

\end{itemize}

\paragraph{Returns:}
\begin{itemize}[noitemsep]
\item \code{enum hyperdex\_client\_returncode* status}\\
The status of the operation.  The client library will fill in this variable
before returning this operation's request id from \code{hyperdex\_client\_loop}.
The pointer must remain valid until the operation completes, and the pointer
should not be aliased to the status for any other outstanding operation.

\end{itemize}

%%%%%%%%%%%%%%%%%%%% cond_map_string_prepend %%%%%%%%%%%%%%%%%%%%
\pagebreak
\subsubsection{\code{cond\_map\_string\_prepend}}
\label{api:c:cond_map_string_prepend}
\index{cond\_map\_string\_prepend!C API}
Conditionally prepend the specified string to the value of the key-value pair
for each map.

%%% Generated below here
\paragraph{Behavior:}
\begin{itemize}[noitemsep]
This operation requires a pre-existing object in order to complete successfully.
If no object exists, the operation will fail with \code{NOTFOUND}.

This operation will succeed if and only if the predicates specified by
\code{checks} hold on the pre-existing object.  If any of the predicates are not
true for the existing object, then the operation will have no effect and fail
with \code{CMPFAIL}.

All checks are atomic with the write.  HyperDex guarantees that no other
operation will come between validating the checks, and writing the new version
of the object.

\item This operation mutates the value of a key-value pair in a map.  This call
    is similar to the equivalent call without the \code{map\_} prefix, but
    operates on the value of a pair in a map, instead of on an attribute's
    value.  If there is no pair with the specified map key, a new pair will be
    created and initialized to its default value.  If this is undesirable, it
    may be avoided by using a conditional operation that requires that the map
    contain the key in question.

\end{itemize}


\paragraph{Definition:}
\begin{ccode}
int64_t hyperdex_client_cond_map_string_prepend(struct hyperdex_client* client,
        const char* space,
        const char* key, size_t key_sz,
        const struct hyperdex_client_attribute_check* checks, size_t checks_sz,
        const struct hyperdex_client_map_attribute* mapattrs, size_t mapattrs_sz,
        enum hyperdex_client_returncode* status);
\end{ccode}

\paragraph{Parameters:}
\begin{itemize}[noitemsep]
\item \code{struct hyperdex\_client* client}\\
The HyperDex client connection to use for the operation.

\item \code{const char* space}\\
The name of the space as a string or symbol.

\item \code{const char* key, size\_t key\_sz}\\
The key for the operation as a Python type.

\item \code{const struct hyperdex\_client\_attribute\_check* checks, size\_t checks\_sz}\\
A hash of predicates to check against.

\item \code{const struct hyperdex\_client\_map\_attribute* mapattrs, size\_t mapattrs\_sz}\\
A hash specifying map attributes to modify and their respective key/values.

\end{itemize}

\paragraph{Returns:}
\begin{itemize}[noitemsep]
\item \code{enum hyperdex\_client\_returncode* status}\\
The status of the operation.  The client library will fill in this variable
before returning this operation's request id from \code{hyperdex\_client\_loop}.
The pointer must remain valid until the operation completes, and the pointer
should not be aliased to the status for any other outstanding operation.

\end{itemize}

%%%%%%%%%%%%%%%%%%%% map_string_append %%%%%%%%%%%%%%%%%%%%
\pagebreak
\subsubsection{\code{map\_string\_append}}
\label{api:c:map_string_append}
\index{map\_string\_append!C API}
Append the specified string to the value of the key-value pair for each map.

%%% Generated below here
\paragraph{Behavior:}
\begin{itemize}[noitemsep]
This operation requires a pre-existing object in order to complete successfully.
If no object exists, the operation will fail with \code{NOTFOUND}.

\item This operation mutates the value of a key-value pair in a map.  This call
    is similar to the equivalent call without the \code{map\_} prefix, but
    operates on the value of a pair in a map, instead of on an attribute's
    value.  If there is no pair with the specified map key, a new pair will be
    created and initialized to its default value.  If this is undesirable, it
    may be avoided by using a conditional operation that requires that the map
    contain the key in question.

\end{itemize}


\paragraph{Definition:}
\begin{ccode}
int64_t hyperdex_client_map_string_append(struct hyperdex_client* client,
        const char* space,
        const char* key, size_t key_sz,
        const struct hyperdex_client_map_attribute* mapattrs, size_t mapattrs_sz,
        enum hyperdex_client_returncode* status);
\end{ccode}

\paragraph{Parameters:}
\begin{itemize}[noitemsep]
\item \code{struct hyperdex\_client* client}\\
The HyperDex client connection to use for the operation.

\item \code{const char* space}\\
The name of the space as a string or symbol.

\item \code{const char* key, size\_t key\_sz}\\
The key for the operation as a Python type.

\item \code{const struct hyperdex\_client\_map\_attribute* mapattrs, size\_t mapattrs\_sz}\\
A hash specifying map attributes to modify and their respective key/values.

\end{itemize}

\paragraph{Returns:}
\begin{itemize}[noitemsep]
\item \code{enum hyperdex\_client\_returncode* status}\\
The status of the operation.  The client library will fill in this variable
before returning this operation's request id from \code{hyperdex\_client\_loop}.
The pointer must remain valid until the operation completes, and the pointer
should not be aliased to the status for any other outstanding operation.

\end{itemize}

%%%%%%%%%%%%%%%%%%%% cond_map_string_append %%%%%%%%%%%%%%%%%%%%
\pagebreak
\subsubsection{\code{cond\_map\_string\_append}}
\label{api:c:cond_map_string_append}
\index{cond\_map\_string\_append!C API}
Conditionally append the specified string to the value of the key-value pair for
each map.

%%% Generated below here
\paragraph{Behavior:}
\begin{itemize}[noitemsep]
This operation requires a pre-existing object in order to complete successfully.
If no object exists, the operation will fail with \code{NOTFOUND}.

This operation will succeed if and only if the predicates specified by
\code{checks} hold on the pre-existing object.  If any of the predicates are not
true for the existing object, then the operation will have no effect and fail
with \code{CMPFAIL}.

All checks are atomic with the write.  HyperDex guarantees that no other
operation will come between validating the checks, and writing the new version
of the object.

\item This operation mutates the value of a key-value pair in a map.  This call
    is similar to the equivalent call without the \code{map\_} prefix, but
    operates on the value of a pair in a map, instead of on an attribute's
    value.  If there is no pair with the specified map key, a new pair will be
    created and initialized to its default value.  If this is undesirable, it
    may be avoided by using a conditional operation that requires that the map
    contain the key in question.

\end{itemize}


\paragraph{Definition:}
\begin{ccode}
int64_t hyperdex_client_cond_map_string_append(struct hyperdex_client* client,
        const char* space,
        const char* key, size_t key_sz,
        const struct hyperdex_client_attribute_check* checks, size_t checks_sz,
        const struct hyperdex_client_map_attribute* mapattrs, size_t mapattrs_sz,
        enum hyperdex_client_returncode* status);
\end{ccode}

\paragraph{Parameters:}
\begin{itemize}[noitemsep]
\item \code{struct hyperdex\_client* client}\\
The HyperDex client connection to use for the operation.

\item \code{const char* space}\\
The name of the space as a string or symbol.

\item \code{const char* key, size\_t key\_sz}\\
The key for the operation as a Python type.

\item \code{const struct hyperdex\_client\_attribute\_check* checks, size\_t checks\_sz}\\
A hash of predicates to check against.

\item \code{const struct hyperdex\_client\_map\_attribute* mapattrs, size\_t mapattrs\_sz}\\
A hash specifying map attributes to modify and their respective key/values.

\end{itemize}

\paragraph{Returns:}
\begin{itemize}[noitemsep]
\item \code{enum hyperdex\_client\_returncode* status}\\
The status of the operation.  The client library will fill in this variable
before returning this operation's request id from \code{hyperdex\_client\_loop}.
The pointer must remain valid until the operation completes, and the pointer
should not be aliased to the status for any other outstanding operation.

\end{itemize}

%%%%%%%%%%%%%%%%%%%% map_atomic_min %%%%%%%%%%%%%%%%%%%%
\pagebreak
\subsubsection{\code{map\_atomic\_min}}
\label{api:c:map_atomic_min}
\index{map\_atomic\_min!C API}
Take the minium of the specified value and existing value for each key-value
pair.
This operation requires a pre-existing object in order to complete successfully.
If no object exists, the operation will fail with \code{NOTFOUND}.



\paragraph{Definition:}
\begin{ccode}
int64_t hyperdex_client_map_atomic_min(struct hyperdex_client* client,
        const char* space,
        const char* key, size_t key_sz,
        const struct hyperdex_client_map_attribute* mapattrs, size_t mapattrs_sz,
        enum hyperdex_client_returncode* status);
\end{ccode}

\paragraph{Parameters:}
\begin{itemize}[noitemsep]
\item \code{struct hyperdex\_client* client}\\
The HyperDex client connection to use for the operation.

\item \code{const char* space}\\
The name of the space as a string or symbol.

\item \code{const char* key, size\_t key\_sz}\\
The key for the operation as a Python type.

\item \code{const struct hyperdex\_client\_map\_attribute* mapattrs, size\_t mapattrs\_sz}\\
A hash specifying map attributes to modify and their respective key/values.

\end{itemize}

\paragraph{Returns:}
\begin{itemize}[noitemsep]
\item \code{enum hyperdex\_client\_returncode* status}\\
The status of the operation.  The client library will fill in this variable
before returning this operation's request id from \code{hyperdex\_client\_loop}.
The pointer must remain valid until the operation completes, and the pointer
should not be aliased to the status for any other outstanding operation.

\end{itemize}

%%%%%%%%%%%%%%%%%%%% cond_map_atomic_min %%%%%%%%%%%%%%%%%%%%
\pagebreak
\subsubsection{\code{cond\_map\_atomic\_min}}
\label{api:c:cond_map_atomic_min}
\index{cond\_map\_atomic\_min!C API}
XXX


\paragraph{Definition:}
\begin{ccode}
int64_t hyperdex_client_cond_map_atomic_min(struct hyperdex_client* client,
        const char* space,
        const char* key, size_t key_sz,
        const struct hyperdex_client_attribute_check* checks, size_t checks_sz,
        const struct hyperdex_client_map_attribute* mapattrs, size_t mapattrs_sz,
        enum hyperdex_client_returncode* status);
\end{ccode}

\paragraph{Parameters:}
\begin{itemize}[noitemsep]
\item \code{struct hyperdex\_client* client}\\
The HyperDex client connection to use for the operation.

\item \code{const char* space}\\
The name of the space as a string or symbol.

\item \code{const char* key, size\_t key\_sz}\\
The key for the operation as a Python type.

\item \code{const struct hyperdex\_client\_attribute\_check* checks, size\_t checks\_sz}\\
A hash of predicates to check against.

\item \code{const struct hyperdex\_client\_map\_attribute* mapattrs, size\_t mapattrs\_sz}\\
A hash specifying map attributes to modify and their respective key/values.

\end{itemize}

\paragraph{Returns:}
\begin{itemize}[noitemsep]
\item \code{enum hyperdex\_client\_returncode* status}\\
The status of the operation.  The client library will fill in this variable
before returning this operation's request id from \code{hyperdex\_client\_loop}.
The pointer must remain valid until the operation completes, and the pointer
should not be aliased to the status for any other outstanding operation.

\end{itemize}

%%%%%%%%%%%%%%%%%%%% map_atomic_max %%%%%%%%%%%%%%%%%%%%
\pagebreak
\subsubsection{\code{map\_atomic\_max}}
\label{api:c:map_atomic_max}
\index{map\_atomic\_max!C API}
Take the maximum of the specified value and existing value for each key-value
pair.
This operation requires a pre-existing object in order to complete successfully.
If no object exists, the operation will fail with \code{NOTFOUND}.



\paragraph{Definition:}
\begin{ccode}
int64_t hyperdex_client_map_atomic_max(struct hyperdex_client* client,
        const char* space,
        const char* key, size_t key_sz,
        const struct hyperdex_client_map_attribute* mapattrs, size_t mapattrs_sz,
        enum hyperdex_client_returncode* status);
\end{ccode}

\paragraph{Parameters:}
\begin{itemize}[noitemsep]
\item \code{struct hyperdex\_client* client}\\
The HyperDex client connection to use for the operation.

\item \code{const char* space}\\
The name of the space as a string or symbol.

\item \code{const char* key, size\_t key\_sz}\\
The key for the operation as a Python type.

\item \code{const struct hyperdex\_client\_map\_attribute* mapattrs, size\_t mapattrs\_sz}\\
A hash specifying map attributes to modify and their respective key/values.

\end{itemize}

\paragraph{Returns:}
\begin{itemize}[noitemsep]
\item \code{enum hyperdex\_client\_returncode* status}\\
The status of the operation.  The client library will fill in this variable
before returning this operation's request id from \code{hyperdex\_client\_loop}.
The pointer must remain valid until the operation completes, and the pointer
should not be aliased to the status for any other outstanding operation.

\end{itemize}

%%%%%%%%%%%%%%%%%%%% cond_map_atomic_max %%%%%%%%%%%%%%%%%%%%
\pagebreak
\subsubsection{\code{cond\_map\_atomic\_max}}
\label{api:c:cond_map_atomic_max}
\index{cond\_map\_atomic\_max!C API}
XXX


\paragraph{Definition:}
\begin{ccode}
int64_t hyperdex_client_cond_map_atomic_max(struct hyperdex_client* client,
        const char* space,
        const char* key, size_t key_sz,
        const struct hyperdex_client_attribute_check* checks, size_t checks_sz,
        const struct hyperdex_client_map_attribute* mapattrs, size_t mapattrs_sz,
        enum hyperdex_client_returncode* status);
\end{ccode}

\paragraph{Parameters:}
\begin{itemize}[noitemsep]
\item \code{struct hyperdex\_client* client}\\
The HyperDex client connection to use for the operation.

\item \code{const char* space}\\
The name of the space as a string or symbol.

\item \code{const char* key, size\_t key\_sz}\\
The key for the operation as a Python type.

\item \code{const struct hyperdex\_client\_attribute\_check* checks, size\_t checks\_sz}\\
A hash of predicates to check against.

\item \code{const struct hyperdex\_client\_map\_attribute* mapattrs, size\_t mapattrs\_sz}\\
A hash specifying map attributes to modify and their respective key/values.

\end{itemize}

\paragraph{Returns:}
\begin{itemize}[noitemsep]
\item \code{enum hyperdex\_client\_returncode* status}\\
The status of the operation.  The client library will fill in this variable
before returning this operation's request id from \code{hyperdex\_client\_loop}.
The pointer must remain valid until the operation completes, and the pointer
should not be aliased to the status for any other outstanding operation.

\end{itemize}

%%%%%%%%%%%%%%%%%%%% search %%%%%%%%%%%%%%%%%%%%
\pagebreak
\subsubsection{\code{search}}
\label{api:c:search}
\index{search!C API}
Return all objects that match the specified \code{checks}.

\paragraph{Behavior:}
\begin{itemize}[noitemsep]
\item XXX % XXX

\item XXX % XXX

\end{itemize}


\paragraph{Definition:}
\begin{ccode}
int64_t hyperdex_client_search(struct hyperdex_client* client,
        const char* space,
        const struct hyperdex_client_attribute_check* checks, size_t checks_sz,
        enum hyperdex_client_returncode* status,
        const struct hyperdex_client_attribute** attrs, size_t* attrs_sz);
\end{ccode}

\paragraph{Parameters:}
\begin{itemize}[noitemsep]
\item \code{struct hyperdex\_client* client}\\
The HyperDex client connection to use for the operation.

\item \code{const char* space}\\
The name of the space as a string or symbol.

\item \code{const struct hyperdex\_client\_attribute\_check* checks, size\_t checks\_sz}\\
A hash of predicates to check against.

\end{itemize}

\paragraph{Returns:}
\begin{itemize}[noitemsep]
\item \code{enum hyperdex\_client\_returncode* status}\\
The status of the operation.  The client library will fill in this variable
before returning this operation's request id from \code{hyperdex\_client\_loop}.
The pointer must remain valid until the operation completes, and the pointer
should not be aliased to the status for any other outstanding operation.

\item \code{const struct hyperdex\_client\_attribute** attrs, size\_t* attrs\_sz}\\
XXX

\end{itemize}

%%%%%%%%%%%%%%%%%%%% search_describe %%%%%%%%%%%%%%%%%%%%
\pagebreak
\subsubsection{\code{search\_describe}}
\label{api:c:search_describe}
\index{search\_describe!C API}
Return a human-readable string suitable for debugging search internals.  This
API is only really relevant for debugging the internals of \code{search}.


\paragraph{Definition:}
\begin{ccode}
int64_t hyperdex_client_search_describe(struct hyperdex_client* client,
        const char* space,
        const struct hyperdex_client_attribute_check* checks, size_t checks_sz,
        enum hyperdex_client_returncode* status,
        const char** description);
\end{ccode}

\paragraph{Parameters:}
\begin{itemize}[noitemsep]
\item \code{struct hyperdex\_client* client}\\
The HyperDex client connection to use for the operation.

\item \code{const char* space}\\
The name of the space as a string or symbol.

\item \code{const struct hyperdex\_client\_attribute\_check* checks, size\_t checks\_sz}\\
A hash of predicates to check against.

\end{itemize}

\paragraph{Returns:}
\begin{itemize}[noitemsep]
\item \code{enum hyperdex\_client\_returncode* status}\\
The status of the operation.  The client library will fill in this variable
before returning this operation's request id from \code{hyperdex\_client\_loop}.
The pointer must remain valid until the operation completes, and the pointer
should not be aliased to the status for any other outstanding operation.

\item \code{const char** description}\\
The description of the search.  This is a c-string that the client must free.

\end{itemize}

%%%%%%%%%%%%%%%%%%%% sorted_search %%%%%%%%%%%%%%%%%%%%
\pagebreak
\subsubsection{\code{sorted\_search}}
\label{api:c:sorted_search}
\index{sorted\_search!C API}
Return all objects that match the specified \code{checks}, sorted according to
\code{attr}.
\item XXX % XXX



\paragraph{Definition:}
\begin{ccode}
int64_t hyperdex_client_sorted_search(struct hyperdex_client* client,
        const char* space,
        const struct hyperdex_client_attribute_check* checks, size_t checks_sz,
        const char* sort_by,
        uint64_t limit,
        int maxmin,
        enum hyperdex_client_returncode* status,
        const struct hyperdex_client_attribute** attrs, size_t* attrs_sz);
\end{ccode}

\paragraph{Parameters:}
\begin{itemize}[noitemsep]
\item \code{struct hyperdex\_client* client}\\
The HyperDex client connection to use for the operation.

\item \code{const char* space}\\
The name of the space as a string or symbol.

\item \code{const struct hyperdex\_client\_attribute\_check* checks, size\_t checks\_sz}\\
A hash of predicates to check against.

\item \code{const char* sort\_by}\\
XXX

\item \code{uint64\_t limit}\\
XXX

\item \code{int maxmin}\\
Maximize (!= 0) or minimize (== 0).

\end{itemize}

\paragraph{Returns:}
\begin{itemize}[noitemsep]
\item \code{enum hyperdex\_client\_returncode* status}\\
The status of the operation.  The client library will fill in this variable
before returning this operation's request id from \code{hyperdex\_client\_loop}.
The pointer must remain valid until the operation completes, and the pointer
should not be aliased to the status for any other outstanding operation.

\item \code{const struct hyperdex\_client\_attribute** attrs, size\_t* attrs\_sz}\\
XXX

\end{itemize}

%%%%%%%%%%%%%%%%%%%% group_del %%%%%%%%%%%%%%%%%%%%
\pagebreak
\subsubsection{\code{group\_del}}
\label{api:c:group_del}
\index{group\_del!C API}
Asynchronously delete all objects that match the specified \code{checks}.

\paragraph{Behavior:}
\begin{itemize}[noitemsep]
\item This operation is roughly equivalent to a client manually deleting every
    object returned from a search, but saves HyperDex from sending to the client
    objects that are soon to be deleted.
\end{itemize}


\paragraph{Definition:}
\begin{ccode}
int64_t hyperdex_client_group_del(struct hyperdex_client* client,
        const char* space,
        const struct hyperdex_client_attribute_check* checks, size_t checks_sz,
        enum hyperdex_client_returncode* status);
\end{ccode}

\paragraph{Parameters:}
\begin{itemize}[noitemsep]
\item \code{struct hyperdex\_client* client}\\
The HyperDex client connection to use for the operation.

\item \code{const char* space}\\
The name of the space as a string or symbol.

\item \code{const struct hyperdex\_client\_attribute\_check* checks, size\_t checks\_sz}\\
A hash of predicates to check against.

\end{itemize}

\paragraph{Returns:}
\begin{itemize}[noitemsep]
\item \code{enum hyperdex\_client\_returncode* status}\\
The status of the operation.  The client library will fill in this variable
before returning this operation's request id from \code{hyperdex\_client\_loop}.
The pointer must remain valid until the operation completes, and the pointer
should not be aliased to the status for any other outstanding operation.

\end{itemize}

%%%%%%%%%%%%%%%%%%%% count %%%%%%%%%%%%%%%%%%%%
\pagebreak
\subsubsection{\code{count}}
\label{api:c:count}
\index{count!C API}
Count the number of objects that match the specified \code{checks}.

\paragraph{Behavior:}
\begin{itemize}[noitemsep]
\item This will return the number of objects counted by the search.  If an error
    occurs during the count, the count will reflect a partial count.  The real
    count will be higher than the returned value.  Some languages will throw an
    exception rather than return the partial count.
\end{itemize}


\paragraph{Definition:}
\begin{ccode}
int64_t hyperdex_client_count(struct hyperdex_client* client,
        const char* space,
        const struct hyperdex_client_attribute_check* checks, size_t checks_sz,
        enum hyperdex_client_returncode* status,
        uint64_t* count);
\end{ccode}

\paragraph{Parameters:}
\begin{itemize}[noitemsep]
\item \code{struct hyperdex\_client* client}\\
The HyperDex client connection to use for the operation.

\item \code{const char* space}\\
The name of the space as a string or symbol.

\item \code{const struct hyperdex\_client\_attribute\_check* checks, size\_t checks\_sz}\\
A hash of predicates to check against.

\end{itemize}

\paragraph{Returns:}
\begin{itemize}[noitemsep]
\item \code{enum hyperdex\_client\_returncode* status}\\
The status of the operation.  The client library will fill in this variable
before returning this operation's request id from \code{hyperdex\_client\_loop}.
The pointer must remain valid until the operation completes, and the pointer
should not be aliased to the status for any other outstanding operation.

\item \code{uint64\_t* count}\\
The number of objects which match the predicates.

\end{itemize}


\section{Data Structures}

HyperDex encodes as a bytestring all data structures passed between the
application and HyperDex.  This format of this bytestring varies according to
its type.  Below we describe the format of HyperDex data structures, and an API
for serializing and deserializing HyperDex's datatypes.

\subsection{Bytestring Format}

The format of the data structures is defined to be the same on all platforms.

For each format, Python-like psuedocode is provided that shows example
encodings.

\subsubsection{string}

A string is an 8-bit byte string.  HyperDex is agnostic to the contents of the
string, and it may contain any bytes, including \texttt{\\x00}.  By convention,
the trailing \texttt{NULL} should be omitted for C-strings to ensure
interoperability across languages.  For example:

\begin{pythoncode}
>>> encode_string('Hello\x00World!')
b'Hello\x00World!'
\end{pythoncode}

\subsubsection{int}

Integers are encoded as signed 8-byte little-endian integers.  For example:

\begin{pythoncode}
>>> encode_int(1)
b'\x01\x00\x00\x00\x00\x00\x00\x00'
>>> encode_int(-1)
b'\xff\xff\xff\xff\xff\xff\xff\xff'
>>> encode_int(0xdeadbeef)
b'\xef\xbe\xad\xde\x00\x00\x00\x00'
\end{pythoncode}

\subsubsection{float}

Floats are encoded as IEEE 754 binary64 values in little-endian format.  For
example:

\begin{pythoncode}
>>> encode_double(0)
b'\x00\x00\x00\x00\x00\x00\x00\x00'
>>> encode_double(3.1415)
b'o\x12\x83\xc0\xca!\t@'
\end{pythoncode}

\subsubsection{list(string)}

Lists of strings are encoded by concatenating the encoding of each string,
prefixed by an unsigned 4-byte little endian integer indicating the length of
the string.  For example:

\begin{pythoncode}
>>> encode_list_string([])
b''
>>> encode_list_string(['hello', 'world'])
b'\x05\x00\x00\x00hello\x05\x00\x00\x00world'
\end{pythoncode}

\subsubsection{list(int)}

Lists of integers are encoded by concatenating the encoded form of each integer.
For example:

\begin{pythoncode}
>>> encode_list_int([])
b''
>>> encode_list_int([1, -1, 0xdeadbeef])
b'\x01\x00\x00\x00\x00\x00\x00\x00' \
b'\xff\xff\xff\xff\xff\xff\xff\xff' \
b'\xef\xbe\xad\xde\x00\x00\x00\x00'
\end{pythoncode}

\subsubsection{list(floats)}

Lists of floats are encoded by concatenating the encoded form of each float.
For example:

\begin{pythoncode}
>>> encode_list_float([])
b''
>>> encode_list_float([0, 3.1415])
b'\x00\x00\x00\x00\x00\x00\x00\x00' \
b'o\x12\x83\xc0\xca!\t@'
\end{pythoncode}

\subsubsection{set(string)}

Sets of strings are encoded by concatenating the encoding of each string in
sorted order, where each string is prefixed by an unsigned 4-byte little endian
integer indicating the length of the string.  For example:

\begin{pythoncode}
>>> encode_set_string([])
b''
>>> encode_set_string(['world', 'hello'])
b'\x05\x00\x00\x00hello\x05\x00\x00\x00world'
\end{pythoncode}

\subsubsection{set(int)}

Sets of integers are encoded by concatenating the encoded form of each integer
in sorted order.  For example:

\begin{pythoncode}
>>> encode_set_int([])
b''
>>> encode_set_int([1, -1, 0xdeadbeef])
b'\xff\xff\xff\xff\xff\xff\xff\xff' \
b'\x01\x00\x00\x00\x00\x00\x00\x00' \
b'\xef\xbe\xad\xde\x00\x00\x00\x00'
\end{pythoncode}

\subsubsection{set(float)}

Sets of floats are encoded by concatenating the encoded form of each float in
sorted order.  For example:

\begin{pythoncode}
>>> encode_set_float([])
b''
>>> encode_set_float([3.1415, 0])
b'\x00\x00\x00\x00\x00\x00\x00\x00' \
b'o\x12\x83\xc0\xca!\t@'
\end{pythoncode}

\subsubsection{map(string, string)}

Maps from strings to strings are formed by encoding the individual elements,
each prefixed by an unsigned 4-byte little endian integer indicating their
length.  The pairs of elements are stored in sorted order according to the first
element of the pair (the map's key).  For example:

\begin{pythoncode}
>>> encode_map_string_string({})
b''
>>> encode_map_string_string({'hello': 'world',
...                           'map key': 'map val',
...                           'map', 'encoding'})
b'\x05\x00\x00\x00hello\x05\x00\x00\x00world' \
b'\x03\x00\x00\x00map\x08\x00\x00\x00encoding' \
b'\x07\x00\x00\x00map key\x07\x00\x00\x00map val'
\end{pythoncode}

\subsubsection{map(string, int)}

Maps from strings to ints are formed by encoding the individual elements, where
keys are prefixed by an unsigned 4-byte little endian integer indicating their
length.  The pairs of elements are stored in sorted order according to the first
element of the pair (the map's key).  For example:

\begin{pythoncode}
>>> encode_map_string_int({})
b''
>>> encode_map_string_int({'world': -1,
...                        'hello': 1})
b'\x05\x00\x00\x00hello\x01\x00\x00\x00\x00\x00\x00\x00' \
b'\x05\x00\x00\x00world\xff\xff\xff\xff\xff\xff\xff\xff'
\end{pythoncode}

\subsubsection{map(string, float)}

Maps from strings to ints are formed by encoding the individual elements, where
keys are prefixed by an unsigned 4-byte little endian integer indicating their
length.  The pairs of elements are stored in sorted order according to the first
element of the pair (the map's key).  For example:

\begin{pythoncode}
>>> encode_map_string_float({})
b''
>>> encode_map_string_float({'zero': 0,
...                          'pi': 3.1415})
b'\x02\x00\x00\x00pio\x12\x83\xc0\xca!\t@' \
b'\x04\x00\x00\x00zero\x00\x00\x00\x00\x00\x00\x00\x00'
\end{pythoncode}

\subsubsection{map(int, string)}

Maps from ints to strings are formed by encoding the individual elements, where
values are prefixed by an unsigned 4-byte little endian integer indicating their
length.  The pairs of elements are stored in sorted order according to the first
element of the pair (the map's key).  For example:

\begin{pythoncode}
>>> encode_map_int_string({})
b''
>>> encode_map_int_string({1: 'hello',
...                        -1: 'world'})
b'\xff\xff\xff\xff\xff\xff\xff\xff\x05\x00\x00\x00world' \
b'\x01\x00\x00\x00\x00\x00\x00\x00\x05\x00\x00\x00hello'
\end{pythoncode}

\subsubsection{map(int, int)}

Maps from ints to ints are formed by encoding the individual elements.  The
pairs of elements are stored in sorted order according to the first element of
the pair (the map's key).  For example:

\begin{pythoncode}
>>> encode_map_int_int({})
b''
>>> encode_map_int_int({1: 0xdeadbeef,
...                     -1: 0x1eaff00d})
b'\xff\xff\xff\xff\xff\xff\xff\xff\x0d\xf0\xaf\x1e\x00\x00\x00\x00' \
b'\x01\x00\x00\x00\x00\x00\x00\x00\xef\xbe\xad\xde\x00\x00\x00\x00'
\end{pythoncode}

\subsubsection{map(int, float)}

Maps from ints to floats are formed by encoding the individual elements.  The
pairs of elements are stored in sorted order according to the first element of
the pair (the map's key).  For example:

\begin{pythoncode}
>>> encode_map_int_float({})
b''
>>> encode_map_int_float({1: 0,
...                       -1: 3.1415})
b'\xff\xff\xff\xff\xff\xff\xff\xffo\x12\x83\xc0\xca!\t@' \
b'\x01\x00\x00\x00\x00\x00\x00\x00\x00\x00\x00\x00\x00\x00\x00\x00'
\end{pythoncode}

\subsubsection{map(float, string)}

Maps from floats to strings are formed by encoding the individual elements,
where values are prefixed by an unsigned 4-byte little endian integer indicating
their length.  The pairs of elements are stored in sorted order according to the
first element of the pair (the map's key).  For example:

\begin{pythoncode}
>>> encode_map_float_string({})
b''
>>> encode_map_float_string({0: 'hello',
...                          3.1415: 'world'})
b'\x00\x00\x00\x00\x00\x00\x00\x00\x05\x00\x00\x00hello' \
b'o\x12\x83\xc0\xca!\t@\x05\x00\x00\x00world'
\end{pythoncode}

\subsubsection{map(float, int)}

Maps from floats to ints are formed by encoding the individual elements.  The
pairs of elements are stored in sorted order according to the first element of
the pair (the map's key).  For example:

\begin{pythoncode}
>>> encode_map_float_int({})
b''
>>> encode_map_float_int({0: 1,
...                       3.1415: -1})
b'\x00\x00\x00\x00\x00\x00\x00\x00\x01\x00\x00\x00\x00\x00\x00\x00' \
b'o\x12\x83\xc0\xca!\t@\xff\xff\xff\xff\xff\xff\xff\xff'
\end{pythoncode}

\subsubsection{map(float, float)}

Maps from floats to floats are formed by encoding the individual elements.  The
pairs of elements are stored in sorted order according to the first element of
the pair (the map's key).  For example:

\begin{pythoncode}
>>> encode_map_float_float({})
b''
>>> encode_map_float_float({0: 1,
...                         3.1415: -1})
b'\x00\x00\x00\x00\x00\x00\x00\x00\x00\x00\x00\x00\x00\x00\xf0?' \
b'o\x12\x83\xc0\xca!\t@\x00\x00\x00\x00\x00\x00\xf0\xbf'
\end{pythoncode}

\subsection{Serialization API}

The serialization API supports serialization of all datatypes supported by
HyperDex.  Of course, feel free to manually encode data structures, especially
where doing so can make use of efficient stack-allocated data structures.

\subsubsection{struct hyperdex\_ds\_arena}

The packing routines described below may occasionally have to allocate memory
into which the encoded forms of the datatypes are copied.  To free the
programmer from the burden of having to manually allocate and free each of these
pieces of memory, the data structures API allocates all memory via an instance
of \texttt{struct hyperdex\_ds\_arena}.  Via a single call to
\texttt{hyperdex\_ds\_arena\_destroy}, all memory allocated via the arena is
free'd.

\texttt{struct hyperdex\_ds\_arena} is intentionally defined as an incomplete
type because its internals are subject to change.  To create an arena, call
\texttt{hyperdex\_ds\_arena\_create}.  The arena should subsequently be
destroyed via \texttt{hyperdex\_ds\_arena\_destroy}.

\begin{ccode}
struct hyperdex_ds_arena* hyperdex_ds_arena_create();
\end{ccode}
\funcdesc Create a new arena for alocating memory.  On success, this function
returns a non-null pointer for the new arena.  On failure, the function returns
\texttt{NULL}, indicating that memory allocation failed.  It is the caller's
responsibility to pass this function to \texttt{hyperdex\_ds\_arena\_destroy}
when finished.

\funcsep
\begin{ccode}
void hyperdex_ds_arena_destroy(struct hyperdex_ds_arena* arena);
\end{ccode}
\funcdesc Free all memory associated with \texttt{arena}.  This function always
succeeds.

\subsubsection{string}

No serialization is necessary for string data types.  For convenience, the
serialization API provides a copy function that copies a string into
arena-allocated memory.

\begin{ccode}
int hyperdex_ds_copy_string(struct hyperdex_ds_arena* arena, const char* str,
                            size_t str_sz, enum hyperdex_ds_returncode* status,
                            const char** value, size_t* value_sz);
\end{ccode}
\funcdesc  Copy the string \texttt{str}/\texttt{str\_sz} into memory allocated
via \texttt{arena} and return the copy via \texttt{value} and
\texttt{value\_sz}.  This function will fail and return -1 if there is
insufficient memory available for copying the string.  All pointers returned by
this function remain valid until \texttt{arena} is destroyed.  The client should
not attempt to free the returned copy.

\subsubsection{int}

\begin{ccode}
void hyperdex_ds_pack_int(int64_t num, char* value);
\end{ccode}
\funcdesc Packs \texttt{num} into the bytes pointed to by \texttt{buf}.  This
function always succeeds.  It is the caller's responsibility to ensure that
\texttt{buf} points to at least \unit{8}{\byte}.

\funcsep
\begin{ccode}
int hyperdex_ds_copy_int(struct hyperdex_ds_arena* arena, int64_t num,
                         enum hyperdex_ds_returncode* status,
                         const char** value, size_t* value_sz);
\end{ccode}
\funcdesc Encode \texttt{num} into memory allocated via \texttt{arena} and
return the value via \texttt{value} and \texttt{value\_sz}.  This function will
fail and return -1 if there is insufficient memory available for encoding the
nubmer.  All pointers returned by this function remain valid until
\texttt{arena} is destroyed.  The client should not attempt to free the returned
copy.

\subsubsection{float}

\begin{ccode}
void hyperdex_ds_pack_float(double num, char* value);
\end{ccode}
\funcdesc Pack \texttt{num} into the bytes pointed to by \texttt{buf}.  This
function always succeeds.  It is the caller's responsibility to ensure that
\texttt{buf} points to at least \unit{8}{\byte}.

\funcsep
\begin{ccode}
int hyperdex_ds_copy_float(struct hyperdex_ds_arena* arena, double num,
                           enum hyperdex_ds_returncode* status,
                           const char** value, size_t* value_sz);
\end{ccode}
\funcdesc Encode \texttt{num} into memory allocated via \texttt{arena} and
return the value via \texttt{value} and \texttt{value\_sz}.  This function will
fail and return -1 if there is insufficient memory available for encoding the
nubmer.  All pointers returned by this function remain valid until
\texttt{arena} is destroyed.  The client should not attempt to free the returned
copy.

\subsubsection{lists}

The below functions incrementally build lists, performing all relevant error
checking to ensure that the resuling HyperDex list is well-formed.  The first
element appended to the list implicitly determines the type of the list.  All
subsequent calls that push elements of a different type will fail.

\begin{ccode}
struct hyperdex_ds_list* hyperdex_ds_allocate_list(struct hyperdex_ds_arena* arena);
\end{ccode}
\funcdesc Create a new dynamic list.  This function will fail and return
\texttt{NULL} should memory allocation fail.

\funcsep
\begin{ccode}
int hyperdex_ds_list_insert_string(struct hyperdex_ds_list* list,
                                   const char* str, size_t str_sz,
                                   enum hyperdex_ds_returncode* status);
\end{ccode}
\funcdesc Append the string \texttt{str}/\texttt{str\_sz} to \texttt{list}.
This function will fail and return -1 if memory allocation fails, or the list is
not a list of strings.

\funcsep
\begin{ccode}
int hyperdex_ds_list_insert_int(struct hyperdex_ds_list* list, int64_t num,
                                enum hyperdex_ds_returncode* status);
\end{ccode}
\funcdesc Append the integer \texttt{num} to \texttt{list}.  This function will
fail and return -1 if memory allocation fails, or the list is not a list of
integers.

\funcsep
\begin{ccode}
int hyperdex_ds_list_insert_float(struct hyperdex_ds_list* list, double num,
                                  enum hyperdex_ds_returncode* status);
\end{ccode}
\funcdesc Append the float \texttt{num} to \texttt{list}.  This function will
fail and return -1 if memory allocation fails or the list is not a list of
floats.

\funcsep
\begin{ccode}
int hyperdex_ds_list_finalize(struct hyperdex_ds_list*,
                              enum hyperdex_ds_returncode* status,
                              const char** value, size_t* value_sz,
                              enum hyperdatatype* datatype);
\end{ccode}
\funcdesc Finalize the list by writing its elements into a bytestring.  This
function returns the bytestring and the list type.  It will fail and return -1
if memory allocation fails.

\subsubsection{sets}

The below functions incrementally build sets, performing all relevant error
checking to ensure that the resuling HyperDex set is well-formed.  The first
element inserted into the set implicitly determines the type of the set.  All
subsequent calls that insert elements of different types will fail.

\begin{ccode}
struct hyperdex_ds_set* hyperdex_ds_allocate_set(struct hyperdex_ds_arena* arena);
\end{ccode}
\funcdesc Create a new dynamic set.  This function will fail and return
\texttt{NULL} should memory allocation fail.

\funcsep
\begin{ccode}
int hyperdex_ds_set_insert_string(struct hyperdex_ds_set* set,
                                  const char* str, size_t str_sz,
                                  enum hyperdex_ds_returncode* status);
\end{ccode}
\funcdesc Insert the string \texttt{str}/\texttt{str\_sz} into \texttt{set}.
This function will fail and return -1 if memory allocation fails, or the set is
not a set of strings.

\funcsep
\begin{ccode}
int hyperdex_ds_set_insert_int(struct hyperdex_ds_set* set, int64_t num,
                               enum hyperdex_ds_returncode* status);
\end{ccode}
\funcdesc Insert the integer \texttt{num} into \texttt{set}.  This function will
fail and return -1 if memory allocation fails, or the set is not a set of
integers.

\funcsep
\begin{ccode}
int hyperdex_ds_set_insert_float(struct hyperdex_ds_set* set, double num,
                                 enum hyperdex_ds_returncode* status);
\end{ccode}
\funcdesc Insert the float \texttt{num} into \texttt{set}.  This function will
fail and return -1 if memory allocation fails, or the set is not a set of
floats.

\funcsep
\begin{ccode}
int hyperdex_ds_set_finalize(struct hyperdex_ds_set*,
                             enum hyperdex_ds_returncode* status,
                             const char** value, size_t* value_sz,
                             enum hyperdatatype* datatype);
\end{ccode}
\funcdesc Finalize the set by writing its elements into a bytestring.  This
function returns the bytestring and the set type.  It will fail and return -1 if
memory allocation fails.

\subsubsection{maps}

The below functions incrementally build maps, performing all relevant error
checking to ensure that the resuling HyperDex map is well-formed.  The first
key/value-pair inserted into the map implicitly determines the type of the map.  All
subsequent calls that insert elements of different types will fail.

The map is built by alternating calls to the key/value functions described
below, starting with a key-based function.  This keeps the number of cases
linear in the number of primitive types a map may contain, rather than appending
key-value pairs directly (which would require a quadratic number of calls).

\begin{ccode}
struct hyperdex_ds_map* hyperdex_ds_allocate_map(struct hyperdex_ds_arena* arena);
\end{ccode}
\funcdesc Create a new dynamic map.  This function will fail and return
\texttt{NULL} should memory allocation fail.

\funcsep
\begin{ccode}
int hyperdex_ds_map_insert_key_string(struct hyperdex_ds_map* map,
                                      const char* str, size_t str_sz,
                                      enum hyperdex_ds_returncode* status);
\end{ccode}
\funcdesc Set the key of the next pair to be inserted into \texttt{map} to the
string specified by \texttt{str} and \texttt{str\_sz}.  This function will fail
and return -1 if memory allocation fails, or the map does not use strings for
keys.

\funcsep
\begin{ccode}
int hyperdex_ds_map_insert_val_string(struct hyperdex_ds_map* map,
                                      const char* str, size_t str_sz,
                                      enum hyperdex_ds_returncode* status);
\end{ccode}
\funcdesc Set the value of the next pair to be inserted into \texttt{map} to the
string specified by \texttt{str} and \texttt{str\_sz}, and insert the pair.
This function will fail and return -1 if memory allocation fails, or the map
does not use strings for values.

\funcsep
\begin{ccode}
int hyperdex_ds_map_insert_key_int(struct hyperdex_ds_map* map,
                                   int64_t num,
                                   enum hyperdex_ds_returncode* status);
\end{ccode}
\funcdesc Set the key of the next pair to be inserted into \texttt{map} to the
integer specified by \texttt{num}.  This function will fail and return -1 if
memory allocation fails, or the map does not nuse integers for keys.

\funcsep
\begin{ccode}
int hyperdex_ds_map_insert_val_int(struct hyperdex_ds_map* map,
                                   int64_t num,
                                   enum hyperdex_ds_returncode* status);
\end{ccode}
\funcdesc Set the value of the next pair to be inserted into \texttt{map} to the
integer specified by \texttt{num}, and insert the pair.  This function will fail
and return -1 if memory allocation fails, or the map does not use integers for
values.

\funcsep
\begin{ccode}
int hyperdex_ds_map_insert_key_float(struct hyperdex_ds_map* map,
                                     double num,
                                     enum hyperdex_ds_returncode* status);
\end{ccode}
\funcdesc Set the key of the next pair to be inserted into \texttt{map} to the
float specified by \texttt{num}.  This function will fail and return -1 if
memory allocation fails, or the map does not use floats for keys.

\funcsep
\begin{ccode}
int hyperdex_ds_map_insert_val_float(struct hyperdex_ds_map* map,
                                     double num,
                                     enum hyperdex_ds_returncode* status);
\end{ccode}
\funcdesc Set the value of the next pair to be inserted into \texttt{map} to the
float specified by \texttt{num}.  This function will fail and return -1 if
memory allocation fails, or the map does not use floats for values.

\funcsep
\begin{ccode}
int hyperdex_ds_map_finalize(struct hyperdex_ds_map*,
                             enum hyperdex_ds_returncode* status,
                             const char** value, size_t* value_sz,
                             enum hyperdatatype* datatype);
\end{ccode}
\funcdesc Finalize the map by writing its key/value-pairs  into a bytestring.
This function returns the bytestring and the map type.  It will fail and return
-1 if memory allocation fails, or an uneven number of key/value calls were made.

\subsection{Deserialization API}

The deserialization API provides routines to unpack ints and floats, and iterate
the elements in lists, sets, and maps.  Iterators return elements one-by-one
with a minimal amount of copying and allocation.  All iterators are used in the
same pattern.  For example, to iterate a list of integers:

\inputminted{c}{api/c/iterate.c}

Compile and run this example with:

\begin{consolecode}
$ cc -o iterate iterate.c `pkg-config --cflags --libs hyperdex-client`
$ ./iterate
1
-1
3735928559
\end{consolecode}

The function \texttt{hyperdex\_ds\_iterator\_init} sets up the iterator.  Each
container data type has a specialized iteration function.  All iterators share
the same initialization function.

\begin{ccode}
void hyperdex_ds_iterator_init(struct hyperdex_ds_iterator* iter,
                               enum hyperdatatype datatype,
                               const char* value,
                               size_t value_sz);
\end{ccode}
\funcdesc Initialize an iterator for the given data type/value.  This function
always succeeds.

\subsubsection{string}

No deserialization is necessary for string data types.

\subsubsection{int}

\begin{ccode}
int hyperdex_ds_unpack_int(const char* buf, size_t buf_sz, int64_t* num);
\end{ccode}
\funcdesc Unpack \texttt{num} from \texttt{buf}/\texttt{buf\_sz}.  This function
will fail and return -1 if \texttt{buf\_sz} is not exactly \unit{8}{\byte}.

\subsubsection{float}

\begin{ccode}
int hyperdex_ds_unpack_float(const char* buf, size_t buf_sz, double* num);
\end{ccode}
\funcdesc Unpack \texttt{num} from \texttt{buf}/\texttt{buf\_sz}.  This function
will fail and return -1 if \texttt{buf\_sz} is not exactly \unit{8}{\byte}.

\subsubsection{lists}

\begin{ccode}
int hyperdex_ds_iterate_list_string_next(struct hyperdex_ds_iterator* iter,
                                         const char** str, size_t* str_sz);
\end{ccode}
\funcdesc Return the next string element in the list.  This function will return
1 if an element is returned, 0 if there are no elements to return, and -1 if the
list of strings is malformed.  The falue stored in \texttt{*str} is a pointer
into the list of strings and should not be free'd by the application.

\funcsep
\begin{ccode}
int hyperdex_ds_iterate_list_int_next(struct hyperdex_ds_iterator* iter, int64_t* num);
\end{ccode}
\funcdesc Return the next integer element in the list.  This function will
return 1 if an element is returned, 0 if there are no elements to return, and -1
if the list of integers is malformed.

\funcsep
\begin{ccode}
int hyperdex_ds_iterate_list_float_next(struct hyperdex_ds_iterator* iter, double* num);
\end{ccode}
\funcdesc Return the next float element in the list.  This function will return
1 if an element is returned, 0 if there are no elements to return, and -1 if the
list of floats is malformed.

\subsubsection{sets}

\begin{ccode}
int hyperdex_ds_iterate_set_string_next(struct hyperdex_ds_iterator* iter,
                                        const char** str, size_t* str_sz);
\end{ccode}
\funcdesc Return the next string element in the set.  This function will return
1 if an element is returned, 0 if there are no elements to return, and -1 if the
set of strings is malformed.  The value stored in \texttt{*str} is a pointer
into the set of strings and should not be free'd by the application.

\funcsep
\begin{ccode}
int hyperdex_ds_iterate_set_int_next(struct hyperdex_ds_iterator* iter, int64_t* num);
\end{ccode}
\funcdesc Return the next integer element in the set.  This function will return
1 if an element is returned, 0 if there are no elements to return, and -1 if the
set of ints is malformed.

\funcsep
\begin{ccode}
int hyperdex_ds_iterate_set_float_next(struct hyperdex_ds_iterator* iter, double* num);
\end{ccode}
\funcdesc Return the next float element in the set.  This function will return 1
if an element is returned, 0 if there are no elements to return, and -1 if the
set of ints is malformed.

\subsubsection{maps}

\begin{ccode}
int hyperdex_ds_iterate_map_string_string_next(struct hyperdex_ds_iterator* iter,
                                               const char** key, size_t* key_sz,
                                               const char** val, size_t* val_sz);
\end{ccode}
\funcdesc Return the next pair of (string, string) in the map.  This function
will return 1 if an element is returned, 0 if there are no elements to return,
and -1 if the map is malformed.  The values stored in \texttt{*key} and
\texttt{*val} are pointers into the map and should not be free'd by the
application.

\funcsep
\begin{ccode}
int hyperdex_ds_iterate_map_string_int_next(struct hyperdex_ds_iterator* iter,
                                            const char** key, size_t* key_sz,
                                            int64_t* val);
\end{ccode}
\funcdesc Return the next pair of (string, int) in the map.  This function will
return 1 if an element is returned, 0 if there are no elements to return, and -1
if the map is malformed.  The value stored in \texttt{*key} is a pointer into
the map and should not be free'd by the application.

\funcsep
\begin{ccode}
int hyperdex_ds_iterate_map_string_float_next(struct hyperdex_ds_iterator* iter,
                                              const char** key, size_t* key_sz,
                                              double* val);
\end{ccode}
\funcdesc Return the next pair of (string, float) in the map.  This function
will return 1 if an element is returned, 0 if there are no elements to return,
and -1 if the map is malformed.  The value stored in \texttt{*key} is a pointer
into the map and should not be free'd by the application.

\funcsep
\begin{ccode}
int hyperdex_ds_iterate_map_int_string_next(struct hyperdex_ds_iterator* iter,
                                            int64_t* key,
                                            const char** val, size_t* val_sz);
\end{ccode}
\funcdesc Return the next pair of (int, string) in the map.  This function will
return 1 if an element is returned, 0 if there are no elements to return, and -1
if the map is malformed.  The value stored in \texttt{*val} is a pointer into
the map and should not be free'd by the application.

\funcsep
\begin{ccode}
int hyperdex_ds_iterate_map_int_int_next(struct hyperdex_ds_iterator* iter,
                                         int64_t* key, int64_t* val);
\end{ccode}
\funcdesc Return the next pair of (int, int) in the map.  This function will
return 1 if an element is returned, 0 if there are no elements to return, and -1
if the map is malformed.

\funcsep
\begin{ccode}
int hyperdex_ds_iterate_map_int_float_next(struct hyperdex_ds_iterator* iter,
                                           int64_t* key, double* val);
\end{ccode}
\funcdesc Return the next pair of (int, float) in the map.  This function will
return 1 if an element is returned, 0 if there are no elements to return, and -1
if the map is malformed.

\funcsep
\begin{ccode}
int hyperdex_ds_iterate_map_float_string_next(struct hyperdex_ds_iterator* iter,
                                              double* key,
                                              const char** val, size_t* val_sz);
\end{ccode}
\funcdesc Return the next pair of (float, string) in the map.  This function
will return 1 if an element is returned, 0 if there are no elements to return,
and -1 if the map is malformed.  The value stored in \texttt{*val} is a pointer
into the map and should not be free'd by the application.

\funcsep
\begin{ccode}
int hyperdex_ds_iterate_map_float_int_next(struct hyperdex_ds_iterator* iter,
                                           double* key, int64_t* val);
\end{ccode}
\funcdesc Return the next pair of (float, int) in the map.  This function will
return 1 if an element is returned, 0 if there are no elements to return, and -1
if the map is malformed.

\funcsep
\begin{ccode}
int hyperdex_ds_iterate_map_float_float_next(struct hyperdex_ds_iterator* iter,
                                             double* key, double* val);
\end{ccode}
\funcdesc Return the next pair of (float, float) in the map.  This function will
return 1 if an element is returned, 0 if there are no elements to return, and -1
if the map is malformed.

\subsection{Memory Management Utilties}

The data structures API provides utility functions for allocating structures
from the arena, obviating the need to free them individually.

\funcsep
\begin{ccode}
struct hyperdex_client_attribute*
hyperdex_ds_allocate_attribute(struct hyperdex_ds_arena* arena, size_t sz);
\end{ccode}
\funcdesc Allocate an array of \texttt{struct hyperdex\_client\_attribute}.
On success, this function returns a non-null pointer containing \texttt{sz}
elements.  On failure, the function returns \texttt{NULL}, indicating that
memory allocation failed.  The memory will remain valid until the arena is
destroyed and should not be free'd independently by the application.

\funcsep
\begin{ccode}
struct hyperdex_client_attribute_check*
hyperdex_ds_allocate_attribute_check(struct hyperdex_ds_arena* arena, size_t sz);
\end{ccode}
\funcdesc Allocate an array of \texttt{struct hyperdex\_client\_attribute\_check}.
On success, this function returns a non-null pointer containing \texttt{sz}
elements.  On failure, the function returns \texttt{NULL}, indicating that
memory allocation failed.  The memory will remain valid until the arena is
destroyed and should not be free'd independently by the application.

\funcsep
\begin{ccode}
struct hyperdex_client_map_attribute*
hyperdex_ds_allocate_map_attribute(struct hyperdex_ds_arena* arena, size_t sz);
\end{ccode}
\funcdesc Allocate an array of \texttt{struct hyperdex\_client\_map\_attribute}.
On success, this function returns a non-null pointer containing \texttt{sz}
elements.  On failure, the function returns \texttt{NULL}, indicating that
memory allocation failed.  The memory will remain valid until the arena is
destroyed and should not be free'd independently by the application.
