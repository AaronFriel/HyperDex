\chapter{Working with Docker}
\label{chap:docker}

HyperDex supports running within Docker containers for easy deployment.  In this
chapter, we'll walk through the various ways to integrate HyperDex and Docker
for both testing and production environments.

Our first Docker environment contains an entirely self-sufficient HyperDex
cluster in a single container so that developers may quickly setup transient
single-node cluster for testing and prototyping purposes.  This sample
environment is fully feature-complete, and applications developed against such
an environment will work with distributed and sharded HyperDex clusters with
zero application level changes.

\section{Quickly Setup Transient Clusters}

Docker is a great tool for quickly testing and prototyping new ideas.  With
Docker, an application is isolated in its own container, unable to change the
workstation outside the container.  This makes it easy to rapidly deploy new
applications without worrying about having to cleanup or revert any changes they
may make to the OS.

For HyperDex, we can leverage containers to setup short-lived clusters that may
be used for one-off tasks, rapid prototyping, or for going through the HyperDex
tutorial---all without any commitment or modification to our base OS.  The
\code{hyperdex/quickstart} container includes everything necessary to run a
single-node HyperDex cluster, including the coordinator, daemon, and Python
bindings.

Running the HyperDex quickstart container is a simple three-step process:

\begin{enumerate}
\item We'll need Docker installed to use this container, which may be done by
following the \href{https://docs.docker.com/installation/}{Docker installation
instructions}.
\item We then fetch the HyperDex with the command \code{docker pull
hyperdex/quickstart}.
\item To start the cluster, run \code{docker run --net=host
hyperdex/quickstart}.  When all goes well, you'll see the following message
indicating your cluster is ready for use:

\begin{verbatim}
...
daemon: I1230 19:59:50.547586    21 daemon.cc:492] reconfiguration complete...
The transient HyperDex cluster is now online.

This is a transient HyperDex cluster.
You can connect to this cluster at address=172.17.3.27, port=1982.
To stop this cluster, run 'docker stop -t 0 b6ae7c60c275'.
\end{verbatim}
\end{enumerate}

With this quickstart cluster, both the coordinator and daemon logs will be
displayed on stdout.  You can connect to the cluster at athe address and port
listed, and can stop the cluster at any time using the command displayed on your
console.  When you stop the cluster, the HyperDex cluster will be destroyed, and
taken offline automatically.  To fully clean up any disk space in use by the
cluster, run \code{docker rm b6ae7c60c275}, where b6ae7c60c275 is the ID of your
quickstart container.
