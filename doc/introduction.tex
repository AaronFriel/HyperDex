\chapter{Introduction}
\label{chap:introduction}

\section{Why HyperDex?}
\label{sec:introduction:why}

HyperDex is the next-generation key-value store, designed from the ground up to
provide a convenient user interface without sacrificing reliability, robustness,
or performance.  First and foremost, HyperDex enables developers to create new
applications quickly and with strong correctness guarantees.
HyperDex features:

\begin{description}
\item[Performance:]  HyperDex achieves lower latency and higher throughput than
many other key-value stores.  Deploy fewer machines to support your application,
and save on operational costs.

\item[Consistency:]  HyperDex provides strongly-consistent reads, writes, and
atomic operations, making strong guarantees to your application.  Other
key-value stores promise \texttt{eventual} consistency, leaving your application
to pick up the slack.  HyperDex solves the tough problems, enabling you to write
correct applications quickly with less effort.

\item[Fault Tolerance:]  HyperDex employs a unique, built-in fault-tolerance
mechanism to safe-guard your data.  Simply tell HyperDex how many failures your
cluster is likely to experience, and HyperDex ensures that your data is
replicated to withstand at least that many failures.  Where other systems
require you to fully understand the replication internals to bring
fault-tolerance your data, HyperDex's fault-tolerance mechanism is simple to
understand and straightforward to use.

\item[Scalability:]  HyperDex automatically scales as more machines are added to
the system.  Add additional servers to your HyperDex cluster to increase storage
capacity and achievable throughput.

\item[Easy-to-Use:]  HyperDex is easy to use in both development and production
environments.  Set up a HyperDex cluster quickly, and scale out to multiple
machines without any application-level changes.  Other systems may require that
your application know about internals of the deployment, such as cluster
membership, replication options, and sharding configurations.  HyperDex
internally and automatically manages your cluster so your application doesn't
have to.
\end{description}

\section{About this Book}
\label{sec:introduction:about}

This book is the definitive HyperDex reference manual for users, administrators,
and developers.  The HyperDex developers update this manual in tandem with each
release to ensure that the book serves as the best reference for HyperDex.

This book is divided into three parts, each of which is directed to a different
audience.  Part~\ref{part:for-developers}, ``For Developers'', provides an
introduction to HyperDex suitable for developers looking to build applications
on top of HyperDex.  The ``For Developers'' part of this book is written as a
series of tutorials that acquaint you with HyperDex's API and describe many of
the features available within HyperDex.

Part~\ref{part:for-admins}, ``For Administrators'', describes the day-to-day
maintenance of a HyperDex cluster.  The ``For Administrators'' portion of this
book is written as a series of small, isolated recipes that provide instruction
for maintaining and monitoring HyperDex clusters.

%Part~\ref{part:for-hackers}, ``For Hackers'', should be of interest to
%individuals looking to hack on HyperDex itself.  This portion of the book
%provides pointers to enterprising individuals looking to extend, modify, or
%contribute to HyperDex development.

Finally, Part~\ref{part:api-ref} provides an API reference for HyperDex.  This
section serves primarily as a reference of the developer-visible, and
operations-visible, components of HyperDex and should probably be used for
reference rather than read in its entirety.

This book is a continuous work in progress, and newer versions are available at
the \href{http://hyperdex.org/}{HyperDex website}.  The book's sources are
maintained alongside the HyperDex source code as the de-facto documentation for
HyperDex.  Patches, suggestions, and feedback about this book are welcome and
may be submitted using the contact information in the next section.

\section{Help and Support}
\label{sec:introduction:support}

If you run into trouble while working with HyperDex, don't panic!  The HyperDex
team have put together many support avenues to assist you with HyperDex.  Many
of the available resources are freely available online.  When you need more
support than that, the HyperDex team is on standby with commercial support
options that ensure that you're never left in the dark.

\subsection{Online and Free Resources}
\label{sec:introduction:support:free}

The HyperDex team have assembled several free online resources to help you make
the most of HyperDex:

\begin{itemize}
\item HyperDex Reference Manual:  This document is the best avenue for support
    and assistance.  It serves as the canonical documentation of HyperDex and
    should be your first stop with any questions or problems.
\item \href{https://groups.google.com/group/hyperdex-discuss}{HyperDex Mailing List}:
    This open mailing list is most users' first resource solving problems.
    Messages to the list are archived which makes it easy to search for
    solutions to your problem.  If you don't find a solution in the archives,
    start a new thread with your question and the HyperDex developers will try
    to help answer your questions.
\item \href{http://webchat.freenode.net/?channels=hyperdex\&uio=d4}{HyperDex IRC Channel}:
    The \texttt{\#hyperdex} IRC channel on Freenode is a great place to ask
    questions and interact with the HyperDex developers in real-time.
\item \href{https://github.com/rescrv/HyperDex/issues}{HyperDex Bug Tracker}:
    If you've encountered an error in the code that you think may be a bug,
    check the bug tracker to see if other users have reported the same problem.
    Consider reporting the problem yourself if it looks like the problem has not
    yet been reported.
\item \href{http://hyperdex.org/FAQ/}{HyperDex FAQ}:
    If you've got a question, check out the FAQ for frequently asked (and
    answered) questions.  The FAQ contains pointers to other resources to help
    solve your problems.
\item \href{http://hyperdex.org/papers/}{HyperDex Papers}:
    If you're interested in developing or changing the internals of HyperDex,
    we've published papers that detail the internal architecture and design
    principles of HyperDex.
\end{itemize}

\subsection{Commercial Support}
\label{sec:introduction:support:commercial}

The HyperDex team provides commercial support for HyperDex, which includes
around-the-clock email support and consulting services.  If you're in need of
support, you can contact the HyperDex developers directly at {\em support}@{\em
hyperdex.org}.
